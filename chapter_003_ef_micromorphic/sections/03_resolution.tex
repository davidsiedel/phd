\section{Resolution schemes}
\label{sec:micromorphicdamage:alternate_minimisation}

\subsection{A first alternate minimization scheme}

\paragraph{Alternate scheme}

Observing that the Lagrangian \eqref{eq:micromorphicdamage:Lagrangian} is not convex \textit{globally}, but convex with respect to
each variable taken independently, an alternate minimization scheme is the spirit of that proposed by
Bourdin \textit{et al.} in \cite{bourdin_numerical_2000} is devised.
It consists in the following iterative scheme
%
%
%
\begin{equation}
  \left\{
    \begin{aligned}
      \iter{n+1}{\DisplacementField}
      &
      =
      \underset{\VirtualField{\DisplacementField}}{\argmin} \,
      \LagrangianOperator{\BodyLagrange}{\text{tot}}(
        \VirtualField{\DisplacementField},
        \iter{n}{\DamageField},
        \iter{n}{\MicromorphicDamageField}
      )
      \\
      \iter{n+1}{\MicromorphicDamageField}
      &
      =
      \underset{\VirtualField{\MicromorphicDamageField}}{\argmin} \,
      \LagrangianOperator{\BodyLagrange}{\text{tot}}(
        \iter{n+1}{\DisplacementField},
        \iter{n}{\DamageField},
        \VirtualField{\MicromorphicDamageField},
      )
      \\
      \iter{n+1}{\DamageField}
      &
      =
      \underset{\VirtualField{\DamageField}}{\argmin} \,
      \LagrangianOperator{\BodyLagrange}{\text{tot}}(
        \iter{n+1}{\DisplacementField},
        \VirtualField{\DamageField},
        \iter{n+1}{\MicromorphicDamageField},
      )
    \end{aligned}
  \right.
\end{equation}
% \[
% \left\{
% \begin{aligned}
% \iter{n+1}{\vec{u}} &= \underset{\vec{u}^{\star}\in C.A.}{\argmin}\, \mathcal{L}\paren{\vec{u}^{\star},\iter{n}{d},\iter{n}{d_{\chi}}}\\
% \iter{n+1}{d_{\chi}} &= \argmin \,\mathcal{L}\paren{\iter{n+1}{\vec{u}},\iter{n}{d},d_{\chi}^{\star}} \\
% \iter{n+1}{d} &= \argmin \,\mathcal{L}\paren{\iter{n+1}{\vec{u}},d^{\star},\iter{n+1}{d_{\chi}}} \\
% \end{aligned}
% \right.
% \]
where $\iter{n}{\DisplacementField}$, $\iter{n}{\DamageField}$ and
$\iter{n}{\MicromorphicDamageField}$ denote respectively the estimates of the
displacement field, micromorphic damage field and damage field at the
n\textsuperscript{th} iteration of the algorithm.
In particular, each step of the algorithm diminishes the value of the Lagrangian,
ensuring the convergence of the scheme.
%
%
%
Minimization with respect to $\DisplacementField$ and $\MicromorphicDamageField$ leads to
solving the two standard Partial Differential Equations (PDE)
\eqref{eq:ef_micromorphic:formulation:strong_problem_u:eq2}
and
\eqref{eq:ef_micromorphic:formulation:strong_problem_d:eq2}.
%
%
%
% Note that:
% %
% %
% %
% \begin{itemize}
%   \item The PDE associated with $\vec{u}$ is a linear elastic problem with
%   spatially variable mechanical coefficients. This problem becomes non
%   linear if unilateral effects are taken into account.
%   \item The PDE associated with $d_{\chi}$ is always linear and
%   $\iter{n+1}{d_{\chi}}$ is the solution of (see Equation
%   \eqref{eq:micromorphicdamage:d_chi2}):
%   \[
%   -A_{\chi}\,\LaplacianOperator\,\iter{n+1}{d_{\chi}}+H_{\chi}\,\iter{n+1}{d_{\chi}}=
%   H_{\chi}\,\iter{n}{d}
%   \]
%   Note that this PDE is close to the PDE describing heat transfer and can
%   thus be solved in most finite element solvers. See \cite{azinpour_simple_2018}
%   for an alternative implementation of the phase-field model based on this analogy.
% \end{itemize}

\paragraph{Displacement problem}

Considering a linear thermo-elastic material for the definition of the potential
$\psi_{\tensoriis{F}}$,
the PDE associated with $\DisplacementField$ is a linear elastic problem with
spatially variable mechanical coefficients. This problem becomes non
linear if unilateral effects are taken into account.

\paragraph{Micromorphic damage problem}

The PDE associated with $\MicromorphicDamageField$ is always linear and
$\iter{n+1}{\MicromorphicDamageField}$ is the solution of (see Equation
\eqref{eq:micromorphicdamage:d_chi2})
%
%
%
\begin{equation}
  -A_{\chi} \, \LaplacianOperator \, \iter{n+1}{\MicromorphicDamageField}
  +
  H_{\chi} \, \iter{n+1}{\MicromorphicDamageField}
  =
  H_{\chi} \, \iter{n}{\DamageField}
\end{equation}
%
%
%
Note that this PDE is close to the PDE describing heat transfer and can
thus be solved in most finite element solvers. See \cite{azinpour_simple_2018}
for an alternative implementation of the phase-field model based on this analogy.

% - The PDE associated with $\vec{u}$ is a linear elastic problem with
%   spatially variable mechanical coefficients. This problem becomes non
%   linear if unilateral effects are taking into account.
% - The PDE associated with $d_{\chi}$ is always linear and
%   $\iter{n+1}{d_{\chi}}$ is the solution of (see Equation
%   \eqref{eq:micromorphicdamage:d_chi2}):
%   \[
%   -A_{\chi}\,\LaplacianOperator\,\iter{n+1}{d_{\chi}}+H_{\chi}\,\iter{n+1}{d_{\chi}}=
%   H_{\chi}\,\iter{n}{d}
%   \]
%   Note that this PDE is close to the PDE describing heat transfer and can
%   thus be solved in most finite element solvers. See \cite{azinpour_simple_2018}
%   for an alternative implementation of the phase-field model based on this analogy.

\paragraph{Damage problem}

Minimization with respect to the damage $\DamageField$ is local and depends on the considered model.

% Appendix \ref{sec:micromorphicdamage:damage_evolution} describes
% the local equations that must be solved in each case.

\subsection{A second alternate minimization scheme}

Since an update of the damage variable is computationally inexpensive,
compared to the computation of the displacement and micromorphic damage,
one may consider evaluating its value twice, as follows:
%
%
%
\begin{equation}
  \left\{
    \begin{aligned}
      \iter{n+1}{\DisplacementField}
      &
      =
      \underset{\VirtualField{\DisplacementField}}{\argmin} \,
      \LagrangianOperator{\BodyLagrange}{\text{tot}}(
        \VirtualField{\DisplacementField},
        \iter{n}{\DamageField},
        \iter{n}{\MicromorphicDamageField}
      )
      \\
      \iter{n+1/2}{\DamageField}
      &
      =
      \underset{\VirtualField{\DamageField}}{\argmin} \,
      \LagrangianOperator{\BodyLagrange}{\text{tot}}(
        \iter{n+1}{\DisplacementField},
        \VirtualField{\DamageField},
        \iter{n}{\MicromorphicDamageField},
      )
      \\
      \iter{n+1}{\MicromorphicDamageField}
      &
      =
      \underset{\VirtualField{\MicromorphicDamageField}}{\argmin} \,
      \LagrangianOperator{\BodyLagrange}{\text{tot}}(
        \iter{n+1}{\DisplacementField},
        \iter{n+1/2}{\DamageField},
        \VirtualField{\MicromorphicDamageField},
      )
      \\
      \iter{n+1}{\DamageField}
      &
      =
      \underset{\VirtualField{\DamageField}}{\argmin} \,
      \LagrangianOperator{\BodyLagrange}{\text{tot}}(
        \iter{n+1}{\DisplacementField},
        \VirtualField{\DamageField},
        \iter{n+1}{\MicromorphicDamageField},
      )
    \end{aligned}
  \right.
\end{equation}
%
%
%
where the damage estimate $\iter{n+1/2}{d}$ is given by an appropriate
modification of Equation \eqref{eq:micromorphicdamage:damage_estimate}.

\subsection{A third alternate minimization scheme}
\label{sec:micromorphic:third_scheme}

The third alternate minimization scheme is based on the fact that the
minimization with respect to $\DamageField$ and $\MicromorphicDamageField$ is convex

\paragraph{Dependant variables}

Since the micromorphic damage field $\MicromorphicDamageField$ is an implicit functions of
the damage field $\DamageField$ (See Equation \eqref{eq:micromorphicdamage:damage_estimate}), the
total derivative of the micromorphic thermodynamic force $\MicromorphicDamageForceField$
is given by
%
%
%
\begin{equation}
  \label{eq:ef_micromorphic:formulation:implict_damage}
  \begin{aligned}
    \derivtot{\MicromorphicDamageForceField}{\MicromorphicDamageField}
    =
    \deriv{\MicromorphicDamageForceField}{\MicromorphicDamageField}
    +
    \deriv{\MicromorphicDamageForceField}{\DamageField}
    \deriv{\DamageField}{\MicromorphicDamageField}
    % 0
    % \mathfrak{R}_{T}(\mathfrak{U}_{T}(\mathfrak{U}_{\dCell}),
    % \mathfrak{U}_{\dCell}) = 0
  \end{aligned}
\end{equation}
%
%
%
where the expression of the derivative of the damage field with respect to the micromorphic damage
field $\partial \DamageField / \partial \MicromorphicDamageField$ is deduced from
\eqref{eq:micromorphicdamage:damage_estimate}.
As $\partial \DamageField / \partial \MicromorphicDamageField$ is only linear
by part, Equation \eqref{eq:ef_micromorphic:formulation:strong_problem_d:eq2} is indeed non-linear.
% its resolution is performed in this work using a Newton algorithm.

\paragraph{Non-linear micromorphic damage and displacement resolution scheme}

Consequently, the following iterative scheme is considered where the evolution of $\MicromorphicDamageField$ is still given by Equation
\eqref{eq:ef_micromorphic:formulation:strong_problem_d:eq2} but the where the determination of the conjugated
force $\MicromorphicDamageForceField$ is given by Equation
\eqref{eq:ef_micromorphic:formulation:implict_damage}.
% In practice, non-linear Equation \eqref{eq:ef_micromorphic:formulation:implict_damage} can be
% solved by an iterative method.
%
%
%
\begin{equation}
  \left\{
    \begin{aligned}
      \iter{n+1}{\DisplacementField}
      &
      =
      \underset{\VirtualField{\DisplacementField}}{\argmin} \,
      \LagrangianOperator{\BodyLagrange}{\text{tot}}(
        \VirtualField{\DisplacementField},
        \iter{n}{\DamageField},
        \iter{n}{\MicromorphicDamageField}
      )
      \\
      \iter{n+1}{\MicromorphicDamageField},
      \iter{n+1}{\DamageField}
      &
      =
      \underset{
        \VirtualField{\MicromorphicDamageField},
        \VirtualField{\DamageField}
      }{\argmin} \,
      \LagrangianOperator{\BodyLagrange}{\text{tot}}(
        \iter{n+1}{\DisplacementField},
        \VirtualField{\DamageField},
        \VirtualField{\MicromorphicDamageField},
      )
    \end{aligned}
  \right.
\end{equation}
%
%
%
% where the evolution of $\MicromorphicDamageField$ is still given by Equation
% \eqref{eq:ef_micromorphic:formulation:strong_problem_d:eq2} but the determination of the conjugated
% force $\MicromorphicDamageForceField$ relies locally on the resolution of Equation
% \eqref{eq:micromorphicdamage:damage_estimate}.
% As this equation is only linear
% by part, Equation \eqref{eq:ef_micromorphic:formulation:strong_problem_d:eq2}is indeed non linear and
% its resolution is performed in this work using a Newton algorithm.
% %
% %
% %
% To be more specific, the computation of the stiffness matrix associated
% with Problem \eqref{eq:ef_micromorphic:formulation:strong_problem_d:eq2}
% requires the derivative
% $\deriv{\MicromorphicDamageForceField}{\DamageField}$ which is piece-wise constant.
% In
% our numerical experiments, this Newton algorithm usually converges in
% less than 10 iterations.