\section{The micromorphic approach to brittle fracture}
\label{sec:micromorphicdamage:description}

\subsection{Description of the coupled problem}
\label{sec:ef_micromorphic:coupled_problem_descripion}

\paragraph{Solid body in the current configuration}

Let $\BodyEuler$ a solid body that is subjected to a volumetric load $\tensori{f}{}_v$ in
the current configuration at some time $t > 0$.
A displacement $\tensori{u}{}_{d}$ is prescribed
on the Dirichlet boundary $\BodyEulerDirichletBoundary$ and a surface load $\tensori{t}{}_{n}$ is imposed
on the Neumann boundary $\BodyEulerNeumannBoundary$.

\paragraph{Transformation mapping}

Let $\tensori{\Phi}{}$ the transformation mapping of the solid body from the initial configuration $\BodyLagrange$
to the current configuration $\BodyEuler$.
The displacement field $\DisplacementField$ is such
that $\DisplacementField = \tensori{\Phi}{} - \IdentityTensorI$ where $\IdentityTensorI$
is the identity application on $\BodyLagrange$.
The gradient of the transformation is
denoted $\TransformationGradientField = \nabla \tensori{\Phi}{} = \IdentityTensorII + \nabla \DisplacementField$.
Let $\BodyLagrangeDirichletBoundary$ and $\BodyLagrangeNeumannBoundary$ the images
of $\BodyEulerDirichletBoundary$ and $\BodyEulerNeumannBoundary$ respectively by $\tensori{\Phi}{}^{-1}$.

\paragraph{External loads in the reference configuration} 

In the reference configuration, the solid is subjected to a volumetric load
$\tensori{f}{}_V$, a prescribed displacement $\tensori{u}{}_D$ on $\BodyLagrangeDirichletBoundary$, and a surface load $\tensori{t}{}_N$ on $\BodyLagrangeNeumannBoundary$, where the volumetric and surface loads $\tensori{f}{}_V$ and $\tensori{t}{}_N$ have been obtained from their counterparts
$\tensori{f}{}_v$ and $\tensori{t}{}_n$ respectively, using Nanson formulaes. For the sake of simplicty, they are supposed to be independent
on $\tensorii{F}{}$.

\paragraph{State of the solid} The mechanical state of the solid body $\BodyLagrange$ is characterized by the displacement field $\DisplacementField$,
the damage field $\DamageField$ and a micromorphic damage field $\MicromorphicDamageField$.

\paragraph{Free energy potential}

The free energy potential $\psi_{\BodyLagrange}$ of the body $\BodyLagrange$ reads as a function of the displacement $\DisplacementField$, the (local) damage $d$ and the micromorphic damage $\MicromorphicDamageField$, in the form
%
%
%
\begin{equation}
    \psi_{\BodyLagrange}
    (\TransformationGradientField, \DamageField, \MicromorphicDamageField, \nabla \MicromorphicDamageField)
    =
    \psi_{\tensoriis{F}, \DamageField}
    (\TransformationGradientField, \DamageField)
    +
    \psi_{\DamageField}
    (\DamageField)
    +
    \psi_{\MicromorphicDamageField, \DamageField}
    (\MicromorphicDamageField, \DamageField)
    +
    \psi_{\nabla \MicromorphicDamageField}
    (\nabla \MicromorphicDamageField),
\end{equation}
%
%
%
where $\psi_{\tensoriis{F}, \DamageField}$ denotes the mechanical contribution that takes into account the damage in the medium,
$\psi_{\DamageField}$ is the energy stored during the fracture process,
$\psi_{\DamageField, \MicromorphicDamageField}$ is a coupling term between the damage and micromorphic damage variables, and
$\psi_{\nabla \MicromorphicDamageField}$ defines the micromorphic force.
In practice, $\psi_{\BodyLagrange}$ is a differentiable function.

\paragraph{Stresses}

In the spirit of \cite{forest_micromorphic_2009}, the following stresses are introduced
%
%
%
\begin{equation}
    \begin{aligned}
        \PKIStressField = \deriv{\psi_{\BodyLagrange}}{\TransformationGradientField}
        && &&
        \MicromorphicDamageStressField = \deriv{\psi_{\BodyLagrange}}{\nabla \MicromorphicDamageField}
        && &&
        \MicromorphicDamageForceField = \deriv{\psi_{\BodyLagrange}}{\MicromorphicDamageField}
        && &&
        \DamageForceField = -\deriv{\psi_{\BodyLagrange}}{\DamageField},
    \end{aligned}
\end{equation}
%
%
%
where $\PKIStressField$ is the first Piola-Kirchoff stress tensor, and $\MicromorphicDamageStressField, \MicromorphicDamageForceField$ and $\DamageForceField$ are the thermodynamic
forces associated to $\nabla \MicromorphicDamageField, \MicromorphicDamageField$ and $\DamageField$ respectively.

\paragraph{Dissipation potential}

A dissipation potential $\phi_{\BodyLagrange}(\DamageField)$ accounts for the irreversibility of the fracture process in the medium.
For time independent mechanisms, $\phi_{\BodyLagrange}$ is usually not differentiable. However, it is is assumed to be
an homogeneous function of degree one
such that for any time increment $\Delta \, t > 0$, and any admissible damage field $\DamageField$ at the time $t + \Delta \, t$
%
%
%
\begin{equation}
  \label{eq:ef_micromorphic:formulation:dissipation_potential}
  \begin{aligned}
      \Delta \, t \, \phi_{\BodyLagrange}(\frac{\DamageField - \DamageField \TraceOperator{t}} {\Delta \, t})
      =
      \phi_{\BodyLagrange}(\DamageField - \DamageField \TraceOperator{t}).
  \end{aligned}
\end{equation}
%
%
%
In particular, $\phi_{\bodyLag}(d)$ generally contains an indicator function to enforce the irreversibility of the
damage evolution.

\subsection{Lagrangian formulation of the coupled problem}
\label{sec:ef_micromorphic:coupled_problem_lagrangien}

\paragraph{Total Lagrangian}

In the view of generalized standard materials
\cite{moreau_sur_1970, halphen_sur_1975, ehrlacher_principe_1985, nguyen_standard_2002},
and for any time increment $\Delta \, t > 0$,
the equilibrium of the solid $\BodyLagrange$ at the time $t + \Delta \, t$ is characterized by the stationarity of the total incremental Lagrangian
$\LagrangianOperator{\BodyLagrange}{\text{tot}}$ \cite{lorentz_variational_1999,forest_localization_2004}
such that
%
%
%
\begin{equation}
  \label{eq:ef_micromorphic:formulation:total_lagrangian_1}
  \begin{aligned}
    \LagrangianOperator{\BodyLagrange}{\text{tot}}
    =
    &
    \int_{\BodyLagrange}
    \psi_{\BodyLagrange}
    (\TransformationGradientField, \DamageField, \MicromorphicDamageField, \nabla \MicromorphicDamageField)
    +
    \int_{\BodyLagrange} \Delta \, t \, \phi_{\BodyLagrange}(\frac{\DamageField - \DamageField \TraceOperator{t}} {\Delta \, t})
    -
    \int_{\BodyLagrange} \loadLag{} \cdot \DisplacementField
    -
    \int_{\BodyLagrangeNeumannBoundary} \neumannLag \cdot \DisplacementField \TraceOperator{\BodyLagrangeNeumannBoundary},
  \end{aligned}
\end{equation}
%
%
%
for any admissible displacement, damage and micromorphic damage fields $\DisplacementField, \DamageField$
and $\MicromorphicDamageField$ at the time $t + \Delta \, t$.
%
%
%
To satisfy the Ilyushin-Drucker postulate, the total incremental Lagrangian
$\LagrangianOperator{\BodyLagrange}{\text{tot}}$ must be convex with respect to each variables
$\DisplacementField, \DamageField$
and $\MicromorphicDamageField$ taken independently. It
can be shown that this condition is ensured if the free energy potential $\psi_{\BodyLagrange}$ and
the dissipation potential $\phi_{\BodyLagrange}$ are convex with
respect to their respective arguments.

\paragraph{Total Lagrangian split}

Deriving equilibrium equations from the principle of the minimum of the
Lagrangian is non trivial due to the fact that the dissipation potential
is not differentiable.
%
%
%
It is then convenient to separate the total Lagrangian $\LagrangianOperator{\BodyLagrange}{\text{tot}}$ into two parts
$\LagrangianOperator{\BodyLagrange}{\text{hel}}$ and $\LagrangianOperator{\BodyLagrange}{\text{dis}}$ as follows:
%
%
%
\begin{subequations}
  \label{eq:ef_micromorphic:formulation:total_lagrangian_split}
      \begin{alignat}{3}
        \LagrangianOperator{\BodyLagrange}{\text{hel}}
        &
        =
        \int_{\BodyLagrange} \FreeEnergy_{\BodyLagrange}(\TransformationGradientField, \DamageField, \MicromorphicDamageField, \nabla \MicromorphicDamageField)
        -
        \int_{\BodyLagrange} \loadLag{} \cdot \DisplacementField
        -
        \int_{\BodyLagrangeNeumannBoundary} \neumannLag \cdot \DisplacementField \TraceOperator{\BodyLagrangeNeumannBoundary},
        \label{eq:ef_micromorphic:formulation:total_lagrangian_split:eq0}
        \\
        \LagrangianOperator{\BodyLagrange}{\text{dis}}
        &
        = 
        \int_{\BodyLagrange} \phi_{\BodyLagrange}(\DamageField - \DamageField \TraceOperator{t}),
        \label{eq:ef_micromorphic:formulation:total_lagrangian_split:eq1}
  \end{alignat}
\end{subequations}
%
%
%
where \eqref{eq:ef_micromorphic:formulation:dissipation_potential} has been used in \eqref{eq:ef_micromorphic:formulation:total_lagrangian_split:eq1}.

% \subsubsection{Regular part of the Lagrangian}
% \label{sec:ef_micromorphic:lagrangian_regular_part}

% In this work, the following mathematical results \cite{son_standard_2021} characterizing
% the minima of Lagrangian \eqref{eq:ef_micromorphic:formulation:total_lagrangian_1} are assumed :

% \paragraph{Regular part of the Lagrangian}

% The regular part of the Lagrangian $\LagrangianOperator{\BodyLagrange}{\text{hel}}$ is minimal with
% respect to the displacement field $\DisplacementField$ and the micromorphic
% damage field $\MicromorphicDamageField$.

% \subsubsection{Dissipative part of the Lagrangian}
% \label{sec:ef_micromorphic:lagrangian_regular_part}

% \paragraph{Dissipative part of the Lagrangian}

% At each point, the thermodynamic force $\DamageForceField$ associated with the
% damage is in the subgradient of the dissipation potential:
% %
% %
% %
% \begin{equation}
%   \label{eq:ef_micromorphic:formulation:generalized_standard_0}
%   \DamageForceField \in \partial \phi_{\BodyLagrange}
% \end{equation}

\subsection{The displacement and micromorphic damage problem}
\label{sec:ef_micromorphic:lagrangian_regular_part}

In this section, the condition on the regular part \eqref{eq:ef_micromorphic:formulation:total_lagrangian_split:eq0} of
the Lagrangian \eqref{eq:ef_micromorphic:formulation:total_lagrangian_1} is considered only.
The evolution of the damage is discussed in Section
\ref{sec:ef_micromorphic:formulation:damage_evolution}.

\paragraph{Lagrangian variations}

Following \cite{son_standard_2021}, the solid $\BodyLagrange$ is in equilibrium if
the regular part of the Lagrangian $\LagrangianOperator{\BodyLagrange}{\text{hel}}$ is minimal with
respect to the displacement field $\DisplacementField$ and the micromorphic
damage field $\MicromorphicDamageField$.
% Hence, the solution $(\DisplacementField, \MicromorphicDamageField)$
% satisfying the mechanical equilibrium minimizes the Lagragian $\LagrangianOperator{\BodyLagrange}{\text{hel}}$.
The first order variations of the Lagrangian \eqref{eq:ef_micromorphic:formulation:total_lagrangian_split:eq0} with respect to $\DisplacementField$ and $\MicromorphicDamageField$
respectively yields the weak equations
%
%
%
\begin{subequations}
    \label{eq:ef_micromorphic:formulation:lagragian_variations}
    \begin{alignat}{3}
      \langle \deriv{\LagrangianOperator{\BodyLagrange}{\text{hel}}}{\DisplacementField} , \delta \DisplacementField \rangle
      =
      & \int_{\BodyLagrange} \PKIStressField : \nabla \delta \DisplacementField
      -
      \int_{\BodyLagrange} \tensori{f}_V \cdot \delta \DisplacementField
      -
      \int_{\BodyLagrangeNeumannBoundary} \neumannLag \cdot \delta \DisplacementField \TraceOperator{\BodyLagrangeNeumannBoundary}
      &&
      \qquad
      &&
      \forall \delta \DisplacementField,
      \label{eq:ef_micromorphic:formulation:lagragian_variations:eq0}
      \\
      \langle \deriv{\LagrangianOperator{\BodyLagrange}{\text{hel}}}{\MicromorphicDamageField} , \delta \MicromorphicDamageField \rangle
      =
      & \int_{\BodyLagrange} \MicromorphicDamageForceField \, \delta \MicromorphicDamageField + \int_{\BodyLagrange} \MicromorphicDamageStressField \cdot \nabla \MicromorphicDamageField
      &&
      \qquad
      && \forall \delta \MicromorphicDamageField.
      \label{eq:ef_micromorphic:formulation:lagragian_variations:eq3}
    \end{alignat}
\end{subequations}

\paragraph{Strong equation}

Applying the divergence theorem, the following strong equations for the sole displacement problem are deduced from the weak equation
\eqref{eq:ef_micromorphic:formulation:lagragian_variations:eq0}
%
%
%
\begin{subequations}
    \label{eq:ef_micromorphic:formulation:strong_problem_u}
    \begin{alignat}{3}
    \nabla \cdot \PKIStressField + \tensori{f}{}_{V} & = 0
    &&
    \qquad
    &&
    \textit{balance of momentum},
    \label{eq:ef_micromorphic:formulation:strong_problem_u:eq2}
    \\
    \PKIStressField \cdot \tensori{n}{} - \neumannLag{} & = 0
    &&
    \qquad
    &&
    \textit{continuity of the normal stress}.
    \label{eq:ef_micromorphic:formulation:strong_problem_u:eq3}
    \end{alignat}
\end{subequations}
%
%
%
Similarly, the strong equations governing the sole micromorphic damage problem are deduced from
\eqref{eq:ef_micromorphic:formulation:lagragian_variations:eq3}
%
%
%
\begin{subequations}
    \label{eq:ef_micromorphic:formulation:strong_problem_d}
    \begin{alignat}{3}
        \nabla \cdot \MicromorphicDamageStressField - \MicromorphicDamageForceField & = 0
        &&
        \qquad
        &&
        \textit{balance of micromorphic damage momentum},
        \label{eq:ef_micromorphic:formulation:strong_problem_d:eq2}
        \\
        \MicromorphicDamageStressField \cdot \tensori{n}{} & = 0
        &&
        \qquad
        &&
        \textit{micromorphic damage boundary conditions},
        \label{eq:ef_micromorphic:formulation:strong_problem_d:eq3}
    \end{alignat}
\end{subequations}
%
%
%
where the governing laws of the micromorphic damage variable define a generalized continuum medium as introduced in \cite{forest_micromorphic_2009}.

\subsubsection{Evolution of the displacement}
\label{sec:ef_micromorphic:formulation:displacement_evolution}

\paragraph{Mechanical potential}

$\psi_{\tensoriis{F}, \DamageField}$ determines the expression of the stress and contributes
to the thermodynamic force driving the damage evolution.
A classical choice consists in multiplying the free energy of an undamaged
material $\psi_{\tensoriis{F}}$ by a degradation
function $g(\DamageField)$ such that
%
%
%
\begin{subequations}
    \label{eq:micromorphicdamage:freeenergygel}
    \begin{alignat}{3}
      \psi_{\tensoriis{F}, \DamageField}
      (\TransformationGradientField, \DamageField)
      =
      &
      g(\DamageField) \, \psi_{\tensoriis{F}}(\TransformationGradientField),
      \label{eq:micromorphicdamage:freeenergygel:eq0}
      \\
      \PKIStressField
      =
      &
      g(\DamageField) \, \deriv{\psi_{\tensoriis{F}}}{\TransformationGradientField}.
      \label{eq:micromorphicdamage:freeenergygel:eq1}
    \end{alignat}
\end{subequations}

\paragraph{Mechanical potential deviatoric split}
\label{sec:micromorphic:deviatori_split}

In order to account for the possible dependance of the damage driving force on the stress state,
% Since fracture is mostly a pressure-driven phenomenon (which occurs \textit{e.g.} for high triaxial loads),
a decomposition \cite{alessi_gradient_2015,miehe_phase_2010} of the mechanical free energy density $\psi_{\tensoriis{F}}$
into a deviatoric part $\psi_{\tensoriis{F}}{}_{\text{dev}}$ and into a spherical part
$\psi_{\tensoriis{F}}{}_{\text{sph}}$ is commonly assumed.
The degradation function
$g(\DamageField)$
then acts on either one the two components $\psi_{\tensoriis{F}}{}_{\text{dev}}$ or $\psi_{\tensoriis{F}}{}_{\text{sph}}$
hence defining either \textit{shear driven fracture} or \textit{pressure driven fracture}

\paragraph{Mechanical potential spectral split}

Another classical choice proposed by Miehe \cite{miehe_phase_2010} consists in
using a spectral decomposition of the strain to split the free energy into
a positive $\psi_{\tensoriis{F}}{}_{\text{+}}$ and a negative $\psi_{\tensoriis{F}}{}_{\text{-}}$ part.
In this case, the degradation function is only applied
to the positive part $\psi_{\tensoriis{F}}{}_{\text{+}}$ of the free energy, in order to account
for \textit{traction driven fracture} processes.

\paragraph{Micromorphic damage coupling potential}

The free energy density $\psi_{\MicromorphicDamageField, \DamageField}$ is chosen such that it penalizes
the difference between
the damage field $\DamageField$ and the micromorphic damage field $\MicromorphicDamageField$ such that
%
%
%
\begin{subequations}
    \label{eq:micromorphicdamage:achi}
    \begin{alignat}{3}
      \psi_{\MicromorphicDamageField, \DamageField}
      (\MicromorphicDamageField, \DamageField)
      =
      &
      \Frac{H_{\chi}}{2} \, \paren{\DamageField - \MicromorphicDamageField}^{2},
      \label{eq:micromorphicdamage:achi:eq0}
      \\
      \MicromorphicDamageForceField
      =
      &
      - H_{\chi} \, \paren{\DamageField - \MicromorphicDamageField}.
      \label{eq:micromorphicdamage:achi:eq1}
    \end{alignat}
\end{subequations}

\subsubsection{Evolution of the micromorphic damage}
\label{sec:ef_micromorphic:formulation:micromorphic_damage_evolution}

\paragraph{Micromorphic damage gradient potential}

The free energy density $\psi_{\nabla \MicromorphicDamageField}$ is chosen such that
it penalizes the localization of the micromorphic damage field $\MicromorphicDamageField$ and
%
%
%
\begin{subequations}
    \label{eq:micromorphicdamage:bchi}
    \begin{alignat}{3}
      \psi_{\nabla \MicromorphicDamageField}
      (\nabla \MicromorphicDamageField)
      =
      &
      \Frac{A_{\chi}}{2} \, \nabla \MicromorphicDamageField \cdot \nabla \MicromorphicDamageField,
      \label{eq:micromorphicdamage:bchi:eq0}
      \\
      \MicromorphicDamageStressField
      =
      &
      A_{\chi} \, \nabla \MicromorphicDamageField.
      \label{eq:micromorphicdamage:bchi:eq1}
    \end{alignat}
\end{subequations}

\paragraph{Damage and micromorphic damage relation}

Substituting \eqref{eq:micromorphicdamage:achi:eq1} and \eqref{eq:micromorphicdamage:bchi:eq1} in
the expression of the governing equation \eqref{eq:ef_micromorphic:formulation:strong_problem_d:eq2}
shows that the micromorphic damage $\MicromorphicDamageField$ satisfies the Laplace equation \cite{forest_micromorphic_2009}
%
%
%
\begin{equation}
  \label{eq:micromorphicdamage:d_chi2}
  A_{\chi} \, \LaplacianOperator \, \MicromorphicDamageField
  +
  H_{\chi} \, \paren{\DamageField - \MicromorphicDamageField}
  =
  0,
\end{equation}
%
%
%
where $\LaplacianOperator$ denotes the Laplacian operator.

\subsection{The damage problem}
\label{sec:ef_micromorphic:formulation:damage_evolution}

\paragraph{Damage driving force}

Following the definition of mechanical and micromorphic damage related potentials given in
Section \ref{sec:ef_micromorphic:lagrangian_regular_part}, the expression of the thermodynamic
force $\DamageForceField$ associated with the damage field is given by:
%
%
%
\begin{equation}
  \label{eq:micromorphicdamage:Y}
  \begin{aligned}
    \DamageForceField
    &
    =
    -
    \derivtot{g}{d} \, \psi_{\tensoriis{F}} (\TransformationGradientField)
    -
    \derivtot{\psi_{\DamageField}}{\DamageField}
    -
    \deriv{\psi_{\MicromorphicDamageField, \DamageField}}{\DamageField},
    \\
    &
    =
    -
    \derivtot{g}{d} \, \psi_{\tensoriis{F}} (\TransformationGradientField)
    -
    \derivtot{\psi_{\DamageField}}{\DamageField}
    +
    a_{\chi},
  \end{aligned}
\end{equation}
%
%
%
where \eqref{eq:micromorphicdamage:achi:eq0} has been used.

\paragraph{Dissipative part of the Lagrangian}

Following \cite{son_standard_2021,moreau_sur_1970,halphen_sur_1975}, at each point the thermodynamic force $\DamageForceField$ associated with the
damage is assumed to be in the subgradient of the dissipation potential:
%
%
%
\begin{equation}
  \label{eq:ef_micromorphic:formulation:generalized_standard_0}
  \DamageForceField \in \partial \phi_{\BodyLagrange}.
\end{equation}

\paragraph{Dissipation potential}

A simple choice for the dissipation potential is
%
%
%
\begin{equation}
  \label{eq:micromorphicdamage:dissipationpotential}
  \begin{aligned}
    \phi_{\BodyLagrange} (\dot{\DamageField})
    =
    Y_{0} \, \dot{\DamageField}
    +
    \mathrm{I}_{\mathbb{R}_{+}} (\dot{\DamageField}),
  \end{aligned}
\end{equation}
%
%
%
where $Y_0$ is a scalar threshold value, and $\mathrm{I}_{\mathbb{R}_{+}}$ denotes the characteristic function associated with positive real number such that
%
%
%
\begin{equation}
  \mathrm{I}_{\mathbb{R}_{+}} (x)
  =
  \left\{
    \begin{aligned}
    0 & \quad \text{if} \quad x \geq 0
    \\
    +\infty & \quad \text{if} \quad x < 0
    \end{aligned}
  \right.
\end{equation}

\paragraph{Yield surface}

The dissipation potential \eqref{eq:micromorphicdamage:dissipationpotential} is equivalent to define
the damage surface
%
%
%
\begin{equation}
  \label{eq:micromorphicdamage:yield}
  Y = Y_{0},
\end{equation}
%
%
%
leading to an evolution of the damage driven by the following Karush–Kuhn–Tucker conditions
%
%
%
\begin{subequations}
    \label{eq:micromorphicdamage:KTT}
    \begin{alignat}{3}
      \Delta \, \DamageField \, (\DamageForceField - Y_{0})
      &
      =
      0,
      \label{eq:micromorphicdamage:KTT:eq0}
      \\
      \Delta \, \DamageField
      &
      \geq
      0,
      \label{eq:micromorphicdamage:KTT:eq1}
      \\
      \DamageForceField - Y_{0}
      &
      \leq
      0,
      \label{eq:micromorphicdamage:KTT:eq2}
    \end{alignat}
\end{subequations}
%
%
%
Combining equations \eqref{eq:micromorphicdamage:d_chi2},
\eqref{eq:micromorphicdamage:Y} and \eqref{eq:micromorphicdamage:yield}, the yield
surface may also be written:
%
%
%
\begin{equation}
  \label{eq:micromorphicdamage:damage_estimate}
  \derivtot{g}{d} \, \psi_{\tensoriis{F}} (\TransformationGradientField)
  =
  Y_{0}
  +
  \derivtot{\psi_{\DamageField}}{\DamageField}
  -
  H_{\chi}(\DamageField - \MicromorphicDamageField),
\end{equation}
%
%
%
which yields that $\DamageField$ is an implicit function of $\MicromorphicDamageField$