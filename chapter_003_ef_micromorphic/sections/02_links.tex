\section{Link with classical models of brittle fracture}

\paragraph{Total resulting Lagrangian}

Choices described by equations \eqref{eq:micromorphicdamage:freeenergygel},
\eqref{eq:micromorphicdamage:dissipationpotential} and
\eqref{eq:micromorphicdamage:bchi:eq0} lead to the following expression
of the total Lagrangian \eqref{eq:ef_micromorphic:formulation:total_lagrangian_1}
%
%
%
\begin{equation}
  \label{eq:micromorphicdamage:Lagrangian}
  \begin{aligned}
    \LagrangianOperator{\BodyLagrange}{\text{tot}}
    =
    &
    \int_{\BodyLagrange} g(\DamageField) \, \psi_{\tensoriis{F}} (\TransformationGradientField)
    +
    \int_{\BodyLagrange} \psi_{\DamageField}(\DamageField)
    +
    \int_{\BodyLagrange} Y_{0} \, \DamageField
    +
    \int_{\BodyLagrange} \Frac{A_{\chi}}{2} \, \nabla \MicromorphicDamageField \cdot \nabla \MicromorphicDamageField
    \\
    &
    +
    \int_{\BodyLagrange} \Frac{H_{\chi}}{2} \, \paren{\DamageField - \MicromorphicDamageField}^{2}
    +
    \int_{\BodyLagrange} \mathrm{I}_{\mathbb{R}_{+}} (\DamageField - \DamageField \TraceOperator{t})
    \\
    &
    -
    \int_{\BodyLagrange} \loadLag{} \cdot \DisplacementField
    -
    \int_{\BodyLagrangeNeumannBoundary} \neumannLag \cdot \DisplacementField \TraceOperator{\BodyLagrangeNeumannBoundary},
  \end{aligned}
\end{equation}
%
%
%
where the constant term $Y_{0} \, \DamageField \TraceOperator{t}$, which does not have
any influence of the solution, has been removed from the expression of
the Lagrangian.

\paragraph{Equal damage and micromorphic field limit case}

For high values of $H_{\chi}$, the contribution to the
potential $\psi_{\MicromorphicDamageField, \DamageField}$
may be seen as a
penalization term which ensures that the damage $\DamageField$ and the
micromorphic damage $\MicromorphicDamageField$ are equal in a weak sense.
Intuitively, if $H_{\chi} \rightarrow \infty$, these two must become
equal to ensure a finite energy.
The Lagrangian \eqref{eq:micromorphicdamage:Lagrangian} is thus expected to have the following limit
%
%
%
\begin{equation}
  \label{eq:micromorphicdamage:LagrangianLimit}
  \begin{aligned}
    \LagrangianOperator{\BodyLagrange}{\text{tot}}
    =
    &
    \int_{\BodyLagrange} g(\DamageField) \, \psi_{\tensoriis{F}} (\TransformationGradientField)
    +
    \int_{\BodyLagrange} \psi_{\DamageField}(\DamageField)
    +
    \int_{\BodyLagrange} Y_{0} \, \DamageField
    +
    \int_{\BodyLagrange} \Frac{A_{\chi}}{2} \, \nabla \DamageField \cdot \nabla \DamageField
    \\
    &
    +
    \int_{\BodyLagrange} \mathrm{I}_{\mathbb{R}_{+}} (\DamageField - \DamageField \TraceOperator{t})
    \\
    &
    -
    \int_{\BodyLagrange} \loadLag{} \cdot \DisplacementField
    -
    \int_{\BodyLagrangeNeumannBoundary} \neumannLag \cdot \DisplacementField \TraceOperator{\BodyLagrangeNeumannBoundary},
  \end{aligned}
\end{equation}
%
%
%
and Lagrangian \eqref{eq:micromorphicdamage:LagrangianLimit} can be identified with
Lagrangians describing many classical models of brittle fracture with
appropriate choices of $g(\DamageField)$, $\psi_{\DamageField}$,
$A_{\chi}$ and $Y_{0}$.

\subsection{Link with Ambrosio–Tortorelli regularization models}

% \paragraph{Ambrosio–Tortorelli regularization}

% Several works have shown that the solutions of (2.3) converge in the sense of the so-called Γ-
% convergence, to the solution of the initial Problem (2.1) in some specific cases. For example,
% [Bourdin et al., 2000] studied the anti-plane shear case.
% Those results bridges the conceptual gap between the global approach of fracture based
% on Griffith’ theory and the local approach to fracture, a least for a certain class of non local
% models.

The Ambrosio–Tortorelli \cite{ambrosio_approximation_1990} regularization of the initial
variational approach to fracture \cite{francfort_revisiting_1998},
originally introduced in \cite{bourdin_numerical_2000}, have become
widely used in the computational mechanics community.
Several works (See \textit{e.g.} \cite{bourdin_numerical_2000}) have shown that the regularized solution converges
in the sense of the so-called $\Gamma$-
convergence, to the solution of the initial problem in some specific cases.

\paragraph{Values of the AT1 model}

Two regularization models were introduced in \cite{ambrosio_approximation_1990}.
The first one, namely the AT1 model, allows to recover the initial problem formulated in \cite{francfort_revisiting_1998}
for infinitely small values of some characteristic length $\ell_c$ defining the thickness of the smeared crack, and
such that initiation of damage occurs on the onset of some yield criterion. Choosing the following potentials and values
for the definition of the micromorphic model allows to recover the AT1 regularization model
%
%
%
\begin{subequations}
  \label{eq:ef_micromorphic:formulation:AT1_link}
  \begin{alignat}{3}
    g(\DamageField)
    &
    =
    (1 - \DamageField)^2,
    \label{eq:ef_micromorphic:formulation:AT1_link:eq0}
    \\
    \psi_{\DamageField}(\DamageField)
    &
    =
    \Frac{3 \, G_{c}}{8 \, \ell_{c}} \, \DamageField,
    \label{eq:ef_micromorphic:formulation:AT1_link:eq1}
    \\
    A_{\chi}
    &
    =
    \Frac{3}{4} \, G_{c} \, \ell_{c},
    \label{eq:ef_micromorphic:formulation:AT1_link:eq2}
    \\
    Y_{0}
    &
    =
    0,
    \label{eq:ef_micromorphic:formulation:AT1_link:eq3}
  \end{alignat}
\end{subequations}
%
%
%
where $G_{c}$ is a fracture energy, and $\ell_{c}$ is the characteristic length.

\paragraph{Values of the AT2 model}

The second regularization model in \cite{ambrosio_approximation_1990} introduces a quadratic fracture energy potential,
such that damage occurs on the onset of deformation of the medium. Choosing the following potentials and values
for the definition of the micromorphic model allows to recover the AT2 regularization model
%
%
%
\begin{subequations}
  \label{eq:ef_micromorphic:formulation:AT2_link}
  \begin{alignat}{3}
    g(\DamageField)
    &
    =
    (1 - \DamageField)^2,
    \label{eq:ef_micromorphic:formulation:AT2_link:eq0}
    \\
    \psi_{\DamageField}(\DamageField)
    &
    =
    \Frac{G_{c}}{2 \, \ell_{c}} \, \DamageField,
    \label{eq:ef_micromorphic:formulation:AT2_link:eq1}
    \\
    A_{\chi}
    &
    =
    G_{c} \, \ell_{c},
    \label{eq:ef_micromorphic:formulation:AT2_link:eq2}
    \\
    Y_{0}
    &
    =
    0.
    \label{eq:ef_micromorphic:formulation:AT2_link:eq3}
  \end{alignat}
\end{subequations}

\subsection{Link with Lorentz model}

% \paragraph{The Lorentz model}

\paragraph{Values of Lorentz model}

As mentioned above, AT1 and AT2 regularization models proposed by \cite{bourdin_numerical_2000}
introduce the characteristic length $\ell_c$, that controls the thickness of the smeared crack.
As stated in \cite{pham_approche_2010-1, pham_construction_2010}, this regularization length
defines the yield strength of the medium. In numerical applications, the strength of the material
is thus mesh dependant, since the value of characteristic length must
be greater than the element size in order to be captured.
Therefore, Lorentz \textit{et. al.} \cite{lorentz_gradient_2011,lorentz_convergence_2011}
alleviated this constraint by proposing an extension of the regularization presented in
\cite{bourdin_numerical_2000}, that introduces
a state variable $\gamma$ related to the yield strength of the material. The following values for the
proposed micromorphic approach allow to recover Lorentz's gradient damage model
%
%
%
\begin{subequations}
  \label{eq:ef_micromorphic:formulation:Lorentz_link}
  \begin{alignat}{3}
    g(\DamageField)
    &
    =
    \paren{\Frac{1 - \DamageField}{1 + \gamma \, \DamageField}}^{2},
    \label{eq:ef_micromorphic:formulation:Lorentz_link:eq0}
    \\
    \psi_{\DamageField}(\DamageField)
    &
    =
    \Frac{3 \, G_{c}}{8 \, \ell_{c}} \, \DamageField,
    \label{eq:ef_micromorphic:formulation:Lorentz_link:eq1}
    \\
    A_{\chi}
    &
    =
    \Frac{3}{4} \, G_{c} \, \ell_{c},
    \label{eq:ef_micromorphic:formulation:Lorentz_link:eq2}
    \\
    Y_{0}
    &
    =
    0,
    \label{eq:ef_micromorphic:formulation:Lorentz_link:eq3}
  \end{alignat}
\end{subequations}
%
%
%
where $\gamma$ is the aforementioned parameter of the Lorentz' model.

\paragraph{Alternative choices regarding the role of the fracture energy}

Note that an alternative choice can be made for both AT1 and Lorentz' models such that
%
%
%
\begin{equation}
  \begin{aligned}
    \psi_{\DamageField}(\DamageField) = 0
    &&
    \text{and}
    &&
    Y_{0} = \Frac{3}{8} \, G_{c} \, \ell_{c},
  \end{aligned}
\end{equation}
%
%
%
While leading to the same Lagrangian, this alternative choice has a totally
different physical meaning, since part of the fracture energy is now
considered as dissipated rather than stored. Since neither crack healing
nor coupling with heat transfer are considered, those choices are
equivalent in the context of this paper.