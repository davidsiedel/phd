\section{Conclusions and perspectives}

This work has investigated the use of micromorphic behaviours for the
description of quasi-brittle materials and has shown that those
micromorphic behaviours can be considered as varitionaly consistent
approximations of standard phase-field models.
%
%
%
Three alternate minimization schemes, which are straightforward to
implement in standard FEM or FFT solvers, have been proposed, and their numerical performance has been investigated intensively using representative tests. 
In an industrial context, numerical experiments showed that the proposed micromorphic approach is competitive with the phase field approach
since it allows to deal with
the damage irreversibility constraint at quadrature points, thus avoiding the need to deploy supplementary Lagrange multipliers.
%
%
%
Convergence of those schemes is guaranteed but requires a large number
of fixed-point iterations. Regarding this observation, the third
scheme appears to be more efficient.

The proposed approach can be extended to more complex damage behaviours
and ductile failure.
%
%
%
Finally, as a future work, acceleration schemes could also be
investigated to reduce the number of fixed-point iterations.
