\section{Introduction}

The variational approach to brittle fracture takes its roots in the work of
Francfort and Marigo \cite{francfort_revisiting_1998,francfort_vers_2002},
which recasted the Griffith theory into an energy
minimization problem.
This revisted approach of the Griffith theory is however not tractable
with standard numerical methods
\cite{bourdin_numerical_2000, chambolle_approximation_2018}, in particular
the commonly used finite element method. For this reason, Bourdin \textit{et
al.} developed regularized versions \cite{bourdin_numerical_2000} following
the works of Ambrosio and Tortorelli \cite{ambrosio_approximation_1990}.

\paragraph{Phase field approach to brittle fracture and damage irreversibility consition}

The so-called phase-field approaches to fracture have since become
widely popular. As pointed by Gerasimov and De Lorenzis in their
excellent review \cite{gerasimov_numerical_2020}, one of the main
difficulties in the implementation of those approaches is the treatment
of the irreversibility constraint (the damage can only increase), a
question on which a considerable amount of works has been published
(See \textit{e.g.}
\cite{Bourdin2014, Laurenzis2020, DelPietroLancioniMarch2020, Chambolle2018, Gerasimov2019, Bourdin2008}).
Most of the proposed solutions are not directly implemented in standard
FEM or FFT solvers. An noticeable exception to this statement is the
Miehe' alternative based an the so-called history variable
\cite{miehe_phase_2010}. However, Miehe' alternative is not variationally
consistent.

\paragraph{Micromorphic approach to brittle fracture}

A comprehensive framework for 
micromorphic approaches to various physical problems,
including brittle fracture, has been developed by Forest in
\cite{forest_micromorphic_2009, forest_nonlinear_2016}.
Balance equations arising from
such micromorphic approaches are
standard partial differential equations, that can readily be solved by
most FEM or FFT solvers.
In particular, these micromorphic behaviours allow to
deal with the damage irreversibility constraint at integration points,
instead of having to deal with it in the resolution of balance equations,
as is the case for the classical Phase field approach \cite{gerasimov_numerical_2020}. These
advantages have been highlighted by Rezaee-Hajidehi \textit{et al.} in the
context of phase-field approaches to phase transformation
\cite{rezaee-hajidehi_micromorphic_2021}.
%
%
%
Such micromorphic models were recently investigated by Bharali \textit{et al.}
\cite{bharali_computational_2021} to approximate the AT1 and AT2 models
using a monolithic resolution strategy.

\paragraph{Outline}

Following Forest' micromorphic framework
\cite{forest_micromorphic_2009, forest_nonlinear_2016}, a class of micromorphic
brittle behaviours that can approximate classical models of brittle
fracture in a variationally consistent way is proposed in
Section \ref{sec:micromorphicdamage:description}.
The variational basis of the behaviour is exploited to derive three alternate
minimization schemes in Section
\ref{sec:micromorphicdamage:alternate_minimisation}, whose convergence is guaranteed.
%
%
%
Several test cases comparing the prediction of the classical AT2 model
and its micromorphic counterpart are presented in Section
\ref{sec:micromorphicdamage:test_cases}, where
the choice of the penalization
parameter which at the foundation of the link between the Phase Field approach and the 
micromorphic one is discussed in-depth.
%
%
%
Numerical experiments are presented in Section
\ref{sec:micromorphicdamage:test_cases}, which assess:
%
%
%
\begin{itemize}
    \item The performance of the micromorphic approach with high order finite
    elements in the case of shear driven fracture.
    \item The scalability of the micromorphic approach.
\end{itemize}