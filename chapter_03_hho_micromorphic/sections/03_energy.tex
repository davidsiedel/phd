\section{Description}

\subsection{Free energy potential description}

\paragraph{Mechanical energy}

The mechanical energy is the product of the pure mechanical contribution to which is added
an adimensional \textit{degradation function} $g(d)$ such that
%
%
%
\begin{equation}
    \begin{aligned}
        \psi_{\tensoriis{F}, d} = g(d) \, \psi_{mec}(\tensorii{F}{})
    \end{aligned}
\end{equation}
%
%
%
where different choices arise in terms of the definition of the degradation function.

\paragraph{Dissipation potential}

A dissipation potential $\phi_{\bodyLag}(d)$ accounts for the energy dissipated by the fracture process of the medium, and is assumed to be
an homogeneous function of degree one such that
%
%
%
\begin{equation}
    \begin{aligned}
        \Delta \, t \, \phi_{\bodyLag}(\frac{\hat{d} - d} {\Delta \, t}) = \phi_{\bodyLag}(\hat{d} - d)
        &&
        \forall (\hat{d}, \Delta \, t > 0)
    \end{aligned}
\end{equation}
%
%
%
In particular, the dissipation potential generally contains an indicator function imposing the irreversibility of the
damage evolution.

\begin{equation}
    \begin{aligned}
        \psi_{\bodyLag} = 
    \end{aligned}
\end{equation}