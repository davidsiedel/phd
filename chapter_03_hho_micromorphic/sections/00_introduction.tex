\section{Introduction}
\label{sec_micromorphic_introduction}

\subsection{Outline}
\label{sec_micromorphic_introduction_outline}

\cite{abbas_hybrid_2018}

\section{Model problem}
\label{sec_micromorphic_model_problem}

\paragraph{Solid body in the current configuration}

Let $\BodyEuler$ a solid body that is subjected to a volumetric load $\tensori{f}{}_v$ in the current configuration at some time $t > 0$.
A displacement $\tensori{u}{}_{d}$ is prescribed
on the Dirichlet boundary $\BodyEulerDirichletBoundary$ and a surface load $\tensori{t}{}_{n}$ is imposed
on the Neumann boundary $\BodyEulerNeumannBoundary$.

\paragraph{Transformation mapping}

Let $\tensori{\Phi}{}$ the transformation mapping of the solid body from the initial configuration $\BodyLagrange$ to the current configuration $\BodyEuler$.
The displacement field $\DisplacementField$ is such that $\DisplacementField = \tensori{\Phi}{} - \IdentityTensorI$ where $\IdentityTensorI$
is the identity application on $\BodyLagrange$.
The gradient of the transformation is denoted $\TransformationGradientField = \nabla \tensori{\Phi}{} = \IdentityTensorII + \DisplacementGradientField$
where $\DisplacementGradientField = \nabla \DisplacementField$ is the gradient of the displacement.
Let $\BodyLagrangeDirichletBoundary$ and $\BodyLagrangeNeumannBoundary$ the images of $\BodyEulerDirichletBoundary$ and $\BodyEulerNeumannBoundary$ respectively by $\tensori{\Phi}{}^{-1}$.

\paragraph{External loads in the reference configuration} 

In the reference configuration, the solid is subjected to a volumetric load
$\tensori{f}{}_V$, a prescribed displacement $\tensori{u}{}_D$ on $\BodyLagrangeDirichletBoundary$, and a surface load $\tensori{t}{}_N$ on $\BodyLagrangeNeumannBoundary$, where the volumetric and surface loads $\tensori{f}{}_V$ and $\tensori{t}{}_N$ have been obtained from their counterparts
$\tensori{f}{}_v$ and $\tensori{t}{}_n$ respectively, using Nanson formulaes. For the sake of simplicty, they are supposed to be independent
on $\tensorii{F}{}$.

\paragraph{State of the solid} The mechanical state of the solid body $\BodyLagrange$ is characterized by the displacement field $\DisplacementField$,
the damage field $\DamageField$ and a micromorphic damage field $\MicromorphicDamageField$.
In the following, let $\tensori{g}{}_{\chi} = \nabla \MicromorphicDamageField$ the gradient of the micromorphic damage variable.

\paragraph{Free energy potential}

The free energy potential $\psi_{\BodyLagrange}$ of the body $\BodyLagrange$ reads as a function of the displacement $\DisplacementField$, the (local) damage $d$ and the micromorphic damage $\MicromorphicDamageField$, in the form
%
%
%
\begin{equation}
    \psi_{\BodyLagrange}
    (\TransformationGradientField, \DamageField, \MicromorphicDamageField, \MicromorphicDamageGradientField)
    =
    \psi_{\tensoriis{F}, \DamageField}
    (\TransformationGradientField, \DamageField)
    +
    \psi_{\DamageField}
    (\DamageField)
    +
    \psi_{\MicromorphicDamageField, \DamageField}
    (\MicromorphicDamageField, \DamageField)
    +
    \psi_{\MicromorphicDamageGradientField}
    (\MicromorphicDamageGradientField)
\end{equation}
%
%
%
where $\psi_{\tensoriis{F}, \DamageField}$ denotes the mechanical contribution that takes into account the damage in the medium,
$\psi_{\DamageField}$ is the energy stored during the fracture process,
$\psi_{\DamageField, \MicromorphicDamageField}$ is a coupling term between the damage and micromorphic damage variables, and
$\psi_{\MicromorphicDamageGradientField}$ defines the micromorphic force.

\paragraph{Stresses}

The following stresses are introduced
%
%
%
\begin{equation}
    \begin{aligned}
        \PKIStressField = \deriv{\psi_{\BodyLagrange}}{\TransformationGradientField}
        &&
        \MicromorphicDamageStressField = \deriv{\psi_{\BodyLagrange}}{\MicromorphicDamageGradientField}
        &&
        \MicromorphicDamageForceField = \deriv{\psi_{\BodyLagrange}}{\MicromorphicDamageField}
        &&
        \DamageForceField = \deriv{\psi_{\BodyLagrange}}{\DamageField}
    \end{aligned}
\end{equation}
%
%
%
where $\PKIStressField$ is the first Piola-Kirchoff stress tensor, and $\MicromorphicDamageStressField, \MicromorphicDamageForceField$ and $\DamageForceField$ are the thermodynamic
forces associated to $\tensori{g}{}_{\chi}, d_{\chi}$ and $d$ respectively.

\paragraph{Dissipation potential}

A dissipation potential $\phi_{\bodyLag}(d)$ accounts for the energy dissipated by the fracture process in the medium, and is assumed to be
an homogeneous function of degree one such that
%
%
%
\begin{equation}
    \begin{aligned}
        \Delta \, t \, \phi_{\BodyLagrange}(\frac{\DamageField - \DamageField \TraceOperator{t}} {\Delta \, t}) = \phi_{\BodyLagrange}(\DamageField - \DamageField \TraceOperator{t})
        &&
        \forall \Delta \, t > 0
    \end{aligned}
\end{equation}
%
%
%
In particular, the dissipation potential generally contains an indicator function imposing the irreversibility of the
damage evolution.

\paragraph{Total Lagrangian}

The total Hu-Washizu Lagrangian of the body $\BodyLagrange$ is defined
%
%
%
\begin{equation}
    \LagrangianOperator{\BodyLagrange}{tot}
    =
    \LagrangianOperator{\BodyLagrange}{HW}
    +
    \int_{\BodyLagrange} \phi_{\BodyLagrange}(\DamageField - \DamageField \TraceOperator{t})
\end{equation}
%
%
%
where
%
%
%
\begin{equation}
    \LagrangianOperator{\BodyLagrange}{HW}
    =
    \int_{\BodyLagrange} \psi_{\BodyLagrange}
    +
    \int_{\BodyLagrange} (\nabla \DisplacementField - \DisplacementGradientField) : \PKIStressField
    +
    \int_{\BodyLagrange} (\nabla \MicromorphicDamageField - \MicromorphicDamageGradientField) \cdot \MicromorphicDamageStressField
    -
    \int_{\BodyLagrange} \tensori{f}{}_V \cdot \DisplacementField
    -
    \int_{\BodyLagrangeNeumannBoundary} \tensori{t}{}_{N} \cdot \DisplacementField \TraceOperator{\BodyLagrangeNeumannBoundary}
\end{equation}
%
%
%
The solution $(\DisplacementField, \DisplacementGradientField, \PKIStressField, \DamageField, \MicromorphicDamageField, \MicromorphicDamageGradientField, \MicromorphicDamageStressField)$
satisfying the mechanical equilibrium minimizes the Lagragian $\LagrangianOperator{\BodyLagrange}{HW}$. The first order variation with respect to each variables yields the weak equations
%
%
%
\begin{subequations}
    \label{eq_micromorphic_hu_washizu_derivative_0}
    \begin{alignat}{3}
      \langle \deriv{\LagrangianOperator{\BodyLagrange}{HW}}{\DisplacementField} , \delta \DisplacementField \rangle
      =
      & \int_{\BodyLagrange} \PKIStressField : \nabla \delta \DisplacementField
      -
      \int_{\BodyLagrange} \tensori{f}_V \cdot \delta \DisplacementField
      -
      \int_{\BodyLagrangeNeumannBoundary} \neumannLag \cdot \delta \DisplacementField \vert_{\BodyLagrangeNeumannBoundary}
      &&
      \ \ \ \ \ \ \ \
      &&
      \forall \delta \DisplacementField
      \label{eq_micromorphic_hu_washizu_derivative_0:eq0}
      \\
      \langle \deriv{\LagrangianOperator{HW}{\BodyLagrange}}{\PKIStressField} , \delta \PKIStressField \rangle
      =
      & \int_{\BodyLagrange} (\nabla \DisplacementField - \DisplacementGradientField ) : \delta \PKIStressField
      &&
      \ \ \ \ \ \ \ \
      &&
      \forall \delta \PKIStressField
      \label{eq_micromorphic_hu_washizu_derivative_0:eq1}
      \\
      \langle \deriv{\LagrangianOperator{\BodyLagrange}{HW}}{\DisplacementGradientField} , \delta \DisplacementGradientField \rangle
      =
      & \int_{\BodyLagrange} (\deriv{\psi_{\BodyLagrange}}{\DisplacementGradientField} - \PKIStressField) : \delta \DisplacementGradientField
      &&
      \ \ \ \ \ \ \ \
      && \forall \delta \DisplacementGradientField
      \label{eq_micromorphic_hu_washizu_derivative_0:eq2}
      \\
      \langle \deriv{\LagrangianOperator{\BodyLagrange}{HW}}{\MicromorphicDamageField} , \delta \MicromorphicDamageField \rangle
      =
      & \int_{\BodyLagrange} \MicromorphicDamageForceField \, \delta \MicromorphicDamageField + \int_{\BodyLagrange} \MicromorphicDamageStressField \cdot \nabla \MicromorphicDamageField
      &&
      \ \ \ \ \ \ \ \
      && \forall \delta d^\chi
      \label{eq_micromorphic_hu_washizu_derivative_0:eq3}
      \\
      \langle \deriv{\LagrangianOperator{\BodyLagrange}{HW}}{\MicromorphicDamageStressField} , \delta \MicromorphicDamageStressField \rangle
      =
      & \int_{\BodyLagrange} (\nabla \MicromorphicDamageField - \MicromorphicDamageGradientField) \cdot \delta \MicromorphicDamageStressField
      &&
      \ \ \ \ \ \ \ \
      && \forall \delta \MicromorphicDamageStressField
      \label{eq_micromorphic_hu_washizu_derivative_0:eq4}
      \\
      \langle \deriv{\LagrangianOperator{\BodyLagrange}{HW}}{\MicromorphicDamageGradientField} , \delta \MicromorphicDamageGradientField \rangle
      =
      & \int_{\BodyLagrange} (\deriv{\mecPotential}{\MicromorphicDamageGradientField} - \MicromorphicDamageStressField) \cdot \delta \MicromorphicDamageGradientField
      &&
      \ \ \ \ \ \ \ \
      && \forall \delta \tensori{g}{}_{\chi}
      \label{eq_micromorphic_hu_washizu_derivative_0:eq5}
    \end{alignat}
\end{subequations}

\paragraph{Strong equation}

The following strong equations for the sole displacement problem are deduced from the weak equation
\eqref{eq_micromorphic_hu_washizu_derivative_0:eq1} ,
\eqref{eq_micromorphic_hu_washizu_derivative_0:eq2}
and
\eqref{eq_micromorphic_hu_washizu_derivative_0:eq0}
respectively
%
%
%
\begin{subequations}
    \label{eq_micromorphic_strong_equations_meca}
    \begin{alignat}{3}
    \DisplacementGradientField - \nabla \DisplacementField & = 0
    &&
    \ \ \ \ \ \ \ \
    &&
    \textit{displacement gradient}
    \label{eq_micromorphic_strong_equations_meca:eq0}
    \\
    \PKIStressField - \deriv{\psi_{\BodyLagrange}}{\DisplacementGradientField} & = 0
    &&
    \ \ \ \ \ \ \ \
    &&
    \textit{constitutive equation}
    \label{eq_micromorphic_strong_equations_meca:eq1}
    \\
    \nabla \cdot \PKIStressField + \tensori{f}{}_{V} & = 0
    &&
    \ \ \ \ \ \ \ \
    &&
    \textit{balance of momentum}
    \label{eq_micromorphic_strong_equations_meca:eq2}
    \\
    \PKIStressField \cdot \tensori{n}{} - \neumannLag{} & = 0
    &&
    \ \ \ \ \ \ \ \
    &&
    \textit{continuity of the normal stress}
    \label{eq_micromorphic_strong_equations_meca:eq3}
    \end{alignat}
\end{subequations}
%
%
%
Similarly, the strong equations governing the sole micromorphic damage problem are deduced from
\eqref{eq_micromorphic_hu_washizu_derivative_0:eq4} ,
\eqref{eq_micromorphic_hu_washizu_derivative_0:eq5}
and
\eqref{eq_micromorphic_hu_washizu_derivative_0:eq3}
respectively
%
%
%
\begin{subequations}
    \label{eq_micromorphic_strong_equations_damage}
    \begin{alignat}{3}
        \MicromorphicDamageGradientField - \nabla \MicromorphicDamageField & = 0
        &&
        \ \ \ \ \ \ \ \
        &&
        \textit{micromorphic damage gradient}
        \label{eq_micromorphic_strong_equations_damage:eq0}
        \\
        \MicromorphicDamageStressField - \deriv{\psi_{\BodyLagrange}}{\MicromorphicDamageGradientField} & = 0
        &&
        \ \ \ \ \ \ \ \
        &&
        \textit{micromorphic damage constitutive equation}
        \label{eq_micromorphic_strong_equations_damage:eq1}
        \\
        \nabla \cdot \MicromorphicDamageStressField - \MicromorphicDamageForceField & = 0
        &&
        \ \ \ \ \ \ \ \
        &&
        \textit{balance of micromorphic damage momentum}
        \label{eq_micromorphic_strong_equations_damage:eq2}
        \\
        \MicromorphicDamageStressField \cdot \tensori{n}{} & = 0
        &&
        \ \ \ \ \ \ \ \
        &&
        \textit{micromorphic damage boundary conditions}
        \label{eq_micromorphic_strong_equations_damage:eq3}
        \\
    \end{alignat}
\end{subequations}
%
%
%
where the governing laws of the micromorphic damage variable define a generalized continuum medium as introduced in \cite{forest_micromorphic_2009}.