% ---------------------------------------------------------
% ---- SECTION
% ---------------------------------------------------------
\section{Discretization}
\label{sec_discretization}

This section specifies the nature of the so-called hybrid mesh,
and introduces the discretization for both cell and faces displacement fields.
The classical \textit{static condensation} cell unknowns elimination strategy is presented, and a novel elimination scheme
based on the previous Lagrangian formulation of hybrid discontinuous methods is then devised.

% ---------------------------------------------------------
% -- SUBSECTION
% ---------------------------------------------------------
\subsection{Mesh and skeleton}

% ---------------------------------------------------------
% PARAGRAPH
% ---------------------------------------------------------
\paragraph{Faces and skeleton of the mesh}

The boundary $\ElementBoundary$ of each element is decomposed in faces, such
that a face $\ElementFace$ is a subset of $\BodyLagrange$, and either there are two
cells $\Field{\ElementCell}{F}{1}$ and $\Field{\ElementCell}{F}{2}$ such that $F = \Field{\ElementBoundary}{F}{1} \cap \Field{\ElementBoundary}{F}{2}$
($\ElementFace$ is then an interior face), or there is a single cell $\Field{\ElementCell}{F}{}$ such
that $\ElementFace = \Field{\ElementBoundary}{F}{} \cap \BodyLagrangeBoundary$ ($\ElementFace$ is then an exterior face).
Let $\Skeleton(\BodyLagrange) = \{ \Field{\ElementFace}{i}{} \subset \BodyLagrange \ \vert \ 1 \leq i
\leq \Field{N}{\Skeleton}{} \}$ the skeleton of the mesh, collecting all element faces
$\Field{\ElementFace}{i}{}$ in the mesh, where $\Field{N}{\Skeleton}{}$ denotes the number of faces. The set of
faces subjected to Neumann boundary conditions is denoted
$\SkeletonBoundaryNeumann(\BodyLagrange)$, and $\SkeletonBoundaryDirichlet(\BodyLagrange)$
denotes that subjected to Dirichlet boundary conditions.
Moreover, let
$\SkeletonInterior(\BodyLagrange)$ the set of interior faces
and
$\SkeletonBoundary(\BodyLagrange)$ the set of exterior faces. For any cell
$\ElementCell$, let $\Skeleton(\ElementCell) = \{ \ElementFace \in \Skeleton(\BodyLagrange) \ \vert \ \ElementFace
\subset \ElementBoundary \}$ the set of faces composing the boundary of $\ElementCell$,
and let $\Field{N}{\ElementBoundary}{}$ the number of faces in $\ElementBoundary$.

% ---------------------------------------------------------
% PARAGRAPH
% ---------------------------------------------------------
\paragraph{Mesh description}

Likewise, one defines the collection of all cells in the mesh
as $\Mesh(\BodyLagrange) = \{ \Field{\ElementCell}{i}{} \subset \BodyLagrange \ \vert \ 1 \leq i
\leq \Field{N}{\Mesh}{} \}$, where $\Field{N}{\Mesh}{}$ denotes the total number of cells.

% ---------------------------------------------------------
% -- SUBSECTION
% ---------------------------------------------------------
\subsection{Discretization}

% ---------------------------------------------------------
% PARAGRAPH
% ---------------------------------------------------------
\paragraph{Discrete functional space}

Let $\Field{U}{\ElementCell}{\; h}$ and $\Field{D}{\ElementCell}{\; h}$ denote
approximation spaces of finite dimension for the cell displacement and the micromorphic damage respectively,
and $\Field{U}{\ElementFace}{\; h}$ and $\Field{D}{\ElementFace}{\; h}$ those
on a face $\ElementFace \in \Skeleton(\ElementCell)$.
The displacement and micromorphic damage approximation spaces on $\ElementBoundary$ are
$\Field{U}{\ElementBoundary}{\; h} = \prod_{\ElementFace \in \Skeleton(\ElementCell)} \Field{U}{\ElementFace}{\; h}$
and $\Field{D}{\ElementBoundary}{\; h} = \prod_{\ElementFace \in \Skeleton(\ElementCell)} \Field{D}{\ElementFace}{\; h}$
respectively.

\paragraph{Approximation bases and elementary unknowns}

Let $\Field{\Vector{B}}{u, \ElementCell}{\; h}, \Field{\Vector{B}}{u, \ElementFace}{\; h}, \Field{\Vector{B}}{d, \ElementCell}{\; h}, \Field{\Vector{B}}{d, \ElementFace}{\; h}$
denote polynomial bases of $\Field{U}{\ElementCell}{\; h}, \Field{U}{\ElementFace}{\; h}, \Field{D}{\ElementCell}{\; h}, \Field{D}{\ElementFace}{\; h}$ respectively.
Using vector notations, the fields $\Field{\DisplacementField}{\ElementCell}{\; h}, \Field{\DisplacementField}{\ElementFace}{\; h}, \Field{\MicromorphicDamageField}{\ElementCell}{\; h}, \Field{\MicromorphicDamageField}{\ElementFace}{\; h}$
can be represented by vectors of coefficients $\Field{\Vector{U}}{\ElementCell}{}, \Field{\Vector{U}}{\ElementFace}{}, \Field{\Vector{D}}{\ElementCell}{}, \Field{\Vector{D}}{\ElementFace}{}$
in their respective bases such that
%
%
%
\begin{equation}
  \label{eq_bases}
  \begin{aligned}
      \Field{\DisplacementField}{\ElementCell}{\; h} = \Field{\Vector{U}}{\ElementCell}{} \cdot \Field{\Vector{B}}{u, \ElementCell}{\; h}
      &&
      &&
      \Field{\DisplacementField}{\ElementFace}{\; h} = \Field{\Vector{U}}{\ElementFace}{} \cdot \Field{\Vector{B}}{u, \ElementFace}{\; h}
      &&
      &&
      \Field{\MicromorphicDamageField}{\ElementCell}{\; h} = \Field{\Vector{D}}{\ElementCell}{} \cdot \Field{\Vector{B}}{d, \ElementCell}{\; h}
      &&
      &&
      \Field{\MicromorphicDamageField}{\ElementFace}{\; h} = \Field{\Vector{D}}{\ElementFace}{} \cdot \Field{\Vector{B}}{d, \ElementFace}{\; h}
  \end{aligned}
\end{equation}

\paragraph{Global Unknowns}

Let $(\Field{\DisplacementField}{\Mesh}{\; h}, \Field{\DisplacementField}{\Skeleton}{\; h}) \in \Field{U}{\Mesh}{\; h} \times \Field{U}{\Skeleton}{\; h}$
and $(\Field{\MicromorphicDamageField}{\Mesh}{\; h}, \Field{\MicromorphicDamageField}{\Skeleton}{\; h}) \in \Field{D}{\Mesh}{\; h} \times \Field{D}{\Skeleton}{\; h}$
the global
unknowns of problem \eqref{eq_final_problem} in discrete form, where
$\Field{\DisplacementField}{\Mesh}{\; h}, \Field{\DisplacementField}{\Skeleton}{\; h}, \Field{\MicromorphicDamageField}{\Mesh}{\; h}, \Field{\MicromorphicDamageField}{\Skeleton}{\; h}$
are the piece-wise continuous fields such that $\forall \ElementCell \in \Mesh, \forall \ElementFace \in \Skeleton$
%
%
%
\begin{equation}
  \begin{aligned}
    \Field{\DisplacementField}{\Mesh}{\; h} \TraceOperator{\ElementCell} = \Field{\DisplacementField}{\ElementCell}{\; h}
    &&
    &&
    \Field{\DisplacementField}{\Skeleton}{\; h} \TraceOperator{\ElementFace} = \Field{\DisplacementField}{\ElementFace}{\; h}
    &&
    &&
    \Field{\MicromorphicDamageField}{\Mesh}{\; h} \TraceOperator{\ElementCell} = \Field{\MicromorphicDamageField}{\ElementCell}{\; h}
    &&
    &&
    \Field{\MicromorphicDamageField}{\Skeleton}{\; h} \TraceOperator{\ElementFace} = \Field{\MicromorphicDamageField}{\ElementFace}{\; h}
  \end{aligned}
\end{equation}
% 
% 
% 
with $
\Field{U}{\Mesh}{\; h} = \prod_{\ElementCell \in \Mesh} \Field{U}{\ElementCell}{\; h}, \;
\Field{U}{\Skeleton}{\; h} = \prod_{\ElementFace \in \Skeleton} \Field{U}{\ElementFace}{\; h}, \;
\Field{D}{\Mesh}{\; h} = \prod_{\ElementCell \in \Mesh} \Field{D}{\ElementCell}{\; h}, \;
\Field{D}{\Skeleton}{\; h} = \prod_{\ElementFace \in \Skeleton} \Field{D}{\ElementFace}{\; h}
$.
In the following, let
$\Field{\Vector{U}}{\Mesh}{}, \Field{\Vector{U}}{\Skeleton}{}, \Field{\Vector{D}}{\Mesh}{}, \Field{\Vector{D}}{\Skeleton}{}$ the unknown coefficient vectors associated
to $\Field{\DisplacementField}{\Mesh}{\; h}, \Field{\DisplacementField}{\Skeleton}{\; h}, \Field{\MicromorphicDamageField}{\Mesh}{\; h}, \Field{\MicromorphicDamageField}{\Skeleton}{\; h}$ respectively.

\paragraph{Elementary boundary unknown}

Likewise, let $\Field{\DisplacementField}{\ElementBoundary}{\; h} \in \Field{U}{\ElementBoundary}{\; h}$ and $\Field{\MicromorphicDamageField}{\ElementBoundary}{\; h} \in \Field{D}{\ElementBoundary}{\; h}$ such that $\forall \ElementFace \in \Skeleton(\ElementCell)$
%
%
%
\begin{equation}
  \begin{aligned}
    \Field{\DisplacementField}{\ElementBoundary}{\; h} \TraceOperator{\ElementFace} = \Field{\DisplacementField}{\ElementFace}{\; h}
    &&
    &&
    \Field{\MicromorphicDamageField}{\ElementBoundary}{\; h} \TraceOperator{\ElementFace} = \Field{\MicromorphicDamageField}{\ElementFace}{\; h}
  \end{aligned}
\end{equation}
%
%
%
and let $\Field{\Vector{U}}{\ElementBoundary}{}$ and $\Field{\Vector{D}}{\ElementBoundary}{}$
the unknown coefficient vectors associated to $\Field{\DisplacementField}{\ElementBoundary}{\; h}$ and $\Field{\MicromorphicDamageField}{\ElementBoundary}{\; h}$.

% ---------------------------------------------------------
% -- SUBSECTION
% ---------------------------------------------------------
\subsection{Local and global discrete problems}

% ---------------------------------------------------------
% PARAGRAPH
% ---------------------------------------------------------
\subsubsection{Local residual}

As in a functional space of finite dimension, the restriction of a linear
form can be represented by a vector in the dual space, let
$\Field{\Vector{R}}{u, \ElementCell}{}, \Field{\Vector{R}}{u, \ElementBoundary}{}, \Field{\Vector{R}}{d, \ElementCell}{}, \Field{\Vector{R}}{d, \ElementBoundary}{}$ the residual vectors
associated with the the variations of the total
Lagrangian~\eqref{eq_total_lagragian_bis}
%
% 
% 
\begin{subequations}
  \label{eq_final_problem_00}
  \begin{alignat}{3}
    \Field{\Vector{R}}{u, \ElementCell}{}(\Field{\Vector{U}}{\ElementCell}{}, \Field{\Vector{U}}{\ElementBoundary}{}) \cdot \Field{\VirtualField{\Vector{U}}}{\ElementCell}{}
    =
    &
    \int_{\ElementCell} \Field{\PKIStressField}{\ElementCell}{\; h} : \nabla \Field{\VirtualField{\DisplacementField}}{\ElementCell}{\; h}
    -
    \int_{\ElementCell} \tensori{f}{}_V \cdot \Field{\VirtualField{\DisplacementField}}{\ElementCell}{\; h}
    -
    \int_{\ElementBoundary} \Field{\DisplacementTractionField}{\ElementCell}{\; h} \cdot \Field{\VirtualField{\DisplacementField}}{\ElementCell}{\; h} \TraceOperator{\ElementBoundary}
    &&
    \qquad \forall \Field{\VirtualField{\DisplacementField}}{\ElementCell}{\; h}
    \label{eq_final_problem_00:eq0}
    \\
    \Field{\Vector{R}}{u, \ElementBoundary}{}(\Field{\Vector{U}}{\ElementCell}{}, \Field{\Vector{U}}{\ElementBoundary}{}) \cdot \Field{\VirtualField{\Vector{U}}}{\ElementBoundary}{}
    =
    &
    \int_{\neumannCell} (\Field{\DisplacementTractionField}{\ElementCell}{\; h} - \tensori{t}{}_{\neumannCell}) \cdot \delta \Field{\DisplacementField}{\ElementBoundary}{\; h}
    &&
    \qquad \forall \Field{\VirtualField{\DisplacementField}}{\ElementBoundary}{\; h}
    \label{eq_final_problem_00:eq1}
    \\
    \Field{\Vector{R}}{d, \ElementCell}{}(\Field{\Vector{D}}{\ElementCell}{}, \Field{\Vector{D}}{\ElementBoundary}{}) \cdot \Field{\VirtualField{\Vector{D}}}{\ElementCell}{}
    =
    & \int_{\ElementCell} \Field{\MicromorphicDamageStressField}{\ElementCell}{\; h} : \nabla \Field{\VirtualField{\MicromorphicDamageField}}{\ElementCell}{\; h}
    +
    \int_{\ElementCell} \Field{\MicromorphicDamageForceField}{\ElementCell}{\; h} \cdot \Field{\VirtualField{\MicromorphicDamageField}}{\ElementCell}{\; h}
    -
    \int_{\ElementBoundary} \Field{\MicromorphicDamageTractionField}{\ElementCell}{\; h} \, \Field{\VirtualField{\MicromorphicDamageField}}{\ElementCell}{\; h} \TraceOperator{\ElementBoundary}
    &&
    \qquad \forall \delta \Field{\VirtualField{\MicromorphicDamageField}}{\ElementCell}{\; h}
    \label{eq_final_problem_00:eq04}
    \\
    \Field{\Vector{R}}{d, \ElementBoundary}{}(\Field{\Vector{D}}{\ElementCell}{}, \Field{\Vector{D}}{\ElementBoundary}{}) \cdot \Field{\VirtualField{\Vector{D}}}{\ElementBoundary}{}
    = &
    \int_{\neumannCell} \Field{\MicromorphicDamageTractionField}{\ElementCell}{\; h} \, \Field{\VirtualField{\MicromorphicDamageField}}{\ElementBoundary}{\; h}
    &&
    \qquad \forall \delta \Field{\VirtualField{\MicromorphicDamageField}}{\ElementBoundary}{\; h}
    \label{eq_final_problem_00:eq12}
  \end{alignat}
\end{subequations}
where the discrete stress tensor $\Field{\PKIStressField}{\ElementCell}{\; h}$ and the
discrete reconstructed gradient
$\Field{\DisplacementGradientField}{\ElementCell}{\; h}(\Field{\DisplacementField}{\ElementCell}{\; h}, \Field{\DisplacementField}{\ElementBoundary}{\; h})$ are defined by the discrete forms of
equations \eqref{eq_stress} and \eqref{eq_00172} respectively such that
%
%
%
\begin{equation}
  \label{eq_stress_discrete}
  \begin{aligned}
    \Field{\PKIStressField}{\ElementCell}{\; h}
    =
    \deriv{\psi_{\BodyLagrange}}{\Field{\DisplacementGradientField}{\ElementCell}{\; h}}
    &&
    \text{with}
    &&
    \int_{\ElementCell} \Field{\DisplacementGradientField}{\ElementCell}{\; h} : \Field{\tensorii{\tau}}{\ElementCell}{\; h}
    =
    \int_{\ElementCell}  \nabla \Field{\VirtualField{\DisplacementField}}{\ElementCell}{\; h} : \Field{\tensorii{\tau}}{\ElementCell}{\; h}
    +
    \int_{\ElementBoundary} (\Field{\VirtualField{\DisplacementField}}{\ElementBoundary}{\; h} - \Field{\VirtualField{\DisplacementField}}{\ElementCell}{\; h} \TraceOperator{\ElementBoundary}) \cdot \Field{\tensorii{\tau}}{\ElementCell}{\; h} \TraceOperator{\ElementBoundary} \cdot \NormalVector
    &&
    \qquad \forall \Field{\tensorii{\tau}}{\ElementCell}{\; h}
    \\
    \Field{\MicromorphicDamageStressField}{\ElementCell}{\; h} = \deriv{\mecPotential_{\BodyLagrange}}{\Field{\MicromorphicDamageGradientField}{\ElementCell}{\; h}}
    &&
    \text{with}
    &&
    \int_{\ElementCell} \Field{\MicromorphicDamageGradientField}{\ElementCell}{\; h} \cdot \Field{\tensori{\zeta}}{\ElementCell}{\; h}
    =
    \int_{\ElementCell}  \nabla \Field{\VirtualField{\MicromorphicDamageField}}{\ElementCell}{\; h} \cdot \Field{\tensori{\zeta}}{\ElementCell}{\; h}
    +
    \int_{\ElementBoundary} (\Field{\VirtualField{\MicromorphicDamageField}}{\ElementBoundary}{\; h} - \Field{\VirtualField{\MicromorphicDamageField}}{\ElementCell}{\; h} \TraceOperator{\ElementBoundary}) \, \Field{\tensori{\zeta}}{\ElementCell}{\; h} \TraceOperator{\ElementBoundary} \cdot \NormalVector
    &&
    \qquad \forall \Field{\tensori{\zeta}}{\ElementCell}{\; h}
  \end{aligned}
\end{equation}
%
%
%
and the discrete reconstructed tractions write
$\Field{\DisplacementTractionField}{\ElementCell}{\; h} = \Field{\PKIStressField}{\ElementCell}{\; h} \TraceOperator{\ElementBoundary} \cdot \NormalVector + (\DisplacementStabilizationFactor / h_{\ElementCell}) \Field{\DisplacementJumpField}{}{\; h}(\Field{\DisplacementField}{\ElementCell}{\; h}, \Field{\DisplacementField}{\ElementBoundary}{\; h})$
and
$\Field{\MicromorphicDamageTractionField}{\ElementCell}{\; h} = \Field{\MicromorphicDamageStressField}{\ElementCell}{\; h} \TraceOperator{\ElementBoundary} \cdot \NormalVector + (\MicromorphicDamageStabilizationFactor / h_{\ElementCell}) \Field{\MicromorphicDamageJumpField}{}{\; h}(\Field{\MicromorphicDamageField}{\ElementCell}{\; h}, \Field{\MicromorphicDamageField}{\ElementBoundary}{\; h})$ respectively.
In particular, the expression of the discrete jump function $\Field{\DisplacementJumpField}{}{\; h}$
is the key ingredient that defines the HHO method (see Section \ref{sec_appendix_implementation} for more details on this note).
The solution of the discrete problem $(\Field{\DisplacementField}{\ElementCell}{\; h}, \Field{\DisplacementField}{\ElementBoundary}{\; h}, \Field{\MicromorphicDamageField}{\ElementCell}{\; h}, \Field{\MicromorphicDamageField}{\ElementBoundary}{\; h})$
% $(\mathfrak{U}_{\ElementCell}, \mathfrak{U}_{\ElementBoundary})$
is defined by the fact that the
residuals $\Field{\Vector{R}}{u, \ElementCell}{}, \Field{\Vector{R}}{u, \ElementBoundary}{}, \Field{\Vector{R}}{d, \ElementCell}{}, \Field{\Vector{R}}{d, \ElementBoundary}{}$ must be
zero.

% ---------------------------------------------------------
% PARAGRAPH
% ---------------------------------------------------------
\subsubsection{Global residuals and face assembly}

\paragraph{Global problem}

At the global scale, the solution
$(\Field{\Vector{U}}{\Mesh}{}, \Field{\Vector{U}}{\Skeleton}{}, \Field{\Vector{D}}{\Mesh}{}, \Field{\Vector{D}}{\Skeleton}{})$ of the
discrete problem satisfies
%
% 
% 
\begin{subequations}
  \label{eq_final_problem_003}
  \begin{alignat}{3}
    \Field{\Vector{R}}{u, \ElementCell}{}(\Field{\Vector{U}}{\ElementCell}{}, \Field{\Vector{U}}{\ElementBoundary}{})
    \cdot
    \Field{\VirtualField{\Vector{U}}}{\ElementCell}{} & = 0
    &&
    \qquad \forall \ElementCell \in \Mesh(\BodyLagrange), \; \forall \Field{\VirtualField{\Vector{U}}}{\ElementCell}{} \;
    \label{eq_final_problem_003:eq0}
    \\
    \Field{\Vector{R}}{d, \ElementCell}{}(\Field{\Vector{D}}{\ElementCell}{}, \Field{\Vector{D}}{\ElementBoundary}{})
    \cdot
    \Field{\VirtualField{\Vector{D}}}{\ElementCell}{} & = 0
    &&
    \qquad \forall \ElementCell \in \Mesh(\BodyLagrange), \; \forall \Field{\VirtualField{\Vector{U}}}{\ElementCell}{} \;
    \label{eq_final_problem_003:eq1}
    \\
    \Field{\Vector{R}}{u, \Skeleton}{}(\Field{\Vector{U}}{\Mesh}{}, \Field{\Vector{U}}{\Skeleton}{})
    \cdot
    \Field{\VirtualField{\Vector{U}}}{\Skeleton}{} & = 0
    &&
    \qquad \forall \Field{\VirtualField{\Vector{U}}}{\Skeleton}{} \;
    \label{eq_final_problem_003:eq2}
    \\
    \Field{\Vector{R}}{u, \Skeleton}{}(\Field{\Vector{D}}{\Mesh}{}, \Field{\Vector{D}}{\Skeleton}{})
    \cdot
    \Field{\VirtualField{\Vector{D}}}{\Skeleton}{} & = 0
    &&
    \qquad \forall \Field{\VirtualField{\Vector{D}}}{\Skeleton}{} \;
    \label{eq_final_problem_003:eq3}
  \end{alignat}
\end{subequations}
%
%
%
where the vectors
$\Field{\Vector{R}}{u, \Skeleton}{}(\Field{\Vector{U}}{\Mesh}{}, \Field{\Vector{U}}{\Skeleton}{})$ and $\Field{\Vector{R}}{d, \Skeleton}{}(\Field{\Vector{D}}{\Mesh}{}, \Field{\Vector{D}}{\Skeleton}{})$ are the skeleton residuals such that
%
% 
% 
\begin{subequations}
  \label{eq_final_problem_00333}
  \begin{alignat}{3}
    \Field{\Vector{R}}{u, \Skeleton}{}(\Field{\Vector{U}}{\Mesh}{}, \Field{\Vector{U}}{\Skeleton}{})
    \cdot
    \Field{\VirtualField{\Vector{U}}}{\Skeleton}{}
    &
    =
    \sum_{\ElementFace \in \SkeletonInterior(\BodyLagrange)} \int_{\ElementFace}
    (
      \Field{\DisplacementTractionField}{\Field{\ElementCell}{\ElementFace}{1}}{\; h}
      +
      \Field{\DisplacementTractionField}{\Field{\ElementCell}{\ElementFace}{2}}{\; h}
    )
    \cdot \Field{\VirtualField{\DisplacementField}}{\ElementFace}{\; h}
    +
    \sum_{\ElementFace \in \SkeletonBoundaryNeumann(\BodyLagrange)}
    \int_{\ElementFace}
    (
      \Field{\DisplacementTractionField}{\Field{\ElementCell}{\ElementFace}{}}{\; h}
      - \neumannLag
    )
    \cdot \Field{\VirtualField{\DisplacementField}}{\ElementFace}{\; h}
    &&
    \qquad \forall \Field{\VirtualField{\Vector{U}}}{\ElementFace}{}
    \label{eq_final_problem_00333:eq0}
    \\
    \Field{\Vector{R}}{d, \Skeleton}{}(\Field{\Vector{D}}{\Mesh}{}, \Field{\Vector{D}}{\Skeleton}{})
    \cdot
    \Field{\VirtualField{\Vector{U}}}{\Skeleton}{}
    &
    =
    \sum_{\ElementFace \in \SkeletonInterior(\BodyLagrange)} \int_{\ElementFace}
    (
      \Field{\MicromorphicDamageTractionField}{\Field{\ElementCell}{\ElementFace}{1}}{\; h}
      +
      \Field{\MicromorphicDamageTractionField}{\Field{\ElementCell}{\ElementFace}{2}}{\; h}
    )
    \,
    \Field{\VirtualField{\MicromorphicDamageField}}{\ElementFace}{\; h}
    +
    \sum_{\ElementFace \in \SkeletonBoundary(\BodyLagrange)}
    \int_{\ElementFace}
    \Field{\MicromorphicDamageTractionField}{\Field{\ElementCell}{\ElementFace}{}}{\; h}
    \,
    \Field{\VirtualField{\MicromorphicDamageField}}{\ElementFace}{\; h}
    &&
    \qquad \forall \Field{\VirtualField{\Vector{D}}}{\ElementFace}{}
    \label{eq_final_problem_00333:eq1}
  \end{alignat}
\end{subequations}
%
%
%
and which result in the assembly of faces unknowns only.

\paragraph{Assembly over the skeleton}

An interior face $\ElementFace$ is linked to two adjacent cells $\Field{\ElementCell}{}{1}$ and $\Field{\ElementCell}{}{2}$, and
each of these cells applies to the other a surface load $\pm
\tensori{t}{}_{\Field{\ElementCell}{}{1} \cap \Field{\ElementCell}{}{2}}$ through $\ElementFace$, which is identified with
$\tensori{t}{}_{\neumannCell}$ on $\ElementFace$ in \eqref{eq_final_problem_00:eq1} for the displacement field, and which is zero
for the micromorphic damage field.
By summation over each face of the structure, these equal contributions
cancel out, which yields the expression of the first argument in the
right-hand side of \eqref{eq_final_problem_00333:eq0} and \eqref{eq_final_problem_00333:eq1}.
%
%
%
Since exterior faces subjected to Neumann boundary conditions are
linked to a single cell only, the local surface forces $\tensori{t}{}_{\neumannCell}$ are equal to $\neumannLag$ on $\neumannBoundaryLag{}$,
which yields the expression of
the second argument in the right-hand side of
\eqref{eq_final_problem_00333:eq0}.