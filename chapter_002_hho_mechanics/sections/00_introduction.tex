\section{Introduction}
\label{sec_introduction}

\subsection{Outline}
\label{sec_introduction_outline}

% This paper aims at outlining a link between the Hu–Washizu
% variational principle and that of so called relaxed formulations
% % ,
% % from which are derived \textit{e.g.} Hybrid Discontinuous Galerkin
% % (HDG) methods, and, in particular, one of their latest refinement, the
% % Hybrid High Order (HHO) method.
% among which the family of Discontinuous Galerkin (DG) and Hybrid Discontinuous Galerkin
% (HDG) methods.
% This paper focuses on the specific case of Hybrid High Order (HHO) methods, that are a specific.
% , and, in particular, one of their latest refinement, the
% Hybrid High Order (HHO) method.

When solving quasi-incompressible boundary-value problems with the
Finite Element Method (FEM) using standard Lagrange interpolation, high
oscillations of the hydrostatic pressure might occur. This numerical
artifact, also known as volumetric locking is discussed in depth in
classical textbooks of computational
mechanics~\cite{simo_computational_1998, belytschko_nonlinear_nodate, neto_computational_2008}.
An enriched approach is required to resolve or, at least, alleviate
this issue.

Most works in the literature rely on either one of these two
following approaches to derive numerical methods robust to volumetric
locking:
\begin{itemize}
  \item The Hu-Washizu variational principle.
  \item Relaxed formulations, among which the family
  of Discontinuous Galerkin (DG) and Hybrid Discontinuous Galerkin (HDG)
  methods, including one of their latest refinement, the Hybrid High Order
  (HHO) method.
\end{itemize}

% Both these approaches introduce supplementary unknowns to the problem, and many works in the literature rely on either one of these two approaches to derive numerical methods that prove to be robust to volumetric locking phenomena.
% Both these methods are at the foundation of numerical methods

Both approaches introduce supplementary unknowns to the problem:
\begin{itemize}
  \item On the one hand, the Hu-Washizu variational principle
  extends that of Virtual Work by introducing supplementary stress and
  strain fields, that act as Lagrange multipliers to enforce
  respectively the constitutive and strain-displacement relations.
  \item On the other hand, relaxed formulations provide a richer
  kinematics by introducing a displacement jump at element boundaries,
  hence allowing for the definition of enhanced and local strain and
  stress fields.
\end{itemize}

The present paper proposes a variational formulation for HDG methods
based on a Hu-Washizu approach, where the discontinuity of the
displacement field is retrieved by introducing a linear elastic
interface between elements, hence outlining a common framework to both
Hu-Washizu based approaches for continuous displacement fields, and
hybrid discontinuous methods. A novel resolution algorithm, based on the
proposed variational approach is devised and numerically tested, using
the proposed HHO method for axi-symmetric problems.

% The following introduction outlines in a first part the development and applications of The Hu–Washizu variational
% formulation and in a second part that of discontinuous methods.

% DIRE HW A ETE REGARDE POUR LINCOMPRESSIBILITE PLASTIQUE ETC, AVANTDE PARLER DES LETHDODES
% -> INTRODUIRE LES 
% -> NOTATIONS (voir papier micromorphe)
% -> charges mortes pour les temres de passage en configurartion initiale, pas celles de Nanson
% -> les brackets : dimension finie, produit scalair intégrale, nsur des vectiueyrs : produit scalair vectoriel
% -> le potentiel élastique, non-standrad mais pour retrouver les expressions exactes des HDG, passer par vrai potentiel élastique , à tester. Passage en zone cohézive, extension intéressante pas traitée dans le document

% ---------------------------------------------------------
% -- SUBSECTION
% ---------------------------------------------------------
\subsection{From the Hellinger-Reissner principle to the Hu–Washizu variational formulation}

% Before 1950, variational principles considered only displacement
% as a single independent field.
% Generalized variational principles began in the 1950's with the
% breakthrough works of Reissner \cite{reissner_variational_1950} on two-field variational principles for elasticity problems, in
% which the displacement and stress were considered independent fields. Subsequently, de Vebeuke \cite{fraeijs_de_veubeke_diffusion_1951} constructed a
% four-field variational principle, and Hu \cite{hu_variational_1954} and Washizu \cite{washizu_variational_1955} established three-field variational
% principle independently.
% In 1983, Chien \cite{chien_method_1983} first pointed out that the three kinds of
% variables in Hu-Washizu principle are but nct independent
% of each other. Stress-strain relations are still its variational
% constraints, which could be removed only by high-order
% Lagrange multiplier method.

The principle of Virtual Work, which is at the foundation of the Finite Element theory, considers the displacement field alone as the unknown of the mechanical problem.
The first generalized variational principle dates back to the work of Reissner \cite{reissner_variational_1950} in the framework of linear elasticity, where the stress field does
not define as a function of the displacement gradient (or strain) anymore, but is now an unknown of the problem, in the same way as the displacement field.
In 1954 and 1955, Hu \cite{hu_variational_1954} and Washizu \cite{washizu_variational_1955} independently established a three-field variational principle, consisting of a stress,
a strain and a displacement field, hence giving rise to the so-called Hu-Washizu principle, which lays the groundwork for the so-called mixed and enhanced assumed strain methods.
However, de Vebeuke had already introduced in the early 50's a four-field principle, consisting of stress, strain, displacement and surface forces fields, and the Hu-Wahsizu principle
can be seen as an application of the de Vebeuke principle, where the surface force field is assimilated to the normal stress.

% ---------------------------------------------------------
% -- SUBSUBSECTION
% ---------------------------------------------------------
\paragraph{Assumed strain methods}

So-called Assumed Strain methods rely on an adaptation of the Hu-Washizu principle \cite{simo_variational_1985}, where a discrete interpolation of the strain is evaluated at the element level. Such a procedure allows,
following an appropriate choice in terms of discretization, to eliminate the stress and stress unknowns from the initial three field problem, resulting in the definition of a discrete differentiation operator,
which coined the name "Assumed strain". Among these methods is the famous "B-bar" method, that proves to be robust to volumetric locking.

% The denomination "assumed strain methods" is intended to encompass a variety of finite element procedures, often proposed on an ad-hoc basis, which are typically characterized by an interpolation of the discrete gradient operator assumed apriori, independently of the interpolation adopted for the displacement field. The often referred to "B-bar method", proposed by Hughes \cite{belytschko_ac0_1984}, offers an example of an assumed strain method which has proven successful in a variety of situations, including widely used structural elements \cite{hughes_finite_1981}.
% For the finite strain incompressible problem, this method has been precisely reformulated by Simo et al. \cite{simo_variational_1985} within the context of the Hu-Washizu principle.
% The so-called mode decomposition technique, proposed by Belytschko (e.g., [\textcolor{blue}{ref}]), furnishes another example of a B-bar type of method that leads to the formulation of successful structural elements.

% In XXX, Hughes and SImon showed that that assumed strain methods can be systematically formulated within the variational framework furnished by the Hu-Washizu principle. A crucial point in this development concerns the role played by the stress field, now entering the formulation as a Lagrange multiplier, and its recovery within the proposed variational structure. It is first noted that the Lagrange multipliers drop out from the formulation leading to a generalized displacement model, provided a certain orthogonality condition on the assumed strain field is satisfied. In addition, as a result of the variational structure, the admissible stress field (Lagrange multipliers) is constrained by an orthogonality condition arising from the Hu-Washizu principle as an Euler-Lagrange equation. These orthogonality conditions result in a single.

% ---------------------------------------------------------
% PARAGRAPH
% ---------------------------------------------------------
\paragraph{Mixed methods}

The term Mixed method denotes families of methods, that derive from the Hu-Washizu principle by introducing lower-order strain-based and stress-based tensorial fields
in place of the whole stress and strain fields. This can be done by introducing \textit{e.g.} a free energy potential split \cite{malkus_mixed_1978, al-akhrass_methodes_nodate}
(see Section \ref{sec_pressure_swelling} for more details on this note),
which allows to reduce the initial supplementary second order tensorial unknowns to \textit{e.g.} scalar pressure and swelling fields when considering incompressibility problems.

% Since volumetric locking is a pressure dependent phenomenon,
% considering for instance a decomposition of the stress and strain fields
% into \textit{e.g.} devatoric and spherical components, one can express a
% mixed problem in terms of pressure and swelling, which is at the origin
% of so-called UPG methods \cite{al_akhrass_integrating_2014,
%   simo_quasi-incompressible_1991,simo_variational_1985}. The scalar
% pressure and swelling unknowns replace respectively the stress and
% strain tensorial unknowns in \eqref{eq_HW_0} and a modified deformation
% gradient is introduced in the constitutive equation.

% In particular, Simo et al. \cite{simo_variational_1985} proved the equivalence of such an approach to that of the  Hu-Washizu principle when considering incompressibility problems.
% Studies concerning the equivalence of the modified displacement approaches and Hu-Washizu
% mixed approaches were conducted by Simo et al. \cite{simo_variational_1985} with regard to incompressibility problems. \cite{hughes_equivalence_1977,oden_observations_1975, shimodaira_equivalence_1985}

% At the same time,
% studies concerning the equivalence of the modified displacement approaches and Hu-Washizu
% mixed approaches were conducted by Simo et al. \cite{simo_variational_1985} with regard to incompressibility
% problems, and by Simo and Hughes \cite{simo_variational_1985} in a more general context of the assumed strain
% (B-bar) approach. Other related works are \cite{hughes_equivalence_1977,oden_observations_1975, shimodaira_equivalence_1985}

% ---------------------------------------------------------
% -- SUBSECTION
% ---------------------------------------------------------
\subsection{From Discontinuous Galerkin methods to Hybrid Discontinuous Galerkin methods}

% The Hybrid High Order method (HHO) is a discontinuous discretization
% method, that takes root in the Discontinuous Galerkin method (DG).
From the physical standpoint, discontinuous methods ensure the continuity of the flux
across interfaces, by seeking the solution element-wise, hence allowing
jumps of the potential across elements. They can be seen as a
generalization of Finite Volume methods, and are able to capture
physically relevant discontinuities without producing spurious
oscillations, such as volumetric locking.

% ---------------------------------------------------------
% -- SUBSUBSECTION
% ---------------------------------------------------------
\paragraph{Discontinuous Galerkin (DG) methods}

The origin of DG methods dates back to the pioneering work of
\cite{reed_triangular_1973}, where an hyperbolic formualtion is used to
solve the neutron transport equation. The first application of the
method to elliptic problems originates in \cite{babuska_finite_1973}
where Nitsche's method \cite{nitsche_uber_1970} is used to weakly impose
continuity of the flux across interfaces.
% ---> TODO : reformulate
In 2002, Hansbo and Larson \cite{hansbo_discontinuous_2002-1} were the first to
consider the Nitsche's classical DG method for nearly incompressible
elasticity. They showed, theoretically and numerically, that this
method is free from volumetric locking.
% --->

% % ---------------------------------------------------------
% % PARAGRAPH
% % ---------------------------------------------------------
% \paragraph{Application in Linear Elasticity}

%
%

% ---------------------------------------------------------
% PARAGRAPH
% ---------------------------------------------------------
\paragraph{Symmetric interior penalty}

Since the resulting bilinear form
arising from the initial formulation of DG methods is not symmetric, a so called interior
penalty term has been introduced in \cite{wheeler_elliptic_1978},
leading to the so-called Symmetric Interior Penalty (SIP) DG method. A first study
of the method to linear elasticity has been devised by
\cite{riviere_optimal_2000}, where optimal error estimate has been
proved.
%
%
%
% ---> TODO : reformulate
\cite{lew_optimal_2004} generalized the
Symmetric Interior Penalty method to linear elasticity.
In about the same
period of time, DG methods were proposed for other linear problems in
solid mechanics, such as Timoshenko beams
\cite{celiker_locking-free_2006}, Bernoulli-Euler beam and the
Poisson-Kirchhoff plate \cite{brenner_balancing_1999,
  engel_continuousdiscontinuous_2002} and Reissner-Mindlin plates
\cite{arnold_family_2005}. In the mid 2000's, the first applications
of DG methods to nonlinear elasticity problems was undertaken by
\cite{ten_eyck_discontinuous_2006, noels_general_2006}, and in 2007,
Ortner and Süli \cite{ortner_discontinuous_2007} carried out the a
priori error analysis of DG methods for nonlinear elasticity.
% --->

% % ---------------------------------------------------------
% % PARAGRAPH
% % ---------------------------------------------------------
% \paragraph{DG methods in solid mechanics}
% %
% %
% %
% DG methods then sollicitated a vigourus interest, mostly in fluid dynamics \cite{shahbazi_high-order_2007, persson_discontinuous_2009} due to their local conservative property and stability in convection domniated problems. However, except some applications for instance in fracture mechanics using XFEM methods \cite{gracie_blending_2008, shen_stability_2010}, or gradient plasticity \cite{djoko_discontinuous_2007,djoko_discontinuous_2007-1} DG methods did not break through in computational solid mechanics because of their numerical cost, since nodal unknowns need be duplicated to define local basis functions in each element.

% ---------------------------------------------------------
% -- SUBSUBSECTION
% ---------------------------------------------------------
\paragraph{Hybrid Discontinuous Galerkin (HDG) methods}

DG methods then sollicitated a vigourus interest, mostly in fluid dynamics \cite{shahbazi_high-order_2007, persson_discontinuous_2009} due to their local conservative property and stability in convection domniated problems.
However, except some applications for instance in fracture mechanics using XFEM methods
\cite{gracie_blending_2008, shen_stability_2010}, or gradient plasticity \cite{djoko_discontinuous_2007,djoko_discontinuous_2007-1} DG methods did not break through in computational solid mechanics because of their numerical cost, since nodal unknowns need be duplicated to define local basis functions in each element.
To adress this problem, in the early 2010's, \cite{cockburn_unified_2009, soon_hybridizable_2009} introduced additional faces unknowns on element interfaces for linear elastic problem, hence leading to the hybridization of DG methods, or Hybridizable Discontinuous Galerkin method (HDG). By adding supplementary boundary unknowns, the authors actually allowed to eliminate original cell unknowns by a static condensation process, in order to express the global problem on faces ones only.
% 
% % ---------------------------------------------------------
% % PARAGRAPH
% % ---------------------------------------------------------
% \paragraph{HDG methods in solid mechanics}
% 
Extension of HDG methods to non-linear elasticity were first undertaken in \cite{soon_hybridizable_2008} and have then fueled intense reaserch works for various applications such as linear and non-linear convection-diffusion problems \cite{nguyen_implicit_2009,nguyen_implicit_2009-1,nguyen_hybridizable_2010}, incompressible stokes flows \cite{nguyen_hybridizable_2010, nguyen_implicit_2011} and non-linear mechanics \cite{nguyen_hybridizable_2012}.

% ---------------------------------------------------------
% PARAGRAPH
% ---------------------------------------------------------
\paragraph{HHO methods}

In \cite{di_pietro_hybrid_2015, di_pietro_arbitrary-order_2014}, the authors introduced a higher order potential reconstruction operator in the classical HDG formulation for elliptic problems, providing a $h^{k+1} H^1$-norm convergence rate as compared to the ususal $h^k$-rate. This higher order term coined the name for the so called HHO method.
Recent developments of HHO methods in
computational mechanics include the incompressible Stokes
equations (with possibly large irrotational forces) \cite{di_pietro_discontinuous_2016}, the
incompressible Navier–Stokes equations \cite{di_pietro_hybrid_2018}, Biot’s consolidation problem \cite{boffi_nonconforming_2016}, and nonlinear elasticity with small
deformations \cite{botti_hybrid_2017}

% \subsection{Aim of this paper}

% \subsection{Outline}

% In the first part of this paper, the Hu-Washizu variational principle is presented, and the Principle of Virtual Work is introduced as specification of the Hu-Washizu principle.
% The description of an element 