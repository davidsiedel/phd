% ---------------------------------------------------------
% ---- SECTION
% ---------------------------------------------------------
\section{The standard Hu-Washizu variational approach}
\label{sec_model_problem}

This section introduces the standard Hu–Washizu three field
principle. For the sake of simplicity, and without loss of generality,
let consider the case of an hyperelastic material. Extensions to
mechanical behaviours with internal state variables are treated in
classical textbooks of computational mechanics
\cite{belytschko_nonlinear_nodate,besson_non-linear_2010}, and is outlined in Section \ref{sec:discretization:extension_to_non_linear_materials}
in the context of the proposed variational formulation for HDG methods.
% This extension in Section~\ref{sec:discretization:extension_to_non_linear_materials} for
% theorical aspects and in Section~\ref{sec_appendix_implementation} which
% discusses our numerical implementation of the Hybrid High Order method.
% This implementation is used in Section~\ref{sec_numerical_examples}
% which provides several examples in finite strain plasticity.

% ---------------------------------------------------------
% -- SUBSECTION
% ---------------------------------------------------------
\subsection{Description of the mechanical problem and notations}

% ---------------------------------------------------------
% PARAGRAPH
% ---------------------------------------------------------
\paragraph{Solid body}

Let us consider a solid body whose reference configuration is denoted
$\bodyLag$. At a given time $t > 0$, the body is in the current
configuration $\bodyEul$.

% ---------------------------------------------------------
% PARAGRAPH
% ---------------------------------------------------------
\paragraph{Mechanical loading}

The body $\bodyEul$ is assumed to be submitted to a body force $\loadEul$,
a prescribed displacement $\dirichletEul$ on the
Dirichlet boundary $\dirichletBoundaryEul$, and a surface load
$\neumannEul{}$ on the Neumann boundary $\neumannBoundaryEul$.

% ---------------------------------------------------------
% PARAGRAPH
% ---------------------------------------------------------
\paragraph{Deformation}

The transformation mapping 
$\tensori{\Phi}$ takes a point from the reference configuration $\bodyLag$ to the current
configuration $\bodyEul$ such that
%
%
%
\begin{equation}
    \tensori{\Phi}\paren{\tensori{X}} = \tensori{x} = \tensori{X}+\tensori{u}\paren{\tensori{X}}
\end{equation}
%
%
%
where $\tensori{X}$, $\tensori{x}$ and $\tensori{u}$ denote respectively
the position in the reference configuration $\bodyLag$, the position
in the current configuration $\bodyEul$ and the displacement.

% ---------------------------------------------------------
% PARAGRAPH
% ---------------------------------------------------------
\paragraph{Deformation gradient, gradient of the displacement}

The transformation gradient $\tensorii{F}$ is defined as
%
%
%
\begin{equation}
    \tensorii{F} = \nabla \tensori{\Phi} = \tensorii{I} + \tensorii{G}
\end{equation}
%
%
%
where $\nabla$ is the gradient operator in the
reference configuration and 
%
%
%
\begin{equation}
    \label{eq_grad_def}
    \tensorii{G} = \nabla \tensori{u}
\end{equation}
%
%
%
denotes the gradient of the
displacement.

% ---------------------------------------------------------
% PARAGRAPH
% ---------------------------------------------------------
\paragraph{Stress tensor}

The body is assumed made of an hyperelastic material described by a
free energy $\mecPotential_{\bodyLag{}}$ which relates the deformation gradient
$\tensorii{F}$ and the first Piola-Kirchhoff stress tensor $\tensorii{P}$ such that
%
%
%
\begin{equation}
    \label{eq_stress_def}
  \tensorii{P}=\deriv{\mecPotential_{\bodyLag{}}}{\tensorii{F}}
\end{equation}

% ---------------------------------------------------------
% PARAGRAPH
% ---------------------------------------------------------
\paragraph{Mechanical loading in the reference configuration}

$\bodyLag$ morphs under the action of volumetric forces $\tensori{f}_{V}$ and tractions forces
$\neumannLag$, that have been obtained from
their counterparts $\tensori{f}_{v}$ and $\neumannEul$ using the
Nanson formulae. In the following, for simplicity and without loss of generality,
they are assumed independent on the transformation gradient.
A prescribed displacement $\dirichletLag$ is imposed on $\dirichletBoundaryLag{}$.

% ---------------------------------------------------------
% -- SUBSECTION
% ---------------------------------------------------------
\subsection{Primal problem and principle of Virtual Work}

% ---------------------------------------------------------
% PARAGRAPH
% ---------------------------------------------------------
\paragraph{Total lagrangian}

The total Lagrangian $L^{VW}_{\bodyLag{}}$ of the body is defined as
the stored energy minus the work of external loadings, as follows:
%
%
%
\begin{equation}
\label{eq_Lagrangian}
L^{VW}_{\bodyLag{}}
% \paren{\tensori{u}}
= \int_{\Omega}\mecPotential_{\bodyLag{}}
(\tensorii{F}(\tensori{u}))
% (\tensorii{I} + \nabla \tensori{u})
% \paren{\tensorii{F}\paren{\tensori{u}}}
- \int_{\bodyLag} \tensori{f}{}_V \cdot \tensori{u}{}
- \int_{\neumannBoundaryLag} \neumannLag{} \cdot \tensori{u}{}
\vert_{\neumannBoundaryLag}
\end{equation}
%
%
%
% where the body forces $\tensori{f}_{V}$ and conctat tractions
% $\neumannLag$ in the reference configuration have been obtained from
% their counterparts $\tensori{f}_{v}$ and $\neumannEul$ using the
% Nanson formulae,
% and where the displacement field is prescribed by $\dirichletLag$ on $\dirichletBoundaryLag{}$ in the reference configuration.
% In the following, for simplicity and without loss of generality,
% Dirichlet boundary conditions are omitted from the developments, and volumetric and surface loads are assumed independent on the transformation gradient.
% 
% % ---------------------------------------------------------
% % PARAGRAPH
% % ---------------------------------------------------------
% \paragraph{Principle of Virtual Works}
% 
The displacement $\tensori{u}$ satisfying
the mechanical equilibrium minimizes the Lagragian $L^{VW}_{\bodyLag{}}$.
The first order variation of the Lagrangian is given by
%
%
%
\begin{equation}
  \label{eq_virtual_works_0}
  % \langle \frac{\partial L^{VW}_{\bodyLag{}}}{\partial \tensori{u}} \cdot \delta \tensori{u} \rangle
  \langle \frac{\partial L^{VW}_{\bodyLag{}}}{\partial \tensori{u}} , \delta \tensori{u} \rangle
  =
  \int_{\bodyLag} \tensorii{P} : \nabla \delta \tensori{u} -
  \int_{\bodyLag} \tensori{f}_V \cdot \delta \tensori{u} -
  \int_{\neumannBoundaryLag} \neumannLag{} \cdot \delta \tensori{u}
  \vert_{\neumannBoundaryLag}
\end{equation}
%
%
%
which must be null for the the solution displacement, and thus yields the well known principle of \textit{Virtual Work},
where the notation $\langle \cdot , \cdot \rangle$ defines the usual duality pairing.
% where the left-hand side of \eqref{eq_virtual_works_0} reads as the well known \textit{Principle of Virtual Work}.
% . The solution
% displacement thus satisfies the principle of virtual work
% %
% %
% %
% \[
% \int_{\bodyLag} \tensorii{P} : \nabla \delta \tensori{u} =
% \int_{\bodyLag} \tensori{f}_V \cdot \delta \tensori{u} +
% \int_{\neumannBoundaryLag} \neumannLag{} \cdot \delta \tensori{u}
% \vert_{\neumannBoundaryLag}
% \quad
% \forall \delta \tensori{u}{}
% \]

% ---------------------------------------------------------
% -- SUBSECTION
% ---------------------------------------------------------
\subsection{The Hu-Washizu variational approach}
\label{sec_HW_lagrangian}

\paragraph{Hu-Washizu Lagrangian}

The Hu-Washizu Lagrangian $L^{HW}_{\bodyLag{}}$
\cite{hu_variational_1954, washizu_variational_1955, washizu_variational_1974} generalizes
the previous variational principle by considering that the gradient of
the displacement $\tensorii{G}$ and the first Piola-Kirchoff
$\tensorii{P}$ stress are independent unknowns of the problem, such
that:
%
%
%
\begin{equation}
  \label{eq_HW_0} L^{HW}_{\bodyLag{}}
  % \paren{\tensori{u},\tensorii{G}, \tensorii{P}}
  = \int_{\bodyLag{}}
  \mecPotential_{\bodyLag{}}
  % (\tensorii{I}+\tensorii{G})
  (\tensorii{F}(\tensorii{G}))
  + \int_{\bodyLag{}}  (\nabla
  \tensori{u}{} - \tensorii{G}{})\,\colon\,\tensorii{P} -
  \int_{\bodyLag{}} \loadLag \cdot \tensori{u}{} -
  \int_{\neumannBoundaryLag{}} \neumannLag{} \cdot \tensori{u}
  \vert_{\neumannBoundaryLag}
\end{equation}
%
%
%
The solution $(\tensori{u}, \tensorii{G}, \tensorii{P})$
satisfying the mechanical equilibrium minimizes the Lagragian
$L^{HW}_{\bodyLag{}}$. The first order variation of the Hu-Washizu
Lagragian with respect to $\tensori{u}, \tensorii{G}$, and
$\tensorii{P}$ yields % % %
\begin{subequations}
  \label{eq_hu_washizu_derivative_0}
  \begin{alignat}{3}
    \langle \frac{\partial L^{HW}_{\bodyLag{}}}{\partial \tensori{u}} , \delta \tensori{u} \rangle
    =
    % \deriv{L^{HW}_{\bodyLag{}}}{\tensori{u}}
    % \cdot \delta \tensori{u}
    % =
    & \int_{\bodyLag} \tensorii{P} : \nabla
    \delta \tensori{u} - \int_{\bodyLag} \tensori{f}_V \cdot \delta
    \tensori{u} - \int_{\neumannBoundaryLag} \neumannLag \cdot \delta
    \tensori{u} \vert_{\neumannBoundaryLag} && \ \ \ \ \ \ \ \ &&
    \forall \delta \tensori{u}{} \label{eq_hu_washizu_derivative_0:eq0}
    \\
    \langle \frac{\partial L^{HW}_{\bodyLag{}}}{\partial \tensorii{P}} , \delta \tensorii{P} \rangle
    % \deriv{L^{HW}_{\bodyLag{}}}{\tensorii{P}}
    % : \delta \tensorii{P}
    =
    & \int_{\bodyLag} ( \nabla \tensori{u} -
    \tensorii{G} ) : \delta \tensorii{P} && \ \ \ \ \ \ \ \ && \forall
    \delta \tensorii{P}{} \label{eq_hu_washizu_derivative_0:eq2}
    \\
    \langle \frac{\partial L^{HW}_{\bodyLag{}}}{\partial \tensorii{G}} , \delta \tensorii{G} \rangle
    % \deriv{L^{HW}_{\bodyLag{}}}{\tensorii{G}}
    % : \delta \tensorii{G}
    = & \int_{\bodyLag} (\frac{\partial
      \mecPotential}{\partial \tensorii{G}} - \tensorii{P}) : \delta
    \tensorii{G} && \ \ \ \ \ \ \ \ && \forall \delta \tensorii{G}{}
    \label{eq_hu_washizu_derivative_0:eq3}
  \end{alignat}
\end{subequations}
where Equations~\eqref{eq_hu_washizu_derivative_0:eq2}
and~\eqref{eq_hu_washizu_derivative_0:eq3} account
for~\eqref{eq_grad_def} and~\eqref{eq_stress_def} respectively in a weak
sense.

% ---------------------------------------------------------
% -- SUBSECTION
% ---------------------------------------------------------
\subsection{Classical applications of the Hu-Washizu variational
  approach in computational mechanics to circumvent volumetric locking}

In the continuous framework, the Hu-Washizu functional is not
relevant, since Equations~\eqref{eq_hu_washizu_derivative_0:eq2}
and~\eqref{eq_hu_washizu_derivative_0:eq3} would lead to the following
equalities:
%
%
%
\begin{equation}
  \begin{aligned}
    \tensorii{G} = \nabla\tensori{u} && \text{and} && \tensorii{P}=\frac{\partial \mecPotential}{\partial \tensorii{G}}
  \end{aligned}
\end{equation}
%
%
%
% \[
% \tensorii{G} = \nabla\tensori{u}
% \quad\text{and}\quad
% \tensorii{P}=\frac{\partial \mecPotential}{\partial \tensorii{G}}
% \]
% 
However, considering finite-dimensional functional spaces, multiple choices arise
for the specification of the Lagrangian \eqref{eq_HW_0}.

% ---------------------------------------------------------
% PARAGRAPH
% ---------------------------------------------------------
\paragraph{Pressure swelling formulations}
\label{sec_pressure_swelling}

Since volumetric locking is a pressure dependent phenomenon,
considering for instance a decomposition of the stress and strain fields
into \textit{e.g.} devatoric and spherical components, one can express a
mixed problem in terms of pressure and swelling, which is at the origin
of so-called UPG methods \cite{al_akhrass_integrating_2014,
  simo_quasi-incompressible_1991, simo_variational_1985}. The scalar
pressure and swelling unknowns replace respectively the stress and
strain tensorial unknowns in \eqref{eq_HW_0} and a modified deformation
gradient is introduced in the constitutive equation.

% ---------------------------------------------------------
% -- SUBSECTION
% ---------------------------------------------------------
\paragraph{Enhanced assumed strains formulations}
\label{sec_eas}

Another approach of the use of the Hu-Washizu consists in
studying the equilibrium of a single element. Such a framework falls
into the scope of so-called Enhanced Assumed Strains methods
\cite{simo_variational_1986, simo_class_1990}, which result for instance
in the "B-bar" method.
It consists in eliminating both the stress and gradient field, by defining a discrete gradient operator
that verifies a certain orthogonality condition with the stress field,
such that the sole displacement unknowns remains in the formulation.
Such a procedure is somehow similar to the one described in Section \ref{sec_hdg_element_equilibrium},
where both the stress and gradient fields are also eliminated from the problem.

% ---------------------------------------------------------
% -- SUBSECTION
% ---------------------------------------------------------
\paragraph{Towards Discontinuous methods}

In the present document, an introduction to so-called
\textit{non-conformal} methods is proposed, by means of the Hu–Washizu Lagrangian.
At the origin of these methods is the Discontinuous Galerkin (DG)
method, which postulates the discontinuity of the displacement across
elements. This feature allows the method to be robust to volumetric
locking. However, its formulation takes root in a possibly dry
mathematical background, and the ingredients of the method are not
introduced in the literature through physical arguments. The
next section aims at introducing the whole framework of non-conformal
methods, including the displacement discontinuity, through the
Hu–Washizu Lagragian.
The main difference with former works using both a Hu-Washizu Lagrangian in the context of Discontinuous methods is that the present document
introduces the discontinuity of the displacement field from the application of the Hu-Washizu Lagrangian to an element surrounded by an interface as defined in Section \ref{sec_appendix_composite_demo},
whereas \textit{e.g.} \cite{noels_general_2006} and \cite{neunteufel_three-field_2021} formulate a Hu-Washizu Lagrangian using a discretization that already incorporates the discontinuity of the displacement field.
In particular, the proposed variational formulation in Section \ref{sec_hdg_element_equilibrium} naturally introduces the main ingredients of HDG methods (namely the \textit{reconstructed gradient} and the \textit{reconstructed traction force}) as a consequence of the minimization of some Lagrangian, instead of introducing them \textit{a priori}.

% Though one counts a few occurances of the use of the Hu–Washizu Lagragian in the context of discontinuous methods \cite{noels_general_2006,neunteufel_three-field_2021}, none, to our knowledge, introduce all the ingredients of the method through the 
% sole Hu–Washizu Lagragian.

% and though one counts a few applications of the Hu–Washizu Lagrangian for Discontinuous Galerkin methods \cite{}, none of them exploit the 
% Its application in mechanics had not resulted in a break through, and
% so did not its variants, among which the Hybird Discontinuous Galerkin method and the Hybird High Order method.
% Though one counts a few applications of the Hu–Washizu Lagrangian for Discontinuous Galerkin methods 

% En continue, aucun intérêt. Par contre, très puissant une fois les
% bases d'approximations choisies.

% Many variants:

% - Ne considérer uniquement l'espace sphérique.

% - Gardez des champs globaux: U-P-G inconnues nodales. Variantes liées aux choix des espaces d'approximations de U, P et G (Al-Akrass).

% - Travailler par éléments: Assumed strain (c.f. Belytchko).
