\documentclass[anti,rapport,anglais]{note_technique_2018}

\newcommand\hmmax{0}
\newcommand\bmmax{0}

\titre{Hybrid High Order methods applied to incompressible
(visco-)plastic flows and phase field approaches of fracture.
Application de la méthode Hybrid High Order aux écoulements
(visco-)plastiques incompressibles et aux approches variationnelles de
la rupture.}

\titrecourt{Second year report of D. Siedel' PhD. Rapport de seconde
année de la thèse de D. Siedel.}

\input{LSC}

\auteurs{David Siedel, Thomas Helfer, Olivier Fandeur (*)}

\affiliation{(*) DES/ISAS/DM2S/SEMT/LM2S}

% \codebarre{img/barcode.jpg}

\redacteur{\mbox{D. Siedel}}

\fredacteur{\mbox{Doctorant}}

\verificateur{M. Josien (DEC/SESC/LM2C)}

% \dateversion{05/2021}

%\numero{21-005}

%\chrono{CEA/DES/IRESNE/DEC/SESC DO 54}

\indice{A}


\numeroaffaire{A-SICOM-02-01-01}

\clients{EDF-Framatome}

\domaine{SIMU}

\programmerecherche{SICOM}

\accords{CEA-EDF-FRAMATOME F35158}

\typeaction{80\% CEA -- 20\% EDF}

\usepackage{tcolorbox}
\usepackage[bbgreekl]{mathbbol}
\usepackage[eulergreek]{sansmath}
\usepackage{stmaryrd}

\newtcolorbox{myquote}{colback=blue!5!white, colframe=blue!75!black}
\renewenvironment{quote}{\begin{myquote}}{\end{myquote}}
\NoLogoAFAQ
\CoupeListeDiffusion
\diffusioninterne{%
DES/EC/DSE              & É. Proust           & 1 & Diffusion courriel\\
DES/EC/DPE/SIMU         & X. Raepsaet         & 1 & \\
                        & T. Laporte          & 1 & \\
DES/EC/DGCP/SA2P        & A. Soniak-Defresne  & 1 & \\
DES/ISAS/               & C. Santucci         & 1 & \\
                        & M. Serre            & 1 & \\
DES/ISAS/DMN            & P. Bossis           & 1 & \\
DES/ISAS/DPC/SECR/LECBA & B. Bary             & 1 & \\
DES/ISAS/DM2S           & P. Blanc-Tranchant  & 1 & \\
                        & D. Caruge           & 1 & \\
                        & E. Foerster         & 1 & \\
                        & G. Rampal           & 1 & \\
DES/ISAS/DM2S/STMF      & P. Gavoille         & 1 & \\
DES/ISAS/DM2S/STMF/LGLS & E. Adam             & 1 & \\
DES/ISAS/DM2S/SEMT      & S. Naury            & 1 & \\
                        & Y. Kaiser           & 1 & \\
DES/ISAS/DM2S/SEMT/LISN & C. Gourdin          & 1 & \\
DES/ISAS/DM2S/SEMT/DYN  & C. Gauthier         & 1 & \\
                        & B. Prabel           & 1 & \\
DES/ISAS/DM2S/SEMT/LM2S & J.-C. Le Pallec     & 1 & \\
                        & O. Fandeur          & 1 & \\
                        & F. Dipoala          & 1 & \\
IRESNE/DIR      & J-M. Morey & 1 &\\
                & C. Dellis & 1 &\\
                & P. Dumaz & 1 &\\
                & I. Tkatchenko & 1 &\\
DES/IRESNE/DEC          & M. Delpech   &   & Document disponible\\
                        & O. Charlent &   & sur intradec  \\
                        & C. Valot   &   & \\
DES/IRESNE/DEC/CP       & L. Sauvage     &   & \\
                        & É. Fédérici    &   & \\
DES/IRESNE/DEC/SESC     & M. Bauer       &   & \\
                        & R. Eschbach    &   & \\
DES/IRESNE/DEC/SESC/LEVA & A. Monnier    &   & \\
                          & J. Sercombe   &   & \\
DES/IRESNE/DEC/SESC/LECIM & N. Chikhi    &   & \\
DES/IRESNE/DEC/SESC/LM2C & E. Bourasseau &   & \\
DES/IRESNE/DEC/SESC/LSC & B. Collard     &   & \\
                        & V. Marelle     &   & \\
                        & T. Helfer      &   & \\
                        & D. Siedel      &   & \\
}


\diffusionexterne{%
EDF R\&D             & P. Barbrault    & 1 & Diffusion courriel \\
                     & C. Le Maître    & 1 & \\
EDF R\&D Palaiseau   & D. Geoffroy     & 1 & \\
                     & M. Abbas        & 1 & \\
                     & N. Pignet       & 1 & \\
EDF R\&D Renardières & R. Nhili        & 1 & \\
                     & Y. Sirsalane    & 1 & \\
                     & É. Pouillier    & 1 & \\
EDF/R\&D/CADARACHE   & K. Audic        & 1 & \\
                     & R. Largenton    & 1 & \\
                     & L. Parissi      & 1 & \\
                     & A. Kececioglu   & 1 & \\
EDF/DIPNN/DT         & J-J. Vermoyal   & 1 & \\
                     & P. Dias         & 1 & \\
                     & L. Barbié       & 1 & \\
EDF/DCN              & A.-F. Cotte     & 1 & \\
                     & R. Fernandes    & 1 & \\
                     & S. Pélissier    & 1 & \\
École Centrale Nantes/GEM        & N. Moès         & 1 & \\
UPMC/Institut d'Alembert & D. Leguillon & 1 & \\
INSA-Lyon/LAMCOS         & A. Gravouil  & 1 & \\
École des Ponts ParisTech/Laboratoire Navier         & J. Bleyer & 1 & \\
}


% Fix footnotes in tables (requires footnote package)

\usepackage{graphicx}

\resumeDO{

  This document describes two papers written during the second
  year of the PhD thesis of David Siedel:
  \begin{itemize}
    \item One deriving the Hybrid Discontinuous Galerkin (HDG)
    and Hybrid High Order (HHO) methods from the Hu-Washizu principle.
    \item One exploring a class of micromophorphic damage
    behaviours and their link with phase field approaches to fracture.
  \end{itemize}

  \textbf{Résumé français}

  Ce rapport décrit le travail réalisé au cours et autour de la
  seconde année du doctorat de D. Siedel. Ce travail s'est focalisé sur
  la rédaction de deux papier:

  \begin{itemize}
    \item L'un montrant comment les méthodes Hybrid
    Discontinuous Galerkin (HDG) et Hybrid High Order (HHO) peuvent être
    déduites du principe de Hu-Washizu.
    \item L'autre étudiant une classe de modèles micromorphes
    de l'endommagement et leurs liens avec les approaches par champs de
    phase.
  \end{itemize}
}

\resumeDR{
Sans objet.
}

\motsclefs{Méthode Hybrid High Order method. Approches variationnelles de
la rupture.}

\usepackage{amsmath}
\usepackage{amssymb}
\usepackage[euler]{textgreek}
\usepackage{soul}

\usepackage{graphicx}
\usepackage{../notations}

\newcommand*{\refname}{Bibliography}

\begin{document}

\chapter{Introduction}

This document describes two papers written during the second year of
the PhD thesis of David Siedel:
\begin{itemize}
  \item One deriving the Hybrid Discontinuous Galerkin (HDG) and
  Hybrid High Order (HHO) methods from the Hu-Washizu principle. This
  paper is reported in Chapter \ref{chapter:hho} of this report.
  \item One exploring a class of micromophorphic damage
  behaviours and their link with phase field approaches to fracture.
  This paper is reported in Chapter \ref{chapter:micromorphic_damage} of
  this report.
\end{itemize}

\chapter{Hybrid High Order methods for non-linear solid mechanics}
\label{chapter:hho}

\section{Introduction}
\label{sec_introduction}

\subsection{Outline}
\label{sec_introduction_outline}

% This paper aims at outlining a link between the Hu–Washizu
% variational principle and that of so called relaxed formulations
% % ,
% % from which are derived \textit{e.g.} Hybrid Discontinuous Galerkin
% % (HDG) methods, and, in particular, one of their latest refinement, the
% % Hybrid High Order (HHO) method.
% among which the family of Discontinuous Galerkin (DG) and Hybrid Discontinuous Galerkin
% (HDG) methods.
% This paper focuses on the specific case of Hybrid High Order (HHO) methods, that are a specific.
% , and, in particular, one of their latest refinement, the
% Hybrid High Order (HHO) method.

When solving quasi-incompressible boundary-value problems with the
Finite Element Method (FEM) using standard Lagrange interpolation, high
oscillations of the hydrostatic pressure might occur. This numerical
artifact, also known as volumetric locking is discussed in depth in
classical textbooks of computational
mechanics~\cite{simo_computational_1998, belytschko_nonlinear_nodate, neto_computational_2008}.
An enriched approach is required to resolve or, at least, alleviate
this issue.

Most works in the literature rely on either one of these two
following approaches to derive numerical methods robust to volumetric
locking:
\begin{itemize}
  \item The Hu-Washizu variational principle.
  \item Relaxed formulations, among which the family
  of Discontinuous Galerkin (DG) and Hybrid Discontinuous Galerkin (HDG)
  methods, including one of their latest refinement, the Hybrid High Order
  (HHO) method.
\end{itemize}

% Both these approaches introduce supplementary unknowns to the problem, and many works in the literature rely on either one of these two approaches to derive numerical methods that prove to be robust to volumetric locking phenomena.
% Both these methods are at the foundation of numerical methods

Both approaches introduce supplementary unknowns to the problem:
\begin{itemize}
  \item On the one hand, the Hu-Washizu variational principle
  extends that of Virtual Work by introducing supplementary stress and
  strain fields, that act as Lagrange multipliers to enforce
  respectively the constitutive and strain-displacement relations.
  \item On the other hand, relaxed formulations provide a richer
  kinematics by introducing a displacement jump at element boundaries,
  hence allowing for the definition of enhanced and local strain and
  stress fields.
\end{itemize}

The present paper proposes a variational formulation for HDG methods
based on a Hu-Washizu approach, where the discontinuity of the
displacement field is retrieved by introducing a linear elastic
interface between elements, hence outlining a common framework to both
Hu-Washizu based approaches for continuous displacement fields, and
hybrid discontinuous methods. A novel resolution algorithm, based on the
proposed variational approach is devised and numerically tested, using
the proposed HHO method for axi-symmetric problems.

% The following introduction outlines in a first part the development and applications of The Hu–Washizu variational
% formulation and in a second part that of discontinuous methods.

% DIRE HW A ETE REGARDE POUR LINCOMPRESSIBILITE PLASTIQUE ETC, AVANTDE PARLER DES LETHDODES
% -> INTRODUIRE LES 
% -> NOTATIONS (voir papier micromorphe)
% -> charges mortes pour les temres de passage en configurartion initiale, pas celles de Nanson
% -> les brackets : dimension finie, produit scalair intégrale, nsur des vectiueyrs : produit scalair vectoriel
% -> le potentiel élastique, non-standrad mais pour retrouver les expressions exactes des HDG, passer par vrai potentiel élastique , à tester. Passage en zone cohézive, extension intéressante pas traitée dans le document

% ---------------------------------------------------------
% -- SUBSECTION
% ---------------------------------------------------------
\subsection{From the Hellinger-Reissner principle to the Hu–Washizu variational formulation}

% Before 1950, variational principles considered only displacement
% as a single independent field.
% Generalized variational principles began in the 1950's with the
% breakthrough works of Reissner \cite{reissner_variational_1950} on two-field variational principles for elasticity problems, in
% which the displacement and stress were considered independent fields. Subsequently, de Vebeuke \cite{fraeijs_de_veubeke_diffusion_1951} constructed a
% four-field variational principle, and Hu \cite{hu_variational_1954} and Washizu \cite{washizu_variational_1955} established three-field variational
% principle independently.
% In 1983, Chien \cite{chien_method_1983} first pointed out that the three kinds of
% variables in Hu-Washizu principle are but nct independent
% of each other. Stress-strain relations are still its variational
% constraints, which could be removed only by high-order
% Lagrange multiplier method.

The principle of Virtual Work, which is at the foundation of the Finite Element theory, considers the displacement field alone as the unknown of the mechanical problem.
The first generalized variational principle dates back to the work of Reissner \cite{reissner_variational_1950} in the framework of linear elasticity, where the stress field does
not define as a function of the displacement gradient (or strain) anymore, but is now an unknown of the problem, in the same way as the displacement field.
In 1954 and 1955, Hu \cite{hu_variational_1954} and Washizu \cite{washizu_variational_1955} independently established a three-field variational principle, consisting of a stress,
a strain and a displacement field, hence giving rise to the so-called Hu-Washizu principle, which lays the groundwork for the so-called mixed and enhanced assumed strain methods.
However, de Vebeuke had already introduced in the early 50's a four-field principle, consisting of stress, strain, displacement and surface forces fields, and the Hu-Wahsizu principle
can be seen as an application of the de Vebeuke principle, where the surface force field is assimilated to the normal stress.

% ---------------------------------------------------------
% -- SUBSUBSECTION
% ---------------------------------------------------------
\paragraph{Assumed strain methods}

So-called Assumed Strain methods rely on an adaptation of the Hu-Washizu principle \cite{simo_variational_1985}, where a discrete interpolation of the strain is evaluated at the element level. Such a procedure allows,
following an appropriate choice in terms of discretization, to eliminate the stress and stress unknowns from the initial three field problem, resulting in the definition of a discrete differentiation operator,
which coined the name "Assumed strain". Among these methods is the famous "B-bar" method, that proves to be robust to volumetric locking.

% The denomination "assumed strain methods" is intended to encompass a variety of finite element procedures, often proposed on an ad-hoc basis, which are typically characterized by an interpolation of the discrete gradient operator assumed apriori, independently of the interpolation adopted for the displacement field. The often referred to "B-bar method", proposed by Hughes \cite{belytschko_ac0_1984}, offers an example of an assumed strain method which has proven successful in a variety of situations, including widely used structural elements \cite{hughes_finite_1981}.
% For the finite strain incompressible problem, this method has been precisely reformulated by Simo et al. \cite{simo_variational_1985} within the context of the Hu-Washizu principle.
% The so-called mode decomposition technique, proposed by Belytschko (e.g., [\textcolor{blue}{ref}]), furnishes another example of a B-bar type of method that leads to the formulation of successful structural elements.

% In XXX, Hughes and SImon showed that that assumed strain methods can be systematically formulated within the variational framework furnished by the Hu-Washizu principle. A crucial point in this development concerns the role played by the stress field, now entering the formulation as a Lagrange multiplier, and its recovery within the proposed variational structure. It is first noted that the Lagrange multipliers drop out from the formulation leading to a generalized displacement model, provided a certain orthogonality condition on the assumed strain field is satisfied. In addition, as a result of the variational structure, the admissible stress field (Lagrange multipliers) is constrained by an orthogonality condition arising from the Hu-Washizu principle as an Euler-Lagrange equation. These orthogonality conditions result in a single.

% ---------------------------------------------------------
% PARAGRAPH
% ---------------------------------------------------------
\paragraph{Mixed methods}

The term Mixed method denotes families of methods, that derive from the Hu-Washizu principle by introducing lower-order strain-based and stress-based tensorial fields
in place of the whole stress and strain fields. This can be done by introducing \textit{e.g.} a free energy potential split \cite{malkus_mixed_1978, al-akhrass_methodes_nodate}
(see Section \ref{sec_pressure_swelling} for more details on this note),
which allows to reduce the initial supplementary second order tensorial unknowns to \textit{e.g.} scalar pressure and swelling fields when considering incompressibility problems.

% Since volumetric locking is a pressure dependent phenomenon,
% considering for instance a decomposition of the stress and strain fields
% into \textit{e.g.} devatoric and spherical components, one can express a
% mixed problem in terms of pressure and swelling, which is at the origin
% of so-called UPG methods \cite{al_akhrass_integrating_2014,
%   simo_quasi-incompressible_1991,simo_variational_1985}. The scalar
% pressure and swelling unknowns replace respectively the stress and
% strain tensorial unknowns in \eqref{eq_HW_0} and a modified deformation
% gradient is introduced in the constitutive equation.

% In particular, Simo et al. \cite{simo_variational_1985} proved the equivalence of such an approach to that of the  Hu-Washizu principle when considering incompressibility problems.
% Studies concerning the equivalence of the modified displacement approaches and Hu-Washizu
% mixed approaches were conducted by Simo et al. \cite{simo_variational_1985} with regard to incompressibility problems. \cite{hughes_equivalence_1977,oden_observations_1975, shimodaira_equivalence_1985}

% At the same time,
% studies concerning the equivalence of the modified displacement approaches and Hu-Washizu
% mixed approaches were conducted by Simo et al. \cite{simo_variational_1985} with regard to incompressibility
% problems, and by Simo and Hughes \cite{simo_variational_1985} in a more general context of the assumed strain
% (B-bar) approach. Other related works are \cite{hughes_equivalence_1977,oden_observations_1975, shimodaira_equivalence_1985}

% ---------------------------------------------------------
% -- SUBSECTION
% ---------------------------------------------------------
\subsection{From Discontinuous Galerkin methods to Hybrid Discontinuous Galerkin methods}

% The Hybrid High Order method (HHO) is a discontinuous discretization
% method, that takes root in the Discontinuous Galerkin method (DG).
From the physical standpoint, discontinuous methods ensure the continuity of the flux
across interfaces, by seeking the solution element-wise, hence allowing
jumps of the potential across elements. They can be seen as a
generalization of Finite Volume methods, and are able to capture
physically relevant discontinuities without producing spurious
oscillations, such as volumetric locking.

% ---------------------------------------------------------
% -- SUBSUBSECTION
% ---------------------------------------------------------
\paragraph{Discontinuous Galerkin (DG) methods}

The origin of DG methods dates back to the pioneering work of
\cite{reed_triangular_1973}, where an hyperbolic formualtion is used to
solve the neutron transport equation. The first application of the
method to elliptic problems originates in \cite{babuska_finite_1973}
where Nitsche's method \cite{nitsche_uber_1970} is used to weakly impose
continuity of the flux across interfaces.
% ---> TODO : reformulate
In 2002, Hansbo and Larson \cite{hansbo_discontinuous_2002-1} were the first to
consider the Nitsche's classical DG method for nearly incompressible
elasticity. They showed, theoretically and numerically, that this
method is free from volumetric locking.
% --->

% % ---------------------------------------------------------
% % PARAGRAPH
% % ---------------------------------------------------------
% \paragraph{Application in Linear Elasticity}

%
%

% ---------------------------------------------------------
% PARAGRAPH
% ---------------------------------------------------------
\paragraph{Symmetric interior penalty}

Since the resulting bilinear form
arising from the initial formulation of DG methods is not symmetric, a so called interior
penalty term has been introduced in \cite{wheeler_elliptic_1978},
leading to the so-called Symmetric Interior Penalty (SIP) DG method. A first study
of the method to linear elasticity has been devised by
\cite{riviere_optimal_2000}, where optimal error estimate has been
proved.
%
%
%
% ---> TODO : reformulate
\cite{lew_optimal_2004} generalized the
Symmetric Interior Penalty method to linear elasticity.
In about the same
period of time, DG methods were proposed for other linear problems in
solid mechanics, such as Timoshenko beams
\cite{celiker_locking-free_2006}, Bernoulli-Euler beam and the
Poisson-Kirchhoff plate \cite{brenner_balancing_1999,
  engel_continuousdiscontinuous_2002} and Reissner-Mindlin plates
\cite{arnold_family_2005}. In the mid 2000's, the first applications
of DG methods to nonlinear elasticity problems was undertaken by
\cite{ten_eyck_discontinuous_2006, noels_general_2006}, and in 2007,
Ortner and Süli \cite{ortner_discontinuous_2007} carried out the a
priori error analysis of DG methods for nonlinear elasticity.
% --->

% % ---------------------------------------------------------
% % PARAGRAPH
% % ---------------------------------------------------------
% \paragraph{DG methods in solid mechanics}
% %
% %
% %
% DG methods then sollicitated a vigourus interest, mostly in fluid dynamics \cite{shahbazi_high-order_2007, persson_discontinuous_2009} due to their local conservative property and stability in convection domniated problems. However, except some applications for instance in fracture mechanics using XFEM methods \cite{gracie_blending_2008, shen_stability_2010}, or gradient plasticity \cite{djoko_discontinuous_2007,djoko_discontinuous_2007-1} DG methods did not break through in computational solid mechanics because of their numerical cost, since nodal unknowns need be duplicated to define local basis functions in each element.

% ---------------------------------------------------------
% -- SUBSUBSECTION
% ---------------------------------------------------------
\paragraph{Hybrid Discontinuous Galerkin (HDG) methods}

DG methods then sollicitated a vigourus interest, mostly in fluid dynamics \cite{shahbazi_high-order_2007, persson_discontinuous_2009} due to their local conservative property and stability in convection domniated problems.
However, except some applications for instance in fracture mechanics using XFEM methods
\cite{gracie_blending_2008, shen_stability_2010}, or gradient plasticity \cite{djoko_discontinuous_2007,djoko_discontinuous_2007-1} DG methods did not break through in computational solid mechanics because of their numerical cost, since nodal unknowns need be duplicated to define local basis functions in each element.
To adress this problem, in the early 2010's, \cite{cockburn_unified_2009, soon_hybridizable_2009} introduced additional faces unknowns on element interfaces for linear elastic problem, hence leading to the hybridization of DG methods, or Hybridizable Discontinuous Galerkin method (HDG). By adding supplementary boundary unknowns, the authors actually allowed to eliminate original cell unknowns by a static condensation process, in order to express the global problem on faces ones only.
% 
% % ---------------------------------------------------------
% % PARAGRAPH
% % ---------------------------------------------------------
% \paragraph{HDG methods in solid mechanics}
% 
Extension of HDG methods to non-linear elasticity were first undertaken in \cite{soon_hybridizable_2008} and have then fueled intense reaserch works for various applications such as linear and non-linear convection-diffusion problems \cite{nguyen_implicit_2009,nguyen_implicit_2009-1,nguyen_hybridizable_2010}, incompressible stokes flows \cite{nguyen_hybridizable_2010, nguyen_implicit_2011} and non-linear mechanics \cite{nguyen_hybridizable_2012}.

% ---------------------------------------------------------
% PARAGRAPH
% ---------------------------------------------------------
\paragraph{HHO methods}

In \cite{di_pietro_hybrid_2015, di_pietro_arbitrary-order_2014}, the authors introduced a higher order potential reconstruction operator in the classical HDG formulation for elliptic problems, providing a $h^{k+1} H^1$-norm convergence rate as compared to the ususal $h^k$-rate. This higher order term coined the name for the so called HHO method.
Recent developments of HHO methods in
computational mechanics include the incompressible Stokes
equations (with possibly large irrotational forces) \cite{di_pietro_discontinuous_2016}, the
incompressible Navier–Stokes equations \cite{di_pietro_hybrid_2018}, Biot’s consolidation problem \cite{boffi_nonconforming_2016}, and nonlinear elasticity with small
deformations \cite{botti_hybrid_2017}

% \subsection{Aim of this paper}

% \subsection{Outline}

% In the first part of this paper, the Hu-Washizu variational principle is presented, and the Principle of Virtual Work is introduced as specification of the Hu-Washizu principle.
% The description of an element 
% ---------------------------------------------------------
% ---- SECTION
% ---------------------------------------------------------
\section{The standard Hu-Washizu variational approach}
\label{sec_model_problem}

This section introduces the standard Hu–Washizu
principle. For the sake of simplicity, and without loss of generality,
let consider the case of an hyperelastic material. Extensions to
mechanical behaviours with internal state variables are treated in
classical textbooks of computational mechanics
\cite{belytschko_nonlinear_nodate,besson_non-linear_2010}, and is outlined in Section \ref{sec:discretization:extension_to_non_linear_materials}
in the context of the proposed variational formulation for HDG methods.
% This extension in Section~\ref{sec:discretization:extension_to_non_linear_materials} for
% theorical aspects and in Section~\ref{sec_appendix_implementation} which
% discusses our numerical implementation of the Hybrid High Order method.
% This implementation is used in Section~\ref{sec_numerical_examples}
% which provides several examples in finite strain plasticity.

% ---------------------------------------------------------
% -- SUBSECTION
% ---------------------------------------------------------
\subsection{Description of the mechanical problem and notations}

% ---------------------------------------------------------
% PARAGRAPH
% ---------------------------------------------------------
\paragraph{Solid body}

Let us consider a solid body whose reference configuration is denoted
$\bodyLag$. At a given time $t > 0$, the body is in the current
configuration $\bodyEul$.

% ---------------------------------------------------------
% PARAGRAPH
% ---------------------------------------------------------
\paragraph{Mechanical loading}

The body $\bodyEul$ is assumed to be submitted to a body force $\loadEul$,
a prescribed displacement $\dirichletEul$ on the
Dirichlet boundary $\dirichletBoundaryEul$, and a surface load
$\neumannEul{}$ on the Neumann boundary $\neumannBoundaryEul$.

% ---------------------------------------------------------
% PARAGRAPH
% ---------------------------------------------------------
\paragraph{Deformation}

The transformation mapping 
$\tensori{\Phi}$ takes a point from the reference configuration $\bodyLag$ to the current
configuration $\bodyEul$ such that
%
%
%
\begin{equation}
    \tensori{\Phi}\paren{\tensori{X}} = \tensori{x} = \tensori{X}+\tensori{u}\paren{\tensori{X}},
\end{equation}
%
%
%
where $\tensori{X}$, $\tensori{x}$ and $\tensori{u}$ denote respectively
the position in the reference configuration $\bodyLag$, the position
in the current configuration $\bodyEul$ and the displacement.

% ---------------------------------------------------------
% PARAGRAPH
% ---------------------------------------------------------
\paragraph{Deformation gradient, gradient of the displacement}

The transformation gradient $\tensorii{F}$ is defined as
%
%
%
\begin{equation}
    \tensorii{F} = \nabla \tensori{\Phi} = \tensorii{I} + \tensorii{G},
\end{equation}
%
%
%
where $\nabla$ is the gradient operator in the
reference configuration and 
%
%
%
\begin{equation}
    \label{eq_grad_def}
    \tensorii{G} = \nabla \tensori{u},
\end{equation}
%
%
%
denotes the gradient of the
displacement.

% ---------------------------------------------------------
% PARAGRAPH
% ---------------------------------------------------------
\paragraph{Stress tensor}

The body is assumed made of an hyperelastic material described by a
free energy $\mecPotential_{\bodyLag{}}$ which relates the deformation gradient
$\tensorii{F}$ and the first Piola-Kirchhoff stress tensor $\tensorii{P}$ such that
%
%
%
\begin{equation}
    \label{eq_stress_def}
  \tensorii{P}=\deriv{\mecPotential_{\bodyLag{}}}{\tensorii{F}}.
\end{equation}

% ---------------------------------------------------------
% PARAGRAPH
% ---------------------------------------------------------
\paragraph{Mechanical loading in the reference configuration}

$\bodyLag$ morphs under the action of volumetric forces $\tensori{f}_{V}$ and tractions forces
$\neumannLag$, that have been obtained from
their counterparts $\tensori{f}_{v}$ and $\neumannEul$ using the
Nanson formulae. In the following, for simplicity and without loss of generality,
they are assumed independent on the transformation gradient.
A prescribed displacement $\dirichletLag$ is imposed on $\dirichletBoundaryLag{}$.

% ---------------------------------------------------------
% -- SUBSECTION
% ---------------------------------------------------------
\subsection{Primal problem and principle of Virtual Works}

% ---------------------------------------------------------
% PARAGRAPH
% ---------------------------------------------------------
\paragraph{Total lagrangian}

The total Lagrangian $L^{VW}_{\bodyLag{}}$ of the body is defined as
the stored energy minus the work of external loadings, as follows:
%
%
%
\begin{equation}
\label{eq_Lagrangian}
L^{VW}_{\bodyLag{}}
% \paren{\tensori{u}}
= \int_{\Omega}\mecPotential_{\bodyLag{}}
(\tensorii{F}(\tensori{u}))
% (\tensorii{I} + \nabla \tensori{u})
% \paren{\tensorii{F}\paren{\tensori{u}}}
- \int_{\bodyLag} \tensori{f}{}_V \cdot \tensori{u}{}
- \int_{\neumannBoundaryLag} \neumannLag{} \cdot \tensori{u}{}
\vert_{\neumannBoundaryLag}.
\end{equation}
%
%
%
% where the body forces $\tensori{f}_{V}$ and conctat tractions
% $\neumannLag$ in the reference configuration have been obtained from
% their counterparts $\tensori{f}_{v}$ and $\neumannEul$ using the
% Nanson formulae,
% and where the displacement field is prescribed by $\dirichletLag$ on $\dirichletBoundaryLag{}$ in the reference configuration.
% In the following, for simplicity and without loss of generality,
% Dirichlet boundary conditions are omitted from the developments, and volumetric and surface loads are assumed independent on the transformation gradient.
% 
% % ---------------------------------------------------------
% % PARAGRAPH
% % ---------------------------------------------------------
% \paragraph{Principle of Virtual Works}
% 
The displacement $\tensori{u}$ satisfying
the mechanical equilibrium minimizes the Lagragian $L^{VW}_{\bodyLag{}}$.
The first order variation of the Lagrangian is given by
%
%
%
\begin{equation}
  \label{eq_virtual_works_0}
  % \langle \frac{\partial L^{VW}_{\bodyLag{}}}{\partial \tensori{u}} \cdot \delta \tensori{u} \rangle
  \langle \frac{\partial L^{VW}_{\bodyLag{}}}{\partial \tensori{u}} , \delta \tensori{u} \rangle
  =
  \int_{\bodyLag} \tensorii{P} : \nabla \delta \tensori{u} -
  \int_{\bodyLag} \tensori{f}_V \cdot \delta \tensori{u} -
  \int_{\neumannBoundaryLag} \neumannLag{} \cdot \delta \tensori{u}
  \vert_{\neumannBoundaryLag},
\end{equation}
%
%
%
which must be null for the the solution displacement, and thus yields the well known principle of \textit{Virtual Work},
where the notation $\langle \cdot , \cdot \rangle$ defines the usual duality pairing.
% where the left-hand side of \eqref{eq_virtual_works_0} reads as the well known \textit{Principle of Virtual Work}.
% . The solution
% displacement thus satisfies the principle of virtual work
% %
% %
% %
% \[
% \int_{\bodyLag} \tensorii{P} : \nabla \delta \tensori{u} =
% \int_{\bodyLag} \tensori{f}_V \cdot \delta \tensori{u} +
% \int_{\neumannBoundaryLag} \neumannLag{} \cdot \delta \tensori{u}
% \vert_{\neumannBoundaryLag}
% \quad
% \forall \delta \tensori{u}{}
% \]

% ---------------------------------------------------------
% -- SUBSECTION
% ---------------------------------------------------------
\subsection{The Hu-Washizu variational approach}
\label{sec_HW_lagrangian}

\paragraph{Hu-Washizu Lagrangian}

The Hu-Washizu Lagrangian $L^{HW}_{\bodyLag{}}$
\cite{hu_variational_1954, washizu_variational_1955, washizu_variational_1974} generalizes
the previous variational principle by considering that the gradient of
the displacement $\tensorii{G}$ and the first Piola-Kirchoff
$\tensorii{P}$ stress are independent unknowns of the problem, such
that:
%
%
%
\begin{equation}
  \label{eq_HW_0} L^{HW}_{\bodyLag{}}
  % \paren{\tensori{u},\tensorii{G}, \tensorii{P}}
  = \int_{\bodyLag{}}
  \mecPotential_{\bodyLag{}}
  % (\tensorii{I}+\tensorii{G})
  (\tensorii{F}(\tensorii{G}))
  + \int_{\bodyLag{}}  (\nabla
  \tensori{u}{} - \tensorii{G}{})\,\colon\,\tensorii{P} -
  \int_{\bodyLag{}} \loadLag \cdot \tensori{u}{} -
  \int_{\neumannBoundaryLag{}} \neumannLag{} \cdot \tensori{u}
  \vert_{\neumannBoundaryLag}.
\end{equation}
%
%
%
The solution $(\tensori{u}, \tensorii{G}, \tensorii{P})$
satisfying the mechanical equilibrium minimizes the Lagragian
$L^{HW}_{\bodyLag{}}$. The first order variation of the Hu-Washizu
Lagragian with respect to $\tensori{u}, \tensorii{G}$, and
$\tensorii{P}$ yields % % %
\begin{subequations}
  \label{eq_hu_washizu_derivative_0}
  \begin{alignat}{3}
    \langle \frac{\partial L^{HW}_{\bodyLag{}}}{\partial \tensori{u}} , \delta \tensori{u} \rangle
    =
    % \deriv{L^{HW}_{\bodyLag{}}}{\tensori{u}}
    % \cdot \delta \tensori{u}
    % =
    & \int_{\bodyLag} \tensorii{P} : \nabla
    \delta \tensori{u} - \int_{\bodyLag} \tensori{f}_V \cdot \delta
    \tensori{u} - \int_{\neumannBoundaryLag} \neumannLag \cdot \delta
    \tensori{u} \vert_{\neumannBoundaryLag} && \qquad &&
    \forall \delta \tensori{u}{} \label{eq_hu_washizu_derivative_0:eq0},
    \\
    \langle \frac{\partial L^{HW}_{\bodyLag{}}}{\partial \tensorii{P}} , \delta \tensorii{P} \rangle
    % \deriv{L^{HW}_{\bodyLag{}}}{\tensorii{P}}
    % : \delta \tensorii{P}
    =
    & \int_{\bodyLag} ( \nabla \tensori{u} -
    \tensorii{G} ) : \delta \tensorii{P} && \qquad && \forall
    \delta \tensorii{P}{} \label{eq_hu_washizu_derivative_0:eq2},
    \\
    \langle \frac{\partial L^{HW}_{\bodyLag{}}}{\partial \tensorii{G}} , \delta \tensorii{G} \rangle
    % \deriv{L^{HW}_{\bodyLag{}}}{\tensorii{G}}
    % : \delta \tensorii{G}
    = & \int_{\bodyLag} (\frac{\partial
      \mecPotential}{\partial \tensorii{G}} - \tensorii{P}) : \delta
    \tensorii{G} && \qquad && \forall \delta \tensorii{G}{}
    \label{eq_hu_washizu_derivative_0:eq3},
  \end{alignat}
\end{subequations}
where Equations~\eqref{eq_hu_washizu_derivative_0:eq2}
and~\eqref{eq_hu_washizu_derivative_0:eq3} account
for~\eqref{eq_grad_def} and~\eqref{eq_stress_def} respectively in a weak
sense.

% ---------------------------------------------------------
% -- SUBSECTION
% ---------------------------------------------------------
\subsection{Classical applications of the Hu-Washizu variational
  approach in computational mechanics to circumvent volumetric locking}

In the continuous framework, the Hu-Washizu functional is not
relevant, since Equations~\eqref{eq_hu_washizu_derivative_0:eq2}
and~\eqref{eq_hu_washizu_derivative_0:eq3} would lead to the following
equalities:
%
%
%
\begin{equation}
  \begin{aligned}
    \tensorii{G} = \nabla\tensori{u} && \text{and} && \tensorii{P}=\frac{\partial \mecPotential}{\partial \tensorii{G}},
  \end{aligned}
\end{equation}
%
%
%
% \[
% \tensorii{G} = \nabla\tensori{u}
% \quad\text{and}\quad
% \tensorii{P}=\frac{\partial \mecPotential}{\partial \tensorii{G}}
% \]
% 
However, considering finite-dimensional functional spaces, multiple choices arise
for the specification of the Lagrangian \eqref{eq_HW_0}.

% ---------------------------------------------------------
% PARAGRAPH
% ---------------------------------------------------------
\paragraph{Pressure swelling formulations}
\label{sec_pressure_swelling}

Since volumetric locking is a pressure dependent phenomenon,
considering for instance a decomposition of the stress and strain fields
into \textit{e.g.} devatoric and spherical components, one can express a
mixed problem in terms of pressure and swelling, which is at the origin
of so-called UPG methods \cite{al_akhrass_integrating_2014,
  simo_quasi-incompressible_1991, simo_variational_1985}. The scalar
pressure and swelling unknowns replace respectively the stress and
strain tensorial unknowns in \eqref{eq_HW_0} and a modified deformation
gradient is introduced in the constitutive equation.

% ---------------------------------------------------------
% -- SUBSECTION
% ---------------------------------------------------------
\paragraph{Enhanced assumed strains formulations}
\label{sec_eas}

Another approach of the use of the Hu-Washizu consists in
studying the equilibrium of a single element. Such a framework falls
into the scope of so-called Enhanced Assumed Strains methods
\cite{simo_variational_1986, simo_class_1990}, which result for instance
in the "B-bar" method.
It consists in eliminating both the stress and gradient field, by defining a discrete gradient operator
that verifies a certain orthogonality condition with the stress field,
such that the sole displacement unknowns remains in the formulation.
Such a procedure is somehow similar to the one described in Section \ref{sec_hdg_element_equilibrium},
where both the stress and gradient fields are also eliminated from the problem.

% ---------------------------------------------------------
% -- SUBSECTION
% ---------------------------------------------------------
\paragraph{Towards Discontinuous methods}

In the present document, an introduction to so-called
\textit{non-conformal} methods is proposed, by means of the Hu–Washizu Lagrangian.
At the origin of these methods is the Discontinuous Galerkin (DG)
method, which postulates the discontinuity of the displacement across
elements. This feature allows the method to be robust to volumetric
locking. However, its formulation takes root in a possibly harsh
mathematical background, and the ingredients of the method are not
introduced in the literature through physical arguments. The
next section aims at introducing the whole framework of non-conformal
methods, including the displacement discontinuity, through the
Hu–Washizu Lagragian.
The main difference with former works using both a Hu-Washizu Lagrangian in the context of Discontinuous methods is that the present document
introduces the discontinuity of the displacement field from the application of the Hu-Washizu Lagrangian to an element surrounded by an interface as defined in Section \ref{sec_appendix_composite_demo},
whereas \textit{e.g.} \cite{noels_general_2006} and \cite{neunteufel_three-field_2021} formulate a Hu-Washizu Lagrangian using a discretization that already incorporates the discontinuity of the displacement field.
In particular, the proposed variational formulation in Section \ref{sec_hdg_element_equilibrium} naturally introduces the main ingredients of HDG methods (namely the \textit{reconstructed gradient} and the \textit{reconstructed traction force}) as a consequence of the minimization of some Lagrangian, instead of introducing them \textit{a priori}.

% Though one counts a few occurances of the use of the Hu–Washizu Lagragian in the context of discontinuous methods \cite{noels_general_2006,neunteufel_three-field_2021}, none, to our knowledge, introduce all the ingredients of the method through the 
% sole Hu–Washizu Lagragian.

% and though one counts a few applications of the Hu–Washizu Lagrangian for Discontinuous Galerkin methods \cite{}, none of them exploit the 
% Its application in mechanics had not resulted in a break through, and
% so did not its variants, among which the Hybird Discontinuous Galerkin method and the Hybird High Order method.
% Though one counts a few applications of the Hu–Washizu Lagrangian for Discontinuous Galerkin methods 

% En continue, aucun intérêt. Par contre, très puissant une fois les
% bases d'approximations choisies.

% Many variants:

% - Ne considérer uniquement l'espace sphérique.

% - Gardez des champs globaux: U-P-G inconnues nodales. Variantes liées aux choix des espaces d'approximations de U, P et G (Al-Akrass).

% - Travailler par éléments: Assumed strain (c.f. Belytchko).

% ---------------------------------------------------------
% ---- SECTION
% ---------------------------------------------------------
\section{Introduction to the Hybrid Discontinuous Galerkin methods through the Hu-Washizu variational principle}
\label{sec_appendix_composite_demo}

In this section, let $\cell$ a subpart of the body \(\bodyLag\), called a \textit{cell}.
% In the following, one assumes that the
% cell $\cell$ is located inside the body $\bodyLag{}$
% , such that its boundary $\dCell{}$ bears contact loads only.
The cell $\cell$ is in
equilibrium with the rest of the body \(\Omega\backslash T\) if the
displacements and the normal traction are continuous at the boundary
$\dCell{}$.

% ---------------------------------------------------------
% PARAGRAPH
% ---------------------------------------------------------
\paragraph{Conformal methods} Enforcing the displacement continuity at the interface leads to
so-called conformal methods, to which the standard Lagrange Finite Element
method belongs (see Figure \ref{fig_02}(a)).

% ---------------------------------------------------------
% PARAGRAPH
% ---------------------------------------------------------
\paragraph{Discontinuous Galerkin methods} On
the contrary, this condition can be weakened by introducing an elastic
interface of negligible size between \(T\) and \(\Omega\backslash T\).
This representation is at the basis of Discontinuous Galerkin methods
(see Figure \ref{fig_02}(b)).

% ---------------------------------------------------------
% PARAGRAPH
% ---------------------------------------------------------
\paragraph{Hybird Discontinuous Galerkin methods} In
this paper, HDG methods are considered,
where two elastic interfaces are introduced: one between \(T\) and its
boundary \(\partial T\) and a second one between \(\Omega\backslash T\)
and \(\partial T\) (see Figure \ref{fig_02}(c)).
Following this idea,
this section outlines how the use of the Hu-Hashizu Lagrangian allows to
introduce both the
\textit{reconstructed gradient} and the \textit{reconstructed traction force}.

% ---------------------------------------------------------
% -- SUBSECTION
% ---------------------------------------------------------
\subsection{Element description}

% ---------------------------------------------------------
% PARAGRAPH
% ---------------------------------------------------------
\paragraph{Element geometry}

In the following, the cell $\cell$ is assumed to be convex.
It is split into a core part $\Bulk \subset \cell$ with boundary $\dBulk$, and into an interface part $\Crown{} \subset \cell$ with boundary $\dCrown = \dBulk \cup \dCell$, as shown in Figure \ref{fig_02}. The interface $\Crown{}$ has some thickness $\ell > 0$ that is supposed to be small compared to $h_{\cell}$ the diameter of $\cell$.
From a geometrical standpoint, the core part of the element $\Bulk{}$ is an homothety of $\cell$ by some ratio less than $1$.

% ---------------------------------------------------------
% PARAGRAPH
% ---------------------------------------------------------
\paragraph{Element boundary description} The boundary $\dCell{}$ of $\cell$ is the composition of a Neumann boundary $\neumannCell{}$ and a Dirichlet $\dirichletCell{}$, if the element $\cell$ shares a boundary with $\dirichletBoundaryLag{}$. In the following, for the sake of simplicity, the element is assumed to be located inside the body $\bodyLag{}$, such that is only subjected to imposed traction forces on $\neumannCell{} = \dCell{}$ with $\dirichletCell{} = \emptyset$.
%
% 
% 
\begin{figure}[H]
    \centering
    \includegraphics[width=14.cm]{../chapter_002_hho_mechanics/figures/ef_dg_hdg.png}
    \caption{schematic representation of a cell and its surrounding depending on the continuity requirement of the displacement field}
    \label{fig_02}
\end{figure}
%
%
%

% ---------------------------------------------------------
% PARAGRAPH
% ---------------------------------------------------------
\paragraph{Element behaviour}

The core of the element $\Bulk{}$ is made out of the same material that composes $\bodyLag$ and behaves according to the free energy potential $\mecPotential{}_{\bodyLag{}}$. The interface $\Crown{}$ is made out of a pseudo linear elastic material of Young modulus $\beta (\ell / h_{\cell})$ with a zero Poisson ratio and its behavior is defined by the free energy potential $\mecPotential{}_{\Crown{}}$ such that
%
%
%
\begin{equation}
    \label{eq_0009}
        \mecPotential{}_{\Crown} = \frac{1}{2} \beta \frac{\ell}{h_{\cell}} \nabla \tensori{u}{}_{\Crown} : \nabla \tensori{u}{}_{\Crown}
\end{equation}
%
%
%
where the dimensionless ratio $\ell / h_{\cell}$ balances the accumulated energy with the size of the domain $\cell$.

% ---------------------------------------------------------
% PARAGRAPH
% ---------------------------------------------------------
\paragraph{Element loading}

The core $\Bulk$ is subjected to the volumetric loading $\loadLag{}$, and to the traction force applied by the interface $\Crown{}$ onto $\dBulk{}$. By continuity, $\Bulk{}$ applies the opposite traction force on $\Crown{}$ through $\dBulk{}$. The interface $\Crown{}$ is also subjected to the exterior traction force $\neumannCellLoad{}$ acting on $\neumannCell{}$, that accounts for the action of the rest of the solid $\bodyLag{}$ onto the boundary $\dCell$.

% ---------------------------------------------------------
% PARAGRAPH
% ---------------------------------------------------------
\paragraph{Discplacement, displacement gradient and stress fields}

Let note $\tensori{u}{}_{\Bulk}$ the displacement field, $\tensorii{G}{}_{\Bulk}$ the displacement gradient field and $\tensorii{P}{}_{\Bulk}$ the stress field in $\Bulk{}$. Similarly, let $\tensori{u}{}_{\Crown{}}$ the displacement field, $\tensorii{G}{}_{\Crown}$ the displacement gradient field and $\tensorii{P}{}_{\Crown}$ the stress field in $\Crown{}$.
The displacement of the boundary $\dCell{}$ is denoted $\tensori{u}{}_{\dCell{}}$.
By continuity of the displacement field between $\Bulk{}$ and $\dCell$,  the displacement $\tensori{u}{}_{\Crown{}}$ verifies
%
% 
% 
\begin{subequations}
    \label{eq_conformity}
        \begin{alignat}{2}
        \tensori{u}{}_{\Crown} \vert_{\dBulk} & = \tensori{u}{}_{\Bulk} \vert_{\dBulk}
        \label{eq_conformity:eq1}
        \\
        \tensori{u}{}_{\Crown} \vert_{\dCell} & = \tensori{u}{}_{\dCell}
        \label{eq_conformity:eq2}
    \end{alignat}
\end{subequations}

% ---------------------------------------------------------
% PARAGRAPH
% ---------------------------------------------------------
\paragraph{Hu-Washizu Lagrangian of the element}

By combining both the Lagragian of the core $\Bulk{}$ and that of the interface $\Crown{}$, one obtains the total Lagragian $L_{\cell}^{HW}$ over the element such that
%
%
%
\begin{equation}
    \label{eq_hu_washizu_split}
    L_{\cell}^{HW}
    % (\tensori{u}{}_{\cell}, \tensorii{G}{}_{\cell}, \tensorii{P}{}_{\cell})
    =
    \int_{\Bulk} \mecPotential_{\bodyLag{}} + \int_{\Bulk} (\nabla \tensori{u}{}_{\Bulk} - \tensorii{G}{}_{\Bulk}) : \tensorii{P}{}_{\Bulk}
    +
    \int_{\Crown} \mecPotential_{\Crown{}} + \int_{\Crown} (\nabla \tensori{u}{}_{\Crown} - \tensorii{G}{}_{\Crown}) : \tensorii{P}{}_{\Crown}
    -
    \int_{\Bulk} \loadLag \cdot \tensori{u}{}_{\Bulk}
    % -
    % \int_{\Crown} \loadLag \cdot \tensori{u}{}_{\Crown}
    -
    \int_{\neumannCell} \neumannCellLoad \cdot \tensori{u}{}_{\dCell}
\end{equation}

% ---------------------------------------------------------
% -- SUBSECTION
% ---------------------------------------------------------
\subsection{Hypotheses}
\label{sec_assumtions}

% Since the interface is of negligible volume compared to that of the core, the following assumptions are made on the displacement and the stress fields in the interface:

% ---------------------------------------------------------
% PARAGRAPH
% ---------------------------------------------------------
\paragraph{Displacement in the interface}

Since the interface is of negligible volume compared to that of the core, the displacement in the interface $\Crown$ is assumed to be linear with respect to $\tensori{n}$, such that
its gradient is homogeneous in $\Crown{}$ along $\tensori{n}$
%
% 
% 
\begin{equation}
    \label{eq_crown_displacement}
    \nabla
    \tensori{u}{}_{\Crown}
    =
    \frac{\tensori{u}{}_{\dCell}
    -
    \tensori{u}{}_{\Bulk} \vert_{\dBulk} }{\ell} \otimes \tensori{n}
\end{equation}
% 
% 
%
That is, the displacement of the interface $\Crown{}$ linearly bridges that of the boundary $\dCell{}$ to that of the bulk $\Bulk{}$.

% ---------------------------------------------------------
% PARAGRAPH
% ---------------------------------------------------------
\paragraph{Stress in the interface}

Likewise, let assume that $\tensorii{P}{}_{\Crown}$ is constant along the direction $\tensori{n}{}$ in $\Crown{}$. By continuity of the traction force across $\dBulk$, the following equality holds true
%
% 
% 
\begin{equation}
    \label{eq_continuity_traction_force}
    \begin{aligned}
        (\tensorii{P}{}_{\Crown} - \tensorii{P}{}_{\Bulk} \vert_{\dBulk{}}) \cdot \tensori{n}{} = 0
        &&
        \text{on}
        &&
        \dBulk{}
        % &&
        % \text{in}
        % &&
        % \Crown{}
    \end{aligned}
\end{equation}

% ---------------------------------------------------------
% -- SUBSECTION
% ---------------------------------------------------------
\subsection{Towards Hybrid discontinuous methods from the Hu-Washizu Lagrangian}

% Using the hypotheses stated in Section \ref{sec_assumtions} on the displacement and the stress field in $\Crown{}$,
% \eqref{eq_hu_washizu_split} can be expressed as a term depending on the thickness of the interface $\ell$ and on the core and boundary unknowns only.
% In particular, making the thickness of the interface $\ell \rightarrow 0$, such that $\Crown{}$ vanishes and the core part $\Bulk{}$ identifies to $\cell$,
% a simplified expression of \eqref{eq_hu_washizu_split} is obtained.
% The reader can refer to \ref{sec_appendix_Hu_Washizu} for more details on technical details.

% ---------------------------------------------------------
% PARAGRAPH
% ---------------------------------------------------------
\paragraph{Simplified Hu–Washizu Lagrangian for a vanishing interface}

% In particular, making the thickness of the interface $\ell \rightarrow 0$, such that $\Crown{}$ vanishes and the core part $\Bulk{}$ identifies to $\cell$,
% The following simplified Hu–Washizu Lagrangian is obtained (See \ref{sec_appendix_Hu_Washizu})
% Using hypotheses expressed in Section \ref{sec_assumtions} and making $\ell \rightarrow 0$, the following simplified Hu–Washizu Lagrangian arises from \eqref{eq_hu_washizu_split}
Using the hypotheses stated in Section \ref{sec_assumtions} on the displacement and the stress field in $\Crown{}$,
\eqref{eq_hu_washizu_split} can be expressed as a term depending on the thickness of the interface $\ell$ and on the core and boundary unknowns only.
In particular, making the thickness of the interface $\ell \rightarrow 0$, such that $\Crown{}$ vanishes and the core part $\Bulk{}$ identifies to $\cell$,
a simplified expression of \eqref{eq_hu_washizu_split} is obtained
% 
% 
%
\begin{equation}
    \label{eq_0015}
    \begin{aligned}
        L_{\cell}^{HW}
        = &
        \int_{\cell{}} \mecPotential{}_{\bodyLag{}} + \int_{\cell{}} (\nabla \tensori{u}{}_{\cell{}} - \tensorii{G}{}_{\cell{}}) : \tensorii{P}{}_{\cell}
        % \\
        % &
        + \int_{\dCell{}} (\tensori{u}{}_{\dCell} - \tensori{u}{}_{\cell} \vert_{\dCell}) \cdot \tensorii{P}{}_{\cell} \vert_{\dCell{}} \cdot \tensori{n}{}
        % \\
        % &
        + \int_{\dCell} \frac{\beta}{2 h_{\cell}} \lVert \tensori{u}{}_{\dCell{}} - \tensori{u}{}_{\cell{}} \vert_{\dCell{}} \rVert^2
        \\
        &
        -
        \int_{\cell} \loadLag{} \cdot \tensori{u}{}_{\cell{}}
        -
        \int_{\neumannCell{}} \neumannCellLoad{} \cdot \tensori{u}{}_{\dCell{}}
    \end{aligned}
\end{equation}
%
%
%
such that \eqref{eq_0015} fully defines the equilibrium of an element in the context of hybrid discontinuous methods. The reader can refer to \ref{sec_appendix_Hu_Washizu} for more details on technical details.

% ---------------------------------------------------------
% PARAGRAPH
% ---------------------------------------------------------
\paragraph{HDG methods and hybridization of the primal unknown}

Since the interface $\Crown{}$ has vanished, both $\tensori{u}{}_{\cell} \vert_{\dCell{}}$ the trace of the displacement of the core part $\cell$ onto $\dCell{}$ and $\tensori{u}{}_{\dCell{}}$ the displacement of the boundary coexist on $\dCell{}$. The displacement of the element $\cell$ is thus said to be \textit{hybrid}, and is denoted by the pair $(\tensori{u}{}_{\cell}, \tensori{u}{}_{\dCell})$.

% ---------------------------------------------------------
% PARAGRAPH
% ---------------------------------------------------------
\paragraph{The special case of DG methods}

Replacing $\tensori{u}{}_{\dCell}$ by $\tensori{u}{}_{\cell'} \vert_{\dCell}$ for any neighboring cell $\cell'$ amounts to describe the framework of Discontinuous Galerkin methods, where only the core unknown $\tensori{u}{}_{\cell}$ is considered, and the displacement jump on $\dCell$ depends on $\tensori{u}{}_{\cell'} \vert_{\dCell}$ the trace of the displacement of neighboring cells instead.

% ---------------------------------------------------------
% PARAGRAPH
% ---------------------------------------------------------
\paragraph{Conformal Galerkin formulation}

By strongly enforcing continuity of the displacement across $\dCell{}$ such that $\tensori{u}_{\cell} \vert_{\dCell} = \tensori{u}_{\dCell}$, one recovers the Principle of Virtual Work \eqref{eq_HW_0}, which defines the framework of conformal methods.

% ---------------------------------------------------------
% PARAGRAPH
% ---------------------------------------------------------
\paragraph{Lagrangian variations}

By differentiation of the total Lagrangian \eqref{eq_0015} with respect to each variable of the problem, the following weak equations arise
%
%
%
\begin{subequations}
    \label{eq_0017}
        \begin{alignat}{3}
            \langle \frac{\partial L_{\cell}^{HW}}{\partial \tensori{u}{}_{\cell}} , \delta \tensori{u}{}_{\cell} \rangle
            = & \int_{\cell} \tensorii{P}{}_{\cell} : \nabla \delta \tensori{u}{}_{\cell}
            -
            \int_{\cell} \tensori{f}{}_V \cdot \delta \tensori{u}{}_{\cell}
            -
            \int_{\dCell{}} \tensori{\theta}{}_{\cell} \cdot \delta \tensori{u}{}_{\cell} \vert_{\dCell}
            &&
            \ \ \ \ \ \ \ \ 
            &&
            \forall \delta \tensori{u}{}_{\cell}
            % \in \virtualDisplacementSpaceCell
        \label{eq_0017:eq0}
        \\
            \langle \frac{\partial L_{\cell}^{HW}}{\partial \tensori{u}{}_{\dCell}} , \delta \tensori{u}{}_{\dCell} \rangle
            = &
            \int_{\neumannCell} (\tensori{\theta}{}_{\cell} - \tensori{t}{}_{\neumannCell}) \cdot \delta \tensori{u}{}_{\dCell}
            &&
            \ \ \ \ \ \ \ \ 
            &&
            \forall \delta \tensori{u}{}_{\dCell}
            % \in \virtualDisplacementSpaceDCell
        \label{eq_0017:eq1}
        \\
            \langle \frac{\partial L_{\cell}^{HW}}{\partial \tensorii{P}{}_{\cell}} , \delta \tensorii{P}{}_{\cell} \rangle
            = & \int_{\cell} (\nabla \tensori{u}{}_{\cell} - \tensorii{G}{}_{\cell} ) : \delta \tensorii{P}{}_{\cell}
            +
            \int_{\dCell} (\tensori{u}{}_{\dCell} - \tensori{u}{}_{\cell} \vert_{\dCell}) \cdot \delta \tensorii{P}{}_{\cell} \vert_{\dCell} \cdot \tensori{n}{}
            &&
            \ \ \ \ \ \ \ \ 
            &&
            \forall \delta \tensorii{P}{}_{\cell}
            % \in \stressSpaceCell
        \label{eq_0017:eq3}
        \\
            \langle \frac{\partial L_{\cell}^{HW}}{\partial \tensorii{G}{}_{\cell}} , \delta \tensorii{G}{}_{\cell} \rangle
            = &
            \int_{\cell} (\frac{\partial \mecPotential_{\bodyLag}}{\partial \tensorii{G}{}_{\cell}} - \tensorii{P}{}_{\cell}) : \delta \tensorii{G}{}_{\cell}
            &&
            \ \ \ \ \ \ \ \ 
            &&
            \forall \delta \tensorii{G}{}_{\cell}
            % \in \gradSpaceCell
        \label{eq_0017:eq2}
    \end{alignat}
\end{subequations}
% 
% 
%
where the \textit{reconstructed traction force} $\tensori{\theta}{}_{\cell} = \tensorii{P}{}_{\cell} \vert_{\dCell} \cdot \tensori{n}{} + (\beta / h_{\cell}) \tensori{J}(\tensori{u}{}_{\cell}, \tensori{u}{}_{\dCell})$ is introduced, with
$\tensori{J}(\tensori{u}{}_{\cell}, \tensori{u}{}_{\dCell}) = \tensori{u}{}_{\dCell} - \tensori{u}{}_{\cell} \vert_{\dCell}$ the jump function on the boundary $\dCell$.
Following discretization, multiple jump function choices are available. The reader can refer to \ref{sec_stabilization} for more details regarding implementation aspects.
In particular, \eqref{eq_0017:eq0} is the expression of the Principle of Virtual Work in $\cell$, where the \textit{reconstructed traction force} $\tensori{\theta}{}_{\cell}$ replaces the usual expression $\tensorii{P}{}_{\cell} \cdot \tensori{n}{}$ in the external contribution. \eqref{eq_0017:eq1} denotes a supplementary equation to the usual continuous problem as described in \eqref{eq_hu_washizu_derivative_0}, to account for the continuity of the flux $\tensori{\theta}{}_{\cell}$ across the cell boundary.
\eqref{eq_0017:eq2} accounts for the constitutive equation in a weak sense, and \eqref{eq_0017:eq3} defines the equation of an enhanced gradient field, that does not reduce to the projection of $\nabla \tensori{u}{}_{\cell}$ as in \eqref{eq_hu_washizu_derivative_0:eq3}, since it is enriched by a boundary component that depends on the displacement jump.
This feature is at the origin of the robustness of non-conformal methods to volumetric locking (see \ref{sec_appendix_gradient} for more details on this note).

% ---------------------------------------------------------
% -- SUBSECTION
% ---------------------------------------------------------
\subsection{Problem in primal form}
\label{sec_hdg_element_equilibrium}

% ---------------------------------------------------------
% PARAGRAPH
% ---------------------------------------------------------
\paragraph{Reconstructed gradient}

Since minimization of \eqref{eq_0017:eq3} defines a linear problem with any displacement pair $(\tensori{v}{}_{\cell}, \tensori{v}{}_{\dCell})$, one can eliminate \eqref{eq_0017:eq3} from the system \eqref{eq_0017}. The resulting equation defines the so-called \textit{reconstructed gradient} $\tensorii{G}{}_{\cell}(\tensori{v}{}_{\cell}, \tensori{v}{}_{\dCell})$ associated with any displacement pair $(\tensori{v}{}_{\cell}, \tensori{v}{}_{\dCell})$, that satisfies
%
%
%
\begin{equation}
    \label{eq_grad}
    \begin{aligned}
        \int_{\cell} \tensorii{G}{}_{\cell} : \tensorii{\tau}{}_{\cell}
        =
        \int_{\cell}  \nabla \tensori{v}{}_{\cell} : \tensorii{\tau}{}_{\cell}
        +
        \int_{\dCell} (\tensori{v}{}_{\dCell} - \tensori{v}{}_{\cell} \vert_{\dCell}) \cdot \tensorii{\tau}{}_{\cell} \vert_{\dCell} \cdot \tensori{n}{}
        &&
        \forall \tensorii{\tau}{}_{\cell}
        % \in \stressSpaceCell
    \end{aligned}
\end{equation}
%
%
%
where $\tensorii{\tau}{}_{\cell}$ denotes an arbitrary kinematically admissible stress field.

% -> expliquer que quand saut tend vers 0, on retrouve le projection normale
%  ordre du gradient -> dire que même ordre que approximation primale, renvoie aux annexes

% ---------------------------------------------------------
% PARAGRAPH
% ---------------------------------------------------------
\paragraph{Stress tensor}

Likewise, \eqref{eq_0017:eq2} is eliminated from \eqref{eq_0017} since it is linear with $\tensorii{G}{}_{\cell}$. Assuming in addition that the space of kinematically admissible stress fields is included in that of kinematically admissible displacement gradient fields, \eqref{eq_0017:eq2} holds in a strong sense such that
%
%
%
\begin{equation}
    \label{eq_stress}
    \begin{aligned}
        \tensorii{P}{}_{\cell} = \frac{\partial \mecPotential_{\bodyLag}}{\partial \tensorii{G}{}_{\cell}}
    \end{aligned}
\end{equation}

% ---------------------------------------------------------
% PARAGRAPH
% ---------------------------------------------------------
\paragraph{Lagrangian variations in primal form}

Using \eqref{eq_stress} and \eqref{eq_grad}, problem \eqref{eq_0017} depends on the displacement unknowns only.
% and the only remaining variations of the total Lagrangian \eqref{eq_0015} are those with respect to both displacement variables.
A new total Lagrangian $L_{\cell}^{HDG}$ arises from the simplified problem such that
%
%
%
\begin{equation}
    \label{eq_total_lagragian_bis}
    \begin{aligned}
        L_{\cell}^{HDG}
        = &
        \int_{\cell{}} \mecPotential_{\bodyLag}
        +
        \int_{\dCell} \frac{\beta}{2 h_{\cell}} \lVert \tensori{J}(\tensori{u}{}_{\cell{}}, \tensori{u}{}_{\dCell{}}) \rVert^2
        -
        \int_{\cell} \loadLag{} \cdot \tensori{u}{}_{\cell{}}
        -
        \int_{\neumannCell{}} \neumannCellLoad{} \cdot \tensori{u}{}_{\dCell{}}
    \end{aligned}
\end{equation}
%
%
%
with respective cell and boundary displacement variations:
\begin{subequations}
    \label{eq_final_problem}
        \begin{alignat}{3}
            \langle \frac{\partial L_{\cell}^{HDG}}{\partial \tensori{u}{}_{\cell}} , \delta \tensori{u}{}_{\cell} \rangle
            = & \int_{\cell} \tensorii{P}{}_{\cell} : \nabla \delta \tensori{u}{}_{\cell}
            -
            \int_{\cell} \tensori{f}{}_V \cdot \delta \tensori{u}{}_{\cell}
            -
            \int_{\dCell{}} \tensori{\theta}{}_{\cell} \cdot \delta \tensori{u}{}_{\cell} \vert_{\dCell}
            &&
            \ \ \ \ \ \ \ \ 
            &&
            \forall \delta \tensori{u}{}_{\cell}
            % \in \virtualDisplacementSpaceCell
        \label{eq_final_problem:eq0}
        \\
            \langle \frac{\partial L_{\cell}^{HDG}}{\partial \tensori{u}{}_{\dCell}} , \delta \tensori{u}{}_{\dCell} \rangle
            = &
            \int_{\neumannCell} (\tensori{\theta}{}_{\cell} - \tensori{t}{}_{\neumannCell}) \cdot \delta \tensori{u}{}_{\dCell}
            &&
            \ \ \ \ \ \ \ \ 
            % \in \virtualDisplacementSpaceDCell
        \label{eq_final_problem:eq1}
    \end{alignat}
\end{subequations}
where $\tensorii{P}{}_{\cell}$ is defined by \eqref{eq_stress} and
depends on $\tensorii{G}{}_{\cell}$ which solves \eqref{eq_grad}.

\subsection{Restriction to the Small strain hypothesis}

The proposed large deformation formulation also allows for a natural transition to the small deformation framework. In this context, the gradient of the transformation $\tensorii{F}{}_{\cell}$ is assumed to be small compared to the identity $\tensorii{I}$, and the infinitesimal deformation field $\tensorii{\varepsilon}_{\cell}$ is sought in place of $\tensorii{F}{}_{\cell}$ such that
% \eqref{eq_grad_def} writes
%
%
%
\begin{equation}
    \tensorii{\varepsilon}{}_{\cell} - \nabla^s \tensori{u}{}_{\cell} = 0
\end{equation}
%
%
%
where the symmetric displacement gradient $\nabla^s \tensori{u}{}_{\cell}$ replaces 
the usual displacement gradient in \eqref{eq_grad_def}.
The stress tensor $\tensorii{P}{}_{\cell}$ is then identified with the Cauchy stress $\tensorii{\sigma}{}_{\cell}$, such that the Lagrangian \eqref{eq_0015} becomes
%
%
%
\begin{equation}
    \label{eq_small_defs}
    \begin{aligned}
        L_{\cell}^{HW}
        = &
        \int_{\cell{}} \mecPotential{}_{\bodyLag{}} + (\nabla^s \tensori{u}{}_{\cell{}} - \tensorii{\varepsilon}{}_{\cell{}}) : \tensorii{\sigma}{}_{\cell}
        % \\
        % &
        + \int_{\dCell{}} (\tensori{u}{}_{\dCell} - \tensori{u}{}_{\cell} \vert_{\dCell}) \cdot \tensorii{\sigma}{}_{\cell} \vert_{\dCell{}} \cdot \tensori{n}{}
        % \\
        % &
        + \int_{\dCell} \frac{\beta}{2 h_{\cell}} \lVert \tensori{u}{}_{\dCell{}} - \tensori{u}{}_{\cell{}} \vert_{\dCell{}} \rVert^2
        \\
        &
        -
        \int_{\cell} \loadLag{} \cdot \tensori{u}{}_{\cell{}}
        -
        \int_{\neumannCell{}} \neumannCellLoad{} \cdot \tensori{u}{}_{\dCell{}}
    \end{aligned}
\end{equation}
%
%
%
where for any pair $(\tensori{v}{}_{\cell}, \tensori{v}{}_{\dCell})$ the \textit{reconstructed strain} $\tensorii{\varepsilon}{}_{\cell}(\tensori{v}{}_{\cell}, \tensori{v}{}_{\dCell})$ is evaluated against any kinematically admissible symmetric stress field $\tensorii{\tau}{}_{\cell}$ such that
%
%
%
\begin{equation}
    \label{eq_grad_ss}
    \begin{aligned}
        \int_{\cell} \tensorii{\varepsilon}{}_{\cell} : \tensorii{\tau}{}_{\cell}
        =
        \int_{\cell}  \nabla^s \tensori{v}{}_{\cell} : \tensorii{\tau}{}_{\cell}
        +
        \int_{\dCell} (\tensori{v}{}_{\dCell} - \tensori{v}{}_{\cell} \vert_{\dCell}) \cdot \tensorii{\tau}{}_{\cell} \vert_{\dCell} \cdot \tensori{n}{}
        &&
        \forall \tensorii{\tau}{}_{\cell}
        % \in \stressSpaceCell
    \end{aligned}
\end{equation}
%
%
%
and the Cauchy stress is the derivative of the potential $\mecPotential_{\bodyLag{}}$ with respect to $\tensorii{\varepsilon}{}_{\cell}$
%
%
%
\begin{equation}
    \label{eq_stress_ss}
    \begin{aligned}
        \tensorii{\sigma}{}_{\cell} = \frac{\partial \mecPotential_{\bodyLag}}{\partial \tensorii{\varepsilon}{}_{\cell}}
    \end{aligned}
\end{equation}

% \subsection{Extension to the axi-symmetric framework}

% In the following section, we devise a Hybrid High order method for an axi-symmetric framework. Owing to geometrical assumptions on the displacement and its gradient, the definition of the reconstructed gradient \eqref{eq_grad} and of that of the higher order displacement \eqref{eq_potential} needs be modified accordingly. Details about the definitions of these ingredients can be found in \ref{sec_appendix_axi}.

% \paragraph{Axi-symmetric framework}

% The cartesian space is expressed in cylindrical coordinates and a point $\tensori{X} \in \bodyLag$ has coordinates $\tensori{X} = (r, z, \theta)$ where $r$ denotes the radial component, $z$ the ordinate one, and $\theta$ is the angular component describing a revolution around the axis $r = 0$. By cylindrical symmetry, the angular displacement $\tensoro{u}{}_{\theta}$ is supposed to be zero, and both components $u_r$ and $u_z$ do not depend on the angular coordinate $\theta$.

% % \paragraph{Cell displacement gradient}

% % % Adopting notations introduced in Section \ref{sec_composite_demo}, let $\cell$ an open subset of $\bodyLag \subset \mathbb{R}^2$ in the $(r,z)$ plane with cell displacement $\tensori{u}{}_{\cell} \in \displacementSpaceCell$ and boundary displacement $\tensori{u}{}_{\dCell} \in \displacementSpaceDCell$.
% % The partial derivatives of $\tensori{u}{}_{\cell}$ with respect to the cylindrical coordinates are given by
% % %
% % %
% % %
% % \begin{equation}
% %     \begin{aligned}
% %         \forall i, j \in \{ r,z \}, \tensoro{u}{}_{\cell i,j} = \frac{\partial u_{\cell i}}{\partial j} && \text{and} && \tensoro{u}{}_{\cell \theta, \theta} = \frac{u_{\cell r}}{r}
% %     \end{aligned}
% % \end{equation}

% \paragraph{Axis faces treatment}

% Since in cylindrical coordinates, all integrals depend on the radial component $r$, boundary integrals vanish at $r = 0$ on the symmetry axis.
% Therefore, the reconstructed gradient (and the stabilization) do not depend on a closed surface wrapping a cell $\cell$ located on the symmetry axis.
% However, this feature is necessary to prove the robustness of the HHO method to volumetric locking (see \ref{sec_appendix_gradient}).
% Therefore, in order to restore full mobility of a face located on the symmetry axis, we consider infinitely thin cylindrical faces wrapping it, that are subjected to Dirichlet boundary conditions along the radial direction.


%%
%
%
% ---------------------------------------------------------
% ---- SECTION
% ---------------------------------------------------------
\section{Discretization}
\label{sec-discretization}

This section specifies the nature of the so-called hybrid mesh,
and introduces the discretization for both cell and faces displacement fields.
The classical \textit{static condensation} cell unknowns elimination strategy is presented, and a novel elimination scheme
based on the previous Lagrangian formulation of hybrid discontinuous methods is then devised.

% ---------------------------------------------------------
% -- SUBSECTION
% ---------------------------------------------------------
\subsection{Mesh and skeleton}

% ---------------------------------------------------------
% PARAGRAPH
% ---------------------------------------------------------
\paragraph{Faces and skeleton of the mesh}

The boundary $\dCell{}$ of each element is decomposed in faces, such
that a face $F$ is a subset of $\bodyLag$, and either there are two
cells $\cell_F$ and $\cell_F'$ such that $F = \dCell_F \cap \dCell_F'$
($F$ is then an interior face), or there is a single cell $\cell_F$ such
that $F = \dCell_F \cap \partial \Omega$ ($F$ is then an exterior face).
Let $\dHybridMesh(\bodyLag) = \{ F_i \subset \bodyLag \ \vert \ 1 \leq i
\leq N_{F} \}$ the skeleton of the mesh, collecting all element faces
$F_i$ in the mesh, where $N_{F}$ denotes the number of faces. The set of
faces subjected to Neumann boundary conditions is denoted
$\dHybridMesh{}_{N}^e(\bodyLag)$, and $\dHybridMesh{}_{D}^e(\bodyLag)$
denotes that subjected to Dirichlet boundary conditions. Moreover, let
$\mathcal{F}^i(\bodyLag)$ the set of interior faces. For any cell
$\cell$, let $\mathcal{F}(\cell) = \{ F \in \dHybridMesh \ \vert \ F
\subset \dCell \}$ the set of faces composing the boundary of $\cell$,
and let $N_{\dCell}$ the number of faces in $\dCell$.

% ---------------------------------------------------------
% PARAGRAPH
% ---------------------------------------------------------
\paragraph{Mesh description}

Likewise, one defines the collection of all cells in the mesh
as $\HybridMesh(\bodyLag) = \{ \matI \subset \bodyLag \ \vert \ 1 \leq i
\leq N_{\cell} \}$, where $N_T$ denotes the total number of cells. The
composition of both $\mathcal{T}(\bodyLag)$ and
$\dHybridMesh{}(\bodyLag)$ forms the hybrid mesh
$\HybridMeshWhole({\bodyLag}) = \{ \mathcal{T}(\bodyLag),
\mathcal{F}(\bodyLag) \}$.

% ---------------------------------------------------------
% -- SUBSECTION
% ---------------------------------------------------------
\subsection{Discretization}

% ---------------------------------------------------------
% PARAGRAPH
% ---------------------------------------------------------
\paragraph{Discrete functional space}

Let $\discreteDisplacementSpaceCell$ denote an
approximation space of finite dimension for the displacement in the
cell, and $V^h(F)$ that on a face $F \in \mathcal{F}(\cell)$. The
approximation space on $\dCell$ is $V^h(\dCell) = \prod_{F \in
  \mathcal{F}(\cell)} V^h(F)$. Similarly, let $\discreteGradSpaceCell$
the approximation space of the reconstructed gradient and
$\discreteStressSpaceCell$ that chosen for the stress.

\paragraph{Approximation bases and elementary unknowns}

Let $\mathfrak{B}_T^h$ denote a polynomial basis of
$U^h(\cell)$, and $\mathfrak{B}_F^h$ a polynomial basis of $V^h(F)$, with respective dimensions $N_T^h$ and $N_F^h$.
The specific choice of monomial bases for
$\mathfrak{B}_T^h$ and $\mathfrak{B}_F^h$ is discussed in depth in
\ref{sec_appendix_implementation}, though other bases can be chosen.
Let $\tensori{u}{}_{T}^h \in U^h(T)$ the polynomial displacement field in $\cell$
and $\tensori{u}{}_{F}^h \in V^h(F)$ that of a face $F \subset \dCell$.
Using vector notations, $\tensori{u}{}_{T}^h$ (respectively $\tensori{u}{}_{F}^h$) can be represented by a vector of coefficients $\mathfrak{U}_T$ (respectively $\mathfrak{U}_F$)  in $\mathfrak{B}_T^h$ (respectively $\mathfrak{B}_F^h$) such that
% The displacement field $\tensori{u}{}_{T}^h \in U^h(T)$ (respectively $\tensori{u}{}_{F}^h \in V^h(F)$) is represented by a vector of
% coefficients $\mathfrak{U}_T$ of size $N_T^h$ in $\mathfrak{B}_T^h$ (respectively
% $\mathfrak{U}_F$ of size $N_F^h$ in $\mathfrak{B}_F^h$) such that
%
%
%
\begin{equation}
  \begin{aligned}
    \tensori{u}{}_{\cell}^h = \mathfrak{U}_T \cdot \mathfrak{B}_T^h
    &&
    \forall \cell \in \mathcal{T}(\bodyLag)
    &&
    \text{and}
    &&
    \tensori{u}{}_{F}^h = \mathfrak{U}_F \cdot \mathfrak{B}_F^h
    &&
    \forall F \in \mathcal{F}(\bodyLag).
  \end{aligned}
\end{equation}

\paragraph{Global Unknowns}

Let $(\tensori{u}{}_{\mathcal{T}}^h, \tensori{u}{}_{\mathcal{F}}^h)
\in U^h(\mathcal{T}) \times U^h(\mathcal{F})$ be the global displacement
unknown of problem \eqref{eq_final_problem} in discrete form, where
$\tensori{u}{}_{\mathcal{T}}^h$ and $\tensori{u}{}_{\mathcal{F}}^h$ are
the piece-wise continuous displacements such that:
%
%
%
\begin{equation}
  \begin{aligned}
    \tensori{u}{}_{\mathcal{T}}^h
    \vert_{\cell} = \tensori{u}{}_{\cell}^h
    % = \mathfrak{U}_T \cdot \mathfrak{B}_T^h
    &&
    \forall \cell \in \mathcal{T}
    &&
    \text{and}
    &&
    \tensori{u}{}_{\mathcal{F}}^h \vert_{F}
    % = \tensori{u}{}_{F}^h = \mathfrak{U}_F \cdot \mathfrak{B}_F^h
    &&
    \forall F \in \mathcal{F},
  \end{aligned}
\end{equation}
% 
% 
% 
with $U^h(\mathcal{T}) = \prod_{T \in \mathcal{T}} U^h(T)$ and
$U^h(\mathcal{F}) = \prod_{F \in \mathcal{F}} U^h(F)$.
In the following, let
$\mathfrak{U}_{\mathcal{T}}$ the unknown coefficient vector associated
to $\tensori{u}{}^h_{\mathcal{T}}$ and $\mathfrak{U}_{\mathcal{F}}$ that to
$\tensori{u}{}^h_{\mathcal{F}}$.

\paragraph{Elementary boundary unknown}

Likewise, let $\tensori{u}{}_{\dCell}^h \in V^h(\dCell)$ such
that
%
%
%
\begin{equation}
  \begin{aligned}
    \tensori{u}{}_{\dCell}^h \vert_F = \tensori{u}{}_{F}^h
    &&
    \forall F \in \mathcal{F}(T),
  \end{aligned}
\end{equation}
% $\tensori{u}{}_{\dCell}^h \vert_F = \tensori{u}{}_{F}^h, \forall F
% \in \mathcal{F}(T)$.
%
%
%
% In the following, let
% $\mathfrak{U}_{\mathcal{T}}$ the unknown coefficient vector associated
% to $\tensori{u}{}^h_{\mathcal{T}}$, $\mathfrak{U}_{\mathcal{F}}$ that to
% $\tensori{u}{}^h_{\mathcal{F}}$, and $\mathfrak{U}_{\mathcal{\dCell}}$
% that to $\tensori{u}{}^h_{\dCell}$.
and let $\mathfrak{U}_{\mathcal{\dCell}}$
the unknown coefficient vector associated to $\tensori{u}{}^h_{\dCell}$.

% ---------------------------------------------------------
% -- SUBSECTION
% ---------------------------------------------------------
\subsection{Local and global discrete problems}

% ---------------------------------------------------------
% PARAGRAPH
% ---------------------------------------------------------
\subsubsection{Local residual}

As in a functional space of finite dimension, the restriction of a linear
form can be represented by a vector in the dual space, let
$\mathfrak{R}_{\cell}$ and $\mathfrak{R}_{\dCell}$ the residual vectors
associated with the the variations of the total
Lagrangian~\eqref{eq_total_lagragian_bis}:
\begin{subequations}
  \label{eq_final_problem_00}
  \begin{alignat}{3}
    \mathfrak{R}_{\cell}(\mathfrak{U}_{\cell},
    \mathfrak{U}_{\dCell}) \cdot \mathfrak{\hat{U}}_{T}
    % \int_{\mathcal{T}} R_{\mathcal{T}}(\tensori{u}{}_{\mathcal{T}}, \tensori{u}{}_{\mathcal{F}}) \cdot \tensori{\hat{u}}{}_{\mathcal{T}}
 = & \int_{\cell} \tensorii{P}{}_{\cell}^h : \nabla
    \tensori{\hat{u}}{}_{\cell}^h - \int_{\cell} \tensori{f}{}_V \cdot
    \tensori{\hat{u}}{}_{\cell} - \int_{\dCell{}}
    \tensori{\theta}{}_{\cell}^h \cdot \tensori{\hat{u}}{}_{\cell}^h
    \vert_{\dCell} && \qquad && \forall
    \hat{\mathfrak{U}}_{\cell}, % \in \virtualDisplacementSpaceCell
    \label{eq_final_problem_00:eq0} \\
    \mathfrak{R}_{\dCell}(\mathfrak{U}_{\cell},
    \mathfrak{U}_{\dCell}) \cdot \mathfrak{\hat{U}}_{\dCell}
    % \int_{\mathcal{F}} R_{\mathcal{F}}(\tensori{u}{}_{\mathcal{T}}, \tensori{u}{}_{\mathcal{F}}) \cdot \tensori{\hat{u}}{}_{\mathcal{F}}
 = & \int_{\dCell} (\tensori{\theta}{}_{\cell}^h -
    \tensori{t}{}_{\neumannCell}) \cdot \tensori{\hat{u}}{}_{\dCell}^h
    && \qquad && \forall \hat{\mathfrak{U}}_{\dCell},
    % \in \virtualDisplacementSpaceDCell \label{eq_final_problem_00:eq1}
    \label{eq_final_problem_00:eq1} \\
  \end{alignat}
\end{subequations}
where the discrete stress tensor $\tensorii{P}{}_{\cell}^h$ and the
discrete reconstructed gradient
$\tensorii{G}{}_{\cell}^h(\tensori{v}{}_{\cell}^h,
\tensori{v}{}_{\dCell}^h)$ are defined by the discrete forms of
equations \eqref{eq_stress} and \eqref{eq_grad} respectively such that:
\begin{equation}
  \label{eq_stress_discrete}
  \begin{aligned}
    \tensorii{P}{}_{\cell}^h =
    \frac{\partial \mecPotential_{\bodyLag}}{\partial
      \tensorii{G}{}_{\cell}^h} && \text{and} && \int_{\cell}
    \tensorii{G}{}_{\cell}^h : \tensorii{\tau}{}_{\cell}^h =
    \int_{\cell} \nabla \tensori{v}{}_{\cell}^h :
    \tensorii{\tau}{}_{\cell}^h + \int_{\dCell}
    (\tensori{v}{}_{\dCell}^h - \tensori{v}{}_{\cell}^h \vert_{\dCell})
    \cdot \tensorii{\tau}{}_{\cell}^h \vert_{\dCell} \cdot \tensori{n}{}
    && \forall \tensorii{\tau}{}_{\cell}^h \in S^h(\cell),
  \end{aligned}
\end{equation}
%
%
%
and the discrete reconstructed traction writes
$\tensori{\theta}{}_{\cell}^h = \tensorii{P}{}_{\cell}^h \cdot \tensori{n} + (\beta / h_{\cell})
\tensori{J}^h(\tensori{v}{}_{\cell}^h, \tensori{v}{}_{\dCell}^h)$. In particular, the
expression of the discrete jump function $\tensori{J}^h(\tensori{v}{}_{\cell}^h, \tensori{v}{}_{\dCell}^h)$
is the key ingredient that defines the HHO method (see Section \ref{sec_appendix_implementation} for more details on this note).
The solution of the discrete problem
% $(\mathfrak{U}_{\cell}, \mathfrak{U}_{\dCell})$
is defined by the fact that the
residuals $\mathfrak{R}_{\cell}$ and $\mathfrak{R}_{\dCell}$ must be
zero
\begin{equation}
  \label{eq_final_problem_000}
  \begin{aligned}
    \mathfrak{R}_{\cell}(\mathfrak{U}_{\cell}, \mathfrak{U}_{\dCell})
    = 0 && \text{and} &&
    \mathfrak{R}_{\dCell}(\mathfrak{U}_{\cell}, \mathfrak{U}_{\dCell})
     = 0,
  \end{aligned}
\end{equation}
%
%
%
In practice, the computation of $\mathfrak{R}_{\cell}$ and
$\mathfrak{R}_{\dCell}$ is discussed in depth in \ref{sec_appendix_implementation}.

% ---------------------------------------------------------
% PARAGRAPH
% ---------------------------------------------------------
\subsubsection{Global residuals and face assembly}

\paragraph{Global problem}

At the global scale, the solution
$(\mathfrak{U}_{\mathcal{T}}, \mathfrak{U}_{\mathcal{F}})$ of the
discrete problem satisfies:
\begin{equation}
  \label{eq_final_global_problem_0}
  \begin{aligned}
    \forall \cell \in \mathcal{T}(\bodyLag), \forall
    \hat{\mathfrak{U}}_{\cell},
    \mathfrak{R}_{\cell}(\mathfrak{U}_{\cell}, \mathfrak{U}_{\dCell})
    \cdot \mathfrak{\hat{U}}_{T} & = 0 && \text{and} && \forall
    \hat{\mathfrak{U}}_{\mathcal{F}},
    \mathfrak{R}_{\mathcal{F}}(\mathfrak{U}_{\mathcal{T}},
    \mathfrak{U}_{\mathcal{F}}) \cdot \mathfrak{\hat{U}}_{\mathcal{F}} =
    0,
  \end{aligned}
\end{equation}
where the vector
$\mathfrak{R}_{\mathcal{F}}(\mathfrak{U}_{\mathcal{T}},
\mathfrak{U}_{\mathcal{F}})$ is the skeleton residual such that
\begin{equation}
  \label{eq_final_problem_0}
  \begin{aligned}
    \mathfrak{R}_{\mathcal{F}}(\mathfrak{U}_{\mathcal{T}},
    \mathfrak{U}_{\mathcal{F}}) \cdot \mathfrak{\hat{U}}_{\mathcal{F}}
    % \int_{\mathcal{F}} R_{\mathcal{F}}(\tensori{u}{}_{\mathcal{T}}, \tensori{u}{}_{\mathcal{F}}) \cdot \tensori{\hat{u}}{}_{\mathcal{F}}
 = & \sum_{F \in \mathcal{F}^i(\bodyLag{})} \int_{F}
    (\tensori{\theta}{}_{\cell_F}^h + \tensori{\theta}{}_{\cell_F '}^h)
    \cdot \tensori{\hat{u}}{}_{F}^h + \sum_{F \in
      \mathcal{F}^e_N(\bodyLag{})} \int_{F}
    (\tensori{\theta}{}_{\cell_F}^h - \neumannLag) \cdot
    \tensori{\hat{u}}{}_{F}^h && \forall
    \hat{\mathfrak{U}}_{\mathcal{F}},
  \end{aligned}
\end{equation}
%
%
%
and which results in the assembly of faces unknowns only.

\paragraph{Assembly over the skeleton}

An interior face $F$ is linked to two adjacent cells $T$ and $T'$, and
each of these cells applies to the other a surface load $\pm
\tensori{t}{}_{T\cap T'}$ through $F$, which is identified with
$\tensori{t}{}_{\neumannCell}$ on $F$ in \eqref{eq_final_problem_00:eq1}.
By summation over each face of the structure, these equal contributions
cancel out, which yields the expression of the first argument in the
right-hand side of \eqref{eq_final_problem_0}.
%
%
%
Since exterior faces subjected to Neumann boundary conditions are
linked to a single cell only, the local surface forces $\tensori{t}{}_{\neumannCell}$ are equal to $\neumannLag$ on $\neumannBoundaryLag{}$,
which yields the expression of
the second argument in the right-hand side of
\eqref{eq_final_problem_00:eq1}.

% ---------------------------------------------------------
% -- SUBSECTION
% ---------------------------------------------------------
\subsection{Cell unknowns elimination}
\label{sec_cell_unknowns_elimination}

As the cell number of unknowns grows rapidly with the polynomial order
as compared to that in the face (See \ref{sec_shape_functions} for further details), cell
unknowns must be eliminated for the for the method to be numerically
attractive.

In this section, two elimination strategies are examinated. The first
one, presented as the \textit{Cell equilibrium} strategy, has, to our
knowledge, never been introduced in the literature, and arises from the
previous total Lagrangian formulation of HDG methods. The second one
consists in performing a static condensation operation
\cite{abbas_hybrid_2018, abbas_hybrid_2019,di_pietro_hybrid_2015}
following linearization of the problem.

% ---------------------------------------------------------
% PARAGRAPH
% ---------------------------------------------------------
\subsubsection{Cell equilibrium}
\label{sec_cell_equilibrium_scheme}

This first algorithm considers that the cell displacement unknown
$\mathfrak{U}_{T}$ are defined as implicit functions of the boundary
displacement unknown $\mathfrak{U}_{\dCell}$ as follows:
\begin{equation}
  \label{eq_cell_equilibrium_1}
  \begin{aligned}
    \mathfrak{R}_{T}(\mathfrak{U}_{T}(\mathfrak{U}_{\dCell}),
    \mathfrak{U}_{\dCell}) = 0
  \end{aligned}
\end{equation}
In practice, Nonlinear Equation~\eqref{eq_cell_equilibrium_1} can be
solved by an iterative method.
%
%
%
At the global scale, the face residual can thus be expressed as
function of the face displacement only, and satisfies:
\begin{equation}
  \label{eq_cell_equilibrium_face_residual}
  \mathfrak{R}_{\mathcal{F}}(\mathfrak{U}_{\mathcal{T}}\paren{\mathfrak{U}_{\mathcal{F}}},
 \mathfrak{U}_{\mathcal{F}})=0
\end{equation}
%
%
%
Likewise, equation~\eqref{eq_cell_equilibrium_face_residual} is also
generally solved using an iterative method, where the face displacement is
the only unknown. Two algorithm are now presented:

\begin{itemize}
  \item The standard Newton algoritmh.
  \item A fixed point algorithm combined with an acceleration scheme.
\end{itemize}

\paragraph{Resolution of Nonlinear Equation~\eqref{eq_cell_equilibrium_face_residual} using the Newton algorithm}

Let \(\iter{n}{\mathfrak{U}_{\mathcal{F}}}\) be the current estimate
of the solution. The correction
\(\iter{n}{\delta}\mathfrak{U}_{\mathcal{F}}\) of this estimate is given
by:
\[
\iter{n}{\delta}\mathfrak{U}_{\mathcal{F}}=
-\left( \frac{d\mathfrak{R}_{\mathcal{F}}}{d \mathfrak{U}_{\mathcal{F}}}
\right)^{-1} \,\cdot\,\iter{n}{\mathfrak{R}}_{\mathcal{F}},
\]
with
\(\iter{n}{\delta}\mathfrak{U}_{\mathcal{F}}=\iter{n+1}{\mathfrak{U}_{\mathcal{F}}}-\iter{n}{\mathfrak{U}_{\mathcal{F}}}\).
%
%
%
The jacobian matrix \(\frac{d\mathfrak{R}_{\mathcal{F}}}{d
  \mathfrak{U}_{\mathcal{F}}}\) can be determined using the chain rule,
as follows:
\begin{equation}
  \label{eq_cell_equilibrium_0}
  \begin{aligned}
    \frac{d\mathfrak{R}_{\mathcal{F}}}{d
      \mathfrak{U}_{\mathcal{F}}}
    = \frac{\partial
      \mathfrak{R}_{\mathcal{F}}}{\partial \mathfrak{U}_{\mathcal{T}}}
    \frac{\partial \mathfrak{U}_{\mathcal{T}}}{\partial
      \mathfrak{U}_{\mathcal{F}}} \cdot \delta
    \mathfrak{U}_{\mathcal{F}} + \frac{\partial
      \mathfrak{R}_{\mathcal{F}}}{\partial \mathfrak{U}_{\mathcal{F}}}
    \cdot \delta \mathfrak{U}_{\mathcal{F}},
  \end{aligned}
\end{equation}
Applying the implicit function theorem to
\eqref{eq_cell_equilibrium_1} yields:
\begin{equation}
  \label{eq_cell_equilibrium_2}
  \begin{aligned}
    \frac{\partial
      \mathfrak{U}_{\mathcal{T}}}{\partial \mathfrak{U}_{\mathcal{F}}} =
    - \frac{\partial \mathfrak{U}_{\mathcal{T}}}{\partial
      \mathfrak{R}_{\mathcal{T}}} \frac{\partial
      \mathfrak{R}_{\mathcal{T}}}{\partial \mathfrak{U}_{\mathcal{F}}}.
  \end{aligned}
\end{equation}
Injecting \eqref{eq_cell_equilibrium_2} in
\eqref{eq_cell_equilibrium_0} yields the expression of the derivative of
the face resiudal with respect to the face displacement
\begin{equation}
  \label{eq_cell_equilibrium_3}
  \begin{aligned}
    \frac{d\mathfrak{R}_{\mathcal{F}}}{d
      \mathfrak{U}_{\mathcal{F}}} = \frac{\partial
      \mathfrak{R}_{\mathcal{T}}}{\partial \mathfrak{U}_{\mathcal{F}}}
    \cdot \delta \mathfrak{U}_{\mathcal{F}} - \frac{\partial
      \mathfrak{R}_{\mathcal{T}}}{\partial \mathfrak{U}_{\mathcal{T}}}
    \frac{\partial \mathfrak{U}_{\mathcal{T}}}{\partial
      \mathfrak{R}_{\mathcal{T}}} \frac{\partial
      \mathfrak{R}_{\mathcal{T}}}{\partial \mathfrak{U}_{\mathcal{F}}}
    \cdot \delta \mathfrak{U}_{\mathcal{F}}.
  \end{aligned}
\end{equation}

\paragraph{Resolution of Nonlinear
  Equation~\eqref{eq_cell_equilibrium_face_residual} using a accelerated
  fixed point algorithm} Nonlinear
Equation~\eqref{eq_cell_equilibrium_face_residual} can be rewritten as
follows:
\[
\mathfrak{U}_{\mathcal{F}}=
\mathfrak{U}_{\mathcal{F}}-\hat{\mathbb{K}}^{-1}\,\cdot\,\mathfrak{R}_{\mathcal{F}}\paren{\mathfrak{U}_{\mathcal{T}}\paren{\mathfrak{U}_{\mathcal{F}}},
 \mathfrak{U}_{\mathcal{F}}},
\]
where \(\hat{\mathbb{K}}\) is a preconditionning matrix. This equation leads to the following fixed iteration scheme:
\[
\iter{n+1}{\mathfrak{U}_{\mathcal{F}}}=\iter{n}{\mathfrak{U}_{\mathcal{F}}}
- \hat{\mathbb{K}}^{-1}\,\cdot\,\mathfrak{R}_{\mathcal{F}}\paren{\mathfrak{U}_{\mathcal{T}}\paren{\iter{n}{\mathfrak{U}_{\mathcal{F}}}},
  \iter{n}{\mathfrak{U}_{\mathcal{F}}}}.
\]
%
%
%
It is clear that the Newton algorithm is recovered if
\(\hat{\mathbb{K}}\) is egal to
\(\derivtot{\mathfrak{R}_{\mathcal{F}}}{\mathfrak{U}_{\mathcal{F}}}\).
The idea of this algorithm is however to use a constant preconditionning
matrix.

In the authors' experience, the initial elastic stiffness matrix is an
appropriate choice for \(\hat{\mathbb{K}}\) in most cases. This strategy
has been used by default in the {\tt Cast3M} finite element solver for
years. One advantage of this method is that if a direct solver is used
to compute \(\hat{\mathbb{K}}^{-1}\,\cdot\,\mathfrak{R}_{\mathcal{F}}\),
the matrix \(\hat{\mathbb{K}}^{-1}\) only have to be decomposed once.
%
%
%
As expected, this fixed point algorithm leads to a linear convergence
(compared to the quadratic convergence of the Newton algorithm).
However, combined with acceleration schemes, this algorithm can be
competive with the Newton algorithm in many cases. The reader can refer to
\cite{ramiere_iterative_2015} for a discussion on this subject.
Numerical examples using such acceleration schemes are
evaluated in Section \ref{sec_numerical_examples}.
%
%
%
It is worth noticing that if efficient preconditionning operators
where available, this accelerated fixed-point scheme could be the basis
of a truly matrix-free algorithm. Using an incomplete decomposition of
the elastic stiffness matrix may be a first step in this direction.
Exploring this line of research is left for future works.

\subsubsection{Static condensation}
\label{sec_static_condesnation_scheme}

In this section, $\mathfrak{U}_{\mathcal{T}}$ and
$\mathfrak{U}_{\mathcal{F}}$ are assumed to be independent variables.
Again, a Newton method is considered.
%
%
%
At the cell level, the correction of the cell unknown
$\iter{n}{\delta}\mathfrak{U}_{\cell}$ and face unknowns
$\iter{n}{\delta}\mathfrak{U}_{\dCell}$ are given by:
\begin{equation}
\label{eq_static_condensation_final}
\mathbb{K}\,\cdot\,
\begin{pmatrix}
  \iter{n}{\delta}\mathfrak{U}_{\cell}
  \\
  \iter{n}{\delta}\mathfrak{U}_{\dCell}
\end{pmatrix}
= -
\begin{pmatrix}
  \iter{n}{\mathfrak{R}}_{\mathcal{\cell}}
  \\
  \iter{n}{\mathfrak{R}}_{\mathcal{\dCell}}
\end{pmatrix}
\quad\text{with}\quad \mathbb{K} =
\begin{pmatrix}
  \deriv{\mathfrak{R}_{\cell}}{\mathfrak{U}_{\cell}}
  & \deriv{\mathfrak{R}_{\cell}}{\mathfrak{U}_{\dCell}} \\
  \deriv{\mathfrak{R}_{\dCell}}{\mathfrak{U}_{\cell}}
  & \deriv{\mathfrak{R}_{\dCell}}{\mathfrak{U}_{\dCell}} \\
\end{pmatrix}
=
\begin{pmatrix}
  \mathbb{K}_{\cell\cell} &
  \mathbb{K}_{\cell\dCell} \\
  \mathbb{K}_{\dCell\cell} &
  \mathbb{K}_{\dCell\dCell} \\
\end{pmatrix},
\end{equation}
%
%
%
where the blocks \(\mathbb{K}_{\cell\cell}\),
\(\mathbb{K}_{\cell\dCell}\), \(\mathbb{K}_{\dCell\cell}\) and
\(\mathbb{K}_{\dCell\dCell}\) have been introduced for convenience.
%
%
%
The correction of the cell unknown
$\iter{n}{\delta}\mathfrak{U}_{\cell}$ can thus be expressed as:
\begin{equation}
  \label{eq:cell_unknown_correction}
  \iter{n}{\delta}\mathfrak{U}_{\cell}=
  -\mathbb{K}_{\cell\cell}^{-1}\,\cdot\,\,\iter{n}{\mathfrak{R}}_{\mathcal{\cell}}
  -\mathbb{K}_{\cell\cell}^{-1}\,\cdot\,\mathbb{K}_{\cell\dCell}\,\cdot\,\iter{n}{\delta}\mathfrak{U}_{\dCell},
\end{equation}
%
%
%
such that the correction of the face unknown
$\iter{n}{\delta}\mathfrak{U}_{\dCell}$ satisfies:
\[
\paren{ \mathbb{K}_{\dCell\dCell}\,
  -\mathbb{K}_{\dCell\cell}\,\cdot\,\mathbb{K}_{\cell\cell}^{-1}\,\cdot\,\mathbb{K}_{\cell\dCell}
 }\,\cdot\, \iter{n}{\delta}\mathfrak{U}_{\dCell}=
-\iter{n}{\mathfrak{R}}_{\mathcal{\dCell}}
+\mathbb{K}_{\dCell\cell}\,\cdot\,\mathbb{K}_{\cell\cell}^{-1}\,\cdot\,\,\iter{n}{\mathfrak{R}}_{\mathcal{\cell}},
\]
%
%
%
or, introducing the condensed quantities
%
%
%
\(\iter{n}{\left.\mathfrak{R}^{c}_{\mathcal{\dCell}}\right.}\) and
\(\mathbb{K}_{\dCell\cell}^{c}\):
\[
\mathbb{K}_{\dCell\dCell}^{c}\,\cdot\,\iter{n}{\delta}\mathfrak{U}_{\dCell}=
-\iter{n}{\left.\mathfrak{R}^{c}_{\mathcal{\dCell}}\right.},
\]
%
%
%
where
%
%
%
\begin{equation}
  \begin{aligned}
    \mathbb{K}_{\dCell\dCell}^{c}
    =
    \mathbb{K}_{\dCell\dCell}
    -
    \mathbb{K}_{\dCell\cell} \cdot \mathbb{K}_{\cell\cell}^{-1} \cdot \mathbb{K}_{\cell\dCell}
    &&
    \text{and}
    &&
    \mathfrak{R}^{(n), c}_{\mathcal{\dCell}}
    =
    -\iter{n}{\mathfrak{R}}_{\mathcal{\dCell}}
    +\mathbb{K}_{\dCell\cell}\,\cdot\,\mathbb{K}_{\cell\cell}^{-1}\,\cdot\,\,\iter{n}{\mathfrak{R}}_{\mathcal{\cell}},
  \end{aligned}
  % \iter{n}{\left.\mathfrak{R}^{c}_{\mathcal{\dCell}}\right.}
\end{equation}
%
%
%
and the element contributions
\(\iter{n}{\left.\mathfrak{R}^{c}_{\mathcal{\dCell}}\right.}\) and
\(\mathbb{K}_{\dCell\cell}^{c}\) are then assembled to form the linear
system giving \(\iter{n}{\delta}\mathfrak{U}_{\mathcal{F}}\). Once
\(\iter{n}{\delta}\mathfrak{U}_{\dCell}\) is known, a decondensation
step is performed to compute \(\iter{n}{\delta}\mathfrak{U}_{\cell}\)
using \eqref{eq:cell_unknown_correction}.


\subsection{Extension to non linear materials with internal state variables}
\label{sec:discretization:extension_to_non_linear_materials}

This section is devoted to the extension of the method to non linear
materials with local internal state variables. Let \(\mathfrak{I}\) be the
set of internal state variables describing the material. Each cell is
assumed to describe a unique material.
%
%
%
To simplify the presentation and preseve a variational framework, the
behaviour of the material is assumed to be standard
generalized~\cite{moreau_sur_1970,halphen_sur_1975}. The evolution of
the material can thus be described by an incremental lagrangian
\(L_{\cell}^{HDG}\) defined as
follows~\cite{lorentz_variational_1999,forest_localization_2004}:
% \begin{equation}
%   L_{\cell}^{HDG} = \displaystyle \int_{\cell} \left[
%   \mecPotential{}_{\bodyLag{}}+\Delta\,t\,\dissipationPotential\paren{\Frac{\vec{Y}^{\star}-\bts{\vec{Y}}}{\Delta\,t}}
%  \right] + \int_{\dCell} \frac{\beta}{2 h_{\cell}} \lVert
%   \tensori{J}(\tensori{u}{}_{\cell{}}, \tensori{u}{}_{\dCell{}})
%   \rVert^2 - \int_{\cell} \loadLag{} \cdot \tensori{u}{}_{\cell{}} -
%   \int_{\neumannCell{}} \neumannCellLoad{} \cdot
%   \tensori{u}{}_{\dCell{}}
% \end{equation}
%
%
%
\begin{equation}
  L_{\cell}^{HDG}
  =
  \int_{\cell} \mecPotential{}_{\bodyLag{}}
  +
  \int_{\cell} \Delta t \, \dissipationPotential \paren{\frac{\hat{\mathfrak{I}} - \mathfrak{I} \vert_t}{\Delta t}}
  +
  \int_{\dCell} \frac{\beta}{2 h_{\cell}} \lVert \tensori{J}(\tensori{u}{}_{\cell{}}, \tensori{u}{}_{\dCell{}}) \rVert^2
  -
  \int_{\cell} \loadLag{} \cdot \tensori{u}{}_{\cell{}}
  -
  \int_{\neumannCell{}} \neumannCellLoad{} \cdot \tensori{u}{}_{\dCell{}},
\end{equation}
%
%
%
% where:
% \begin{itemize}
%   \item \(\dissipationPotential\) denotes the dissipation
%   potential.
%   \item \(\Delta\,t\) denotes the time increment.
%   \item \(\bts{\vec{Y}}\) denotes the value of the internal state
%   variables at the beginning of the time step.
% \end{itemize}
%
%
%
where \(\dissipationPotential\) denotes the dissipation
potential, \(\Delta t\) the time increment and \(\mathfrak{I} \vert_t\) the value of the internal state
variables at the beginning of the time step.

At this stage, two strategies can be set-up to eliminate the internal
state variables:
\begin{enumerate}
  \item Classically, state variables are assumed to be defined at
  the integration points (or, expressed differently, to be approximated
  in \(L^{2}\)). This strategy is the one used by most finite element
  solvers. In pratice, given the increment of the reconstructed
  gradient, the constitutive equations, expressed as a system of
  ordinary differential equations, are integrated to compute the new
  value of the stress and the consistent tangent operator. This
  strategy, already used by Abbas et
  al.~\cite{abbas_hybrid_2018,abbas_hybrid_2019}, is used in the
  numerical examples of this paper.
  \item The state variables can also be approximated in some
  discrete space on the cell. In this case, the cell resolution
  algorithm could be extended to define the cell displacements and the
  state variables as implicit functions of the face displacements. This
  approach seems \emph{a priori} interesting in at least two cases:
  \begin{itemize}
    \item Applied to plasticity, this approach lead to a en
    potentially efficient multi-field
    plasticity~\cite{simo_computational_1998} method with a low
    computational cost as the extra degrees of freedoms (associated with
    the plastic strains) can be eliminated at the cell level.
    % Static condensation in the context of multi-field plasticty \cite{schroder_static_2015}.
    \item Applied to phase-field damage problems, the
    irreversibility constraint could be treated at the element level by
    defining appropriate Lagrange multiplier that can be eliminated.
    % A similar idea was developped by Cicuttin et al. for the elliptic
    % obstacle problem~\cite{cicuttin_hybrid_2020}.
  \end{itemize}
  Exploring those two lines of research is left for future
  works.
\end{enumerate}

%Moreover, it allows to consider extending the present cell correction
%iterative resolution to \textit{e.g.} constrained resolution algorithm,
%in order to solve inequality constrained problems, as encountered in
%multi-field plasticity~\cite{schroder_small_2015} for instance.

\subsection{Comparison between both schemes}
\label{par_cell_eq}

\paragraph{Static condensation}

The static condensation algorithm is the one used in the literature
\cite{di_pietro_discontinuous-skeletal_2015,cockburn_algorithm_2019,abbas_hybrid_2019-1,abbas_hybrid_2018}
to eliminate cell unknowns. Contrary to the introduced cell resolution
algorithm, this scheme needs not iterate at the cell level to accomodate
the cell correction. The actualization of the cell unknown displacement
by its correction requires that the quantities $\partial
\mathfrak{U}{}_{\cell} / \partial \mathfrak{R}_{\cell}$ and $\partial
\mathfrak{R}_{\cell} / \partial \mathfrak{U}{}_{\dCell}^k$ computed at the
previous iteration are known. From a numerical standpoint, this results
in keeping stiffness matrices blocks (see
Section~\ref{sec_appendix_implementation2}) from an iteration to the
other.

\paragraph{Cell equilibrium}

The novel cell resolution scheme needs iterate at the cell level. It
may require to integrate the constitutive equation more times than the
static condensation algorithm does (See
paragraph~\ref{sec:discretization:extension_to_non_linear_materials}).
However, it allows to exactly evaluate the equilibrium of the cell with
its boundary, what does not the former.

% \begin{figure}[H]
%     \centering
%     \includegraphics[width=15.cm]{chapter_002_hho_mechanics/figures/resolution.png}
%     \caption{Schematic representation of both resolution schemes}
%     \label{fig_resolution}
% \end{figure}


% NOTE : COMMENT ON TRACE LE DEPLACREMENT -> MOYENNE AU NOEUDS A PRECISER DANS LA LEGENDE DE LA FIGURE
% NOTE : LE CODE, CE QU'ON UTILISE, MFRONT MGIS -> IMPLEMENTATION PORTEE DANS CAST3M -> IMPLEMENTATION PLUGGABLE DANS AUTRE CHOSE -> 
% NOTE : ON PARLE PAS DES TEMPS DE CALCULS DANS LA SUITE parce que c du python, on montr que les 2 schémas d'intégration convergecne de manière qudartique, localement et globalement
% POUR ETRE COHERENT EN CONDENSATION, IL FAUT ACCLERER LES FACES +  LES CELLULES. POUR NOUS C QUE LES FACES


% ---------------------------------------------------------
% ---- SECTION
% ---------------------------------------------------------
\section{Numerical examples in the axisymmetric modelling hypothesis}
\label{sec_numerical_examples}

In this section, the proposed axi-symmetric HHO method is evaluated on
classical test cases taken from the literature to emphasize robustness
to volumetric locking.
The reader can refer to \ref{sec_appendix_axi} regarding technical aspects on the formulation of the proposed HHO method for the axisymmetric modelling hypothesis. 
Both the small and large strains
frameworks are considered, for elasto-plastic behaviors. The notation
HHO($k,l$) relates to the HHO element of order $k$ on faces, and order $l$ in the
cell.

The tests presented in this section have been performed using an
\texttt{python} implementation freely available on github: \url{https://github.com/davidsiedel/h2o_paper}.
The results of this implementations were thoroughly compared to the
results obtained with the reference implementation provided by the
\texttt{Disk++} solver \cite{cicuttin_implementation_2018} in the cartesian framework.

% ---------------------------------------------------------
% PARAGRAPH
% ---------------------------------------------------------
\paragraph{Stabilization parameter}

To ensure coercivity of the HHO method, the so-called \textit{stabilization} parameter
$\beta$, that relates to the Young modulus of the elastic interface introduced in Section \ref{sec_hdg_element_equilibrium}, needs be chosen according to the material under study. In \cite{di_pietro_hybrid_2015}, a value of
order $2 \mu$ is advocated, where $\mu$ denotes the shear modulus of the
material. This value is used for all test cases in the present section.

% ---------------------------------------------------------
% -- SUBSECTION
% ---------------------------------------------------------
\subsection{The free dilatation test}
\label{sec_satoh_test}

The first test case of the following benchmark aims at displaying the
robustness of the HHO method for coupled mechanical-thermal problems.

% ---------------------------------------------------------
% PARAGRAPH
% ---------------------------------------------------------
\paragraph{Specimen and loading}

For this test case, the unit box is fixed on both the right and
bottom boundaries in their respective normal directions, and a quadratic
thermal load depending on the $r$-coordinate is imposed in the solid
(see Figure \ref{fig_satoh_setting}). The mesh is composed of 400
quadrangles. The thermal loading is given by:
%
%
%
\begin{equation}
    T(r,z) = 4 (T_{max} - T_{min}) \, r \, (1 - r) + T_{min},
\end{equation}
%
%
%
with temperature values $T_{max} = 2000$ K and $T_{min} = 293.15$ K.

\begin{figure}[H]
    \centering
    \includegraphics[width=10.cm]{../chapter_002_hho_mechanics/drawings/satoh_mesh.png}
    \caption{Geometry, displacement boundary conditions and temperature loading for the free dilatation test case}
    \label{fig_satoh_setting}
\end{figure}

% ---------------------------------------------------------
% PARAGRAPH
% ---------------------------------------------------------
\paragraph{Material behavior}

A linear thermo-elastic energy potential is considered with a Young's
modulus $E$ equal to $150$ GPa. The material is quasi-incompressible
with a Poisson's ratio $\nu$ equal to $0.499$. The dilatation parameter
is taken as $\alpha = 1e^{-6}$ K$^{-1}$.

% ---------------------------------------------------------
% PARAGRAPH
% ---------------------------------------------------------
\paragraph{Volumetric locking and polynomial approximation for the strain and temperature fields}

A sign of volumetric locking is the presence of strong oscillations in the trace of the Cauchy stress (or, equivalently, the hydrostatic pressure) within elements.
Using Lagrange finite elements of order respectively $1, 2$, the
strain approximation is of order $0, 1$, whereas the thermal
strain adopted in this example is of order $2$, which results in strong signs of volumetric locking. For Lagrange
elements to accurately model the hydrostatic pressure, one needs to choose a polynomial approximation of the temperature field that is one order lower than that of the
displacement field. This feature is of major importance for mixed
elements, where quadratic displacement elements needs be employed together with linear ones for the pressure/volume variation field.

\begin{figure}[H]
    \centering
    \includegraphics[width=10.cm]{../chapter_002_hho_mechanics/drawings/satoh_res.png}
    \caption{Map fo the Trace of the Cauchy stress at quadrature points for the Free dilatation test case at the last time step}
    \label{fig_satoh_calc}
\end{figure}

% ---------------------------------------------------------
% PARAGRAPH
% ---------------------------------------------------------
\paragraph{Comparison of FE and HHO methods}

In Figure \ref{fig_satoh_calc}, one can observe that the pressure map is completely smooth for HHO computations, even for a quadratic temperature field acting on a linear gradient using HHO(1,1). As expected, the results display mild signs of volumetric locking for the quadratic finite element
approximation, and strong oscillations are noted for the linear finite element solution.

% ---------------------------------------------------------
% -- SUBSECTION
% ---------------------------------------------------------
\subsection{Perfect plastic radially loaded sphere}
\label{sec_swelling_sphere}

% ---------------------------------------------------------
% PARAGRAPH
% ---------------------------------------------------------
\paragraph{Specimen and loading}

This benchmark consists in a quasi-incompressible sphere under uniform internal loading (see Figure \ref{fig_sphere_mesh}).
This test case has an analytical solution and the state of the specimen is known when the plastic region has reached the external border of the sphere.
The sphere has an inner radius $r_{int} = 0.8$ mm and an outer
radius $r_{ext} = 1$ mm. An internal radial displacement $u$ is imposed.
The simulation is performed until the limit load corresponding to an internal displacement of $0.2$ mm is reached.

\begin{figure}[H]
    \centering
    \includegraphics[width=12.cm]{../chapter_002_hho_mechanics/drawings/sphere_mesh.png}
    \caption{Geometry and loadings for the radially loaded sphere test case}
    \label{fig_sphere_mesh}
\end{figure}

% ---------------------------------------------------------
% PARAGRAPH
% ---------------------------------------------------------
\paragraph{Material behavior}

An isotropic hardening energy potential $\mecPotential{}_{\bodyLag{}}^p$ is chosen for the description of the plastic evolution of the material such that

\begin{equation}
    \mecPotential{}_{\bodyLag{}}^p(p)
    =
    \sigma_0 p + \frac{1}{2} H p^2 + (\sigma_{\infty} - \sigma_0)(p - \frac{1 - e^{-\delta p}}{\delta}),
\end{equation}
%
%
%
where the parameter $p$ denotes the equivalent plastic strain and a Von Mises yields function $f$ describes the flow rule
%
%
%
\begin{equation}
    f = \sqrt{\frac{3}{2}} \rVert \text{dev} (\tensorii{\sigma}) \lVert - \frac{\partial \mecPotential{}_{\bodyLag{}}^p(p)}{\partial p}.
\end{equation}
%
%
%
Moreover, the small strain hypothesis is assumed, and
perfect plasticity is considered for this test case. The saturation parameter $\delta$ and the hardening parameter $H$ are taken to $0$, the yield stresses are such that $\sigma_0 = \sigma_{\infty} = 6$ MPa, and the elastic potential parameters are the Young modulus $E = 28.85$ MPa and the Poisson ratio $\nu = 0.499$, such that the material is quasi-incompressible.

\begin{figure}[H]
    \centering
    % \includegraphics[width=12.cm]{../chapter_002_hho_mechanics/figures/sphere_mesh.png}
    \includegraphics[width=12.cm]{../chapter_002_hho_mechanics/drawings/sphere_res.png}
    \caption{Final displacement along the radius of the sphere, and final equivalent plastic strain map at quadrature points for the radially loaded sphere test case}
    \label{fig_sphere_res}
\end{figure}

% ---------------------------------------------------------
% PARAGRAPH
% ---------------------------------------------------------
\paragraph{Displacement along the radius}

Since an analytical solution is known for this test case, it is compared to the proposed HHO method numerical results. The displacement of the section of the sphere at cell nodes
is plotted in Figure \ref{fig_sphere_res}, along with the analytical one, and it is noticed that the obtained results are in agreement with the analytical ones.
Figure \ref{fig_sphere_res} mentions the label HHO without specifying approximation orders for all computations deliver the same result.

% ---------------------------------------------------------
% PARAGRAPH
% ---------------------------------------------------------
\paragraph{Trace of the Cauchy stress}

As for the displacement, the analytical solution for the trace of the Cauchy stress tensor is compared to the one computed using the proposed HHO method for three approximation orders.
Numerical results at quadrature points fit the analytical curve, and display no sign of volumetric locking. The computed solution is however less smooth
at the borders of the specimen for higher orders, a phenomenon that was pointed out in \cite{abbas_hybrid_2019-1} for the three dimensional case, and attributed to the fact that planar faces are considered.

\begin{figure}[H]
    \centering
    \includegraphics[width=15.cm]{../chapter_002_hho_mechanics/drawings/sphere_pressures.png}
    \caption{trace of the Cauchy stress tensor along the radius of the sphere at quadrature points}
    \label{fig_sphere_pressure}
\end{figure}

% ---------------------------------------------------------
% -- SUBSECTION
% ---------------------------------------------------------
\subsection{Necking of a notched bar}
\label{sec:hho_meca:notched_bar}

% ---------------------------------------------------------
% PARAGRAPH
% ---------------------------------------------------------
\paragraph{Specimen and loading}

The last application consists of a notched bar that is subjected to uniaxial
extension.
The bar has a length of $30$ mm, a top section of radius $5$ mm and a bottom section of radius $3$ mm.
A vertical
displacement $u_z = 0.8$ mm is imposed at the top, as shown in Figure \ref{fig_ssna_mesh}.
For symmetry reasons, only one-quarter of the
bar is discretized.

\begin{figure}[H]
    \centering
    % \includegraphics[width=12.cm]{../chapter_002_hho_mechanics/figures/ssna_mesh.png}
    \includegraphics[width=12.cm]{../chapter_002_hho_mechanics/drawings/ssna_mesh.png}
    \caption{
        Geometry, loadings and load deflection curve for the notched specimen test case
    }
    \label{fig_ssna_mesh}
\end{figure}

% ---------------------------------------------------------
% PARAGRAPH
% ---------------------------------------------------------
\paragraph{Material behavior}

The same constitutive equation as that in \ref{sec_swelling_sphere} is considered for the present test case. 
However, the finite strain hypothesis is chosen, based on a logarithmic decomposition of the stress \cite{miehe_anisotropic_2002}.

% ---------------------------------------------------------
% PARAGRAPH
% ---------------------------------------------------------
\paragraph{Material parameters}

Materials parameters are taken as
$\sigma_0 = 450$ MPa, $\sigma_{\infty} = 715$ MPa with a saturation parameter $\delta = 16.93$. The Young modulus is $E = 206.9$ GPa, and the Poisson ratio is $\nu = 0.29$.

% ---------------------------------------------------------
% PARAGRAPH
% ---------------------------------------------------------
\paragraph{Load deflection curve}

The load-displacement curve is plotted
in Figure \ref{fig_ssna_mesh}, and gives similar results to that obtained with quadratic reduced integration elements.

% ---------------------------------------------------------
% PARAGRAPH
% ---------------------------------------------------------
\paragraph{Equivalent plastic strain}

Moreover, the equivalent
plastic strain $p$ at quadrature points and at the final load is plotted Figure \ref{fig_ssna_res}.
% It has been observed that the equivalent plastic strain might suffer some oscillations at a certain limit load with UPG methods.
One notices through the present example, that the proposed HHO method displays no oscillations of the equivalent plastic strain as well.

\begin{figure}[H]
    \centering
    % \includegraphics[width=12.cm]{../chapter_002_hho_mechanics/figures/ssna_plastic.png}
    \includegraphics[width=12.cm]{../chapter_002_hho_mechanics/drawings/ssna_res.png}
    \caption{
        final hydrostatic pressure map and equivalent plastic strain map at quadrature points in the notch region
    }
    \label{fig_ssna_res}
\end{figure}

% ---------------------------------------------------------
% PARAGRAPH
% ---------------------------------------------------------
\paragraph{Hydrostatic pressure}

The hydrostatic pressure map at quadrature points and at the final load is shown Figure \ref{fig_ssna_res} for three HHO element orders (respectively $1, 2$ and $3$).
As for the radially loaded sphere test case, one notices that the hydrostatic pressure map is
fairly smooth over the whole structure at all approximation orders, even at the bottom left corner where plasticity is confined.

% ---------------------------------------------------------
% ---- SUBSECTION
% ---------------------------------------------------------
\subsection{Comparison of cell unknowns elimination schemes}
\label{sec_num_example_part_2}

In this section, both elimination strategies
introduced in Section \ref{sec_cell_unknowns_elimination} are evaluated. In particular, six different variants of the usual
Newton algorithm
are studied, namely :
% \begin{itemize}
%     \item (R) the Static condensation scheme using a consistent Jacobian operator
%     \item (M1) the Cell equilibrium scheme using a consistent Jacobian operator
%     \item (M2) the Static condensation scheme using an elastic Jacobian operator
%     \item (M3) the Cell equilibrium scheme using an elastic Jacobian operator
%     \item (M4) the Static condensation scheme using an elastic Jacobian operator and coupled with an Anderson acceleration algorithm, that acts on both cell and faces unknowns at the global level
%     \item (M5) the Cell equilibrium scheme using an elastic Jacobian operator coupled with an Anderson acceleration algorithm on faces unknowns only
% \end{itemize}
(Sc) a Static condensation scheme using a consistent Jacobian operator,
(Cc) a Cell equilibrium scheme using a consistent Jacobian operator,
(Se) a Static condensation scheme using an elastic Jacobian operator,
(Ce) a Cell equilibrium scheme using an elastic Jacobian operator,
(Sa) a Static condensation scheme using an elastic Jacobian operator and coupled with an Anderson acceleration algorithm, that acts on both cell and faces unknowns at the global level,
(Ca) a Cell equilibrium scheme using an elastic Jacobian operator coupled with an Anderson acceleration algorithm on faces unknowns only.

The last two test cases that showcase non-linear behaviors are considered for the evaluation of all these variants. The static condensation resolution scheme that uses a consistent stiffness matrix is taken for reference, since it is the
one used in the literature \cite{abbas_hybrid_2019,abbas_hybrid_2018}.
The estimation of the Anderson acceleration is performed every iteration and is based on the last three residual and unknown vectors.

\paragraph{Number of iterations per time step}

The number of iterations per time step for each variant
is displayed in Figure \ref{fig_acceleration_res_0} for both the radially loaded sphere test case
and the notched rod one.
At each time steps, the number of iteration is normalized by the number of iterations obtained with the
Static condensation variant (Sc).
As expected, the number of iteration per time step is similar for Cell resolution scheme based variants and Static condensation ones, for all polynomial orders and all alternatives listed above.

\begin{figure}[H]
    \centering
    \includegraphics[width=12.cm]{../chapter_002_hho_mechanics/figures/plot_global_iterations__4_ordn.png}
    \caption{Normalized number of iterations per time step for non-linear test cases and all resolution schemes variants}
    \label{fig_acceleration_res_0}
\end{figure}

\paragraph{Accelereated variants comparison}

The key argument in using the
Cell equilibrium scheme with an acceleration step is that 
only faces unknowns are accelerated.
Using the static condensation scheme demands to accelerate both cell and faces unknowns simultaneously.
However, using the Cell equilibrium scheme, the update of the cell unknown after a sole face acceleration step is performed at the cell level.
From a mechanical standpoint, faces unknowns are perturbed, and solving non-linear equation \eqref{eq_cell_equilibrium_1} by a local Newton algorithm as depicted in Section \ref{sec_cell_equilibrium_scheme} allows to retrieve the cell unknown increment verifying the equilibrium of the element.

\paragraph{Memory footprint}

In addition, local matrices used for the decondensation step in the Static condensation scheme are not kept from one iteration
to another using the Cell equilibrium scheme, hence drastically reducing the total memory footprint of the computation.
The number of scalar values to be stored between each iteration for a cell $\cell$ and for each variant is given in Table \ref{table_memory_footprint}, where $d$ is the number of components of the unknown field,
$M^l_T$ is the size of a scalar cell polynomial unknown, $M^l_T$ is that of a face, $N_{\dCell}$ is the number of faces for the cell $T$, and
$N_{it}$ denotes the number of residual and associated unknowns to be stored by the Anderson acceleration algorithm.

\begin{table}[H]
    \centering
    \begin{tabular}{||c c||} 
        \hline
        Resolution Method & Memory footprint
        \\
        [0.5ex] 
        \hline\hline
        (Sc), (Se) & $d \, M^l_T (M^k_{\dCell} \, N_{\dCell} + M^l_T)$
        \\ \hline
        (Cc), (Ce) & 0
        \\ \hline
        (Sa) & $d \, M^l_T (M^k_{\dCell} \, N_{\dCell} + M^l_T) + d \, 2 \, N_{it} (M^k_{\dCell} \, N_{\dCell} + M^l_T) $
        \\ \hline
        (Ca) & $d \, 2 \, N_{it} (M^k_{\dCell} \, N_{\dCell})$
        \\ \hline
    \end{tabular}
    \caption{Memory footprint for each alternative and for an element}
    \label{table_memory_footprint}
\end{table}

The number of scalar values that need be stored from one iteration to another for a single quandrangular element, depending on the discretization, and on the chosen acceleration scheme (\textit{i.e.} on the number of resiudal and unknown vectors to store)
is given in Figure \ref{fig_acceleration_res_memory} for different acceleration strategies, cell and faces polynomial orders.

\begin{figure}[H]
    \centering
    \includegraphics[width=8.cm]{../chapter_002_hho_mechanics/figures/plot_memory.png}
    \caption{Number of scalar entries to store from one iteration to another for different polynomial orders and resolution schemes variants}
    \label{fig_acceleration_res_memory}
\end{figure}

\paragraph{Local iterations}

The price for alleviating the memory footprint of the Static condensation scheme
is paid by systematic calls to the local Newton algorithm.
As a consequence, one also notices an increasing number of calls to the constitutive equation.
The mean number of local cell iterations per time time step is given in Figure \ref{fig_acceleration_res_1} for both test cases. The high number of cell iterations for the radially loaded sphere test case is
due to the fact that the plastic front involves a linearly increasing number of cells submitted to a perfect plastic beahviour.
Considering \textit{e.g.} a linear hardening constitutive equation leads to a stabilized number of cell iterations once the whole domain is plastic.

\begin{figure}[H]
    \centering
    \includegraphics[width=12.cm]{../chapter_002_hho_mechanics/figures/plot_cell_iterations__4_cell_iters.png}
    \caption{Normalized number of cell iterations per time step for non-linear test cases and Cell equilibrium based resolution schemes variants}
    \label{fig_acceleration_res_1}
\end{figure}

\paragraph{Local system}

The local system to solve with each cell scales with the size of a single cell unknown. An overview of the values taken using a monomial shape function can be found in \ref{sec_appendix_implementation}. These local resolution procedures consist in solving dense systems and are completely
parallelizable.

% \subsubsection{Quadratic and accelerated schemes}

% Both the Cell equilibrium and Static condensation resolution schemes are compared in terms of number of global iterations, using either
% a consistent stiffness matrix, an elastic operator, and an elastic operator coupled with an Anderson acceleration
% algorithm.
% The static condensation resolution scheme that uses a consistent stiffness matrix is taken for reference, since it is the
% one used in the literature \cite{abbas_hybrid_2019,abbas_hybrid_2018}.

% \paragraph{Resolution schemes using a consitent stiffness matrix}

% The number of iterations per time step for the Cell equilibrium algorithm using a consistent tangent operator
% is showcased by the dashed blue line in Figure \ref{fig_acceleration_res_0} for the radially loaded sphere test case,
% and in Figure \ref{fig_acceleration_res_1} for the notched rod one.
% Results are normalized by those obtained with the Static condensation algorithm.
% As expected, the number of iteration per time is roughly the same between the Cell resolution scheme
% and the Static condensation one, for all polynomial orders.

% \paragraph{Resolution schemes using an elastic stiffness matrix}

% Keeping the first elastic stiffness matrix for the whole computation showcases assets that are discussed in Section \ref{sec_cell_unknowns_elimination}. In this section, four resolution methods based on the elastic stiffness matrix are studied :
% \begin{itemize}
%     \item (M1) the Static condensation scheme
%     \item (M2) the Cell equilibrium scheme
%     \item (M3) the Static condensation scheme, coupled with an Anderson acceleration algorithm, that acts on both cell and faces unknowns at the global level
%     \item (M4) the Cell equilibrium scheme, coupled with an Anderson acceleration algorithm on faces unknowns only
% \end{itemize}
% %
% %
% %
% The number of iterations 
% (M1) the Static condensation scheme, (M2) the Cell equilibrium scheme, (M3) the Static condensation scheme, coupled with
% an Anderson acceleration algorithm, that acts on both cell and faces unknowns at the global level and (M4) the Cell
% equilibrium scheme, coupled with an Anderson acceleration algorithm for faces unknowns only.

% \paragraph{}

% Figure \ref{fig_acceleration_res_0} gives the number of iterations needed to achieve convergence
% at the skeletal level, with respect to the current pseudo-time step. The computation is performed
% in $40$ uniform steps from an initial steady state up to the limit radial load of $0.2$ mm (see \ref{sec_swelling_sphere}).
% For this experiment, three variants of both cell elimination schemes are considered :
% \begin{itemize}
%     \item A first scheme S1 consists in using a consistent tangent operator for the computation of the stiffness matrix
%     \item A second scheme S2 consists in using an elastic tangent operator for the computation of the stiffness matrix
%     \item A last scheme consists in us
% \end{itemize}

% ---------------------------------------------------------
% ---- SECTION
% ---------------------------------------------------------
\section{Conclusion}

% Dire de maniere explicte, mettre les éléments virtuels dans le même cadre -> dire quil reste à examiner le lien avec les éléments viertuels 
% Le HW permet de retrouver les principes HDG, HHO qui sont au coeur de ce paprier mais aussi les cG
% les VEM notn pas ete bconsideres, bien que le cadrez propos semblke sadapter à leur formalmisme
% descxirption du code, trouvaable sur github, avec chaque exemple

An introduction to HDG and HHO methods has been proposed, based on the minimization of a Hu-Washizu Lagrangian. The expression of the method arising from this approach allows to introduce naturally all the ingredients of the method, as well as the displacement discontinuity, in a unified framework.
This formulation also allows to draw a connection between HDG methods and other locking-free methods based on the minimization of a Hu-Washizu Lagrangian.
A natural cell-based resolution scheme emerged from this formulation, that has been tested and evaluated.
Finally, a HHO method to account for mechanical problems in the axisymmetric framework has been devised and evaluated numerically, for both linear thermoelastic behaviours, and plastic behaviours under both the small and finite strain hypotheses.
The proposed HHO method exhibits a robust behaviour to volumetric locking for strain-hardening plasticity as well as for perfect plasticity in primal formulation, with a moderate number of degrees of freedom.

This work can be pursued in several directions. One could use the cell resolution algorithm to address local resolution problems, such as those encountered with \textit{e.g.} damage irreversibility in phase field fracture mechanics, or multi field plasticity. Moreover, an adaptation of the HHO method to
reconstruct pressure-driven gradient terms only could lead to a simpler formulation, closer to that of mixed methods \cite{simo_quasi-incompressible_1991}.

\chapter{Micromorphic damage behaviours for quasi-brittle materials}
\label{chapter:micromorphic_damage}

\section{Introduction}

The variational approach to brittle fracture takes its roots in the work of
Francfort and Marigo \cite{francfort_revisiting_1998,francfort_vers_2002},
which recasted the Griffith theory into an energy
minimization problem.
This revisted approach of the Griffith theory is however not tractable
with standard numerical methods
\cite{bourdin_numerical_2000, chambolle_approximation_2018}, in particular
the commonly used finite element method. For this reason, Bourdin \textit{et
al.} developed regularized versions \cite{bourdin_numerical_2000} following
the works of Ambrosio and Tortorelli \cite{ambrosio_approximation_1990}.

\paragraph{Phase field approach to brittle fracture and damage irreversibility consition}

The so-called phase-field approaches to fracture have since become
widely popular. As pointed by Gerasimov and De Lorenzis in their
excellent review \cite{gerasimov_numerical_2020}, one of the main
difficulties in the implementation of those approaches is the treatment
of the irreversibility constraint (the damage can only increase), a
question on which a considerable amount of works has been published.
Most of the proposed solutions are not directly implemented in standard
FEM or FFT solvers. An noticeable exception to this statement is the
Miehe' alternative based an the so-called history variable
\cite{miehe_phase_2010}. However, Miehe' alternative is not variationally
consistent.

\paragraph{Micromorphic approach to brittle fracture}

A comprehensive framework for 
micromorphic approaches to various physical problems,
including brittle fracture, has been developed by Forest in
\cite{forest_micromorphic_2009, forest_nonlinear_2016}.
Balance equations arising from
such micromorphic approaches are
standard partial differential equations, that can readily be solved by
most FEM or FFT solvers.
In particular, these micromorphic behaviours allow to
deal with the damage irreversibility constraint at integration points,
instead of having to deal with it in the resolution of balance equations,
as is the case for the classical Phase field approach \cite{gerasimov_numerical_2020}. These
advantages have been highlighted by Rezaee-Hajidehi \textit{et al.} in the
context of phase-field approaches to phase transformation
\cite{rezaee-hajidehi_micromorphic_2021}.
%
%
%
Such micromorphic models were recently investigated by Bharali \textit{et al.}
\cite{bharali_computational_2021} to approximate the AT1 and AT2 models
using a monolithic resolution strategy.

\paragraph{Outline}

Following Forest' micromorphic framework
\cite{forest_micromorphic_2009, forest_nonlinear_2016}, a class of micromorphic
brittle behaviours that can approximate classical models of brittle
fracture in a variationally consistent way is proposed in
Section \ref{sec:micromorphicdamage:description}.
The variational basis of the behaviour is exploited to derive three alternate
minimization schemes in Section
\ref{sec:micromorphicdamage:alternate_minimisation}, whose convergence is guaranteed.
%
%
%
Several test cases comparing the prediction of the classical AT2 model
and its micromorphic counterpart are presented in Section
\ref{sec:micromorphicdamage:test_cases}, where
the choice of the penalization
parameter which at the foundation of the link between the Phase Field approach and the 
micromorphic one is discussed in-depth.
%
%
%
Two numerical experiments are presented in Section
\ref{sec:micromorphicdamage:numerical_experiments}, which assess:
%
%
%
\begin{itemize}
    \item The performance of the micromorphic approach with high order finite
    elements in the case of shear driven fracture.
    \item The scalability of the micromorphic approach.
\end{itemize}
\section{The micromorphic approach to brittle fracture}
\label{sec:micromorphicdamage:description}

\subsection{Description of the coupled problem}
\label{sec:ef_micromorphic:coupled_problem_descripion}

\paragraph{Solid body in the current configuration}

Let $\BodyEuler$ a solid body that is subjected to a volumetric load $\tensori{f}{}_v$ in
the current configuration at some time $t > 0$.
A displacement $\tensori{u}{}_{d}$ is prescribed
on the Dirichlet boundary $\BodyEulerDirichletBoundary$ and a surface load $\tensori{t}{}_{n}$ is imposed
on the Neumann boundary $\BodyEulerNeumannBoundary$.

\paragraph{Transformation mapping}

Let $\tensori{\Phi}{}$ the transformation mapping of the solid body from the initial configuration $\BodyLagrange$
to the current configuration $\BodyEuler$.
The displacement field $\DisplacementField$ is such
that $\DisplacementField = \tensori{\Phi}{} - \IdentityTensorI$ where $\IdentityTensorI$
is the identity application on $\BodyLagrange$.
The gradient of the transformation is
denoted $\TransformationGradientField = \nabla \tensori{\Phi}{} = \IdentityTensorII + \nabla \DisplacementField$.
Let $\BodyLagrangeDirichletBoundary$ and $\BodyLagrangeNeumannBoundary$ the images
of $\BodyEulerDirichletBoundary$ and $\BodyEulerNeumannBoundary$ respectively by $\tensori{\Phi}{}^{-1}$.

\paragraph{External loads in the reference configuration} 

In the reference configuration, the solid is subjected to a volumetric load
$\tensori{f}{}_V$, a prescribed displacement $\tensori{u}{}_D$ on $\BodyLagrangeDirichletBoundary$, and a surface load $\tensori{t}{}_N$ on $\BodyLagrangeNeumannBoundary$, where the volumetric and surface loads $\tensori{f}{}_V$ and $\tensori{t}{}_N$ have been obtained from their counterparts
$\tensori{f}{}_v$ and $\tensori{t}{}_n$ respectively, using Nanson formulaes. For the sake of simplicty, they are supposed to be independent
on $\tensorii{F}{}$.

\paragraph{State of the solid} The mechanical state of the solid body $\BodyLagrange$ is characterized by the displacement field $\DisplacementField$,
the damage field $\DamageField$ and a micromorphic damage field $\MicromorphicDamageField$.

\paragraph{Free energy potential}

The free energy potential $\psi_{\BodyLagrange}$ of the body $\BodyLagrange$ reads as a function of the displacement $\DisplacementField$, the (local) damage $d$ and the micromorphic damage $\MicromorphicDamageField$, in the form
%
%
%
\begin{equation}
    \psi_{\BodyLagrange}
    (\TransformationGradientField, \DamageField, \MicromorphicDamageField, \nabla \MicromorphicDamageField)
    =
    \psi_{\tensoriis{F}, \DamageField}
    (\TransformationGradientField, \DamageField)
    +
    \psi_{\DamageField}
    (\DamageField)
    +
    \psi_{\MicromorphicDamageField, \DamageField}
    (\MicromorphicDamageField, \DamageField)
    +
    \psi_{\nabla \MicromorphicDamageField}
    (\nabla \MicromorphicDamageField)
\end{equation}
%
%
%
where $\psi_{\tensoriis{F}, \DamageField}$ denotes the mechanical contribution that takes into account the damage in the medium,
$\psi_{\DamageField}$ is the energy stored during the fracture process,
$\psi_{\DamageField, \MicromorphicDamageField}$ is a coupling term between the damage and micromorphic damage variables, and
$\psi_{\nabla \MicromorphicDamageField}$ defines the micromorphic force.
In practice, $\psi_{\BodyLagrange}$ is a differentiable function.

\paragraph{Stresses}

In the spirit of \cite{forest_micromorphic_2009}, the following stresses are introduced
%
%
%
\begin{equation}
    \begin{aligned}
        \PKIStressField = \deriv{\psi_{\BodyLagrange}}{\TransformationGradientField}
        && &&
        \MicromorphicDamageStressField = \deriv{\psi_{\BodyLagrange}}{\nabla \MicromorphicDamageField}
        && &&
        \MicromorphicDamageForceField = \deriv{\psi_{\BodyLagrange}}{\MicromorphicDamageField}
        && &&
        \DamageForceField = \deriv{\psi_{\BodyLagrange}}{\DamageField}
    \end{aligned}
\end{equation}
%
%
%
where $\PKIStressField$ is the first Piola-Kirchoff stress tensor, and $\MicromorphicDamageStressField, \MicromorphicDamageForceField$ and $\DamageForceField$ are the thermodynamic
forces associated to $\nabla \MicromorphicDamageField, \MicromorphicDamageField$ and $\DamageField$ respectively.

\paragraph{Dissipation potential}

A dissipation potential $\phi_{\BodyLagrange}(\DamageField)$ accounts for the irreversibility of the fracture process in the medium.
For time independent mechanisms, $\phi_{\BodyLagrange}$ is usually not differentiable. However, it is is assumed to be
an homogeneous function of degree one
such that for any time increment $\Delta \, t > 0$, and any admissible damage field $\DamageField$ at the time $t + \Delta \, t$
%
%
%
\begin{equation}
  \label{eq:ef_micromorphic:formulation:dissipation_potential}
  \begin{aligned}
      \Delta \, t \, \phi_{\BodyLagrange}(\frac{\DamageField - \DamageField \TraceOperator{t}} {\Delta \, t})
      =
      \phi_{\BodyLagrange}(\DamageField - \DamageField \TraceOperator{t})
  \end{aligned}
\end{equation}
%
%
%
In particular, $\phi_{\bodyLag}(d)$ generally contains an indicator function to enforce the irreversibility of the
damage evolution.

\subsection{Lagrangian formulation of the coupled problem}
\label{sec:ef_micromorphic:coupled_problem_lagrangien}

\paragraph{Total Lagrangian}

In the view of generalized standard materials
\cite{moreau_sur_1970, halphen_sur_1975, ehrlacher_principe_1985, nguyen_standard_2002},
and for any time increment $\Delta \, t > 0$,
the equilibrium of the solid $\BodyLagrange$ at the time $t + \Delta \, t$ is characterized by the stationarity of the total incremental Lagrangian
$\LagrangianOperator{\BodyLagrange}{\text{tot}}$ \cite{lorentz_variational_1999,forest_localization_2004}
such that
%
%
%
\begin{equation}
  \label{eq:ef_micromorphic:formulation:total_lagrangian_1}
  \begin{aligned}
    \LagrangianOperator{\BodyLagrange}{\text{tot}}
    =
    &
    \int_{\BodyLagrange}
    \psi_{\BodyLagrange}
    (\TransformationGradientField, \DamageField, \MicromorphicDamageField, \nabla \MicromorphicDamageField)
    +
    \int_{\BodyLagrange} \Delta \, t \, \phi_{\BodyLagrange}(\frac{\DamageField - \DamageField \TraceOperator{t}} {\Delta \, t})
    -
    \int_{\BodyLagrange} \loadLag{} \cdot \DisplacementField
    -
    \int_{\BodyLagrangeNeumannBoundary} \neumannLag \cdot \DisplacementField \TraceOperator{\BodyLagrangeNeumannBoundary}
  \end{aligned}
\end{equation}
%
%
%
for any admissible displacement, damage and micromorphic damage fields $\DisplacementField, \DamageField$
and $\MicromorphicDamageField$ at the time $t + \Delta \, t$.
%
%
%
To satisfy the Ilyushin-Drucker postulate, the total incremental Lagrangian
$\LagrangianOperator{\BodyLagrange}{\text{tot}}$ must be convex with respect to each variables
$\DisplacementField, \DamageField$
and $\MicromorphicDamageField$ taken independently. It
can be shown that this condition is ensured if the free energy potential $\psi_{\BodyLagrange}$ and
the dissipation potential $\phi_{\BodyLagrange}$ are convex with
respect to their respective arguments.

\paragraph{Total Lagrangian split}

Deriving equilibrium equations from the principle of the minimum of the
Lagrangian is non trivial due to the fact that the dissipation potential
is not differentiable.
%
%
%
It is then convenient to separate the total Lagrangian $\LagrangianOperator{\BodyLagrange}{\text{tot}}$ into two parts
$\LagrangianOperator{\BodyLagrange}{\text{hel}}$ and $\LagrangianOperator{\BodyLagrange}{\text{dis}}$ as follows:
%
%
%
\begin{subequations}
  \label{eq:ef_micromorphic:formulation:total_lagrangian_split}
      \begin{alignat}{3}
        \LagrangianOperator{\BodyLagrange}{\text{hel}}
        &
        =
        \int_{\BodyLagrange} \FreeEnergy_{\BodyLagrange}(\TransformationGradientField, \DamageField, \MicromorphicDamageField, \nabla \MicromorphicDamageField)
        -
        \int_{\BodyLagrange} \loadLag{} \cdot \DisplacementField
        -
        \int_{\BodyLagrangeNeumannBoundary} \neumannLag \cdot \DisplacementField \TraceOperator{\BodyLagrangeNeumannBoundary}
        \label{eq:ef_micromorphic:formulation:total_lagrangian_split:eq0}
        \\
        \LagrangianOperator{\BodyLagrange}{\text{dis}}
        &
        = 
        \int_{\BodyLagrange} \phi_{\BodyLagrange}(\DamageField - \DamageField \TraceOperator{t})
        \label{eq:ef_micromorphic:formulation:total_lagrangian_split:eq1}
  \end{alignat}
\end{subequations}
%
%
%
where \eqref{eq:ef_micromorphic:formulation:dissipation_potential} has been used in \eqref{eq:ef_micromorphic:formulation:total_lagrangian_split:eq1}.

% \subsubsection{Regular part of the Lagrangian}
% \label{sec:ef_micromorphic:lagrangian_regular_part}

% In this work, the following mathematical results \cite{son_standard_2021} characterizing
% the minima of Lagrangian \eqref{eq:ef_micromorphic:formulation:total_lagrangian_1} are assumed :

% \paragraph{Regular part of the Lagrangian}

% The regular part of the Lagrangian $\LagrangianOperator{\BodyLagrange}{\text{hel}}$ is minimal with
% respect to the displacement field $\DisplacementField$ and the micromorphic
% damage field $\MicromorphicDamageField$.

% \subsubsection{Dissipative part of the Lagrangian}
% \label{sec:ef_micromorphic:lagrangian_regular_part}

% \paragraph{Dissipative part of the Lagrangian}

% At each point, the thermodynamic force $\DamageForceField$ associated with the
% damage is in the subgradient of the dissipation potential:
% %
% %
% %
% \begin{equation}
%   \label{eq:ef_micromorphic:formulation:generalized_standard_0}
%   \DamageForceField \in \partial \phi_{\BodyLagrange}
% \end{equation}

\subsection{The displacement and micromorphic damage problem}
\label{sec:ef_micromorphic:lagrangian_regular_part}

In this section, the condition on the regular part \eqref{eq:ef_micromorphic:formulation:total_lagrangian_split:eq0} of
the Lagrangian \eqref{eq:ef_micromorphic:formulation:total_lagrangian_1} is considered only.
The evolution of the damage is discussed in Section
\ref{sec:ef_micromorphic:formulation:damage_evolution}.

\paragraph{Lagrangian variations}

Following \cite{son_standard_2021}, the solid $\BodyLagrange$ is in equilibrium if
the regular part of the Lagrangian $\LagrangianOperator{\BodyLagrange}{\text{hel}}$ is minimal with
respect to the displacement field $\DisplacementField$ and the micromorphic
damage field $\MicromorphicDamageField$.
% Hence, the solution $(\DisplacementField, \MicromorphicDamageField)$
% satisfying the mechanical equilibrium minimizes the Lagragian $\LagrangianOperator{\BodyLagrange}{\text{hel}}$.
The first order variations of the Lagrangian \eqref{eq:ef_micromorphic:formulation:total_lagrangian_split:eq0} with respect to $\DisplacementField$ and $\MicromorphicDamageField$
respectively yields the weak equations
%
%
%
\begin{subequations}
    \label{eq:ef_micromorphic:formulation:lagragian_variations}
    \begin{alignat}{3}
      \langle \deriv{\LagrangianOperator{\BodyLagrange}{\text{hel}}}{\DisplacementField} , \delta \DisplacementField \rangle
      =
      & \int_{\BodyLagrange} \PKIStressField : \nabla \delta \DisplacementField
      -
      \int_{\BodyLagrange} \tensori{f}_V \cdot \delta \DisplacementField
      -
      \int_{\BodyLagrangeNeumannBoundary} \neumannLag \cdot \delta \DisplacementField \vert_{\BodyLagrangeNeumannBoundary}
      &&
      \qquad
      &&
      \forall \delta \DisplacementField
      \label{eq:ef_micromorphic:formulation:lagragian_variations:eq0}
      \\
      \langle \deriv{\LagrangianOperator{\BodyLagrange}{\text{hel}}}{\MicromorphicDamageField} , \delta \MicromorphicDamageField \rangle
      =
      & \int_{\BodyLagrange} \MicromorphicDamageForceField \, \delta \MicromorphicDamageField + \int_{\BodyLagrange} \MicromorphicDamageStressField \cdot \nabla \MicromorphicDamageField
      &&
      \qquad
      && \forall \delta d^\chi
      \label{eq:ef_micromorphic:formulation:lagragian_variations:eq3}
    \end{alignat}
\end{subequations}

\paragraph{Strong equation}

Applying the divergence theorem, the following strong equations for the sole displacement problem are deduced from the weak equation
\eqref{eq:ef_micromorphic:formulation:lagragian_variations:eq0}
%
%
%
\begin{subequations}
    \label{eq:ef_micromorphic:formulation:strong_problem_u}
    \begin{alignat}{3}
    \nabla \cdot \PKIStressField + \tensori{f}{}_{V} & = 0
    &&
    \qquad
    &&
    \textit{balance of momentum}
    \label{eq:ef_micromorphic:formulation:strong_problem_u:eq2}
    \\
    \PKIStressField \cdot \tensori{n}{} - \neumannLag{} & = 0
    &&
    \qquad
    &&
    \textit{continuity of the normal stress}
    \label{eq:ef_micromorphic:formulation:strong_problem_u:eq3}
    \end{alignat}
\end{subequations}
%
%
%
Similarly, the strong equations governing the sole micromorphic damage problem are deduced from
\eqref{eq:ef_micromorphic:formulation:lagragian_variations:eq3}
%
%
%
\begin{subequations}
    \label{eq:ef_micromorphic:formulation:strong_problem_d}
    \begin{alignat}{3}
        \nabla \cdot \MicromorphicDamageStressField - \MicromorphicDamageForceField & = 0
        &&
        \qquad
        &&
        \textit{balance of micromorphic damage momentum}
        \label{eq:ef_micromorphic:formulation:strong_problem_d:eq2}
        \\
        \MicromorphicDamageStressField \cdot \tensori{n}{} & = 0
        &&
        \qquad
        &&
        \textit{micromorphic damage boundary conditions}
        \label{eq:ef_micromorphic:formulation:strong_problem_d:eq3}
    \end{alignat}
\end{subequations}
%
%
%
where the governing laws of the micromorphic damage variable define a generalized continuum medium as introduced in \cite{forest_micromorphic_2009}.

\subsubsection{Evolution of the displacement}
\label{sec:ef_micromorphic:formulation:displacement_evolution}

\paragraph{Mechanical potential}

$\psi_{\tensoriis{F}, \DamageField}$ determines the expression of the stress and contributes
to the thermodynamic force driving the damage evolution.
A classical choice consists in multiplying the free energy of an undamaged
material $\psi_{\tensoriis{F}}$ by a degradation
function $g(\DamageField)$ such that
%
%
%
\begin{subequations}
    \label{eq:micromorphicdamage:freeenergygel}
    \begin{alignat}{3}
      \psi_{\tensoriis{F}, \DamageField}
      (\TransformationGradientField, \DamageField)
      =
      &
      g(\DamageField) \, \psi_{\tensoriis{F}}(\TransformationGradientField)
      \label{eq:micromorphicdamage:freeenergygel:eq0}
      \\
      \PKIStressField
      =
      &
      g(\DamageField) \, \deriv{\psi_{\tensoriis{F}}}{\TransformationGradientField}
      \label{eq:micromorphicdamage:freeenergygel:eq1}
    \end{alignat}
\end{subequations}

\paragraph{Mechanical potential deviatoric split}
\label{sec:micromorphic:deviatori_split}

In order to account for the possible dependance of the damage driving force on the stress state,
% Since fracture is mostly a pressure-driven phenomenon (which occurs \textit{e.g.} for high triaxial loads),
a decomposition \cite{alessi_gradient_2015,miehe_phase_2010} of the mechanical free energy density $\psi_{\tensoriis{F}}$
into a deviatoric part $\psi_{\tensoriis{F}}{}_{\text{dev}}$ and into a spherical part
$\psi_{\tensoriis{F}}{}_{\text{sph}}$ is commonly assumed.
The degradation function
$g(\DamageField)$
then acts on either one the two components $\psi_{\tensoriis{F}}{}_{\text{dev}}$ or $\psi_{\tensoriis{F}}{}_{\text{sph}}$
hence defining either \textit{shear driven fracture} or \textit{pressure driven fracture}

\paragraph{Mechanical potential spectral split}

Another classical choice proposed by Miehe \cite{miehe_phase_2010} consists in
using a spectral decomposition of the strain to split the free energy into
a positive $\psi_{\tensoriis{F}}{}_{\text{+}}$ and a negative $\psi_{\tensoriis{F}}{}_{\text{-}}$ part.
In this case, the degradation function is only applied
to the positive part $\psi_{\tensoriis{F}}{}_{\text{+}}$ of the free energy, in order to account
for \textit{traction driven fracture} processes.

\paragraph{Micromorphic damage coupling potential}

The free energy density $\psi_{\MicromorphicDamageField, \DamageField}$ is chosen such that it penalizes
the difference between
the damage field $\DamageField$ and the micromorphic damage field $\MicromorphicDamageField$ such that
%
%
%
\begin{subequations}
    \label{eq:micromorphicdamage:achi}
    \begin{alignat}{3}
      \psi_{\MicromorphicDamageField, \DamageField}
      (\MicromorphicDamageField, \DamageField)
      =
      &
      \Frac{H_{\chi}}{2} \, \paren{\DamageField - \MicromorphicDamageField}^{2}
      \label{eq:micromorphicdamage:achi:eq0}
      \\
      \MicromorphicDamageForceField
      =
      &
      - H_{\chi} \, \paren{\DamageField - \MicromorphicDamageField}
      \label{eq:micromorphicdamage:achi:eq1}
    \end{alignat}
\end{subequations}

\subsubsection{Evolution of the micromorphic damage}
\label{sec:ef_micromorphic:formulation:micromorphic_damage_evolution}

\paragraph{Micromorphic damage gradient potential}

The free energy density $\psi_{\nabla \MicromorphicDamageField}$ is chosen such that
it penalizes the localization of the micromorphic damage field $\MicromorphicDamageField$ and
%
%
%
\begin{subequations}
    \label{eq:micromorphicdamage:bchi}
    \begin{alignat}{3}
      \psi_{\nabla \MicromorphicDamageField}
      (\nabla \MicromorphicDamageField)
      =
      &
      \Frac{A_{\chi}}{2} \, \nabla \MicromorphicDamageField \cdot \nabla \MicromorphicDamageField
      \label{eq:micromorphicdamage:bchi:eq0}
      \\
      \MicromorphicDamageStressField
      =
      &
      A_{\chi} \, \nabla \MicromorphicDamageField
      \label{eq:micromorphicdamage:bchi:eq1}
    \end{alignat}
\end{subequations}

\paragraph{Micromorphic damage relation}

Substituting \eqref{eq:micromorphicdamage:achi:eq1} and \eqref{eq:micromorphicdamage:bchi:eq1} in
the expression of the governing equation \eqref{eq:ef_micromorphic:formulation:strong_problem_d:eq2}
shows that the micromorphic damage $\MicromorphicDamageField$ satisfies the Laplace equation \cite{forest_micromorphic_2009}
%
%
%
\begin{equation}
  \label{eq:micromorphicdamage:d_chi2}
  A_{\chi} \, \LaplacianOperator \, \MicromorphicDamageField
  +
  H_{\chi} \, \paren{\DamageField - \MicromorphicDamageField}
  =
  0
\end{equation}
%
%
%
where $\LaplacianOperator$ denotes the Laplacian operator.

\subsection{Sole Damage problem}
\label{sec:ef_micromorphic:formulation:damage_evolution}

\paragraph{Damage driving force}

Following the definition of mechanical and micromorphic damage related potentials given in
Section \ref{sec:ef_micromorphic:lagrangian_regular_part}, the expression of the thermodynamic
force $\DamageForceField$ associated with the damage field is given by:
%
%
%
\begin{equation}
  \label{eq:micromorphicdamage:Y}
  \begin{aligned}
    \DamageForceField
    &
    =
    -
    \derivtot{g}{d} \, \psi_{\tensoriis{F}} (\TransformationGradientField)
    -
    \derivtot{\psi_{\DamageField}}{\DamageField}
    -
    \deriv{\psi_{\MicromorphicDamageField, \DamageField}}{\DamageField}
    \\
    &
    =
    -
    \derivtot{g}{d} \, \psi_{\tensoriis{F}} (\TransformationGradientField)
    -
    \derivtot{\psi_{\DamageField}}{\DamageField}
    +
    a_{\chi}
  \end{aligned}
\end{equation}
%
%
%
where \eqref{eq:micromorphicdamage:achi:eq0} has been used.

\paragraph{Dissipative part of the Lagrangian}

Following \cite{son_standard_2021,moreau_sur_1970,halphen_sur_1975}, at each point the thermodynamic force $\DamageForceField$ associated with the
damage is assumed to be in the subgradient of the dissipation potential:
%
%
%
\begin{equation}
  \label{eq:ef_micromorphic:formulation:generalized_standard_0}
  \DamageForceField \in \partial \phi_{\BodyLagrange}
\end{equation}

\paragraph{Dissipation potential}

A simple choice for the dissipation potential is
%
%
%
\begin{equation}
  \label{eq:micromorphicdamage:dissipationpotential}
  \begin{aligned}
    \phi_{\BodyLagrange} (\dot{\DamageField})
    =
    Y_{0} \, \dot{\DamageField}
    +
    \mathrm{I}_{\mathbb{R}_{+}} (\dot{\DamageField})
  \end{aligned}
\end{equation}
%
%
%
where $Y_0$ is a scalar threshold value, and $\mathrm{I}_{\mathbb{R}_{+}}$ denotes the characteristic function associated with positive real number such that
%
%
%
\begin{equation}
  \mathrm{I}_{\mathbb{R}_{+}} (x)
  =
  \left\{
    \begin{aligned}
    0 & \quad \text{if} \quad x \geq 0
    \\
    +\infty & \quad \text{if} \quad x < 0
    \end{aligned}
  \right.
\end{equation}

\paragraph{Yield surface}

The dissipation potential \eqref{eq:micromorphicdamage:dissipationpotential} is equivalent to define
the damage surface
%
%
%
\begin{equation}
  \label{eq:micromorphicdamage:yield}
  Y = Y_{0}
\end{equation}
%
%
%
leading to an evolution of the damage driven by the following Karush–Kuhn–Tucker conditions
%
%
%
\begin{subequations}
    \label{eq:micromorphicdamage:KTT}
    \begin{alignat}{3}
      \Delta \, \DamageField \, (\DamageForceField - Y_{0})
      &
      =
      0
      \label{eq:micromorphicdamage:KTT:eq0}
      \\
      \Delta \, \DamageField
      &
      \geq
      0
      \label{eq:micromorphicdamage:KTT:eq1}
      \\
      \DamageForceField - Y_{0}
      &
      \leq
      0
      \label{eq:micromorphicdamage:KTT:eq2}
    \end{alignat}
\end{subequations}
%
%
%
Combining equations \eqref{eq:micromorphicdamage:d_chi2},
\eqref{eq:micromorphicdamage:Y} and \eqref{eq:micromorphicdamage:yield}, the yield
surface may also be written:
%
%
%
\begin{equation}
  \label{eq:micromorphicdamage:damage_estimate}
  \derivtot{g}{d} \, \psi_{\tensoriis{F}} (\TransformationGradientField)
  =
  Y_{0}
  +
  \derivtot{\psi_{\DamageField}}{\DamageField}
  -
  H_{\chi}(\DamageField - \MicromorphicDamageField)
\end{equation}
%
%
%
which yields that $\DamageField$ is an implicit function of $\MicromorphicDamageField$
\section{Link with classical models of brittle fracture}

\paragraph{Total resulting Lagrangian}

Choices described by equations \eqref{eq:micromorphicdamage:freeenergygel},
\eqref{eq:micromorphicdamage:dissipationpotential} and
\eqref{eq:micromorphicdamage:bchi:eq0} lead to the following expression
of the total Lagrangian \eqref{eq:ef_micromorphic:formulation:total_lagrangian_1}
%
%
%
\begin{equation}
  \label{eq:micromorphicdamage:Lagrangian}
  \begin{aligned}
    \LagrangianOperator{\BodyLagrange}{\text{tot}}
    =
    &
    \int_{\BodyLagrange} g(\DamageField) \, \psi_{\tensoriis{F}} (\TransformationGradientField)
    +
    \int_{\BodyLagrange} \psi_{\DamageField}(\DamageField)
    +
    \int_{\BodyLagrange} Y_{0} \, \DamageField
    +
    \int_{\BodyLagrange} \Frac{A_{\chi}}{2} \, \nabla \MicromorphicDamageField \cdot \nabla \MicromorphicDamageField
    \\
    &
    +
    \int_{\BodyLagrange} \Frac{H_{\chi}}{2} \, \paren{\DamageField - \MicromorphicDamageField}^{2}
    +
    \int_{\BodyLagrange} \mathrm{I}_{\mathbb{R}_{+}} (\DamageField - \DamageField \TraceOperator{t})
    \\
    &
    -
    \int_{\BodyLagrange} \loadLag{} \cdot \DisplacementField
    -
    \int_{\BodyLagrangeNeumannBoundary} \neumannLag \cdot \DisplacementField \TraceOperator{\BodyLagrangeNeumannBoundary}
  \end{aligned}
\end{equation}
%
%
%
where the constant term $Y_{0} \, \DamageField \TraceOperator{t}$, which thus does not have
any influence of the solution, has been removed from the expression of
the Lagrangian.

\paragraph{Equal damage and micromorphic field limit case}

For high values of $H_{\chi}$, the contribution to the
potential $\psi_{\MicromorphicDamageField, \DamageField}$
may be seen as a
penalization term which ensures that the damage $\DamageField$ and the
micromorphic damage $\MicromorphicDamageField$ are equal in a weak sense.
Intuitively, if $H_{\chi} \rightarrow \infty$, these two must become
equal to ensure a finite energy.
The Lagrangian \eqref{eq:micromorphicdamage:Lagrangian} is thus expected to have the following limit
%
%
%
\begin{equation}
  \label{eq:micromorphicdamage:LagrangianLimit}
  \begin{aligned}
    \LagrangianOperator{\BodyLagrange}{\text{tot}}
    =
    &
    \int_{\BodyLagrange} g(\DamageField) \, \psi_{\tensoriis{F}} (\TransformationGradientField)
    +
    \int_{\BodyLagrange} \psi_{\DamageField}(\DamageField)
    +
    \int_{\BodyLagrange} Y_{0} \, \DamageField
    +
    \int_{\BodyLagrange} \Frac{A_{\chi}}{2} \, \nabla \DamageField \cdot \nabla \DamageField
    \\
    &
    +
    \int_{\BodyLagrange} \mathrm{I}_{\mathbb{R}_{+}} (\DamageField - \DamageField \TraceOperator{t})
    \\
    &
    -
    \int_{\BodyLagrange} \loadLag{} \cdot \DisplacementField
    -
    \int_{\BodyLagrangeNeumannBoundary} \neumannLag \cdot \DisplacementField \TraceOperator{\BodyLagrangeNeumannBoundary}
  \end{aligned}
\end{equation}
%
%
%
and Lagrangian \eqref{eq:micromorphicdamage:LagrangianLimit} can be identified with
Lagrangians describing many classical models of brittle fracture with
appropriate choices of $g(\DamageField)$, $\psi_{\DamageField}$,
$A_{\chi}$ and $Y_{0}$.

\subsection{Link with Ambrosio–Tortorelli regularization models}

% \paragraph{Ambrosio–Tortorelli regularization}

% Several works have shown that the solutions of (2.3) converge in the sense of the so-called Γ-
% convergence, to the solution of the initial Problem (2.1) in some specific cases. For example,
% [Bourdin et al., 2000] studied the anti-plane shear case.
% Those results bridges the conceptual gap between the global approach of fracture based
% on Griffith’ theory and the local approach to fracture, a least for a certain class of non local
% models.

Ambrosio–Tortorelli \cite{ambrosio_approximation_1990} regularization of the initial
variational approach to fracture \cite{francfort_revisiting_1998},
originally introduced in \cite{bourdin_numerical_2000}, have become
widely used in the computational mechanics community.
Several works have shown that the regularized solution converges in the sense of the so-called $\Gamma$-
convergence, to the solution of the initial problem in some specific cases
(See \textit{e.g.} \cite{bourdin_numerical_2000}).

\paragraph{Values of the AT1 model}

Two regularization models were introduced in \cite{ambrosio_approximation_1990}.
The first one, namely the AT1 model, allows to recover the initial problem formulated in \cite{francfort_revisiting_1998}
for infinitely small values of the characteristic length defining the thckness of the smeared crack, and
such that initiation of damage occurs on the onset of some yield criterion. Choosing the following potentials and values
for the definition of the micromorphic model allows to recover the AT1 model
%
%
%
\begin{subequations}
  \label{eq:ef_micromorphic:formulation:AT1_link}
  \begin{alignat}{3}
    g(\DamageField)
    &
    =
    (1 - \DamageField)
    \label{eq:ef_micromorphic:formulation:AT1_link:eq0}
    \\
    \psi_{\DamageField}(\DamageField)
    &
    =
    \Frac{3 \, G_{c}}{8 \, \ell_{c}} \, \DamageField
    \label{eq:ef_micromorphic:formulation:AT1_link:eq1}
    \\
    A_{\chi}
    &
    =
    \Frac{3}{4} \, G_{c} \, \ell_{c}
    \label{eq:ef_micromorphic:formulation:AT1_link:eq2}
    \\
    Y_{0}
    &
    =
    0
    \label{eq:ef_micromorphic:formulation:AT1_link:eq3}
  \end{alignat}
\end{subequations}
%
%
%
where $G_{c}$ is a fracture energy, and $\ell_{c}$ is the characteristic length.

\paragraph{Values of the AT2 model}

The second regularization model in \cite{ambrosio_approximation_1990} introduces a quadratic fracture energy potential,
such that damage occurs on the onset of deformation of the medium. Choosing the following potentials and values
for the definition of the micromorphic model allows to recover the AT2 model
%
%
%
\begin{subequations}
  \label{eq:ef_micromorphic:formulation:AT2_link}
  \begin{alignat}{3}
    g(\DamageField)
    &
    =
    (1 - \DamageField)^2
    \label{eq:ef_micromorphic:formulation:AT2_link:eq0}
    \\
    \psi_{\DamageField}(\DamageField)
    &
    =
    \Frac{G_{c}}{2 \, \ell_{c}} \, \DamageField
    \label{eq:ef_micromorphic:formulation:AT2_link:eq1}
    \\
    A_{\chi}
    &
    =
    G_{c} \, \ell_{c}
    \label{eq:ef_micromorphic:formulation:AT2_link:eq2}
    \\
    Y_{0}
    &
    =
    0
    \label{eq:ef_micromorphic:formulation:AT2_link:eq3}
  \end{alignat}
\end{subequations}

\subsection{Link with Lorentz model}

% \paragraph{The Lorentz model}

\paragraph{Values of Lorentz model}

AT1 and AT2 regularization models proposed by \cite{bourdin_implementation_2000}
introduce the characteristic length $\ell_c$, that controls the thickness of the smeared crack.
As stated in \cite{pham_approche_2010-1, pham_construction_2010}, this regularization length
defines the yield strength of the medium. In numerical applications, the strength of the material
is thus mesh dependant, since the value of characteristic length must
be greater than the element size in order to be captured.
Therefore, Lorentz \textit{et. al.} \cite{lorentz_gradient_2011,lorentz_convergence_2011}
alleviated this constraint by proposing an extension of the regularization proposed in
\cite{bourdin_implementation_2000}, that introduces
a state variable $\gamma$ related to the yield strength of the material. The following values for the
proposed micromorphic approach hence yield Lorentz gradient damage model
%
%
%
\begin{subequations}
  \label{eq:ef_micromorphic:formulation:Lorentz_link}
  \begin{alignat}{3}
    g(\DamageField)
    &
    =
    \paren{\Frac{1 - \DamageField}{1 + \gamma \, \DamageField}}^{2}
    \label{eq:ef_micromorphic:formulation:Lorentz_link:eq0}
    \\
    \psi_{\DamageField}(\DamageField)
    &
    =
    \Frac{3 \, G_{c}}{8 \, \ell_{c}} \, \DamageField
    \label{eq:ef_micromorphic:formulation:Lorentz_link:eq1}
    \\
    A_{\chi}
    &
    =
    \Frac{3}{4} \, G_{c} \, \ell_{c}
    \label{eq:ef_micromorphic:formulation:Lorentz_link:eq2}
    \\
    Y_{0}
    &
    =
    0
    \label{eq:ef_micromorphic:formulation:Lorentz_link:eq3}
  \end{alignat}
\end{subequations}
%
%
%
where $\gamma$ is the aforementioned parameter of the Lorentz' model.

\paragraph{Alternative choices regarding the role of the fracture energy}

Note that an alternative choice can be made for both AT1 and Lorentz' models such that
%
%
%
\begin{equation}
  \begin{aligned}
    \psi_{\DamageField}(\DamageField) = 0
    &&
    \text{and}
    &&
    Y_{0} = \Frac{3}{8} \, G_{c} \, \ell_{c}
  \end{aligned}
\end{equation}
%
%
%
While leading the same Lagrangian, this alternative choice has a totally
different physical meaning, since part of the fracture energy is now
considered as dissipated rather than stored. Since neither crack healing
nor coupling with heat transfer are considered, those choices are
equivalent in the context of this paper.
\section{Resolution schemes}
\label{sec:micromorphicdamage:alternate_minimisation}

\subsection{A first alternate minimization scheme}

\paragraph{Alternate scheme}

Observing that the Lagrangian \eqref{eq:micromorphicdamage:Lagrangian} is not convex \textit{globally}, but convex with respect to
each variable taken independently, an alternate minimization scheme is the spirit of that proposed by
Bourdin \textit{et al.} in \cite{bourdin_numerical_2000} is devised.
It consists in the following iterative scheme
%
%
%
\begin{equation}
  \left\{
    \begin{aligned}
      \iter{n+1}{\DisplacementField}
      &
      =
      \underset{\VirtualField{\DisplacementField}}{\argmin} \,
      \LagrangianOperator{\BodyLagrange}{\text{tot}}(
        \VirtualField{\DisplacementField}, \,
        \iter{n}{\DamageField}, \,
        \iter{n}{\MicromorphicDamageField}
      ),
      \\
      \iter{n+1}{\MicromorphicDamageField}
      &
      =
      \underset{\VirtualField{\MicromorphicDamageField}}{\argmin} \,
      \LagrangianOperator{\BodyLagrange}{\text{tot}}(
        \iter{n+1}{\DisplacementField}, \,
        \iter{n}{\DamageField}, \,
        \VirtualField{\MicromorphicDamageField}
      ),
      \\
      \iter{n+1}{\DamageField}
      &
      =
      \underset{\VirtualField{\DamageField}}{\argmin} \,
      \LagrangianOperator{\BodyLagrange}{\text{tot}}(
        \iter{n+1}{\DisplacementField}, \,
        \VirtualField{\DamageField}, \,
        \iter{n+1}{\MicromorphicDamageField}
      ),
    \end{aligned}
  \right.
\end{equation}
% \[
% \left\{
% \begin{aligned}
% \iter{n+1}{\vec{u}} &= \underset{\vec{u}^{\star}\in C.A.}{\argmin}\, \mathcal{L}\paren{\vec{u}^{\star},\iter{n}{d},\iter{n}{d_{\chi}}}\\
% \iter{n+1}{d_{\chi}} &= \argmin \,\mathcal{L}\paren{\iter{n+1}{\vec{u}},\iter{n}{d},d_{\chi}^{\star}} \\
% \iter{n+1}{d} &= \argmin \,\mathcal{L}\paren{\iter{n+1}{\vec{u}},d^{\star},\iter{n+1}{d_{\chi}}} \\
% \end{aligned}
% \right.
% \]
where $\iter{n}{\DisplacementField}$, $\iter{n}{\DamageField}$ and
$\iter{n}{\MicromorphicDamageField}$ denote respectively the estimates of the
displacement field, micromorphic damage field and damage field at the
n\textsuperscript{th} iteration of the algorithm.
In particular, each step of the algorithm diminishes the value of the
Lagrangian \cite{bourdin_numerical_2000},
ensuring the convergence of the scheme.
%
%
%
Minimization with respect to $\DisplacementField$ and $\MicromorphicDamageField$ leads to
solving the two standard Partial Differential Equations (PDE)
\eqref{eq:ef_micromorphic:formulation:strong_problem_u:eq2}
and
\eqref{eq:ef_micromorphic:formulation:strong_problem_d:eq2}.
%
%
%
% Note that:
% %
% %
% %
% \begin{itemize}
%   \item The PDE associated with $\vec{u}$ is a linear elastic problem with
%   spatially variable mechanical coefficients. This problem becomes non
%   linear if unilateral effects are taken into account.
%   \item The PDE associated with $d_{\chi}$ is always linear and
%   $\iter{n+1}{d_{\chi}}$ is the solution of (see Equation
%   \eqref{eq:micromorphicdamage:d_chi2}):
%   \[
%   -A_{\chi}\,\LaplacianOperator\,\iter{n+1}{d_{\chi}}+H_{\chi}\,\iter{n+1}{d_{\chi}}=
%   H_{\chi}\,\iter{n}{d}
%   \]
%   Note that this PDE is close to the PDE describing heat transfer and can
%   thus be solved in most finite element solvers. See \cite{azinpour_simple_2018}
%   for an alternative implementation of the phase-field model based on this analogy.
% \end{itemize}

\paragraph{Displacement problem}

Considering a linear thermo-elastic material for the definition of the potential
$\psi_{\tensoriis{F}}$,
the PDE associated with $\DisplacementField$ is a linear elastic problem with
spatially variable mechanical coefficients. This problem becomes non
linear if unilateral effects are taken into account.

\paragraph{Micromorphic damage problem}

The PDE associated with $\MicromorphicDamageField$ is always linear and
$\iter{n+1}{\MicromorphicDamageField}$ is the solution of (see Equation
\eqref{eq:micromorphicdamage:d_chi2})
%
%
%
\begin{equation}
  -A_{\chi} \, \LaplacianOperator \, \iter{n+1}{\MicromorphicDamageField}
  +
  H_{\chi} \, \iter{n+1}{\MicromorphicDamageField}
  =
  H_{\chi} \, \iter{n}{\DamageField},
\end{equation}
%
%
%
Note that this PDE is close to the PDE describing heat transfer and can
thus be solved in most finite element solvers. See \cite{azinpour_simple_2018}
for an alternative implementation of the phase-field model based on this analogy.

% - The PDE associated with $\vec{u}$ is a linear elastic problem with
%   spatially variable mechanical coefficients. This problem becomes non
%   linear if unilateral effects are taking into account.
% - The PDE associated with $d_{\chi}$ is always linear and
%   $\iter{n+1}{d_{\chi}}$ is the solution of (see Equation
%   \eqref{eq:micromorphicdamage:d_chi2}):
%   \[
%   -A_{\chi}\,\LaplacianOperator\,\iter{n+1}{d_{\chi}}+H_{\chi}\,\iter{n+1}{d_{\chi}}=
%   H_{\chi}\,\iter{n}{d}
%   \]
%   Note that this PDE is close to the PDE describing heat transfer and can
%   thus be solved in most finite element solvers. See \cite{azinpour_simple_2018}
%   for an alternative implementation of the phase-field model based on this analogy.

\paragraph{Damage problem}

Minimization with respect to the damage $\DamageField$ is local and depends on the considered model.

% Appendix \ref{sec:micromorphicdamage:damage_evolution} describes
% the local equations that must be solved in each case.

\subsection{A second alternate minimization scheme}

Since an update of the damage variable is computationally inexpensive,
compared to the computation of the displacement and micromorphic damage,
one may consider evaluating its value twice, as follows:
%
%
%
\begin{equation}
  \left\{
    \begin{aligned}
      \iter{n+1}{\DisplacementField}
      &
      =
      \underset{\VirtualField{\DisplacementField}}{\argmin} \,
      \LagrangianOperator{\BodyLagrange}{\text{tot}}(
        \VirtualField{\DisplacementField}, \,
        \iter{n}{\DamageField}, \,
        \iter{n}{\MicromorphicDamageField}
      ),
      \\
      \iter{n+1/2}{\DamageField}
      &
      =
      \underset{\VirtualField{\DamageField}}{\argmin} \,
      \LagrangianOperator{\BodyLagrange}{\text{tot}}(
        \iter{n+1}{\DisplacementField}, \,
        \VirtualField{\DamageField}, \,
        \iter{n}{\MicromorphicDamageField}
      ),
      \\
      \iter{n+1}{\MicromorphicDamageField}
      &
      =
      \underset{\VirtualField{\MicromorphicDamageField}}{\argmin} \,
      \LagrangianOperator{\BodyLagrange}{\text{tot}}(
        \iter{n+1}{\DisplacementField}, \,
        \iter{n+1/2}{\DamageField}, \,
        \VirtualField{\MicromorphicDamageField}
      ),
      \\
      \iter{n+1}{\DamageField}
      &
      =
      \underset{\VirtualField{\DamageField}}{\argmin} \,
      \LagrangianOperator{\BodyLagrange}{\text{tot}}(
        \iter{n+1}{\DisplacementField}, \,
        \VirtualField{\DamageField}, \,
        \iter{n+1}{\MicromorphicDamageField}
      ),
    \end{aligned}
  \right.
\end{equation}
%
%
%
where the damage estimate $\iter{n+1/2}{d}$ is given by an appropriate
modification of Equation \eqref{eq:micromorphicdamage:damage_estimate}.

\subsection{A third alternate minimization scheme}
\label{sec:micromorphic:third_scheme}

The third alternate minimization scheme is based on the fact that the
minimization with respect to $\DamageField$ and $\MicromorphicDamageField$ is convex

\paragraph{Dependant variables}

Since the micromorphic damage field $\MicromorphicDamageField$ is an implicit functions of
the damage field $\DamageField$ (See Equation \eqref{eq:micromorphicdamage:damage_estimate}), the
total derivative of the micromorphic thermodynamic force $\MicromorphicDamageForceField$
is given by
%
%
%
\begin{equation}
  \label{eq:ef_micromorphic:formulation:implict_damage}
  \begin{aligned}
    \derivtot{\MicromorphicDamageForceField}{\MicromorphicDamageField}
    =
    \deriv{\MicromorphicDamageForceField}{\MicromorphicDamageField}
    +
    \deriv{\MicromorphicDamageForceField}{\DamageField}
    \deriv{\DamageField}{\MicromorphicDamageField},
    % 0
    % \mathfrak{R}_{T}(\mathfrak{U}_{T}(\mathfrak{U}_{\dCell}),
    % \mathfrak{U}_{\dCell}) = 0
  \end{aligned}
\end{equation}
%
%
%
where the expression of the derivative of the damage field with respect to the micromorphic damage
field $\partial \DamageField / \partial \MicromorphicDamageField$ is deduced from
\eqref{eq:micromorphicdamage:damage_estimate}.
As $\partial \DamageField / \partial \MicromorphicDamageField$ is only linear
by part, Equation \eqref{eq:ef_micromorphic:formulation:strong_problem_d:eq2} is indeed non-linear.
% its resolution is performed in this work using a Newton algorithm.

\paragraph{Non-linear micromorphic damage and displacement resolution scheme}

Consequently, the following iterative scheme is considered where the evolution of $\MicromorphicDamageField$ is still given by Equation
\eqref{eq:ef_micromorphic:formulation:strong_problem_d:eq2} but the where the determination of the conjugated
force $\MicromorphicDamageForceField$ is given by Equation
\eqref{eq:ef_micromorphic:formulation:implict_damage}.
% In practice, non-linear Equation \eqref{eq:ef_micromorphic:formulation:implict_damage} can be
% solved by an iterative method.
%
%
%
\begin{equation}
  \left\{
    \begin{aligned}
      \iter{n+1}{\DisplacementField}
      &
      =
      \underset{\VirtualField{\DisplacementField}}{\argmin} \,
      \LagrangianOperator{\BodyLagrange}{\text{tot}}(
        \VirtualField{\DisplacementField},
        \iter{n}{\DamageField},
        \iter{n}{\MicromorphicDamageField}
      )
      \\
      \iter{n+1}{\MicromorphicDamageField},
      \iter{n+1}{\DamageField}
      &
      =
      \underset{
        \VirtualField{\MicromorphicDamageField},
        \VirtualField{\DamageField}
      }{\argmin} \,
      \LagrangianOperator{\BodyLagrange}{\text{tot}}(
        \iter{n+1}{\DisplacementField},
        \VirtualField{\DamageField},
        \VirtualField{\MicromorphicDamageField},
      )
    \end{aligned}
  \right.
\end{equation}
%
%
%
% where the evolution of $\MicromorphicDamageField$ is still given by Equation
% \eqref{eq:ef_micromorphic:formulation:strong_problem_d:eq2} but the determination of the conjugated
% force $\MicromorphicDamageForceField$ relies locally on the resolution of Equation
% \eqref{eq:micromorphicdamage:damage_estimate}.
% As this equation is only linear
% by part, Equation \eqref{eq:ef_micromorphic:formulation:strong_problem_d:eq2}is indeed non linear and
% its resolution is performed in this work using a Newton algorithm.
% %
% %
% %
% To be more specific, the computation of the stiffness matrix associated
% with Problem \eqref{eq:ef_micromorphic:formulation:strong_problem_d:eq2}
% requires the derivative
% $\deriv{\MicromorphicDamageForceField}{\DamageField}$ which is piece-wise constant.
% In
% our numerical experiments, this Newton algorithm usually converges in
% less than 10 iterations.
\section{Numerical implementations using standard finite elements}
\label{sec:micromorphicdamage:test_cases}

In this section, the proposed micromorphic apporoach to brittle fracture
is evaluated on numerical applications.
classical test cases taken from the literature.
The first three are classical test cases taken from the literature and have been performed using the
\texttt{mgis.fenics} implementation. Implementations are freely available on the following repository: \url{https://github.com/davidsiedel/h2o_paper}.
They aim at comparing both the classical phase field approach to the proposed micromorphic one.
The last two test cases are performed using the \texttt{mfem.fenics} implementation, and showcase respectively
the performance of the micromorphic approach with high order finite
elements in the case of shear driven fracture, and the scalability of the approach.

\paragraph{Hypotheses}

For all the applications in this section, a spectral decomposition of the thermo-elastic energy potential
$\psi_{\tensoriis{\varepsilon}}$ in the framework of
small deformations is considered, where $\tensorii{\varepsilon}$ denotes the linearized strain tensor.
Test cases in 2 dimension assume a plane strain hypothesis, and the framework of
small deformations is adopted.
The penalization parameter $H_{\chi}$ is chosen as
%
%
%
\begin{equation}
  H_{\chi} = \beta\,\Frac{G_{c}}{\ell_{c}}
\end{equation}
%
%
%
where $\beta$ is a normalized penalization parameter \cite{bharali_computational_2021}.

\paragraph{Choice of the convergence criterion of the staggered schemes}

In this paper, the staggered schemes are stopped when the damage becomes
stationnary, i.e. when the absolute difference between two estimates of
the damge is below a given threshold \(\varepsilon_{d}\) at each
integration point.
%
%
%
This criterion is not totally satisfying as it does not ensure that a
true minimum of the Lagrangian is found. Therefore, this second criterion has been carefully checked
for each steps of the numerical experiments described in
this section.

% \subsection{Comparison of the micromorphic solutions to the phase-field solutions on selected test cases  using \texttt{mgis.fenics}}

\subsection{Tensile test on a fiber reinforced matrix}

% The first test case of the following benchmark is taken from 

% ---------------------------------------------------------
% PARAGRAPH
% ---------------------------------------------------------
\paragraph{Specimen and loading}

The considered test case taken from \cite{bourdin_numerical_2000} describes a matrix represented by a square of length 1mm,
that is attached to a fiber in the middle (see Figure \ref{fig:micromorphic_formulation:matrix_fiber}).
The fiber is depicted by a hole
of radius 0.2mm located at the center of the specimen, and is considered to be infinitely stiff with
respect to the matrix.
A vertical displacement of amplitude 0.125mm is applied at the top of the specimen, and the
matrix is clamped around the fiber in both directions.
Damage is enforced to be null at the intersection between the matrix and the fiber, and natural
boundary conditions are considered elsewhere.

\begin{figure}[H]
    \centering
    \includegraphics[width=14.cm]{../chapter_003_ef_micromorphic/drawings/matrix_mesh.png}
    \caption{Geometry, boundary conditions and damage pattern}
    \label{fig:micromorphic_formulation:matrix_fiber}
\end{figure}

% ---------------------------------------------------------
% PARAGRAPH
% ---------------------------------------------------------
\paragraph{Material behaviour}

A linear thermo-elastic energy potential $\psi_{\tensoriis{\varepsilon}}$ is considered,
with a Young's modulus $E$ equal to $200$ MPa. The material displays a compressible
behaviour with a Poisson's ratio $\nu$ equal to $0.2$.
The chosen micromorphic model is that corresponding to a AT2 model, and is described by Equations \eqref{eq:ef_micromorphic:formulation:AT2_link},
where the fracture energy release rate $G_c$ is equal to $1$ J/mm${}^2$, and the characteristic length is $\ell_c = 0.02$ mm.
Moreover, the degradation function acts on the spherical part of free energy potential (See Section \ref{sec:micromorphic:deviatori_split})
% The penalization factor $\beta$ has been tests with several values.

\paragraph{Force delfection curve}

Figure \ref{fig:micromorphic_damage:force} displays the force-deflection curves
for the tensile test on a fiber reinforced matrix, using either the proposed micromorphic
approach, an AT2 phase field approach based on the resolution of the Karush–Kuhn–Tucker conditions
to enforce irreversibility of the damage field, or an AT2 phase field approach based on the definition
of Miehe's history function \cite{miehe_phase_2010}.
The third resolution scheme depicted in Section \ref{sec:micromorphic:third_scheme} is considered for the
micromorphic approach.
As illustrated by Figure \ref{fig:micromorphic_damage:force}, the
force-displacement curve of the third scheme is very similar to the
standard AT2 model based on the resolution of the variational inequality.

\begin{figure}[H]
  \centering
  \includegraphics[width=10.cm]{../chapter_003_ef_micromorphic/figures/FiberReinforcedMatrix-force.pdf}
  \caption{Evolution of the force as a function of the imposed displacement for
  the fiber reinforced matrix test for the third scheme with \(\beta=150\)
  and the standard phase-field schemes based on the resolution of the
  variational inegality or based on Miehe' history
  function \cite{miehe_phase_2010}}
  \label{fig:micromorphic_damage:force}
\end{figure}

\subsection{Shear test on a notched plate}

\paragraph{Specimen and loading}

The second test case describes a square plate length 1mm,
that is clamped at the bottom in the vertical direction (see Figure \ref{fig:micromorphic_formulation:shear_plate}).
A tangential displacement of amplitude 0.125mm is applied at the top of the specimen, and one of the vertices
of the plate is fixed to prevent rigid body motions in the tangential direction.

\begin{figure}[H]
    \centering
    \includegraphics[width=14.cm]{../chapter_003_ef_micromorphic/drawings/shear_plate.png}
    \caption{Geometry, boundary conditions and damage pattern}
    \label{fig:micromorphic_formulation:shear_plate}
\end{figure}

\paragraph{Material behaviour}

As for the precedent test case, a linear elastic energy potential $\psi_{\tensoriis{\varepsilon}}$ is considered,
with a Young's modulus $E=200$ MPa and a Poisson's ratio $\nu=0.2$.
The chosen micromorphic damage is also described by Equations \eqref{eq:ef_micromorphic:formulation:AT2_link},
with a fracture energy release rate $G_c=1$ J/mm${}^2$, and a characteristic length $\ell_c = 0.02$ mm.
The degradation function acts on the spherical part of free energy potential (See Section \ref{sec:micromorphic:deviatori_split})

\paragraph{Force delfection curve}

Figure \ref{fig:micromorphic_damage:beta} displays the force-deflection curve of the shear test, for the
proposed micromorphic approach based on the third resolution scheme described in Section \ref{sec:micromorphic:third_scheme} for several values
of the penalization parameter $\beta$.
As illustrated by Figure \ref{fig:micromorphic_damage:beta}, the
penalization factor plays a major role on the overall
force-deflection curve. Experiments show that a value of
$\beta = 150$ leads to undistinguishable results with those obtains using the AT2 model based
on the resolution of the variational inequality to enforce damage irreversibility.

\begin{figure}[H]
  \centering
  \includegraphics[width=10.cm]{../chapter_003_ef_micromorphic/figures/shear-force.pdf}
  \caption{Evolution of the force as a function of the imposed displacement for
  \(\beta=50\), \(\beta=100\), \(\beta=150\) for the shear test using the
  third staggered scheme}
  \label{fig:micromorphic_damage:beta}
\end{figure}

\paragraph{Number of iterations}

Figure \ref{fig:micromorphic_damage:shear:iterations} gives the number of iterations needed to achieve convergence of the fixed point algorithm
for the shear test, using the
proposed micromorphic approach based on the third resolution scheme, and for the AT2 phase field approach based on the resolution of the Karush–Kuhn–Tucker conditions
to enforce irreversibility of the damage field.
The number of iteration is roughly
similar between the standard AT2 model and the third scheme, although a
bit higher in general, as depicted on Figure
\ref{fig:micromorphic_damage:shear:iterations}.

\begin{figure}[H]
  \centering
  \includegraphics[width=10.cm]{../chapter_003_ef_micromorphic/figures/shear-iterations.pdf}
  \caption{Number of iterations of the fixed point algorithm for the shear test
  as a function of the step number for the standard AT2 model and the
  third scheme for \(\beta=150\) and
  \(\beta=300\)}
  \label{fig:micromorphic_damage:shear:iterations}
\end{figure}

% \paragraph{Coucou}

% However, an higher value of \(300\) is
% required to reproduce closely the evolution of the fracture energy.

\paragraph{Fracture energy}

Figure \ref{fig:micromorphic_damage:beta2} gives the evolution of the fracture energy as a function
of the imposed displacement, for the
proposed micromorphic approach and the third resolution scheme.
Values are compared to those obtained with the standard AT2 phase field model based the variational resolution of the evolution of damage.
A value of $\beta = 300$ is needed here to recover the energy dissipated by fracture for the AT2 model.

\begin{figure}[H]
  \centering
  \includegraphics[width=10.cm]{../chapter_003_ef_micromorphic/figures/DissipatedEnergies.pdf}
  \caption{Evolution of the fracture energy as a function of the imposed
  displacement for \(\beta=50\), \(\beta=100\), \(\beta=150\) for the
  shear test using the third staggered
  scheme}
  \label{fig:micromorphic_damage:beta2}
\end{figure}

\subsection{Shear driven fracture on a tensile test}

\paragraph{Spaciemn and loading}

The considered test case taken from \cite{alessi_phase-field_2020} describes a simple rectangular rod
of length $2$mm and width $1$mm.
A vertical displacement of amplitude $0.125$mm is applied at the top of the specimen, and the
bottom is clamped in the vertical direction.
One of the vertices
of the plate is fixed in the tangential direction to prevent rigid body motions.
Natural
boundary conditions are considered elsewhere.

\begin{figure}[H]
  \centering
  \includegraphics[width=8.cm]{../chapter_003_ef_micromorphic/drawings/alessi_mesh.png}
  \caption{Geometry, boundary conditions and defect zone for the shear driven fracture test case}
  \label{fig:micromorphic_damage:alessi_mesh}
\end{figure}

\paragraph{Material behaviour}

\textit{Alessi et al.} proposed a model to describe deviatoric driven fracture
using the following choice of the elastic free energy $\psi_{\tensoriis{\varepsilon}, \DamageField}$
%
%
%
\begin{equation}
  \psi_{\tensoriis{\varepsilon}, \DamageField}
  (\tensorii{\varepsilon}, \DamageField)
  =
  \frac{\mu}{2} \, g(\DamageField) \, \tensorii{\varepsilon}{}^{\text{dev}} : \tensorii{\varepsilon}{}^{\text{dev}}
  +
  \frac{K}{2} \, \tensorii{\varepsilon}{}^{\text{sph}} : \tensorii{\varepsilon}{}^{\text{sph}}
\end{equation}
%
%
%
where $K = 175$ GPa is the bulk modulus, $\mu = 80.77$ Gpa is the shear modulus, $\tensorii{\varepsilon}{}^{\text{dev}}$ is the
deviatoric part of the elastic strain, and $\tensorii{\varepsilon}{}^{\text{sph}}$ is the spherical
part.
The chosen micromorphic damage is described by Equations \eqref{eq:ef_micromorphic:formulation:AT2_link},
with a fracture energy release rate $G_c=2.7$ J/m2${}^2$, and a characteristic length $\ell_c = 0.025$ mm.
For damage initiation to localize at the center of the specimen, a degraded fracture energy release rate $G_c^{m}=0.99 G_c$
is imposed in a square of size $0.05$mm at the middle of the rod.

\paragraph{Quasi-incompressible behaviour}

As stated by \textit{Alessi et al.}, this model leads to quasi-incompressible
behaviour in highly damaged zones, and Lagrange elements are not able to
properly describe the damage localization band as well as the dissipated energy
by the crack propagation. They then showed that various
classical approaches (selective reduced integration and mixed
displacement/pressure formulation) can overcome this issue.
%
%
%
This test case is adapted to demonstrate that the
micromorphic approach can be used with higher order finite elements in order to alleviate this issue.
In the following, the same order of approximation are used to solve both the mechanical and
the micromorphic problems.

\paragraph{Mesh description}
The rod is discretized with triangle elements. The number of elements
used as a function of the finite element order is given in Table
\ref{tbl:micromorphicdamage:shear_driven_fracture_test_elements}. In
practice, the number of elements have little influence on the results
and the conclusions drawn in the next paragraph.
A very fine
mesh is used for low order finite elements (\(1\) and \(2\)) with
several dozen elements inside the damage band (See Figure
\ref{fig:micromorphicdamage:shear_driven_fracture_test_order1}). A coarser
mesh is used for higher order elements (\(4\) and \(6\)) with \(4\) to
\(6\) elements inside the damage band.
%
%
%
\begin{table}[H]
  \centering
  \begin{tabular}{||c c||} 
      \hline
      Resolution Method & Memory footprint
      \\
      [0.5ex] 
      \hline\hline
      Order 1 & \(102\,272\) elements
      \\ \hline
      Order 2 & \(409\,088\) elements
      \\ \hline
      Order 4 & \(25\,568\) elements 
      \\ \hline
      Order 6 & \(25\,568\) elements 
      \\ \hline
  \end{tabular}
  \caption{Geometry and material parameters for the shear driven fracture test}
  \label{tbl:micromorphicdamage:shear_driven_fracture_test_elements}
\end{table}

\paragraph{Damage pattern for low order elements}

Figure \ref{fig:micromorphicdamage:shear_driven_fracture_test_order1}
describes the damage pattern after the propagation of the crack. As
described by \textit{Alessi et al.}, volumetric locking leads to a spurious
damage localization band with excessive thickness, \textit{i.e.} a thickness
largely greater than the characteristic length \(\ell_{c}\). A very fine
mesh is used to demonstrate that the issue is not solved by mesh
refinement.

\begin{figure}[H]
  \centering
  \includegraphics[width=14.cm]{../chapter_003_ef_micromorphic/figures/shear-driven-fracture-damage-results-order-1.pdf}
  \caption{Spurious damage map obtained with linear elements (left). Zoom on the shear fracture (right)}
  \label{fig:micromorphicdamage:shear_driven_fracture_test_order1}
\end{figure}

\paragraph{Damage pattern for high order elements}

Figure \ref{fig:micromorphicdamage:shear_driven_fracture_test_higher_order}
show the results obtained with higher order elements. While the
simulation with quadratic elements still exhibit a spurious damage
localisation band, similar to the one observed with linear elements in
Figure \ref{fig:micromorphicdamage:shear_driven_fracture_test_order1}, higher
order elements lead to satisfying results, i.e. higher order elements
alleviate issues related to volumetric locking.

\begin{figure}[H]
  \centering
  \includegraphics[width=14.cm]{../chapter_003_ef_micromorphic/figures/shear-driven-fracture-damage-results-higher-orders.pdf}
  \caption{Damage map for higher order elements. Quadradic elements (left), fourth order elements (center), sixth order elements (right). The quadratic mesh is too fine to be shown without hiding the results}
  \label{fig:micromorphicdamage:shear_driven_fracture_test_higher_order}
\end{figure}

\paragraph{Force deflection curves}

Figure \ref{fig:micromorphicdamage:traction_curve} presents the
force/displacement curves as a function of the finite element order.
Quadratic results, which are intermediate between linear and quadratic
results, are not reproduced for the sake of clarity.

\begin{figure}[H]
  \centering
  \includegraphics[width=10.cm]{../chapter_003_ef_micromorphic/figures/shear-driven-fracture-damage-results-force.pdf}
  \caption{Traction curve}
  \label{fig:micromorphicdamage:traction_curve}
\end{figure}

\paragraph{Observations}

All order of approximations give similar results up to the crack propagation.
The traction curve given by linear elements exhibits a residual
stiffness and a spurious dissipation after the crack propagation.
Fourth order and sixth order give undistinguishable results. In both
cases, the force drops to zero after the crack propagation.

\subsection{Fragmentation of a nuclear fuel pellet}

\paragraph{Spaciemn and loading}

This test case describes the fragmentation of a nuclear fuel pellet during the
reactor start-up.
The pellet is a cylinder of width $8.17$mm and height $13.4$mm.
The top and bottom surfaces are recessed by a dishing of diameter $6.1$mm and height $0.32$mm.
The pellet is fixed at one point to prevent rigid body motions.
Assuming a constant thermal conductivity, a temperature load $T(r, t)$ is imposed in the pellet such that
%
%
%
\begin{equation}
  T(r, t) = (T_{\mathrm{c}} - T_{\mathrm{o}}) (t) \, (1-r^{2}) + T_{\mathrm{o}}
\end{equation}
%
%
%
where $T_{c} = 1500$K is the core temperature, $T_{o} = 600$K is the outer surface temperature,
and $r$ is the radial component such that $r=0$ on the symmetry axis of the pellet, and $t$ is a
pseudo-time loading parameter in the range \([0,1]\).
%
%
%

% The considered test case taken from \cite{alessi_phase-field_2020} describes a simple rectangular rod
% of length $2$mm and width $1$mm.
% A vertical displacement of amplitude $0.125$mm is applied at the top of the specimen, and the
% bottom is clamped in the vertical direction.
% One of the vertices
% of the plate is fixed in the tangential direction to prevent rigid body motions.
% Natural
% boundary conditions are considered elsewhere.

\begin{figure}[H]
  \centering
  \includegraphics[width=10.cm]{../chapter_003_ef_micromorphic/drawings/pellet_mesh.png}
  \caption{Geometry, boundary conditions and defect zone for the pellet}
  \label{fig:micromorphic_damage:pellet_mesh}
\end{figure}

\paragraph{Material behaviour}

A linear thermo-elastic free energy potential is considered, such that
%
%
%
\begin{equation}
  \label{eq:micromorphicdamage:freeenergygel_2}
  \psi_{\tensoriis{\varepsilon}, \DamageField}
  (\tensorii{\varepsilon}, \DamageField, T)
  =
  g(\DamageField) \, \psi_{\tensoriis{\varepsilon}}^{\text{th}}(\tensorii{\varepsilon}, T)
\end{equation}
%
%
%
where the thermo-elastic potential $\psi_{\tensoriis{\varepsilon}}^{\text{th}}$ is such that
%
%
%
\begin{equation}
  \psi_{\tensoriis{\varepsilon}}^{\text{th}}(\tensorii{\varepsilon}, T)
  =
  (\tensorii{\varepsilon} - \alpha \, (T - T_{\text{ref}}) \, \tensorii{I} )
  :
  \tensoriv{C}
  :
  (\tensorii{\varepsilon} - \alpha \, (T - T_{\text{ref}}) \, \tensorii{I} )
\end{equation}
%
%
%
with $\tensoriv{C}$ the Hooke tensor defined by the Young modulus $E = 150$GPa and the Poisson ratio
$\nu = 0.3$, the thermal expansion reference temperature $T_{\text{ref}} = 273$K and the linear mean
thermal expansion coefficient $\alpha  = 1 \, \cdot\,10^{-5}$K${}^{-1}$.
%
%
%
The chosen micromorphic damage is described by Equations \eqref{eq:ef_micromorphic:formulation:AT2_link},
with a fracture energy release rate $G_c=2.7$ J/m2${}^2$, and a characteristic length $\ell_c = 0.025$ mm.
For damage initiation to localize anywhere in the specimen, a random uniform distribution of the
fracture energy release rate with a mean value $G_c^{\text{mean}}=G_c$ and a standard deviation $G_c^{\text{std}}=0.2$
is imposed in the pellet.

\paragraph{Large scale computation}

Damage patterns for the cracking of the fuel pellet are given in Figure \ref{fig:micromorphic_damage:pellet_cracked}.
The mesh is composed of \(33\,230\,848\) triangles, \(132\,923\,392\) nodes, leading to a mechanical problem with
\(22.10^6\) degrees of freedom. This large scale test case assesses the adaptivity of the micromorphic approach to industrial
applications. Using the micromorphic approach, the irreversibility condition for the damage field is dealt with at quadrature points,
and is thus fully parallel, hence avoiding the need to deploy the \(132\,923\,392\) Lagrange multipliers that would have been necessary
for the AT2 model based on the variational treatment of the irreversibility condition.
%
%
%
\begin{figure}[H]
  \centering
  \includegraphics[width=14.cm]{../chapter_003_ef_micromorphic/figures/FuelPelletCracking-results.pdf}
  \caption{Crack pattern}
  \label{fig:micromorphic_damage:pellet_cracked}
\end{figure}
\section{Conclusions and perspectives}

This work has investigated the use of micromorphic behaviours for the
description of quasi-brittle materials and has shown that those
micromorphic behaviours can be considered as varitionaly consistent
approximations of standard phase-field models.
%
%
%
Three alternate minimization schemes, which are straightforward to
implement in standard FEM or FFT solvers, have been proposed, and their numerical performance has been investigated intensively using representative tests. 
%
%
%
Convergence of those schemes is guaranteed but requires a large number
of fixed-point iterations. Regarding this observation, the third
scheme appears to be more efficient.

The proposed approach can be extended to more complex damage behaviours
and ductile failure.
%
%
%
Finally, as a future work, acceleration schemes could also be
investigated to reduce the number of fixed-point iterations.


\appendix

\chapter{Notations}
\input{../chapter_002_hho_mechanics/sections/06_notations.tex}

\chapter{A Hu-Washizu Formulation for HDG methods}
\input{../chapter_002_hho_mechanics/sections/07_appendix_hu_washizu.tex}

\chapter{Implementation}
% ---------------------------------------------------------
% ---- SECTION
% demander nicolas pr les questions supplementaires
% ---------------------------------------------------------
\section{Implementation}
\label{sec_appendix_implementation}

This section specifies implementation aspects regarding HDG methods, and in particular those defining the HHO method.

% ---------------------------------------------------------
% -- SUBSECTION
% ---------------------------------------------------------
\subsection{Shape functions}
\label{sec_shape_functions}

Since the displacement is discontinuous, the usual Lagrange basis functions are not necessarily needed for the description of the discrete displacement, gradient ans stress fields. A natural choice amounts to choose monomial basis functions.

% ---------------------------------------------------------
% PARAGRAPH
% ---------------------------------------------------------
\paragraph{Monomial basis functions}

Let $\mathcal{M}_s^l$ the set of natural integer vectors in a $s$-dimensional euclidean space such that
%
%
%
\begin{equation}
    \begin{aligned}
        \mathcal{M}_s^l =
        % \{
        \bigg\{
            \mathcal{M}_{s,p} && \vert && 0 \leq p \leq l
        \bigg\}
        &&
        \text{with}
        &&
        \mathcal{M}_{s,p} =
        \bigg\{
            \tensori{\alpha}{}_m && \vert && \sum_{1 \leq j \leq s} \alpha_{mj} = p
        \bigg\}
        % \}
    \end{aligned}
\end{equation}
%
%
%
and denote $M_s^l$ the cardinality of $\mathcal{M}_s^l$.
Let $D$ some $s$-dimensional euclidean domain, $1 \leq s \leq 3$, and $M^l(D)$ the monomial basis function of order $l$ on $D$.
The value of a scalar polynomial field $a_{M}^l \in M^l(D)$ at some point $\tensori{X} \in D$ is given by
%
%
%
\begin{equation}
    \label{eq_basis_fun3}
    \begin{aligned}
        a_{M}^l(\tensori{X}) = \sum_{0 \leq p \leq M_s^l} a_{p} \prod_{1 \leq j \leq s} \frac{(X_j - X_{Dj})^{\alpha_{pj}}}{h_D}
        &&
        \text{with}
        &&
        \tensori{\alpha}{}_p \in \mathcal{M}_{s}^l
    \end{aligned}
\end{equation}
%
%
%
where $\tensori{X}{}_D$ denotes the centroid of $D$, and the coefficients $a_p$ in $M^l(D)$ form a vector $\mathfrak{A}_D^l$ of size $M_s^l$.

% ---------------------------------------------------------
% PARAGRAPH
% ---------------------------------------------------------
\paragraph{Cell and faces approximation space sizes}

The number of degrees of freedom for a scalar field depends on the polynomial order. An overview of the values taken using monomial shape functions is given in tables \ref{table_num_dofs_2D} and \ref{table_num_dofs_3D} for both a cell and a face up to an approximation of order $4$
%
%
%
\begin{table}[H]
\centering
\begin{tabular}{||c c c||} 
    \hline
    polynomial order & cell dofs & face dofs \\ [0.5ex] 
    \hline\hline
    $0$ & 1 & 1
    \\ \hline
    $1$ & 3 & 2
    \\ \hline
    $2$ & 6 & 3
    \\ \hline
    $3$ & 10 & 4
    \\ \hline
    $4$ & 15 & 5
    \\ \hline
\end{tabular}
\caption{Number of cell and faces degrees of freedom for a scalar unknown in two dimension}
\label{table_num_dofs_2D}
\end{table}
%
%
%
\begin{table}[H]
\centering
\begin{tabular}{||c c c||} 
    \hline
    polynomial order & cell dofs & face dofs \\ [0.5ex] 
    \hline\hline
    $0$ & 1 & 1
    \\ \hline
    $1$ & 4 & 3
    \\ \hline
    $2$ & 10 & 6
    \\ \hline
    $3$ & 19 & 10
    \\ \hline
    $4$ & 35 & 15
    \\ \hline
\end{tabular}
\caption{Number of cell and faces degrees of freedom for a scalar unknown in three dimension}
\label{table_num_dofs_3D}
\end{table}
%
%
%
The need to eliminate cell unknowns justifies by observing that the number of degrees of freedom in cells grows rapidly with the approximation order as compared to that in faces.

% ---------------------------------------------------------
% -- SUBSECTION
% ---------------------------------------------------------
\subsection{Stabilization}
\label{sec_stabilization}

In this section, approximation spaces for unknowns of the global problem are described, which leads to several choices in terms of definition of the jump function. Depending on such a choice, one recovers either the HDG method as defined by \cite{lehrenfeld_hdg_2010}, or the HHO \cite{di_pietro_hybrid_2015} one.

% ---------------------------------------------------------
% PARAGRAPH
% ---------------------------------------------------------
\paragraph{Discrete functional space}

Discrete spaces are chosen such that
%
%
%
\begin{equation*}
    \begin{aligned}
        \discreteDisplacementSpaceCell = P^l(\cell, \mathbb{R}^{d})
        &&
        \discreteDisplacementSpaceDCell = P^k(\dCell, \mathbb{R}^{d})
        &&
        \discreteGradSpaceCell = P^k(\cell, \mathbb{R}^{d \times d})
        &&
        \discreteStressSpaceCell = P^k(\cell, \mathbb{R}^{d \times d})
    \end{aligned}
\end{equation*}
%
%
%
where $P^a(D, \mathbb{R}^{m})$ denotes the space of $m-$ valued polynomials in $D$, spanned by the monomial basis $M^a(D)$. In particular, one notices that the cell displacement polynomial order $l$ might be chosen different from the face displacement order $k$ provided $k - 1 \leq l \leq k + 1$.

% ---------------------------------------------------------
% PARAGRAPH
% ---------------------------------------------------------
\paragraph{HDG jump function}

Accounting for the possible different polynomial order between the cell and faces, one can specify a discrete jump function (or stabilization) in a straightforward way such that it delivers the displacement difference point-wise for any displacement pair $(\tensori{v}{}_{\cell}^l, \tensori{v}{}_{\dCell}^k) \in U^h(\cell) \times V^h(\dCell)$
%
%
%
\begin{equation}
    \begin{aligned}
        \tensori{J}{}_{\dCell}^{HDG}(\tensori{v}{}_{\cell}^l, \tensori{v}{}_{\dCell}^k) = \Pi^k_{\dCell{}} (
            \tensori{v}{}_{\dCell}^k - \tensori{v}{}_{\cell}^l \vert_{\dCell}
        )
    \end{aligned}
\end{equation}
%
%
%
where $\Pi^k_{\dCell{}}$ denotes the orthogonal projector onto $\discreteDisplacementSpaceDCell$.
This discrete jump function is at the origin of Hybrid Discontinuous Galerkin methods \cite{lehrenfeld_hdg_2010}, and grants a convergence of order $k$ in the energy norm.

% ---------------------------------------------------------
% PARAGRAPH
% ---------------------------------------------------------
\paragraph{HHO jump function}

A richer discrete jump function $\tensori{J}{}_{\dCell}^{HHO}$ providing a convergence of order $k + 1$ in the energy norm was introduced in \cite{di_pietro_discontinuous-skeletal_2015}, hence giving the Hybrid High Order method its name, such that
%
%
%
\begin{equation}
    \label{eq_hho_stabilization_vector}
    \begin{aligned}
        \tensori{J}{}_{\dCell}^{HHO}(\tensori{v}{}_{\cell}^l, \tensori{v}{}_{\dCell}^k) = \Pi^k_{\dCell{}} (
            \tensori{v}{}_{\dCell}^k - \tensori{v}{}_{\cell}^l \vert_{\dCell}
            -
            (
                (\tensoro{I}{}_{\cell}^{k + 1} - \Pi_{\cell}^k) (
                    \tensori{w}{}_\cell^{k + 1}
                )
            ) \vert_{\dCell{}}
        )
    \end{aligned}
\end{equation}
%
%
%
where $\Pi_{\cell}^k$ is the projector onto $P^{k}(\cell, \mathbb{R}^{d})$, $\tensoro{I}{}_{\cell}^{k + 1}$ is the identity function in $\discretePotentialSpaceCell = P^{k + 1}(\cell, \mathbb{R}^{d})$.

% ---------------------------------------------------------
% PARAGRAPH
% ---------------------------------------------------------
\paragraph{Reconstructed higher order displacement}

The term $\tensori{w}{}_{\cell}^{k+1}$ in \eqref{eq_hho_stabilization_vector}
% $ \in \discretePotentialSpaceCell$
denotes a higher order discrete displacement in $\discretePotentialSpaceCell$
that solves for a given displacement pair $(\tensori{v}{}_{\cell}^l, \tensori{v}{}_{\dCell}^k) \in U^h(\cell) \times V^h(\dCell)$
%
%
%
\begin{subequations}
    \label{eq_potential}
        \begin{alignat}{3}
            \int_\cell \nabla \tensori{w}{}_{\cell}^{k+1} : \nabla \tensori{d}{}_{\cell}^{k+1}
            & =
            \int_\cell \nabla \tensori{v}{}_{\cell}^l : \nabla \tensori{d}{}_{\cell}^{k+1}
            +
            \int_{\dCell} (\tensori{v}{}_{\dCell}^k - \tensori{v}{}_{\cell}^l) \cdot \nabla \tensori{d}{}_{\cell}^{k+1} \cdot \tensori{n}{}
            \ \ \ \ \ \ \ \ 
            &&
            \forall \tensori{d}{}_{\cell}^{k+1} \in \discretePotentialSpaceCell
            \label{eq_potential:eq0}
            \\
            \int_\cell \tensori{w}{}_{\cell}^{k+1} & = \int_\cell \tensori{v}{}_{\cell}^{l}
            \label{eq_potential:eq1}
    \end{alignat}
\end{subequations}

% ---------------------------------------------------------
% -- SUBSECTION
% ---------------------------------------------------------
\subsection{Algebraic formulation}
\label{sec_appendix_implementation2}

% ---------------------------------------------------------
% PARAGRAPH
% ---------------------------------------------------------
\paragraph{Unknown vector}

A cell displacement component unknown vector is represented by the coefficient vector $\mathfrak{U}_{\cell}^l$ in $M^l(\cell)$, and a face displacement component unknown vector is represented by the coefficient vector $\mathfrak{U}_{F}^k$ in $M^k(F)$, for any $F \subset \dCell$.
The global element displacement unknown vector of size $d M^l_d + d N_{\dCell} M^k_{d - 1}$ is
denoted $\mathfrak{U}_{\ClosedCell{}}$ and is the collection of all cell and faces displacement component vectors.
In the following, the cell coefficients in $\mathfrak{U}_{\ClosedCell{}}$ are denoted $\mathfrak{U}_{\cell}$, and the boundary coefficients $\mathfrak{U}_{\dCell}$.

% ---------------------------------------------------------
% PARAGRAPH
% ---------------------------------------------------------
\paragraph{Quadrature}

Integrals are evaluated numerically by means of a quadrature rule. Hence, let ${Q}_D$ a quadrature rule for the domain $D$ of order at least $2k$. A quadrature point is denoted $\tensori{X}{}_q$ and a quadrature weight $w_q$.

% ---------------------------------------------------------
% PARAGRAPH
% ---------------------------------------------------------
\paragraph{Reconstructed gradient operator}

From an algebraic standpoint, \eqref{eq_grad} defines a linear problem
consisting in inverting a mass matrix in $\discreteGradSpaceCell{}$. One can thus define 
$
{\mathbb{B}}{}_{\cell}
$
the discrete gradient operator acting on the displacement unknowns vector $\mathfrak{U}{}_{\ClosedCell}$ at a quadrature point $\tensori{X}{}_q \in \cellQuadrature$, and ${\mathfrak{G}}{}_{\cell}^k$ the vector representation of the reconstructed gradient $\tensorii{G}{}_{\cell}^k(\tensori{v}{}_{\cell}^l, \tensori{v}{}_{\dCell}^k)$ such that
%
%
%
\begin{equation}
    \label{eq_discrete_gradient_vector}
    \begin{aligned}
        {\mathfrak{G}}{}_{\cell}^k
        (\tensori{X}{}_q)
        =
        {\mathbb{B}}{}_{\cell}
        (\tensori{X}{}_q)
        \cdot
        {\mathfrak{U}}{}_{\ClosedCell}
    \end{aligned}
\end{equation}
%
%
%
where ${\mathbb{B}}{}_{\cell}$ is composed by a cell block and a boundary block of respective size $dM_d^l$ and $dM_{d - 1}^k$.

% ---------------------------------------------------------
% PARAGRAPH
% ---------------------------------------------------------
\paragraph{Stabilization operator}

Similarly, the algebraic realization of \eqref{eq_hho_stabilization_vector} amounts to compute the stabilization operator ${\mathfrak{J}}{}_{\cell}$ such that 
%
%
%
\begin{equation}
    \label{eq_discrete_stabilization_vector}
    \begin{aligned}
        % {\mathcal{J}}{}_{\dCell}^{HHO}
        {\mathfrak{J}}{}_{\dCell}^{HHO}
        =
        {\mathbb{J}}{}_{\cell}
        \cdot
        {\mathfrak{U}}{}_{\ClosedCell}
    \end{aligned}
\end{equation}
%
%
%
where, as for ${\mathbb{B}}{}_{\cell}$, the operator ${\mathbb{J}}{}_{\cell}$ is composed by a cell and a boundary block with same respective sizes.

% ---------------------------------------------------------
% PARAGRAPH
% ---------------------------------------------------------
\paragraph{Offline computation}

Since \eqref{eq_grad} and \eqref{eq_hho_stabilization_vector} depend on the geometry of the element $\cell$ only, one can compute the operators $\mathbb{B}{}_{\cell}$ and $\mathbb{Z}{}_{\cell}$ for each element once and for all in an offline pre-computation step by working in the reference configuration. Once this offline step is performed, the algebraic form of the problem resembles closely to the standard finite element one, where the operator $\mathbb{B}{}_{\cell}$ replaces the usual shape function gradient operator, and the stabilization operator 
$\mathbb{Z}{}_{\cell}$ is incorporated in the expression of the tangent matrix and in that of internal forces.

% ---------------------------------------------------------
% PARAGRAPH
% ---------------------------------------------------------
\paragraph{Internal forces} The internal forces vector $\mathfrak{F}_{\ClosedCell}^{int}$ depends on the product of the stress values with the gradient operator computed at quadrature points, plus a supplementary force corresponding to that of the traction between the cell and its boundary
%
%
%
\begin{equation}
    \label{eq_internal_forces}
    \begin{aligned}
    \mathfrak{F}_{\ClosedCell}^{int}
    % (\mathfrak{U}{}_{\ClosedCell})
    = &
    \sum_{\tensori{X}{}_q \in \cellQuadrature{}}
    (w_q
    {\mathbb{B}}{}_{\cell}^{t}(\tensori{X}{}_q) \cdot
    {\mathfrak{P}}{}_{\cell}^k(\tensori{X}{}_q, \mathfrak{U}{}_{\ClosedCell})
    )
    +
    \frac{\beta}{h_T}
    {\mathbb{J}}{}_{\cell}^t
    \cdot
    {\mathbb{J}}{}_{\cell}
    \cdot
    {\mathfrak{U}}{}_{\ClosedCell}
    \end{aligned}
\end{equation}
%
%
%
where ${\mathfrak{P}}{}_{\cell}^k$ denotes the vector representation of $\tensorii{P}{}_{\cell}^k$, and the superscript $\{\cdot\}^t$ denotes the transposition operator.

% ---------------------------------------------------------
% PARAGRAPH
% ---------------------------------------------------------
\paragraph{External forces}

The external forces vector $\mathfrak{F}_{\ClosedCell}^{ext}$ is the evaluation of the given bulk and boundary loads at respective cell and face quadrature points tested against the respective cell and face shape functions, such that
%
%
%
\begin{equation}
    \label{eq_external_forces}
    \begin{aligned}
        \mathfrak{F}_{\cell}^{ext}
        =
        \sum_{\tensori{X}{}_q \in \cellQuadrature{}}
        (w_q
        \loadLag{}(\tensori{X}{}_q) \cdot
        {\mathfrak{U}}{}_{\cell}^l
        )
        &&
        \text{and}
        &&
        \mathfrak{F}_{\dCell}^{ext}
        =
        \sum_{\tensori{X}{}_q \in Q_{\dCell}}
        (w_q
        \neumannLag{}(\tensori{X}{}_q) \cdot
        {\mathfrak{U}}{}_{\dCell}^k
        )
    \end{aligned}
\end{equation}

% ---------------------------------------------------------
% PARAGRAPH
% ---------------------------------------------------------
\paragraph{Tangent matrix and resiudal}

The elementary residual vector $\mathfrak{R}_{\ClosedCell}$ is such
that $\mathfrak{R}_{\ClosedCell} =
\mathfrak{F}_{\ClosedCell}^{int} -
\mathfrak{F}_{\ClosedCell}^{ext}$. The tangent matrix
$\mathbb{K}_{\ClosedCell}$ expresses the
derivative of $\mathfrak{R}_{\ClosedCell}$
with respect to $\mathfrak{U}{}_{\ClosedCell}$, and writes such that
%
%
%
\begin{equation}
  \label{eq_stiffness_operator}
  \begin{aligned}
    \mathbb{K}_{\ClosedCell}
    = \sum_{\tensori{X}{}_q \in \cellQuadrature{}} (w_q
    {\mathbb{B}}{}_{\cell}^{t}(\tensori{X}{}_q) \cdot
    \mathbb{A}(\tensori{X}{}_q, \mathfrak{U}{}_{\ClosedCell}) \cdot
    {\mathbb{B}}{}_{\cell}(\tensori{X}{}_q) ) + \frac{\beta}{h_T}
    {\mathbb{J}}{}_{\cell}^t \cdot {\mathbb{J}}{}_{\cell}
  \end{aligned}
\end{equation}
%
%
%
where $\mathbb{A}$ is the matrix representation of the
fourth-order tensor $\tensoriv{A}{} = \partial \tensorii{P}{}_{\cell}^k
/ \partial \tensorii{G}{}_{\cell}^k$. The matrix
$\mathbb{K}_{\ClosedCell}$ is block
decomposable such that
%
%
%
\begin{equation}
  \label{eq_stiffness_operator_blocks}
  \begin{aligned}
    \mathbb{K}_{\ClosedCell}
    =
    \begin{pmatrix}
      \mathbb{K}_{\cell
        \cell} && \mathbb{K}_{\cell
        \dCell} \\
      \mathbb{K}_{\dCell
        \cell} && \mathbb{K}_{\dCell
        \dCell}
    \end{pmatrix}
  \end{aligned}
\end{equation}
%
%
%
which leads to the following algebraic expression of
both~\eqref{eq_cell_equilibrium_3}
and~\eqref{eq_static_condensation_final}~:
\begin{equation}
  \label{eq_condensation_matrix}
  \begin{aligned}
    \frac{d \mathfrak{R}_{\mathcal{F}}}{d
      \mathfrak{U}_{\mathcal{F}}} = \frac{d
      \mathfrak{R}_{\mathcal{F}}^c}{d \mathfrak{U}_{\mathcal{F}}} =
    \mathbb{K}_{\cell \dCell} -
    \mathbb{K}_{\dCell \cell}
    \mathbb{K}_{\cell \cell}^{-1}
    \mathbb{K}_{\cell \dCell}
  \end{aligned}
\end{equation}

% ---------------------------------------------------------
% -- SUBSECTION
% ---------------------------------------------------------
\subsection{Operators in the axi-symmetric framework}
\label{sec_appendix_axi}

This part specifies the formulation of HHO operators in the axi-symmetric framework.

% ---------------------------------------------------------
% PARAGRAPH
% ---------------------------------------------------------
\paragraph{Reconstructed gradient}

For any displacement pair $(\tensori{v}{}_{\cell}^l, \tensori{v}{}_{\dCell}^k) \in \discreteDisplacementSpaceCell{} \times \discreteDisplacementSpaceDCell{}$, the component $\tensoro{G}{}_{\cell \theta \theta}(\tensoro{v}{}_{\cell r}, \tensoro{v}{}_{\dCell r})$ solves
%
%
%
\begin{equation}
    \label{axi_symmetric_gradient_theta}
    \begin{aligned}
        \int_{\cell} 2 \pi r \tensoro{G}{}_{\cell \theta \theta}(\tensoro{v}{}_{\cell r}, \tensoro{v}{}_{\dCell r}) \tensoro{\tau}{}_{\cell \theta \theta}
        =
        \int_{\cell} 2 \pi r \frac{\tensoro{u}{}_{\cell r}}{r} \tensoro{\tau}{}_{\cell \theta \theta}
        =
        \int_{\cell} 2 \pi \tensoro{u}{}_{\cell r} \tensoro{\tau}{}_{\cell \theta \theta}
        &&
        \forall \tensorii{\tau}{}_{\cell} \in \stressSpaceCell
    \end{aligned}
\end{equation}
%
%
%
In the radial and ordonal directions, \textit{i.e.} $\forall i, j \in \{ r,z \}$, the expression given in \eqref{eq_grad} is retrieved, and the component $G_{\cell ij}(\tensoro{v}{}_{\cell i}, \tensoro{v}{}_{\dCell i})$ solves
%
%
%
\begin{equation}
    \label{axi_symmetric_gradient_rz}
    \begin{aligned}
    \int_{\cell} 2 \pi r G_{\cell ij}(\tensoro{v}{}_{\cell i}, \tensoro{v}{}_{\dCell i}) \tau_{\cell ij} =
    \int_{\cell} 2 \pi r \frac{\partial \tensoro{u}{}_{\cell i}}{\partial j} \tau_{ij} -
    \int_{\dCell} 2 \pi r (u_{\dCell i} - u_{\cell i} \vert_{\dCell}) \tau_{\cell ij} \vert_{\dCell} n_{j}
    &&
    \forall \tensorii{\tau}{}_{\cell} \in \stressSpaceCell
    \end{aligned}
\end{equation}

% ---------------------------------------------------------
% PARAGRAPH
% ---------------------------------------------------------
\paragraph{Reconstructed higher order displacement}

For any $\tensori{d}{}_{\cell}^{k + 1} \in \discretePotentialSpaceCell$, the radial component $w^{k+1}_{\cell r}$ solves
%
%
%
\begin{equation}
    \label{axi_symmetric_potential_r}
    \begin{aligned}
        \int_{\cell} 2 \pi r (\sum_{i \in \{ r,z \}} \frac{\partial w^{k+1}_{\cell r}}{\partial i} \frac{\partial d^{k+1}_{\cell r}}{\partial i} + \frac{w^{k+1}_{\cell r}}{r} \frac{d^{k+1}_{\cell r}}{r})
        = &
        \int_{\cell} 2 \pi r (\sum_{i \in \{ r,z \}} \frac{\partial u_{\cell r}}{\partial i} \frac{\partial d^{k+1}_{\cell r}}{\partial i} + \frac{u_{\cell r}}{r} \frac{d^{k+1}_{\cell r}}{r})
        \\
        &
        +
        \int_{\dCell} 2 \pi r \sum_{i \in \{ r,z \}} (u_{\dCell r} - u_{\cell r} \vert_{\dCell}) \frac{\partial d^{k+1}_{\cell r}}{\partial i} \vert_{\dCell} n_{i}
    \end{aligned}
\end{equation}
%
%
%
where the mean value condition is not needed on the radial component of the higher order displacement since the left hand side of the system described by \eqref{axi_symmetric_potential_r} depends directly on the displacement unknown and not only on its gradient as in \eqref{axi_symmetric_potential_z}.
The ordinate component $w^{k+1}_{\cell z}$ solves :
%
%
%
\begin{subequations}
    \label{axi_symmetric_potential_z}
        \begin{alignat}{3}
            \int_{\cell} 2 \pi r \sum_{i \in \{ r,z \}}
            \frac{\partial w^{k+1}_{\cell z}}{\partial i} \frac{\partial d^{k+1}_{\cell z}}{\partial i}
            = &
            \int_{\cell} 2 \pi r \sum_{i \in \{ r,z \}} \frac{\partial u_{\cell z}}{\partial i} \frac{\partial d^{k+1}_{\cell z}}{\partial i}
            -
            \int_{\dCell} 2 \pi r \sum_{i \in \{ r,z \}} (u_{\dCell z} - u_{\cell z} \vert_{\dCell})
            \frac{\partial d^{k+1}_{\cell z}}{\partial i} \vert_{\dCell} n_{i}
            \\
            \int_{\cell} 2 \pi r w^{k+1}_{\cell z} = & \int_{\cell} 2 \pi r u_{\cell z}
        \end{alignat}
\end{subequations}

\paragraph{Axis faces treatment}

In cylindrical coordinates, all integrals depend on the radial component $r$, and boundary integrals vanish at $r = 0$ on the symmetry axis.
As a consequence, some information on the boundary displacement is lost in the surface integral term of both the reconstructed gradient and the stabilization of a cell $\cell$ located on the symmetry axis.
% do not depend on the closed surface integral wrapping a cell $\cell$ located on the symmetry axis.
% Therefore, the reconstructed gradient (and the stabilization) do not depend on the closed surface integral wrapping a cell $\cell$ located on the symmetry axis.
However, this feature is necessary to prove the robustness of the HHO method to volumetric locking (see \ref{sec_appendix_gradient}).
In order to circumvent this problem, we consider infinitely thin cylindrical faces wrapping the symmetry axis, that are subjected to Dirichlet boundary conditions along the radial direction.
% Thus, in order to restore full mobility of a face located on the symmetry axis, we consider infinitely thin cylindrical faces wrapping it, that are subjected to Dirichlet boundary conditions along the radial direction.

% ---------------------------------------------------------
% ---- SECTION
% ---------------------------------------------------------
\section{Numerical examples in plane strain and tridimensional modelling hypotheses}
\label{sec_implementation_tridimensional_results}

The section showcases numerical examples
using the proposed implementation of the HHO method
for a cartesian modelling hypothesis.
The test cases under study, namely the
classical Cook membrane, and the radially loaded sphere test cases, show that no
volumetric locking is encountered.

% ---------------------------------------------------------
% -- SUBSECTION
% ---------------------------------------------------------
\subsection{Cook's membrane test case}

% ---------------------------------------------------------
% PARAGRAPH
% ---------------------------------------------------------
\paragraph{Specimen and loading}

Let consider the Cook membrane specimen that is subjected to uniaxial
traction (see Figure \ref{fig_cook_mesh}). The membrane has a width of $48$ mm and a height of $60$ mm.
A vertical traction $t = 3.125 \, 10^{8}$ N/m is imposed on the right of the specimen.
The HHO computation is run on a polygonal mesh (see Figure \ref{fig_cook}) and
is compared with standard QU4 and QU8 formulations (\textit{i.e.} linear
and quadratic approximations) on quadrangular meshes.

\begin{figure}[H]
    \centering
    \includegraphics[width=12.cm]{../chapter_002_hho_mechanics/drawings/cook_mesh.png}
    \caption{Geometry and boundary conditions for the Cook membrane test case}
    \label{fig_cook_mesh}
\end{figure}

% ---------------------------------------------------------
% PARAGRAPH
% ---------------------------------------------------------
\paragraph{Constitutive equation}

The same constitutive equation as that in \ref{sec:hho_meca:notched_bar} is
considered for the present test case.

% ---------------------------------------------------------
% PARAGRAPH
% ---------------------------------------------------------
\paragraph{Material parameters}

Materials parameters are taken as
$\sigma_0 = 450$ MPa, $\sigma_{\infty} = 715$ MPa with a saturation parameter $\delta = 16.93$. The Young modulus is $E = 206.9$ GPa, and the Poisson ratio is $\nu = 0.29$.

\begin{figure}[H]
    \centering
    \includegraphics[width=14.cm]{../chapter_002_hho_mechanics/drawings/cook.png}
    \caption{Hydrostatic pressure map one the reference configuration at the limit load}
    \label{fig_cook}
\end{figure}

% ---------------------------------------------------------
% PARAGRAPH
% ---------------------------------------------------------
\paragraph{Numerical results}

As expected, the linear and quadratic finite element methods display respectively strong and mild oscillations of the pressure, whereas the HHO one shows no sign of locking.

% ---------------------------------------------------------
% -- SUBSECTION
% ---------------------------------------------------------
\subsection{Indentation test case}

% ---------------------------------------------------------
% PARAGRAPH
% ---------------------------------------------------------
\paragraph{Specimen and loading}

The last test case consists in the three dimensional radially loaded sphere test case.
The sphere has an inner radius of $0.8$ mm, and a thickness of $0.2$mm.
A displacement of $0.2$mm is imposed on the inner surface (see Figure \ref{fig_cube}).

% ---------------------------------------------------------
% PARAGRAPH
% ---------------------------------------------------------
\paragraph{Material}

The same perfect plastic material as that in \ref{sec_swelling_sphere} is considered for the present test case.

\begin{figure}[H]
    \centering
    \includegraphics[width=14.cm]{../chapter_002_hho_mechanics/drawings/sphere_appendix.png}
    \caption{Hydrostatic pressure map one the reference configuration at the limit load}
    \label{fig_cube}
\end{figure}

% ---------------------------------------------------------
% PARAGRAPH
% ---------------------------------------------------------
\paragraph{Numerical results}

The pressure map at the end of the computation is displayed in Figure \ref{fig_cube}, and no sign of volumetric locking are present for the HHO computation, as for the axisymmetric case.
\input{../chapter_002_hho_mechanics/sections/08_appendix_rec_grad.tex}

\bibliography{../bib_all}
\bibliographystyle{fr-insa}


\end{document}
