\section{Volumetric locking}
\label{sec:hho:volumetric_locking}

Comme évoqué précédemment, un problème de verrouillage numérique peut apparaitre avec
les éléments finis H 1 -conformes (d’ordre bas) quand les déformations élastiques sont quasi-
incompressibles et/ou les déformations plastiques sont incompressibles. Une raison pour
l’apparition de ce verrouillage est que les éléments H 1 -conformes (d’ordre bas) ne sont pas
assez riches pour approcher correctement une telle solution. Pour résoudre cette difficulté
numérique, de nombreuses méthodes ont été développées. Nous présentons ci-dessous les
principales méthodes ainsi que leurs avantages et inconvénients. Nous commençons par les
méthodes les plus anciennes.
• Méthodes mixtes : L’idée est d’introduire un ou plusieurs multiplicateurs de Lagrange
pour imposer la condition d’incompressibilité. L’ajout de ces multiplicateurs augmente
le coût de la construction et de la résolution du problème global. En outre, celui-ci est
de type point-selle, ce qui peut compliquer la résolution avec une méthode itérative.
De plus, la formulation des méthodes mixtes est dépendante de la loi de comportement
utilisée et ces méthodes sont difficilement extensibles aux maillages polyédriques. Pour
les grandes déformations hyperélastiques, nous mentionnons les travaux de Al Akhrass
and al. [9], Brink and Stein [44], Dobrowolski [99], et Klass, Maniatty and Shephar [150]
et en ce qui concerne la plasticité, nous avons en petites déformations les travaux de
Cervera and al. [53, 52, 51] et Chimumenti and al. [60, 59] et en grandes déformations les
travaux de Al Akhrass and al. [9], Agelet de Saracibar and al. [6], Bargellini and al. [24],
Le Tallec, Rahier and Kaiss [157], de Souza Neto and al. [82], et Simo, Taylor and Pister
[196]. Plus récemment, une analyse de stabilité menée par Aurichhio and al. [16, 17,
15] a montré certaines limitations des méthodes mixtes pour les grandes déformations
hyperélastiques.
• Méthodes de Galerkin discontinu (dG) : Les méthodes dG (Arnold and al. [12],
Cockburn [73]) reposent sur une formulation primale comme les méthodes éléments
finis H 1 -conformes. En revanche, l’espace d’approximation n’est plus continu mais
typiquement polynomial par morceaux. Un inconvénient de ses méthodes dG est qu’elles
sont coûteuses car il y a beaucoup d’inconnues. En outre, dans le cas de la plasticité, le
comportement doit être calculé sur les faces en plus des cellules et la matrice tangente
de Newton n’est pas symétrique. Nous mentionnons pour les grandes déformations
hyperélastique, les travaux de Eyck and Lew [117], Eyck, Celiker and Lew [116, 115], et
Noels and Radovitzsky [174]. Les méthodes Interior penalty dG ont été développées pour
la plasticité classique en petites déformations par Hansbo [131] et Liu and al. [162],
et en grandes déformations par Liu, Wheeler and Yotov [163]; et en ce qui concerne la
plasticité à gradient en petites déformations, nous mentionnons les travaux de Djoko and
al. [98, 97] et en grandes déformations, ceux de McBride and Djoko [166].
• Méthodes d’intégration réduite ou sélective : La technique d’intégration réduite
(Zienkiewicz, Taylor and Too, [217]) consiste à intégrer la matrice de rigidité associée
aux éléments finis H 1 -conformes en utilisant une règle d’intégration d’ordre plus faible
que celui nécessaire pour une intégration exacte, alors que la méthode d’intégration
sélective (Doherty, Wilson and Taylor, [100]) consiste à ne sous-intégrer que les termes
de la matrice de rigidité liés à la contrainte d’incompressibilité. Ces deux méthodes
permettent de réduire en partie les problèmes de verrouillage (cf. Ponthot [179]) mais
présentent des difficultés pour les stabiliser (présence de modes à énergie nulle) et les
étendre à des matériaux anisotropes (cf. les travaux de Hughes [145]). Un lien a pu
être établi sous conditions avec certaines méthodes mixtes par Malkus and Hughes [165]
et l’opérateur de stabilisation des méthodes dG par Reese and al. [27, 182]. En ce qui
concerne les grandes déformations, nous mentionnons pour l’hyperélasticité, les travaux
de Adam and al. [5] et Reese, Küssner and Reddy [5, 183], et pour la plasticité, les travaux
de Reese [181].
• Méthodes Enhanced Assumed Strain (EAS) : Les méthodes EAS ont été intro-
duites par Simo and Rifai [194] où l’idée est de venir enrichir le tenseur des déformations
par l’ajout de variables additionnelles dont certaines peuvent être éliminées par con-
densation statique. Cependant, il est en général nécessaire de stabiliser ces méthodes.
Pour les grandes déformations, ces méthodes ont été utilisées dans les travaux de Eyck
and Lew [118] et Reese and Wriggers [184] en hyperélasticité et dans les travaux de Krysl
[152, 153] et Simo and al. [191, 192] en plasticité.
Depuis quelques années, d’autres méthodes que celles présentées ci-dessus ont été développées
pour résoudre entre autres le problème du verrouillage numérique.
• Méthodes isogéométriques (IGA) : L’idée des méthodes IGA est d’utiliser pour
la discrétisation des fonctions de base qui sont plus régulières que les polynômes de
Lagrange comme les NURBS. Un avantage sur les domaines à frontière courbe, la1.4. Grandes déformations et plasticité
représentation de la géométrie réelle est plus fine, ce qui permet de réduire l’erreur de
discrétisation géométrique. De manière générale, le prix à payer est un stencil plus
étendu et une forte dépendance entre le module géométrique et le solveur éléments
finis. Ces méthodes ont été développées pour la plasticité en petites déformations par
Elguedj and al. [112] et en grandes déformations par Elguedj and Hughes [113]. De plus,
une adaptation de l’intégration sélective aux méthodes IGA a été proposée par Adam
and al. [5].
• Méthodes de type Virtual Element Method (VEM) : Assez récemment, les
méthodes VEM (Beirão da Veiga, Brezzi, Marini and Russo, [28, 29]) ont été développées
afin d’étendre les éléments finis aux maillages polyédriques. Ces méthodes reposent
sur une formulation primale et elles sont robustes dans la limite incompressible. Nous
mentionnons en particulier la méthode VEM d’ordre bas pour les petites déformations
plastiques par Beirão da Veiga, Lovadina and Mora [30] (et son extension bidimensionnelle
d’ordre élevé par Artioli and al. [13]), alors que le cas des grandes déformations plastiques
est traité par Wriggers and Hudobivnik [209] en 2D et Hudobivnik, Aldakheel and Wriggers
[143] en 3D, toujours pour de l’ordre bas. Le cas hyperélastique est étudié par Chi,
Beirão da Veiga and Paulino [58] et Wriggers and al. [210].
• Méthodes hybrides : Ces dernières années, de nombreuse méthodes hybrides ont
été développées en plus de HHO. L’idée est de rajouter des variables globales pour
rendre les autres locales et donc éliminables par condensation statique. Ces méthodes
se différencient par le choix des variables globales et par le lien entre variables globales et
locales. Les méthodes Hybridizable Discontinuous Galerkin (HDG) [76] ont été utilisées
pour l’hyperélasticité par Kabaria, Lew and Cockburn [147] et Nguyen et Peraire [171],
alors que les méthodes hybrid dG (d’ordre bas) avec traces conformes et hybridizable
weakly conforming Galerkin avec traces non-conformes ont été étudiées dans le cadre
de la mécanique des solides linéaire par Krämer and al. [151] et dans le cas non-linéaire
par Bayat and al. [26] et Wulfinghoff and al. [212].
Nous terminons cette brève revue bibliographique en illustrant le phénomène de verrouillage
numérique sur la simulation d’un essai de traction en grandes déformations plastiques, voir
la Fig 1.11. Nous remarquons que la trace du tenseur des contraintes de Cauchy oscille pour
les éléments finis H 1 -conformes linéaires (a) et quadratiques (b) avec intégration complète,
alors que pour les éléments finis quadratiques sous-intégrés (c) et mixtes à trois champs (d),
le verrouillage est éliminé.