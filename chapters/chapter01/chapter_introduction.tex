\chapter{Introduction}

\section{Fibre-reinforced composite materials}

\textit{\textbf{general information: fibre-reinforced composites, failure mechanisms, defects, applications; context - pressure vessel safety}} \\

Fibre-reinforced composite materials consist of two phases: fibres and matrix. Fibres are responsible for carrying most loads. In long fibre composites, used in most high performance applications, the behaviour of the composite is highly anisotropic, with stiffness much higher in the fibre direction. Matrix carries part of the loads in the transverse directions, as well as plays a complimentary role under some other loading conditions, for instance controlling the onset of fibre kinking in axial compression. \\

Composite materials can be analyzed at different scales. First, the purely macroscopic structure scale. Secondly, at the level of individual unidirectional plies. Finally, one can consider the properties of the constituents (fibres and matrix) at the microscale. While structure scale is of most interest in engineering practice, it is the microscale phenomena that govern the damage and failure of the material. It is the gap between these two scales that adds complexity to the analysis of composites. \\

\section{Longitudinal tensile failure}

The failure of composites in axial direction is fibre-driven.

\pagebreak
%\clearpage

\section{Plan de la thèse}

\begin{itemize}
\item Introduction
\begin{itemize}
           \item[.] Enjeux et objectifs industriels
           \item[.] L'étude dans le cadre du projet ITN FiBreMod
           \item[.] Objectifs scientifiques et démarche
\end{itemize}

\item Les matériaux et les structures envisagés. Mise en évidence des problèmes à résoudre
\begin{itemize}
    \item[.]  Description succinte des structures d'application envisagées
    \item[.]  Description succinte des matériaux envisagés
    \item[.]  Les problèmes de calculs liés à la taille des phénomènes importants au premier ordre sur la rupture des structures
\end{itemize}

\item Rappel de l'existant : modèle de rupture de fibres
\begin{itemize}
    \item[.]  Le modèle existant
    \item[.]  Les validations et les calculs réalisés
    \item[.]  Les problèmes de calculs induits par la taille du phénomène
\end{itemize}

\item Une méthode pour réduire les temps de calcul et la taille des problèmes
\begin{itemize}
           \item[.] Rappel du concept d'ergodicité et du concept de portée intégrale
           \item[.] Application : validation et confirmation de la taille du VER
           \item[.] Application au cas des réservoirs hautes pressions
\end{itemize}

\item Tentative de réduction des temps de calcul et de la taille des problèmes
\begin{itemize}
    \item[.] Un problème de calcul de structure avec l'existant
    \item[.] Un problème de calcul de structure avec l'utilisation de la méthode ergodique
\end{itemize}

\item Calculs et confrontations expérimentales. Mise en place d'un outil de dimensionnement
\begin{itemize}
    \item[.] Cas d'un anneau : expérience et calcul
    \item[.] Cas d'un réservoir : expérience et calcul
    \item[.] Mise en place d'un outil de dimensionnement d'une structure composite à fibres continues
\end{itemize}

\item Conclusion
\begin{itemize}
    \item[.] Synthèse et conclusion de l'étude
    \item[.] Bilan de l'apport de l'étude dans le cadre du projet ITN FiBreMod
    \item[.] Perspectives
\end{itemize}

\end{itemize}