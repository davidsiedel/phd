
% ---------------------------------------------------------
% ---- SECTION
% ---------------------------------------------------------
\section{Conclusion}

% Dire de maniere explicte, mettre les éléments virtuels dans le même cadre -> dire quil reste à examiner le lien avec les éléments viertuels 
% Le HW permet de retrouver les principes HDG, HHO qui sont au coeur de ce paprier mais aussi les cG
% les VEM notn pas ete bconsideres, bien que le cadrez propos semblke sadapter à leur formalmisme
% descxirption du code, trouvaable sur github, avec chaque exemple

An introduction to HDG and HHO methods has been proposed, based on the minimization of a Hu-Washizu Lagrangian. The expression of the method arising from this approach allows to introduce naturally all the ingredients of the method, as well as the displacement discontinuity, in a unified framework.
This formulation also allows to draw a connection between HDG methods and other locking-free methods based on the minimization of a Hu-Washizu Lagrangian.
A natural cell-based resolution scheme emerged from this formulation, that has been tested and evaluated.
Finally, a HHO method to account for mechanical problems in the axisymmetric framework has been devised and evaluated numerically, for both linear thermoelastic behaviours, and plastic behaviours under both the small and finite strain hypotheses.
The proposed HHO method exhibits a robust behaviour to volumetric locking for strain-hardening plasticity as well as for perfect plasticity in primal formulation, with a moderate number of degrees of freedom.

This work can be pursued in several directions. One could use the cell resolution algorithm to address local resolution problems, such as those encountered with \textit{e.g.} damage irreversibility in phase field fracture mechanics, or multi field plasticity. Moreover, an adaptation of the HHO method to
reconstruct pressure-driven gradient terms only could lead to a simpler formulation, closer to that of mixed methods \cite{simo_quasi-incompressible_1991}.