\section{The Hybrid High Order method}

The aim of this paragraph is to introduce a new discretization method
for solving partial differential equations, the Hybrid High Order method
(HHO). The target audience is computational mechanicians with some
experience of the finite element method. As in the finite element
method, the body under consideration is discretized into elements. In
HHO, the elements are called cells and the boundaries of the elements
form the skeleton of the mesh.

This introduction is meant to introduce the key elements of the
HHO method:
\begin{itemize}
    \item The hybrid approach means that one considers two kind of
    displacements: the displacements in the cells (the elements) and the
    displacements of the skeleton (the boundaries).
    \item The displacements of the cells and the displacements of the boundaries
    are allowed to be discontinuous, \textit{i.e.} the restriction of the
    displacements on the cell on the boundary (its trace in the
    mathematical sense) is not equal to the displacements of the
    boundaries. In other words, \textit{displacement jumps} are allowed at the
    cells' boundaries.
    \item The displacements jumps are penalized through a so-called
    \textit{stabilization operator}.
    \item The displacement jumps at the boundaries allows the definition of an
    *enriched gradient*. Contrary to the finite element, the displacements
    and the gradients have the same order of approximation. This
    enriched gradient is the key to avoid issues like volumetric
    locking.
    \item A cell is in local equilibrium with its boundary. This will allow an
    important optimisation as the unknowns describing cell' displacements
    can be eliminated locally, a process called \textit{static condensation}.
\end{itemize}

This section is divided in two parts:
\begin{itemize}
    \item Section \ref{} provides an intuitive introduction that
    introduces all the previous concepts. Some mathematical aspects of the
    methods are eluded on purpose for the sake of simplicity.
    \item Evil is in the details. Section \ref{} adds
    rigorous complements to this intuitive introduction by considering
    some important details from the mathematical and numerical point of
    view.
\end{itemize}

\subsection{A intuitive introduction for the mechanicians}
\label{introduction}

\subsubsection{Main ideas of the finite element method}

The standard finite element is Garlerkin conforming approximation
method. More precisely, the domain is discretized in finite regions,
called elements. On each element, an approximated displacements field is
defined using a polynomial basis. Partial Differential Equations are
then expressed in weak form, \textit{i.e.} the principle of virtual work in
mechanics, and this weak form is restrained to the approximation space
of the continuous functions which can be represented on each element by
the associated polynomial basis.

The continuity requirement puts heavy constraints of the choice of the
polynomial basis and limits the geometry of the elements to mostly
simplexes (tetrahedra and triangles) and bricks.

The gradient of the unknowns, \textit{i.e.} the strain in small strain analysis
of the deformation gradient in finite strain analysis, which enters the
principle is defined as the derivative of each element of the polynomial
which approximate of the displacements field.

\subsubsection{Main ideas from the Discontinuous Galerkin and Hybrid Discontinous Galerkin approaches}

Discontinuous Galerkin approaches relaxes the previous continuity
requirement, allowing the approximated displacements fields on the cells
(the elements) to be discontinuous at the elements boundaries, but at
the same time, penalizes either the displacements jumps or the jumps in
normal stress. The choice of the penalisation must meet the important
requirement of being consistent, \textit{i.e.} the penalisation must behave
appropriately when the mesh sizes decreases to ensures optimal
convergence rates and avoid spurious effects.

Hybrid Discontinous Galerkin introduces the displacement of the
boundaries of elements as mediators between elements. Jumps between the
approximated displacements of the boundaries and the trace of the
approximated displacements on the cells at the boundaries are allowed.

\subsubsection{An intuitive derivation of the Hybrid High Order method}

Inspired by the ideas of standard finite element method and the ideas of
the Discontinuous Galerkin and Hybrid Discontinous Galerkin methods, let
us consider an element and its boundary $(\partial T)$. This
boundary may be the boundary of another element or be part of the
boundary of the considered body.

If the cell is considered as a continuous homogeneous mechanical
structure, the displacement field must be continuous at the boundary. To
allow a displacement jump, let us consider that the boundary is a thin
(with respect to the element size) interface $\Gamma$ of width $l$.
Let us denote:

- $T$ the interior of the element, such that the element is the union
  $T\,\cup\,\Gamma$.
- $\partial\,T_{+}$ the intersection of the $\Gamma$ and
  $paren{\partial\,T}$, \textit{i.e.}
  $\partial\,T_{+}=\Gamma\,\cap\,\partial\,T$.
- $\partial\,T_{-}$ the intersection of the $\Gamma$ and $T$, \textit{i.e.}
  $\partial\,T_{-}=T\,\cap\,\Gamma$.

See Figure for details. These notations underline the fact that, as
the length of the interface vanishes:

- $T$ tends to the whole element 
- $\partial\,T_{+}$ and $\partial\,T_{-}$ tend to $\partial\,T$.

\subsubsection{Discontinuity of the displacement field at the boundary}

\subsection{Mathematical complements}
\label{mathematical_complements}

\subsubsection{Restriction of the boundary of element to lines and planes}

\subsubsection{Approximation spaces}

\subsubsection{A more general definition of the stabilisation operator}

\subsubsection{Dirichlet boundary conditions}

\subsubsection{Convergence results in linear elasticity}

\subsection{A first implementation in \texttt{Python}}