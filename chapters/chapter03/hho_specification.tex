\section{The Principle of Virtual Works in the framework of the HHO method}
\label{PVW}

    \subsection{Specifying the local problem}

        Let $\mathcal{T}_h(\Omega)$ a mesh over the domain $\Omega$, and let $T$ a cell in $\mathcal{T}_h(\Omega)$, endowed with a displacement $\tensori{u}\subscript{\bm{T}}$. Let $\Gamma \subset \mathbb{R}^{d-1}$ denote the skeleton of $\Omega$, and $\mathcal{F}_h(\Gamma)$ its segmentation as defined by $\mathcal{T}_h(\Omega)$. Let $\Gamma_N$ the part of $\Gamma$ subjected to Neumann boundary conditions, and $\Gamma_D$ the part of $\Gamma$ subjected to Dirichlet boundary conditions. For any $T$ in $\mathcal{T}_h$, let $\partial T \subset \Gamma$ its boudary endowed with a displacement $\tensori{u}\subscript{\bm{\partial T}}$
        such that for any $F \subset \partial T$ :
        \begin{equation}
            \begin{aligned}
                \tensori{u}\subscript{\bm{\partial T}} : \partial T \ni \lighttensori{s} \mapsto \tensori{u}\subscript{\bm{F}}(\lighttensori{s}) \cdot \delta_F
            \end{aligned}
        \end{equation}
        where $\tensori{u}\subscript{\bm{F}}$ denotes the displacement of the face, and $\delta_F$ is the indicator function on $F$.
        Let $\llbracket \tensori{u}\subscript{\bm{\partial T}} \rrbracket = \tensori{u}\subscript{\bm{\partial T}}-\tensori{u}\subscript{\bm{T}}\vert_{\partial T}$ the displacement jump between the cell and its boundary.
        Let consider the continuous local problem :
        \begin{equation}
            \label{eq_continuous_problem_0}
            \begin{aligned}
                \int_{T} \hat{\tensorii{G}}\subscript{\bm{T}} : \tensorii{\Pi}\subscript{\bm{I}}(\tensorii{G}\subscript{\bm{T}})
                +
                \int_{\partial T}
                \frac{\beta}{h_{\partial T}}
                \llbracket \hat{\tensori{u}}\subscript{\bm{\partial T}} \rrbracket
                \cdot
                \llbracket {\tensori{u}}\subscript{\bm{\partial T}} \rrbracket
                =
                \int_{T} \hat{\tensori{u}}\subscript{\bm{T}} \cdot \tensori{F}\subscript{\bm{V}}
                &&
                &
                \mbox{in $T$}
            \end{aligned}
        \end{equation}
        Where $\tensorii{G}\subscript{\bm{T}}$ is the discrete gradient operator as defined in the introduction section, and where for any face $F$ in $\partial T$ :
        \begin{equation}
            \begin{aligned}
                h_{\partial T} : \partial T \ni \lighttensori{s} \mapsto h_F \cdot \delta_F \in \mathbb{R}
            \end{aligned}
        \end{equation}
        In particular, the face diameter $h_F$ has been introduced in the cohesive force term, in order to ensure stability of the method. Such a requirement derives from Nitche's formulation of enforcing Dirichlet boundary condtions in a weak sense, where the weak Dirichlet condition in the considered setting is the expression of the penalization of the displacement jumps across cells through the choesive traction force, whcih amounts to state that one seeks continuity in a weak sense.
        \newline
        Generalizing \eqref{eq_continuous_problem_0} to possible Dirichlet and Neumann boundary conditions where $\partial_N T$ denotes the part of $\partial T$ subjected to traction forces and $\partial_D T$ that to imposed displacements, one obtains the local PVW in the context of the HHO method :
        \begin{equation}
            \label{eq_continuous_problem_1}
            \begin{aligned}
                \int_{T} \hat{\tensorii{G}}\subscript{\bm{T}} : \tensorii{\Pi}\subscript{\bm{I}}(\tensorii{G}\subscript{\bm{T}})
                +
                \int_{\partial T}
                \frac{\beta}{h_{\partial T}}
                \llbracket \hat{\tensori{u}}\subscript{\bm{{\partial T}}} \rrbracket
                \cdot
                \llbracket {\tensori{u}\subscript{\bm{{\partial T}}}} \rrbracket
                =
                \int_{T} \hat{\tensori{u}}\subscript{\bm{T}} \cdot \tensori{F}\subscript{\bm{V}}
                +
                \int_{{\partial T}} \hat{\tensori{u}}\subscript{\bm{{\partial T}}} \cdot \tensori{T}\subscript{\bm{N}}
                &&
                &
                \mbox{in $T$}
                \\
                \tensori{u}\subscript{\bm{{\partial T}}} = \tensori{U}\subscript{\bm{D}}
                &&
                &
                \mbox{on $\partial_D T$}
            \end{aligned}
        \end{equation}
        where boundary conditions are applied on $\tensori{u}\subscript{\bm{\partial T}}$.

    \subsection{The global Principle of Virtual Works}

        The PVW over the whole structure $\Omega$ is then the assembly of all local problems :
        \begin{equation}
            \label{eq_continuous_problem_2}
            \begin{aligned}
                \sum_{T \in \mathcal{T}_h(\Omega)}
                \Bigg(
                \int_{T} \hat{\tensorii{G}}\subscript{\bm{T}} : \tensorii{\Pi}\subscript{\bm{I}}(\tensorii{G}\subscript{\bm{T}})
                +
                \int_{\partial T}
                \frac{\beta}{h_{\partial T}}
                \llbracket \hat{\tensori{u}}\subscript{\bm{\partial T}} \rrbracket
                \cdot
                \llbracket {\tensori{u}\subscript{\bm{\partial T}}} \rrbracket
                =
                \int_{T} \hat{\tensori{u}}\subscript{\bm{T}} \cdot \tensori{F}\subscript{\bm{V}}
                +
                \int_{\partial T} \hat{\tensori{u}}\subscript{\bm{\partial T}} \cdot \tensori{T}\subscript{\bm{N}}
                \Bigg)
                &&
                &
                \mbox{in $\Omega$}
                \\
                \forall F \in \mathcal{F}_h(\Gamma), \tensori{u}\subscript{\bm{F}} = \tensori{U}\subscript{\bm{D}}
                &&
                &
                % \mbox{on $\partial_D \Omega$}
                \mbox{on $\Gamma_D$}
            \end{aligned}
        \end{equation}

\section{Discretization}
\label{sec_discretization}

    Since the method is non-conformal, a particular functional space needs be considered : that of piece-wise continuous kinematically adissible displacements. Following the Galerkin method that defines the kinematic admissibility of a displacement field by the Sobolev space $H^1(\Omega;\mathbb{R}^d)$, one introduces the broken Sobolev space, which is the set of piece-wise $H^1$ functions in $\mathcal{T}_h(\Omega)$ :
    \begin{equation}
        H^1(\mathcal{T}_h;\mathbb{R}^d) = 
        \Bigg\{
            \tensori{v} \in L^2(\Omega;\mathbb{R}^d)
            \ \ \Bigg\vert \ \ 
            \forall T \in \mathcal{T}_h(\Omega), \tensori{v}\vert_T \in H^1(T;\mathbb{R}^d)
        \Bigg\}
    \end{equation}
    The natural approximation space for $H^1(\mathcal{T}_h(\Omega);\mathbb{R}^d)$ is that then that of piece-wise continuous polynomials of order $k$ in $\mathcal{T}_h(\Omega)$ :
    \begin{equation}
        P^l(\mathcal{T}_h;\mathbb{R}^d) = 
        \Bigg\{
            \tensori{v} \in L^2(\Omega;\mathbb{R}^d)
            \ \ \Bigg\vert \ \ 
            \forall T \in \mathcal{T}_h(\Omega), \tensori{v}\vert_T \in \mathbb{P}^l(T;\mathbb{R}^d)
        \Bigg\}
    \end{equation}
    Similarily, one introduces the approximation space for faces displacements :
    \begin{equation}
        P^k(\mathcal{F}_h;\mathbb{R}^d) = 
        \Bigg\{
            \tensori{v} \in L^2(\Gamma;\mathbb{R}^d)
            \ \ \Bigg\vert \ \ 
            \forall F \in \mathcal{F}_h(\Gamma), \tensori{v}\vert_F \in \mathbb{P}^k(F;\mathbb{R}^d)
        \Bigg\}
    \end{equation}
    One readily notices that the order of approximation of the displacement in a cell $T$ and in a face $F$ can be different. In particular, the polynomial order $l$ in a cell $T$ must be such that $k-1 \leq l \leq k+1$ where $k$ is the polynomial order of approximation in a face $F$.

\section{Stabilization}
    
    Hence, specifying $\tensori{u}\subscript{\bm{T}}$ in $P^l(\mathcal{T}_h(\Omega);\mathbb{R}^d)$ and $\tensori{u}\subscript{\bm{\partial T}}$ in $P^k(\mathcal{F}_h(\Omega);\mathbb{R}^d)$, the displacement jump can be re-written :
    \begin{equation}
        \llbracket \tensori{u}\subscript{\bm{\partial T}} \rrbracket
        =
        \Pi_{\partial T}^k(\tensori{u}\subscript{\bm{T}} \vert_{\partial T}) - \tensori{u}\subscript{\bm{\partial T}}
    \end{equation}
    where $\Pi_{\partial T}^k$ denotes the projector from $L^2(\Omega;\mathbb{R}^d)$ onto $P^k(\mathcal{F}(\Omega);\mathbb{R}^d)$. In the particular case where $l = k+1$, a $h^{k+1}$ convergence can be achieved. Otherwise, the convergence order is $h^k$.
    However, defining a more spfisticated form of displacement, involving a more sofisticated displacement field in the cell, one can ensure a $h^{k+1}$ convergence rate for any $l \in \llbracket k-1, k \rrbracket$.
    Let $\tensori{u}\subsupscript{\bm{T}}{\bm{\star}}$ denote such a displacement, that defines as the vector-valued field that solves :
    \begin{equation}
        \begin{aligned}
            \int_T \nabla^{s} \tensori{w} : \nabla^{s} \tensori{u}\subsupscript{\bm{T}}{\bm{\star}}
            =
            \int_T \nabla^{s} \tensori{w} : \nabla^{s} \tensori{u}\subscript{\bm{T}}
            +
            \int_{\partial T} \nabla^{s} \tensori{w} \lighttensori{n}\subscript{F} \cdot \llbracket \tensori{u}\subscript{\bm{\partial T}} \rrbracket
        \end{aligned}
    \end{equation}
    together with
    \begin{equation}
        \begin{aligned}
            \int_T \tensori{u}\subsupscript{\bm{T}}{\bm{\star}}
            =
            \int_T \tensori{u}\subscript{\bm{T}}
            &&
            \text{and}
            &&
            \int_T \nabla^{ss} \tensori{u}\subsupscript{\bm{T}}{\bm{\star}}
            =
            \int_{\partial T} \frac{1}{2} (\tensori{u}\subscript{\bm{\partial T}} \otimes \lighttensori{n}\subscript{F} - \lighttensori{n}\subscript{F} \otimes \tensori{u}\subscript{\bm{\partial T}})
        \end{aligned}
    \end{equation}
    Using $\tensori{u}\subsupscript{\bm{T}}{\bm{\star}}$, the penalization of the displacement jump writes as :
    \begin{equation}
        \begin{aligned}
            \llbracket \tensori{u}\subscript{\bm{\partial T}} \rrbracket
            =
            \big(
                \tensori{u}\subscript{\bm{\partial T}}
                -
                \Pi_{\partial T}^k(\tensori{u}\subsupscript{\bm{T}}{\bm{\star}}\vert_{\partial T})
            \big)
            -
            \big(
                \tensori{u}\subscript{\bm{T}}
                -
                \Pi_{T}^k(\tensori{u}\subsupscript{\bm{T}}{\bm{\star}})
            \big)
            \vert_{\partial T}
        \end{aligned}
    \end{equation}

\section{Unknown space}

    Following the discretization introduced in \ref{sec_discretization}, one defines the unknwon space for the global HHO unknown, namely the assembly of the cell unknown and of the $N_{\partial T}$ faces unknwon :
    \begin{equation}
        \begin{aligned}
            {P}^{k,l}_{\textnormal{HHO}}(T;\mathbb{R}^d)
            =
            \mathbb{P}^{l}(T;\mathbb{R}^d) \times \mathbb{P}^{k}(\partial T;\mathbb{R}^d)
            =
            \mathbb{P}^{l}(T;\mathbb{R}^d) \times
            \underbrace{
            \mathbb{P}^{k}(F_1;\mathbb{R}^d)
            \times ... \times
            \mathbb{P}^{k}(F_{N_{\partial T}};\mathbb{R}^d)}_{N_{\partial T}}
        \end{aligned}
    \end{equation}
    And the HHO global unknown is the vector :
    \begin{equation}
        \tensori{u}\subscript{\bm{T}, \bm{\partial T}}
        =
        \begin{pmatrix}
            \begin{aligned}
                \tensori{u}\subscript{\bm{T}}
                \\
                \tensori{u}\subscript{\bm{\partial T}}
            \end{aligned}
        \end{pmatrix}
        =
        \begin{pmatrix}
            \begin{aligned}
                \tensori{u}\subscript{\bm{T}}
                \\
                \tensori{u}\subscript{\bm{F_1}}
                \\
                \vdots
                \\
                \tensori{u}\subscript{\bm{F_{N_{\partial T}}}}
            \end{aligned}
        \end{pmatrix}
    \end{equation}