% ---------------------------------------------------------
% ---- SECTION
% ---------------------------------------------------------
\section{From the continuous Hu-Washizu Lagrangian to the HDG Lagrangian}
\label{sec_appendix_Hu_Washizu}

In this part, the development for the expression of the Hu-Washizu Lagrangian \eqref{eq_0015} is exposed, using the assumptions made in Section \ref{sec_assumtions}.

% ---------------------------------------------------------
% PARAGRAPH
% ---------------------------------------------------------
\paragraph{Element geometry}

In the following, the cell $\cell$ is assumed to be convex.
It is split into a core part $\Bulk \subset \cell$ with boundary $\dBulk$, and into an interface part $\Crown{} \subset \cell$ with boundary $\dCrown = \dBulk \cup \dCell$, as shown in Figure \ref{fig_02}. The interface $\Crown{}$ has some thickness $\ell > 0$ that is supposed to be small compared to $h_{\cell}$ the diameter of $\cell$.

% ---------------------------------------------------------
% PARAGRAPH
% ---------------------------------------------------------
\paragraph{Homotethic transformation}

Let $\tensori{\Xi}{}_{\cell}$ the homothety of ratio $(1 - \alpha \ell)$ and center $\tensori{X}{}_{\cell}$ the centroid of $\cell$, with $0 < \alpha < 1 / \ell$ such that $\Bulk$ (respectively $\dBulk$) is the image of $\cell$ (respectively $\dCell$) by $\tensori{\Xi}{}_{\cell}$. Since $\dBulk$ is an homothety of $\dCell$, any point $\tensori{X}{}_{\dCell} \in \dCell$ and $\tensori{X}{}_{\dBulk} = \tensori{\Xi}{}_{\cell}(\tensori{X}{}_{\dCell}) \in \dBulk$ share the same unit outward normal $\tensori{n}{}$.

% ---------------------------------------------------------
% PARAGRAPH
% ---------------------------------------------------------
\paragraph{Change of reference}

Let the change of frame $\tensori{\Psi}$ that takes a point from the reference frame to the local frame with origin on $\dBulk{}$, and whose first direction is given by the normal vector $\tensori{n}$ such that
%
%
%
\begin{equation}
    \tensori{\Psi} : \tensori{X} \mapsto \tensori{x} = \tensorii{Q}{} \tensori{X} + \tensori{c}
\end{equation}
%
%
%
where $\tensorii{Q}{}$ is the rotation matrix whose first row coincides with $\tensori{n}{}$, and $\tensori{c}$ is a constant vector.

% ---------------------------------------------------------
% PARAGRAPH
% ---------------------------------------------------------
\paragraph{Displacement in the interface}

Assuming that the interface $\Crown$ is thin enough (\textit{i.e.} that $\ell$ is small enough) let assume that the displacement $\tensori{u}{}_{\Crown{}}$ in $\Crown$ linearly bridges $\tensori{u}{}_{\Bulk{}} \vert_{\dBulk{}}$ to $\tensori{u}{}_{\dCell{}}$ such that
%
%
%
\begin{equation}
    \tensori{u}{}_{\Crown{}}(\tensori{x}) =
    \frac{
    \tensori{u}{}_{\dCell{}}(\tensori{\Psi}{}^{-1}(\tensori{x}{}_{\ell}))
    -
    \tensori{u}{}_{\Bulk{}} \vert_{\dBulk{}}(\tensori{\Psi}{}^{-1}(\tensori{x}{}_{o}))
    }
    {\ell}
    x_0
    +
    \tensori{u}{}_{\Bulk{}} \vert_{\dBulk{}}(\tensori{\Psi}{}^{-1}(\tensori{x}{}_{o}))
\end{equation}
%
%
%
where $x_0$ is the first coordinate of a point $\tensori{x}$ in the local frame defined by $\tensori{\Psi}$.
The vector $\tensori{x}{}_{o}$ denotes a point located in the plane $x_0 = 0$, and $\tensori{x}{}_{\ell}$ a point on the plane $x_0 = \ell$, such that they share the same coordinates on their respective planes.

% ---------------------------------------------------------
% PARAGRAPH
% ---------------------------------------------------------
\paragraph{Displacement gradient in the interface}

The derivative of $\tensori{u}{}_{\Crown{}}$ with respect to $\tensori{X}$ yields
%
%
%
\begin{equation}
    \frac{
        \partial \tensoro{u}{}_{\Crown{}i}
    }{
        \partial X_j
    }
    =
    \sum_k
    \frac{
        \partial \tensoro{u}{}_{\Crown{}i}
    }{
        \partial x_k
    }
    \frac{
        \partial x_k
    }{
        \partial X_j
    }
    =
    \frac{
    \tensoro{u}{}_{\dCell{}i}(\tensori{\Psi}{}^{-1}(\tensori{x}{}_{\ell}))
    -
    \tensoro{u}{}_{\Bulk{}i} \vert_{\dBulk{}}(\tensori{\Psi}{}^{-1}(\tensori{x}{}_{o}))
    }
    {\ell}
    Q_{0j}
\end{equation}
%
%
%
which reads
%
%
%
\begin{equation}
    \label{eq_grad_displacement_interface}
    \nabla \tensori{u}{}_{\Crown{}}(\tensori{X}) =
    \frac{
    \tensori{u}{}_{\dCell{}}(\tensori{X}{}_{\ell})
    -
    \tensori{u}{}_{\Bulk{}} \vert_{\dBulk{}}(\tensori{X}{}_{o})
    }
    {\ell}
    \otimes
    \tensori{n}{}
\end{equation}
%
%
%
where we have used the fact that the first row of the rotation matrix $\tensorii{Q}$ is given by $\tensori{n}$. The points $\tensori{X}{}_{o}$ and $\tensori{X}{}_{\ell}$ are located on the normal plane to $\tensori{n}$ on $\dBulk{}$ and $\dCell{}$ respectively, in the reference frame.

% ---------------------------------------------------------
% PARAGRAPH
% ---------------------------------------------------------
\paragraph{Stress in the interface}

As introduced in Section \ref{sec_composite_demo}, the stress $\tensorii{P}{}_{\Crown}$ is assumed constant along the direction $\tensori{n}{}$ in $\Crown{}$. By continuity of the traction force across $\dBulk$, the following equality holds true
%
% 
% 
\begin{equation}
    \label{eq_continuity_traction_force_2}
    \begin{aligned}
        (\tensorii{P}{}_{\Crown} - \tensorii{P}{}_{\Bulk} \vert_{\dBulk{}}) \cdot \tensori{n}{} =  0
        &&
        \text{in}
        &&
        \Crown{}
    \end{aligned}
\end{equation}

% ---------------------------------------------------------
% PARAGRAPH
% ---------------------------------------------------------
\paragraph{Internal Hu-Washizu in the interface}

Let $L_{\Crown{}, \text{int}}^{HW}$ the internal contribution of the Hu-Washizu Lagrangian in $\Crown{}$
%
%
%
\begin{equation}
    \label{eq22}
    \begin{aligned}
        L_{\Crown{}, \text{int}}^{HW}
        := &
        \int_{\Crown{}} \mecPotential{}_{\Crown} + (\nabla \tensori{u}{}_{\Crown} - \tensorii{G}{}_{\Crown}) : \tensorii{P}{}_{\Crown}
    \end{aligned}
\end{equation}
%
% 
%
Let $C_\Crown = \{ v \in L^2(\Crown) \ \vert \ v \cdot \tensori{n} = \text{cste} \}$ the set of $L^2$-functions which are constant along the normal axis in $\Crown$. For any function in $C_\Crown$, the following equality holds true:
%
% 
% 
\begin{equation}
    \label{eq_virtual_works0}
        \int_{\Crown} v \ dV
        =
        \int_{\dBulk{}} \int_{\epsilon = 0}^{\ell} v (1 - \alpha \epsilon) \ dS d \epsilon
        =
        \ell (1 - \frac{\alpha}{2} \ell) \int_{\dBulk{}} v \ dS
\end{equation}
%
% 
% 
Noticing that $\nabla \tensori{u}{}_{\Crown} \in C_\Crown$, one has :
%
% 
% 
\begin{equation}
    \begin{aligned}
        \int_{\Crown{}} \mecPotential{}_{\Crown}
        = & 
        \ell (1 - \frac{\alpha}{2} \ell)
        \int_{\dBulk{}} \frac{1}{2} \beta \frac{\ell}{h_{\cell}} \nabla \tensori{u}{}_{\Crown} : \nabla \tensori{u}{}_{\Crown}
        \\
        = & 
        \ell (1 - \frac{\alpha}{2} \ell)
        \int_{\dBulk{}} \frac{\beta}{2 \ell h_{\cell}} (\tensori{u}{}_{\dCell} - \tensori{u}{}_{\Bulk} \vert_{\dBulk}) \otimes
        \tensori{n} : (\tensori{u}{}_{\dCell} - \tensori{u}{}_{\Bulk} \vert_{\dBulk}) \otimes
        \tensori{n}
        \\
        = & 
        \ell (1 - \frac{\alpha}{2} \ell)
        \int_{\dBulk{}} \frac{\beta}{2 \ell h_{\cell}} \lVert \tensori{u}{}_{\dCell} - \tensori{u}{}_{\Bulk}{} \vert_{\dBulk} \lVert {}^2
        \\
        = & 
        (1 - \frac{\alpha}{2} \ell)
        \int_{\dBulk{}} \frac{\beta}{2 h_{\cell}} \lVert \tensori{u}{}_{\dCell} - \tensori{u}{}_{\Bulk}{} \vert_{\dBulk} \lVert {}^2
    \end{aligned}
\end{equation}
%
% 
% 
Moreover, for $\tensorii{P}{}_{\Crown}$ in $C_\Crown{}$ :
%
% 
% 
\begin{equation}
    \begin{aligned}
        \int_{\Crown{}} \nabla \tensori{u}{}_{\Crown} : \tensorii{P}{}_{\Crown}
        = &
        \ell (1 - \frac{\alpha}{2} \ell)
        \int_{\dBulk{}} \nabla \tensori{u}{}_{\Crown} : \tensorii{P}{}_{\Crown}
        \\
        = &
        \ell (1 - \frac{\alpha}{2} \ell)
        \int_{\dBulk{}}
        \frac{1}{\ell}
        (\tensori{u}{}_{\dCell} - \tensori{u}{}_{\Bulk}{} \vert_{\dBulk}) \otimes \tensori{n} : \tensorii{P}{}_{\Crown{}}
        \\
        = &
        (1 - \frac{\alpha}{2} \ell)
        \int_{\dBulk{}}
        (\tensori{u}{}_{\dCell} - \tensori{u}{}_{\Bulk}{} \vert_{\dBulk}) \cdot \tensorii{P}{}_{\Bulk{}} \vert_{\dBulk{}} \cdot \tensori{n}
    \end{aligned}
\end{equation}
% 
% 
% 
where we have used \eqref{eq_continuity_traction_force_2}. And Finally :
%
% 
% 
\begin{equation}
    \begin{aligned}
        L_{\Crown{}, \text{int}}^{HW}
        =
        (1 - \frac{\alpha}{2} \ell)
        \int_{\dBulk{}} \frac{\beta}{2 h_{\cell}} \lVert \tensori{u}{}_{\dCell{}} - \tensori{u}{}_{\Bulk{}} \vert_{\dBulk{}} \rVert^2
        +
        (1 - \frac{\alpha}{2} \ell)
        \int_{\dBulk} (\tensori{u}{}_{\dCell{}} - \tensori{u}{}_{\Bulk{}} \vert_{\dBulk{}}) \cdot \tensorii{P}{}_{\Bulk{}} \vert_{\dBulk{}} \cdot \tensori{n}{}
        -
        \int_{\Crown{}} \tensorii{G}{}_{\Crown{}} : \tensorii{P}{}_{\Crown{}}
    \end{aligned}
\end{equation}

% ---------------------------------------------------------
% PARAGRAPH
% ---------------------------------------------------------
\paragraph{Total Hu-Washizu Lagrangian in the composute element}

Injecting \eqref{eq22} in \eqref{eq_hu_washizu_split} yields
%
% 
% 
\begin{equation}
    \label{eq_0014}
    \begin{aligned}
        L_{\cell}^{HW}
        = &
        \int_{\Bulk} \mecPotential{}_{\bodyLag{}} + (\nabla \tensori{u}{}_{\Bulk} - \tensorii{G}{}_{\Bulk}) : \tensorii{P}{}_{\Bulk}
        % \\
        % &
        +
        (1 - \frac{\alpha}{2} \ell)
        % \Biggl(
        \int_{\dBulk{}} (\tensori{u}{}_{\dCell{}} - \tensori{u}{}_{\Bulk} \vert_{\dBulk}) \cdot \tensorii{P}{}_{\Bulk} \vert_{\dBulk} \cdot \tensori{n}{}
        % \\
        % &
        \\
        &
        +
        (1 - \frac{\alpha}{2} \ell)
        \int_{\dBulk{}} \frac{\beta}{2 h_T} \lVert \tensori{u}{}_{\dCell{}} - \tensori{u}{}_{\Bulk} \vert_{\dBulk{}} \rVert^2
        % \Biggr)
        % \\
        % &
        -
        \int_{\Crown{}} \tensorii{G}{}_{\Crown{}} : \tensorii{P}{}_{\Crown{}}
        % \\
        % &
        -
        \int_{\Bulk} \loadLag \cdot \tensori{u}{}_{\Bulk}
        -
        \int_{\Crown{}} \loadLag \cdot \tensori{u}{}_{\Crown{}}
        -
        \int_{\neumannCell{}} \neumannCellLoad{} \cdot \tensori{u}{}_{\dCell{}}
    \end{aligned}
\end{equation}
%
% 
% 
Since $\ell$ is arbitrary, let $\ell \rightarrow 0$,
the interface region vanishes such that $\Crown{} \rightarrow \emptyset, \Bulk{} \rightarrow \cell$ and $\dBulk{} \rightarrow \dCell$, and the expression of the Hu–Washizu functional over the region $\cell$ writes
% 
% 
%
\begin{equation}
    \label{eq_0015A}
    \begin{aligned}
        L_{\cell}^{HW}
        = &
        \int_{\cell{}} \mecPotential{}_{\bodyLag{}} + (\nabla \tensori{u}{}_{\cell{}} - \tensorii{G}{}_{\cell{}}) : \tensorii{P}{}_{\cell}
        % \\
        % &
        + \int_{\dCell{}} (\tensori{u}{}_{\dCell} - \tensori{u}{}_{\cell} \vert_{\dCell}) \cdot \tensorii{P}{}_{\cell} \vert_{\dCell{}} \cdot \tensori{n}{}
        % \\
        % &
        + \int_{\dCell} \frac{\beta}{2 h_{\cell}} \lVert \tensori{u}{}_{\dCell{}} - \tensori{u}{}_{\cell{}} \vert_{\dCell{}} \rVert^2
        \\
        &
        -
        \int_{\cell} \loadLag{} \cdot \tensori{u}{}_{\cell{}}
        -
        \int_{\neumannCell{}} \neumannCellLoad{} \cdot \tensori{u}{}_{\dCell{}}
    \end{aligned}
\end{equation}

which concludes the development of equation \eqref{eq_0015}.