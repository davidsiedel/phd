% ---------------------------------------------------------
% ---- SECTION
% ---------------------------------------------------------
\section{Implementation}
\label{sec_implementation}

% ---------------------------------------------------------
% -- SUBSECTION
% ---------------------------------------------------------
\subsection{Shape functions}
\label{sec_shape_functions}

Since the displacement is discontinuous, the usual Lagrange basis functions are not necessarily needed for the description of the discrete displacement, gradient ans stress fields. A natural choice amounts to choose monomial basis functions.

% ---------------------------------------------------------
% PARAGRAPH
% ---------------------------------------------------------
\paragraph{Monomial basis functions}

Let $\mathcal{M}_s^l$ the set of natural integer vectors in a $s$-dimensional euclidean space such that
%
%
%
\begin{equation}
    \begin{aligned}
        \mathcal{M}_s^l =
        % \{
        \bigg\{
            \mathcal{M}_{s,p} && \vert && 0 \leq p \leq l
        \bigg\}
        &&
        \text{with}
        &&
        \mathcal{M}_{s,p} =
        \bigg\{
            \tensori{\alpha}{}_m && \vert && \sum_{1 \leq j \leq s} \alpha_{mj} = p
        \bigg\}
        % \}
    \end{aligned}
\end{equation}
%
%
%
and denote $M_s^l$ the cardinality of $\mathcal{M}_s^l$.
Let $D$ some $s$-dimensional euclidean domain, $1 \leq s \leq 3$, and $M^l(D)$ the monomial basis function of order $l$ on $D$.
The value of a scalar polynomial field $a_{M}^l \in M^l(D)$ at some point $\tensori{X} \in D$ is given by
%
%
%
\begin{equation}
    \label{eq_basis_fun3}
    \begin{aligned}
        a_{M}^l(\tensori{X}) = \sum_{0 \leq p \leq M_s^l} a_{p} \prod_{1 \leq j \leq s} \frac{(X_j - X_{Dj})^{\alpha_{pj}}}{h_D}
        &&
        \text{with}
        &&
        \tensori{\alpha}{}_p \in \mathcal{M}_{s}^l
    \end{aligned}
\end{equation}
%
%
%
where $\tensori{X}{}_D$ denotes the centroid of $D$, and the coefficients $a_p$ in $M^l(D)$ form a vector $\mathfrak{A}_D^l$ of size $M_s^l$.

% ---------------------------------------------------------
% PARAGRAPH
% ---------------------------------------------------------
\paragraph{Cell and faces approximation space sizes}

The number of degrees of freedom for a scalar field depends on the polynomial order. An overview of the values taken using monomial shape functions is given in tables \ref{table_num_dofs_2D} and \ref{table_num_dofs_3D} for both a cell and a face up to an approximation of order $4$
%
%
%
\begin{table}[H]
\centering
\begin{tabular}{||c c c||} 
    \hline
    polynomial order & cell dofs & face dofs \\ [0.5ex] 
    \hline\hline
    $0$ & 1 & 1
    \\ \hline
    $1$ & 3 & 2
    \\ \hline
    $2$ & 6 & 3
    \\ \hline
    $3$ & 10 & 4
    \\ \hline
    $4$ & 15 & 5
    \\ \hline
\end{tabular}
\caption{Number of cell and faces degrees of freedom for a scalar unknown in two dimension}
\label{table_num_dofs_2D}
\end{table}
%
%
%
\begin{table}[H]
\centering
\begin{tabular}{||c c c||} 
    \hline
    polynomial order & cell dofs & face dofs \\ [0.5ex] 
    \hline\hline
    $0$ & 1 & 1
    \\ \hline
    $1$ & 4 & 3
    \\ \hline
    $2$ & 10 & 6
    \\ \hline
    $3$ & 19 & 10
    \\ \hline
    $4$ & \textcolor{blue}{XX} & 15
    \\ \hline
\end{tabular}
\caption{Number of cell and faces degrees of freedom for a scalar unknown in three dimension}
\label{table_num_dofs_3D}
\end{table}
%
%
%
The need to eliminate cell unknowns is hence justified by observing that the number of degrees of freedom in cells grows rapidly with the approximation order as compared to that in faces.

% ---------------------------------------------------------
% -- SUBSECTION
% ---------------------------------------------------------
\subsection{Stabilization}
\label{sec_stabilization}

In this section, approximation spaces for unknowns of the global problem are described, which leads to several choices in terms of definition of the stabilization. Depending on such a choice, one recovers either the HDG method, or the HHO one.

% ---------------------------------------------------------
% PARAGRAPH
% ---------------------------------------------------------
\paragraph{Discrete functional space}

Discrete spaces are chosen such that
%
%
%
\begin{equation*}
    \begin{aligned}
        \discreteDisplacementSpaceCell = P^l(\cell, \mathbb{R}^{d})
        &&
        \discreteDisplacementSpaceDCell = P^k(\dCell, \mathbb{R}^{d})
        &&
        \discreteGradSpaceCell = P^k(\cell, \mathbb{R}^{d \times d})
        &&
        \discreteStressSpaceCell = P^k(\cell, \mathbb{R}^{d \times d})
    \end{aligned}
\end{equation*}
%
%
%
where $P^a(D, \mathbb{R}^{m})$ denotes the space of $m-$ valued polynomials in $D$, spanned by the monomial basis $M^a(D)$. In particular, one notices that the cell displacement polynomial order $l$ might be chosen different from the face displacement order $k$ such that $k - 1 \leq l \leq k + 1$.

% ---------------------------------------------------------
% PARAGRAPH
% ---------------------------------------------------------
\paragraph{HDG stabilization}

Accounting for the possible different polynomial order between the cell and faces, one can specify a discrete jump function in a natural way such that it delivers the displacement difference point-wise for any displacement pair $(\tensori{v}{}_{\cell}^l, \tensori{v}{}_{\dCell}^k) \in U^h(\cell) \times V^h(\dCell)$
%
%
%
\begin{equation}
    \begin{aligned}
        \tensori{J}{}_{\dCell}^{HDG}(\tensori{v}{}_{\cell}^l, \tensori{v}{}_{\dCell}^k) = \Pi^k_{\dCell{}} (
            \tensori{v}{}_{\dCell}^k - \tensori{v}{}_{\cell}^l \vert_{\dCell}
        )
    \end{aligned}
\end{equation}
%
%
%
where $\Pi^k_{\dCell{}}$ denotes the orthogonal projector onto $\discreteDisplacementSpaceDCell$.
This straightforward discrete jump function is at the origin of Hybrid Discontinuous Galerkin methods, and grants a convergence of order $k$ in the energy norm.

% ---------------------------------------------------------
% PARAGRAPH
% ---------------------------------------------------------
\paragraph{HHO stabilization}

A richer discrete jump function $\tensori{J}{}_{\dCell}^{HHO}$ providing a convergence of order $k + 1$ in the energy norm was introduced in \cite{di_pietro_discontinuous-skeletal_2015}, hence giving the Hybrid High Order method its name, such that
%
%
%
\begin{equation}
    \label{eq_hho_stabilization_vector}
    \begin{aligned}
        \tensori{J}{}_{\dCell}^{HHO}(\tensori{v}{}_{\cell}^l, \tensori{v}{}_{\dCell}^k) = \Pi^k_{\dCell{}} (
            \tensori{v}{}_{\dCell}^k - \tensori{v}{}_{\cell}^l \vert_{\dCell}
            -
            (
                (\tensoro{I}{}_{\cell}^{k + 1} - \Pi_{\cell}^k) (
                    \tensori{w}{}_\cell^{k + 1}
                )
            ) \vert_{\dCell{}}
        )
    \end{aligned}
\end{equation}
%
%
%
where $\Pi_{\cell}^k$ is the projector onto $P^{k}(\cell, \mathbb{R}^{d})$, $\tensoro{I}{}_{\cell}^{k + 1}$ is the identity function in $\discretePotentialSpaceCell = P^{k + 1}(\cell, \mathbb{R}^{d})$.

% ---------------------------------------------------------
% PARAGRAPH
% ---------------------------------------------------------
\paragraph{Reconstructed higher order displacement}

The term $\tensori{w}{}_{\cell}^{k+1}$ in \eqref{eq_hho_stabilization_vector}
% $ \in \discretePotentialSpaceCell$
denotes a higher order discrete displacement in $\discretePotentialSpaceCell$ that solves for any displacement pair $(\tensori{v}{}_{\cell}^l, \tensori{v}{}_{\dCell}^k) \in \discreteHybridDisplacementSpaceCell$
%
%
%
\begin{subequations}
    \label{eq_potential}
        \begin{alignat}{3}
            \int_\cell \nabla \tensori{w}{}_{\cell}^{k+1} : \nabla \tensori{d}{}_{\cell}^{k+1}
            & =
            \int_\cell \nabla \tensori{v}{}_{\cell}^l : \nabla \tensori{d}{}_{\cell}^{k+1}
            +
            \int_{\dCell} (\tensori{v}{}_{\dCell}^k - \tensori{v}{}_{\cell}^l) \cdot \nabla \tensori{d}{}_{\cell}^{k+1} \cdot \tensori{n}{}
            \ \ \ \ \ \ \ \ 
            &&
            \forall \tensori{d}{}_{\cell}^{k+1} \in \discretePotentialSpaceCell
            \label{eq_potential:eq0}
            \\
            \int_\cell \tensori{w}{}_{\cell}^{k+1} & = \int_\cell \tensori{v}{}_{\cell}^{l}
            \label{eq_potential:eq1}
    \end{alignat}
\end{subequations}

% ---------------------------------------------------------
% -- SUBSECTION
% ---------------------------------------------------------
\subsection{Algebraic formulation}
\label{sec_implementation2}

% ---------------------------------------------------------
% PARAGRAPH
% ---------------------------------------------------------
\paragraph{Unknown vector}

A cell displacement component unknown is represented by the coefficient vector $\mathfrak{U}_{\cell}^l$ in $M^l(\cell)$, and a face displacement component unknown is represented by the coefficient vector $\mathfrak{U}_{F}^k$ in $M^k(F)$, for any $F \subset \dCell$.
The global element displacement unknown vector of size $d M^l_d + d N_{\dCell} M^k_{d - 1}$ is
denoted $\mathfrak{U}_{\ClosedCell{}}$ and is the collection of all cell and faces displacement component vectors.
In the following, the cell coefficients in $\mathfrak{U}_{\ClosedCell{}}$ are denoted $\mathfrak{U}_{\cell}$, and the boundary coefficients $\mathfrak{U}_{\dCell}$.

% ---------------------------------------------------------
% PARAGRAPH
% ---------------------------------------------------------
\paragraph{Quadrature}

Integrals are evaluated numerically by means of a quadrature rule on an element shape. Hence, let ${Q}_D$ a quadrature rule for the domain $D$ of order at least $2k$. A quadrature point is denoted $\tensori{X}{}_q$ and a quadrature weight $w_q$.

% ---------------------------------------------------------
% PARAGRAPH
% ---------------------------------------------------------
\paragraph{Reconstructed gradient operator}

From an algebraic standpoint, \eqref{eq_grad} defines a linear problem
consisting in inverting a mass matrix in $\discreteGradSpaceCell{}$. One can thus define 
$
{\mathbb{B}}{}_{\cell}
$
the discrete gradient operator acting on the displacement unknowns vector $\mathfrak{U}{}_{\ClosedCell}$ at a quadrature point $\tensori{X}{}_q \in \cellQuadrature$, and ${\mathfrak{G}}{}_{\cell}^k$ the vector representation of the reconstructed gradient $\tensorii{G}{}_{\cell}^k(\tensori{v}{}_{\cell}^l, \tensori{v}{}_{\dCell}^k)$ such that
%
%
%
\begin{equation}
    \label{eq_discrete_gradient_vector}
    \begin{aligned}
        {\mathfrak{G}}{}_{\cell}^k
        (\tensori{X}{}_q)
        =
        {\mathbb{B}}{}_{\cell}
        (\tensori{X}{}_q)
        \cdot
        {\mathfrak{U}}{}_{\ClosedCell}
    \end{aligned}
\end{equation}
%
%
%
where ${\mathbb{B}}{}_{\cell}$ is composed by a cell block and a boundary block of respective size $dM_d^l$ and $dM_{d - 1}^k$.

% ---------------------------------------------------------
% PARAGRAPH
% ---------------------------------------------------------
\paragraph{Stabilization operator}

Similarly, the algebraic realization of \eqref{eq_hho_stabilization_vector} amounts to compute the stabilization operator ${\mathfrak{J}}{}_{\cell}$ such that 
%
%
%
\begin{equation}
    \label{eq_discrete_stabilization_vector}
    \begin{aligned}
        % {\mathcal{J}}{}_{\dCell}^{HHO}
        {\mathfrak{J}}{}_{\dCell}^{HHO}
        =
        {\mathbb{J}}{}_{\cell}
        \cdot
        {\mathfrak{U}}{}_{\ClosedCell}
    \end{aligned}
\end{equation}
%
%
%
where, as for ${\mathbb{B}}{}_{\cell}$, the operator ${\mathbb{J}}{}_{\cell}$ is composed by a cell and a boundary block with same respective sizes.

% ---------------------------------------------------------
% PARAGRAPH
% ---------------------------------------------------------
\paragraph{Offline computation}

Since \eqref{eq_grad} and \eqref{eq_hho_stabilization_vector} depend on the geometry of the element $\cell$ only, one can compute the operators $\mathbb{B}{}_{\cell}$ and $\mathbb{Z}{}_{\cell}$ for each element once and for all in an offline pre-computation step by working in the reference configuration. Once this offline step is performed, the algebraic form of the problem resembles closely to the standard finite element one, where the operator $\mathbb{B}{}_{\cell}$ replaces the usual shape function gradient operator, and the stabilization operator 
$\mathbb{Z}{}_{\cell}$ is incorporated in the expression of the tangent matrix and in that of internal forces.

% ---------------------------------------------------------
% PARAGRAPH
% ---------------------------------------------------------
\paragraph{Internal forces} The internal forces vector $\mathfrak{F}_{\ClosedCell}^{int}$ depends on the product of the stress values with the gradient operator computed at quadrature points, plus a supplementary force corresponding to that of the traction between the cell and its boundary
%
%
%
\begin{equation}
    \label{eq_internal_forces}
    \begin{aligned}
    \mathfrak{F}_{\ClosedCell}^{int} (\mathfrak{U}{}_{\ClosedCell})
    = &
    \sum_{\tensori{X}{}_q \in \cellQuadrature{}}
    (w_q
    {\mathbb{B}}{}_{\cell}^{t}(\tensori{X}{}_q) \cdot
    {\mathfrak{P}}{}_{\cell}^k(\tensori{X}{}_q, \mathfrak{U}{}_{\ClosedCell})
    )
    +
    \frac{\beta}{h_T}
    {\mathbb{J}}{}_{\cell}^t
    \cdot
    {\mathbb{J}}{}_{\cell}
    \cdot
    {\mathfrak{U}}{}_{\ClosedCell}
    \end{aligned}
\end{equation}
%
%
%
where ${\mathfrak{P}}{}_{\cell}^k$ denotes the vector representation of $\tensorii{P}{}_{\cell}^k$, and the superscript $\{\cdot\}^t$ denotes the transposition operator.

% ---------------------------------------------------------
% PARAGRAPH
% ---------------------------------------------------------
\paragraph{External forces}

The external forces vector $\mathfrak{F}_{\ClosedCell}^{ext}$ is the evaluation of the given bulk and boundary loads at respective cell and face quadrature points tested against the respective cell and face shape functions, such that
%
%
%
\begin{equation}
    \label{eq_external_forces}
    \begin{aligned}
        \mathfrak{F}_{\cell}^{ext}
        =
        \sum_{\tensori{X}{}_q \in \cellQuadrature{}}
        (w_q
        \loadLag{}(\tensori{X}{}_q) \cdot
        {\mathfrak{U}}{}_{\cell}^l
        )
        &&
        \text{and}
        &&
        \mathfrak{F}_{\dCell}^{ext}
        =
        \sum_{\tensori{X}{}_q \in Q_{\dCell}}
        (w_q
        \neumannLag{}(\tensori{X}{}_q) \cdot
        {\mathfrak{U}}{}_{\dCell}^k
        )
    \end{aligned}
\end{equation}

% ---------------------------------------------------------
% PARAGRAPH
% ---------------------------------------------------------
\paragraph{Tangent matrix and resiudal}

The elementary residual vector $\mathfrak{R}_{\ClosedCell}$ is such
that $\mathfrak{R}_{\ClosedCell}(\mathfrak{U}{}_{\ClosedCell}) =
\mathfrak{F}_{\ClosedCell}^{int}(\mathfrak{U}{}_{\ClosedCell}) -
\mathfrak{F}_{\ClosedCell}^{ext}$. The tangent matrix
$\mathbb{K}_{\ClosedCell}(\mathfrak{U}{}_{\ClosedCell})$ expresses the
derivative of $\mathfrak{R}_{\ClosedCell}(\mathfrak{U}{}_{\ClosedCell})$
with respect to $\mathfrak{U}{}_{\ClosedCell}$, and writes such that % %
%
\begin{equation}
  \label{eq_stiffness_operator}
  \begin{aligned}
    \mathbb{K}_{\ClosedCell}(\mathfrak{U}{}_{\ClosedCell})
    = \sum_{\tensori{X}{}_q \in \cellQuadrature{}} (w_q
    {\mathbb{B}}{}_{\cell}^{t}(\tensori{X}{}_q) \cdot
    \mathbb{A}(\tensori{X}{}_q, \mathfrak{U}{}_{\ClosedCell}) \cdot
    {\mathbb{B}}{}_{\cell}(\tensori{X}{}_q) ) + \frac{\beta}{h_T}
    {\mathbb{J}}{}_{\cell}^t \cdot {\mathbb{J}}{}_{\cell}
  \end{aligned}
\end{equation}
% % % where $\mathbb{A}$ is the matrix representation of the
fourth-order tensor $\tensoriv{A}{} = \partial \tensorii{P}{}_{\cell}^k
/ \partial \tensorii{G}{}_{\cell}^k$. The matrix
$\mathfrak{R}_{\ClosedCell}(\mathfrak{U}{}_{\ClosedCell})$ is block
decomposable such that % % %
\begin{equation}
  \label{eq_stiffness_operator_blocks}
  \begin{aligned}
    \mathbb{K}_{\ClosedCell}(\mathfrak{U}{}_{\ClosedCell})
    =
    \begin{pmatrix}
      \mathbb{K}_{\cell
        \cell}(\mathfrak{U}{}_{\ClosedCell}) && \mathbb{K}_{\cell
        \dCell}(\mathfrak{U}{}_{\ClosedCell}) \\
      \mathbb{K}_{\dCell
        \cell}(\mathfrak{U}{}_{\ClosedCell}) && \mathbb{K}_{\dCell
        \dCell}(\mathfrak{U}{}_{\ClosedCell})
    \end{pmatrix}
  \end{aligned}
\end{equation}
which leads to the following algebraic expression of
both~\eqref{eq_cell_equilibrium_3}
and~\eqref{eq_static_condensation_final}~:
\begin{equation}
  \label{eq_condensation_matrix}
  \begin{aligned}
    \frac{d \mathfrak{R}_{\mathcal{F}}}{d
      \mathfrak{U}_{\mathcal{F}}} = \frac{d
      \mathfrak{R}_{\mathcal{F}}^c}{d \mathfrak{U}_{\mathcal{F}}} =
    \mathbb{K}_{\cell \dCell}(\mathfrak{U}{}_{\ClosedCell}) -
    \mathbb{K}_{\dCell \cell}(\mathfrak{U}{}_{\ClosedCell})
    \mathbb{K}_{\cell \cell}(\mathfrak{U}{}_{\ClosedCell})^{-1}
    \mathbb{K}_{\cell \dCell}(\mathfrak{U}{}_{\ClosedCell})
  \end{aligned}
\end{equation}

% ---------------------------------------------------------
% -- SUBSECTION
% ---------------------------------------------------------
\subsection{Operators in the axi-symmetric framework}
\label{sec_appendix_axi}

This part specifies the formulation of HHO operators in the axi-symmetric framework.

% ---------------------------------------------------------
% PARAGRAPH
% ---------------------------------------------------------
\paragraph{Reconstructed gradient}

For any displacement pair $(\tensori{v}{}_{\cell}^l, \tensori{v}{}_{\dCell}^k) \in \discreteDisplacementSpaceCell{} \times \discreteDisplacementSpaceDCell{}$, the component $\tensoro{G}{}_{\cell \theta \theta}(\tensoro{v}{}_{\cell r}, \tensoro{v}{}_{\dCell r})$ solves
%
%
%
\begin{equation}
    \label{axi_symmetric_gradient_theta}
    \begin{aligned}
        \int_{\cell} 2 \pi r \tensoro{G}{}_{\cell \theta \theta}(\tensoro{v}{}_{\cell r}, \tensoro{v}{}_{\dCell r}) \tensoro{\tau}{}_{\cell \theta \theta}
        =
        \int_{\cell} 2 \pi r \frac{\tensoro{u}{}_{\cell r}}{r} \tensoro{\tau}{}_{\cell \theta \theta}
        =
        \int_{\cell} 2 \pi \tensoro{u}{}_{\cell r} \tensoro{\tau}{}_{\cell \theta \theta}
        &&
        \forall \tensorii{\tau}{}_{\cell} \in \stressSpaceCell
    \end{aligned}
\end{equation}
%
%
%
In the radial and ordonal directions, \textit{i.e.} $\forall i, j \in \{ r,z \}$, the expression given in \eqref{eq_grad} is retrieved, and the component $G_{\cell ij}(\tensoro{v}{}_{\cell i}, \tensoro{v}{}_{\dCell i})$ solves
%
%
%
\begin{equation}
    \label{axi_symmetric_gradient_rz}
    \begin{aligned}
    \int_{\cell} 2 \pi r G_{\cell ij}(\tensoro{v}{}_{\cell i}, \tensoro{v}{}_{\dCell i}) \tau_{\cell ij} =
    \int_{\cell} 2 \pi r \frac{\partial \tensoro{u}{}_{\cell i}}{\partial j} \tau_{ij} -
    \int_{\dCell} 2 \pi r (u_{\dCell i} - u_{\cell i} \vert_{\dCell}) \tau_{\cell ij} \vert_{\dCell} n_{j}
    &&
    \forall \tensorii{\tau}{}_{\cell} \in \stressSpaceCell
    \end{aligned}
\end{equation}

% ---------------------------------------------------------
% PARAGRAPH
% ---------------------------------------------------------
\paragraph{Reconstructed higher order displacement}

For any $\tensori{d}{}_{\cell}^{k + 1} \in \discretePotentialSpaceCell$, the radial component $w^{k+1}_{\cell r}$ solves
%
%
%
\begin{equation}
    \label{axi_symmetric_potential_r}
    \begin{aligned}
        \int_{\cell} 2 \pi r (\sum_{i \in \{ r,z \}} \frac{\partial w^{k+1}_{\cell r}}{\partial i} \frac{\partial d^{k+1}_{\cell r}}{\partial i} + \frac{w^{k+1}_{\cell r}}{r} \frac{d^{k+1}_{\cell r}}{r})
        = &
        \int_{\cell} 2 \pi r (\sum_{i \in \{ r,z \}} \frac{\partial u_{\cell r}}{\partial i} \frac{\partial d^{k+1}_{\cell r}}{\partial i} + \frac{u_{\cell r}}{r} \frac{d^{k+1}_{\cell r}}{r})
        \\
        &
        +
        \int_{\dCell} 2 \pi r \sum_{i \in \{ r,z \}} (u_{\dCell r} - u_{\cell r} \vert_{\dCell}) \frac{\partial d^{k+1}_{\cell r}}{\partial i} \vert_{\dCell} n_{i}
    \end{aligned}
\end{equation}
%
%
%
where the mean value condition is not needed on the radial component of the higher order displacement since the left hand side of the system described by \eqref{axi_symmetric_potential_r} depends directly on the displacement unknown and not only on its gradient as in \eqref{axi_symmetric_potential_z}.
The ordinate component $w^{k+1}_{\cell z}$ solves :
%
%
%
\begin{subequations}
    \label{axi_symmetric_potential_z}
        \begin{alignat}{3}
            \int_{\cell} 2 \pi r \sum_{i \in \{ r,z \}}
            \frac{\partial w^{k+1}_{\cell z}}{\partial i} \frac{\partial d^{k+1}_{\cell z}}{\partial i}
            = &
            \int_{\cell} 2 \pi r \sum_{i \in \{ r,z \}} \frac{\partial u_{\cell z}}{\partial i} \frac{\partial d^{k+1}_{\cell z}}{\partial i}
            -
            \int_{\dCell} 2 \pi r \sum_{i \in \{ r,z \}} (u_{\dCell z} - u_{\cell z} \vert_{\dCell})
            \frac{\partial d^{k+1}_{\cell z}}{\partial i} \vert_{\dCell} n_{i}
            \\
            \int_{\cell} 2 \pi r w^{k+1}_{\cell z} = & \int_{\cell} 2 \pi r u_{\cell z}
        \end{alignat}
\end{subequations}